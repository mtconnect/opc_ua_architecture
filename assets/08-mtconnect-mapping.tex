\clearpage
\section{Mapping the MTConnect Information Model to OPC UA} 
\label{mtconnect-mapping}

This section describes a \gls{uml} representation of MTConnect semantic data models for mapping MTConnect into \gls{opc} \gls{ua}. More detailed information is provided in Section \ref{mtconnect_information_model} for each data type.

OPC UA defines abstractions representing data, relationships, and events from devices. The abstractions do not provide the semantic meaning; they provide a structure to convey the meta-data and the values as they change. The OPC UA model has the base building blocks to represent an ontological model where the specific ontology is povided by companion specification for a specific domain.

MTConnect has similar capabilities but uses a different structural model where the meta-data and the streaming values are in separate documents to normalize the data flow in a similar way that many publish-subscribe protocols separate the structure from the data. MTConnect also supports a store-and-forward capability like many message brokers in a \gls{mom} architecture to enable resilience and recovery of data in the event of connectivity problems.

When translating from MTConnect to \gls{opc} \gls{ua}, the MTConnect abstractions of \glspl{MTDataItem} are converted using the OPC UA  \gls{DataVariable} abstractions as given in \cite{UAPart8}.  The relationships are mapped to multiple \gls{DataVariable} types where the category and the type determine the correct mapping. Conditions are mapped a sub-type of the OPC UA \gls{ConditionType} in a similar way. The behavior of the OPC UA \glspl{Condition} can be found in \cite{UAPart9}.

\subsection{MTConnect UML Representation of OPC}

\FloatBarrier

\subsection{MTConnect Information Model}

The MTConnect information model has the following abstractions:

\begin{enumerate}
  \item \glspl{MTComponent}
  \item \glspl{MTDataItem}
  \item \gls{Configuration}
  \item \glspl{Composition}
  \item \glspl{Asset}
\end{enumerate}

The first concern of the MTConnect OPC UA companion specification is the \gls{MTDevice} model covered in MTConnect \cite{MTCPart2} and \cite{MTCPart3}. The top-level \gls{MTComponent} of any MTConnect information model is the \gls{MTDevice}. A \gls{MTComponent} represents a logical part or a collection of parts of a piece of equipment. The \glspl{MTDataItem} represent information that is communicated from \glspl{MTComponent}, and the representation and communication of the information are covered in \cite{MTCPart3}. The \glspl{Composition} are the lowest level of contextualization for MTConnect \glspl{MTComponent} that do not have any structure but can be associated with \glspl{MTDataItem} to provide additional context. 

The \gls{Configuration} is a collection of information about the component that provides more detail about its capabilities. The standard has only specified the \xml{SensorConfiguration} at this point.

\glspl{Asset} are complex information models that provide a point in time consistent set of information about the use of a physical or logical entity in the manufacturing process. These models, for example, may represent a cutting tool, a program, or a process. The \glspl{Asset} will be covered in a subsequent companion specification. The only assets currently in the MTConnect standard are \mtmodel{CuttingTool} and \xml{CuttingToolArchetype}. Refer to MTConnect Part 4.0 \cite{MTCPart40} and MTConnect Part 4.1 \cite{MTCPart41}.

The specification uses examples to illustrate the process of conversion from XML to OPC UA; the following sections cover the main points and concerns when converting an MTConnect Device model to a Nodeset. Following the metamodel discussion will be a section on the handling of streaming data and mapping to the correct data items. 

\subsection{Mapping The Model}
\lstset{language=XML,numbers=left,xleftmargin=2em}


%%% Local Variables:
%%% mode: latex
%%% TeX-master: "main"
%%% End:
