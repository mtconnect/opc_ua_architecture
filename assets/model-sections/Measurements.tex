% Generated 2019-11-12 19:49:26 -0600
\subsection{Measurements} \label{model:Measurements}
\subsubsection{Defintion of \texttt{ CommonMeasurementType}}
  \label{type:CommonMeasurementType}

\FloatBarrier
\begin{table}[ht]
\centering 
  \caption{\texttt{CommonMeasurementType} Definition}
  \label{table:CommonMeasurementType}
\fontsize{9pt}{11pt}\selectfont
\tabulinesep=3pt
\begin{tabu} to 6in {|X[-1.35]|X[-0.7]|X[-1.75]|X[-1.5]|X[-1]|X[-0.7]|} \everyrow{\hline}
\hline
\rowfont\bfseries {Attribute} & \multicolumn{5}{|l|}{Value} \\
\tabucline[1.5pt]{}
BrowseName & \multicolumn{5}{|l|}{CommonMeasurementType} \\
IsAbstract & \multicolumn{5}{|l|}{False} \\
\tabucline[1.5pt]{}
\rowfont \bfseries References & NodeClass & BrowseName & DataType & Type\-Definition & {Modeling\-Rule} \\
\end{tabu}
\end{table} 


\FloatBarrier
\subsubsection{Defintion of \texttt{ CuttingToolClassType}}
  \label{type:CuttingToolClassType}

\FloatBarrier
\begin{table}[ht]
\centering 
  \caption{\texttt{CuttingToolClassType} Definition}
  \label{table:CuttingToolClassType}
\fontsize{9pt}{11pt}\selectfont
\tabulinesep=3pt
\begin{tabu} to 6in {|X[-1.35]|X[-0.7]|X[-1.75]|X[-1.5]|X[-1]|X[-0.7]|} \everyrow{\hline}
\hline
\rowfont\bfseries {Attribute} & \multicolumn{5}{|l|}{Value} \\
\tabucline[1.5pt]{}
BrowseName & \multicolumn{5}{|l|}{CuttingToolClassType} \\
IsAbstract & \multicolumn{5}{|l|}{True} \\
\tabucline[1.5pt]{}
\rowfont \bfseries References & NodeClass & BrowseName & DataType & Type\-Definition & {Modeling\-Rule} \\
\multicolumn{6}{|l|}{Subtype of BaseConditionClassType (See \cite{UAPart9} Documentation)} \\
HasSubtype & ObjectType & \multicolumn{2}{l}{BodyDiameterMaxClassType} & \multicolumn{2}{|l|}{See section \ref{type:BodyDiameterMaxClassType}} \\
HasSubtype & ObjectType & \multicolumn{2}{l}{BodyLengthMaxClassType} & \multicolumn{2}{|l|}{See section \ref{type:BodyLengthMaxClassType}} \\
HasSubtype & ObjectType & \multicolumn{2}{l}{CuttingDiameterMaxType} & \multicolumn{2}{|l|}{See section \ref{type:CuttingDiameterMaxType}} \\
HasSubtype & ObjectType & \multicolumn{2}{l}{CuttingItemClassType} & \multicolumn{2}{|l|}{See section \ref{type:CuttingItemClassType}} \\
HasSubtype & ObjectType & \multicolumn{2}{l}{DepthOfCutMaxClassType} & \multicolumn{2}{|l|}{See section \ref{type:DepthOfCutMaxClassType}} \\
HasSubtype & ObjectType & \multicolumn{2}{l}{FlangeDiameterMaxClassType} & \multicolumn{2}{|l|}{See section \ref{type:FlangeDiameterMaxClassType}} \\
HasSubtype & ObjectType & \multicolumn{2}{l}{FunctionalLengthClassType} & \multicolumn{2}{|l|}{See section \ref{type:FunctionalLengthClassType}} \\
HasSubtype & ObjectType & \multicolumn{2}{l}{OverallToolLengthClassType} & \multicolumn{2}{|l|}{See section \ref{type:OverallToolLengthClassType}} \\
HasSubtype & ObjectType & \multicolumn{2}{l}{ProtrudingLengthClassType} & \multicolumn{2}{|l|}{See section \ref{type:ProtrudingLengthClassType}} \\
HasSubtype & ObjectType & \multicolumn{2}{l}{ShankDiameterClassType} & \multicolumn{2}{|l|}{See section \ref{type:ShankDiameterClassType}} \\
HasSubtype & ObjectType & \multicolumn{2}{l}{ShankHeightClassType} & \multicolumn{2}{|l|}{See section \ref{type:ShankHeightClassType}} \\
HasSubtype & ObjectType & \multicolumn{2}{l}{ShankLengthClassType} & \multicolumn{2}{|l|}{See section \ref{type:ShankLengthClassType}} \\
HasSubtype & ObjectType & \multicolumn{2}{l}{UsableLengthMaxClassType} & \multicolumn{2}{|l|}{See section \ref{type:UsableLengthMaxClassType}} \\
HasSubtype & ObjectType & \multicolumn{2}{l}{WeightClassType} & \multicolumn{2}{|l|}{See section \ref{type:WeightClassType}} \\
Has\-Property & Variable & Code & String & Property\-Type & Mandatory \\
Has\-Property & Variable & Units & String & Property\-Type & Mandatory \\
\end{tabu}
\end{table} 


\FloatBarrier
\subsubsection{Defintion of \texttt{ BodyDiameterMaxClassType}}
  \label{type:BodyDiameterMaxClassType}

\FloatBarrier
\begin{table}[ht]
\centering 
  \caption{\texttt{BodyDiameterMaxClassType} Definition}
  \label{table:BodyDiameterMaxClassType}
\fontsize{9pt}{11pt}\selectfont
\tabulinesep=3pt
\begin{tabu} to 6in {|X[-1.35]|X[-0.7]|X[-1.75]|X[-1.5]|X[-1]|X[-0.7]|} \everyrow{\hline}
\hline
\rowfont\bfseries {Attribute} & \multicolumn{5}{|l|}{Value} \\
\tabucline[1.5pt]{}
BrowseName & \multicolumn{5}{|l|}{BodyDiameterMaxClassType} \\
IsAbstract & \multicolumn{5}{|l|}{False} \\
\tabucline[1.5pt]{}
\rowfont \bfseries References & NodeClass & BrowseName & DataType & Type\-Definition & {Modeling\-Rule} \\
\multicolumn{6}{|l|}{Subtype of CuttingToolClassType (See section \ref{type:CuttingToolClassType})} \\
Has\-Property & Variable & Code & String & Property\-Type & Mandatory \\
Has\-Property & Variable & Units & String & Property\-Type & Mandatory \\
\end{tabu}
\end{table} 


\FloatBarrier
\subsubsection{Defintion of \texttt{ BodyLengthMaxClassType}}
  \label{type:BodyLengthMaxClassType}

\FloatBarrier
\begin{table}[ht]
\centering 
  \caption{\texttt{BodyLengthMaxClassType} Definition}
  \label{table:BodyLengthMaxClassType}
\fontsize{9pt}{11pt}\selectfont
\tabulinesep=3pt
\begin{tabu} to 6in {|X[-1.35]|X[-0.7]|X[-1.75]|X[-1.5]|X[-1]|X[-0.7]|} \everyrow{\hline}
\hline
\rowfont\bfseries {Attribute} & \multicolumn{5}{|l|}{Value} \\
\tabucline[1.5pt]{}
BrowseName & \multicolumn{5}{|l|}{BodyLengthMaxClassType} \\
IsAbstract & \multicolumn{5}{|l|}{False} \\
\tabucline[1.5pt]{}
\rowfont \bfseries References & NodeClass & BrowseName & DataType & Type\-Definition & {Modeling\-Rule} \\
\multicolumn{6}{|l|}{Subtype of CuttingToolClassType (See section \ref{type:CuttingToolClassType})} \\
Has\-Property & Variable & Code & String & Property\-Type & Mandatory \\
Has\-Property & Variable & Units & String & Property\-Type & Mandatory \\
\end{tabu}
\end{table} 


\FloatBarrier
\subsubsection{Defintion of \texttt{ CuttingDiameterMaxType}}
  \label{type:CuttingDiameterMaxType}

\FloatBarrier
\begin{table}[ht]
\centering 
  \caption{\texttt{CuttingDiameterMaxType} Definition}
  \label{table:CuttingDiameterMaxType}
\fontsize{9pt}{11pt}\selectfont
\tabulinesep=3pt
\begin{tabu} to 6in {|X[-1.35]|X[-0.7]|X[-1.75]|X[-1.5]|X[-1]|X[-0.7]|} \everyrow{\hline}
\hline
\rowfont\bfseries {Attribute} & \multicolumn{5}{|l|}{Value} \\
\tabucline[1.5pt]{}
BrowseName & \multicolumn{5}{|l|}{CuttingDiameterMaxType} \\
IsAbstract & \multicolumn{5}{|l|}{False} \\
\tabucline[1.5pt]{}
\rowfont \bfseries References & NodeClass & BrowseName & DataType & Type\-Definition & {Modeling\-Rule} \\
\multicolumn{6}{|l|}{Subtype of CuttingToolClassType (See section \ref{type:CuttingToolClassType})} \\
Has\-Property & Variable & Code & String & Property\-Type & Mandatory \\
Has\-Property & Variable & Units & String & Property\-Type & Mandatory \\
\end{tabu}
\end{table} 


\FloatBarrier
\subsubsection{Defintion of \texttt{ CuttingItemClassType}}
  \label{type:CuttingItemClassType}

\FloatBarrier
\begin{table}[ht]
\centering 
  \caption{\texttt{CuttingItemClassType} Definition}
  \label{table:CuttingItemClassType}
\fontsize{9pt}{11pt}\selectfont
\tabulinesep=3pt
\begin{tabu} to 6in {|X[-1.35]|X[-0.7]|X[-1.75]|X[-1.5]|X[-1]|X[-0.7]|} \everyrow{\hline}
\hline
\rowfont\bfseries {Attribute} & \multicolumn{5}{|l|}{Value} \\
\tabucline[1.5pt]{}
BrowseName & \multicolumn{5}{|l|}{CuttingItemClassType} \\
IsAbstract & \multicolumn{5}{|l|}{False} \\
\tabucline[1.5pt]{}
\rowfont \bfseries References & NodeClass & BrowseName & DataType & Type\-Definition & {Modeling\-Rule} \\
\multicolumn{6}{|l|}{Subtype of CuttingToolClassType (See section \ref{type:CuttingToolClassType})} \\
HasSubtype & ObjectType & \multicolumn{2}{l}{ChamferFlatLengthClassType} & \multicolumn{2}{|l|}{See section \ref{type:ChamferFlatLengthClassType}} \\
HasSubtype & ObjectType & \multicolumn{2}{l}{ChamferWidthClassType} & \multicolumn{2}{|l|}{See section \ref{type:ChamferWidthClassType}} \\
HasSubtype & ObjectType & \multicolumn{2}{l}{CornerRadiusClassType} & \multicolumn{2}{|l|}{See section \ref{type:CornerRadiusClassType}} \\
HasSubtype & ObjectType & \multicolumn{2}{l}{CuttingDiameterClassType} & \multicolumn{2}{|l|}{See section \ref{type:CuttingDiameterClassType}} \\
HasSubtype & ObjectType & \multicolumn{2}{l}{CuttingEdgeLengthClassType} & \multicolumn{2}{|l|}{See section \ref{type:CuttingEdgeLengthClassType}} \\
HasSubtype & ObjectType & \multicolumn{2}{l}{CuttingHeightClassType} & \multicolumn{2}{|l|}{See section \ref{type:CuttingHeightClassType}} \\
HasSubtype & ObjectType & \multicolumn{2}{l}{CuttingItemFunctionalLengthClassType} & \multicolumn{2}{|l|}{See section \ref{type:CuttingItemFunctionalLengthClassType}} \\
HasSubtype & ObjectType & \multicolumn{2}{l}{CuttingItemWeightClassType} & \multicolumn{2}{|l|}{See section \ref{type:CuttingItemWeightClassType}} \\
HasSubtype & ObjectType & \multicolumn{2}{l}{DriveAngleClassType} & \multicolumn{2}{|l|}{See section \ref{type:DriveAngleClassType}} \\
HasSubtype & ObjectType & \multicolumn{2}{l}{FlangeDiameterClassType} & \multicolumn{2}{|l|}{See section \ref{type:FlangeDiameterClassType}} \\
HasSubtype & ObjectType & \multicolumn{2}{l}{FunctionalWidthClassType} & \multicolumn{2}{|l|}{See section \ref{type:FunctionalWidthClassType}} \\
HasSubtype & ObjectType & \multicolumn{2}{l}{IncribedCircleDiameterClassType} & \multicolumn{2}{|l|}{See section \ref{type:IncribedCircleDiameterClassType}} \\
HasSubtype & ObjectType & \multicolumn{2}{l}{InsertWidthClassType} & \multicolumn{2}{|l|}{See section \ref{type:InsertWidthClassType}} \\
HasSubtype & ObjectType & \multicolumn{2}{l}{PointAngleClassType} & \multicolumn{2}{|l|}{See section \ref{type:PointAngleClassType}} \\
HasSubtype & ObjectType & \multicolumn{2}{l}{StepDiameterLengthClassType} & \multicolumn{2}{|l|}{See section \ref{type:StepDiameterLengthClassType}} \\
HasSubtype & ObjectType & \multicolumn{2}{l}{StepIncludedAngleClassType} & \multicolumn{2}{|l|}{See section \ref{type:StepIncludedAngleClassType}} \\
HasSubtype & ObjectType & \multicolumn{2}{l}{ToolCuttingEdgeAngleClassType} & \multicolumn{2}{|l|}{See section \ref{type:ToolCuttingEdgeAngleClassType}} \\
HasSubtype & ObjectType & \multicolumn{2}{l}{ToolLeadAngleClassType} & \multicolumn{2}{|l|}{See section \ref{type:ToolLeadAngleClassType}} \\
HasSubtype & ObjectType & \multicolumn{2}{l}{ToolOrientationClassType} & \multicolumn{2}{|l|}{See section \ref{type:ToolOrientationClassType}} \\
HasSubtype & ObjectType & \multicolumn{2}{l}{WiperEdgeLengthClassType} & \multicolumn{2}{|l|}{See section \ref{type:WiperEdgeLengthClassType}} \\
\end{tabu}
\end{table} 


\FloatBarrier
\subsubsection{Defintion of \texttt{ ChamferFlatLengthClassType}}
  \label{type:ChamferFlatLengthClassType}

\FloatBarrier
\begin{table}[ht]
\centering 
  \caption{\texttt{ChamferFlatLengthClassType} Definition}
  \label{table:ChamferFlatLengthClassType}
\fontsize{9pt}{11pt}\selectfont
\tabulinesep=3pt
\begin{tabu} to 6in {|X[-1.35]|X[-0.7]|X[-1.75]|X[-1.5]|X[-1]|X[-0.7]|} \everyrow{\hline}
\hline
\rowfont\bfseries {Attribute} & \multicolumn{5}{|l|}{Value} \\
\tabucline[1.5pt]{}
BrowseName & \multicolumn{5}{|l|}{ChamferFlatLengthClassType} \\
IsAbstract & \multicolumn{5}{|l|}{False} \\
\tabucline[1.5pt]{}
\rowfont \bfseries References & NodeClass & BrowseName & DataType & Type\-Definition & {Modeling\-Rule} \\
\multicolumn{6}{|l|}{Subtype of CuttingItemClassType (See section \ref{type:CuttingItemClassType})} \\
Has\-Property & Variable & Code & String & Property\-Type & Mandatory \\
Has\-Property & Variable & Units & String & Property\-Type & Mandatory \\
\end{tabu}
\end{table} 


\FloatBarrier
\subsubsection{Defintion of \texttt{ ChamferWidthClassType}}
  \label{type:ChamferWidthClassType}

\FloatBarrier
\begin{table}[ht]
\centering 
  \caption{\texttt{ChamferWidthClassType} Definition}
  \label{table:ChamferWidthClassType}
\fontsize{9pt}{11pt}\selectfont
\tabulinesep=3pt
\begin{tabu} to 6in {|X[-1.35]|X[-0.7]|X[-1.75]|X[-1.5]|X[-1]|X[-0.7]|} \everyrow{\hline}
\hline
\rowfont\bfseries {Attribute} & \multicolumn{5}{|l|}{Value} \\
\tabucline[1.5pt]{}
BrowseName & \multicolumn{5}{|l|}{ChamferWidthClassType} \\
IsAbstract & \multicolumn{5}{|l|}{False} \\
\tabucline[1.5pt]{}
\rowfont \bfseries References & NodeClass & BrowseName & DataType & Type\-Definition & {Modeling\-Rule} \\
\multicolumn{6}{|l|}{Subtype of CuttingItemClassType (See section \ref{type:CuttingItemClassType})} \\
Has\-Property & Variable & Code & String & Property\-Type & Mandatory \\
Has\-Property & Variable & Units & String & Property\-Type & Mandatory \\
\end{tabu}
\end{table} 


\FloatBarrier
\subsubsection{Defintion of \texttt{ CornerRadiusClassType}}
  \label{type:CornerRadiusClassType}

\FloatBarrier
\begin{table}[ht]
\centering 
  \caption{\texttt{CornerRadiusClassType} Definition}
  \label{table:CornerRadiusClassType}
\fontsize{9pt}{11pt}\selectfont
\tabulinesep=3pt
\begin{tabu} to 6in {|X[-1.35]|X[-0.7]|X[-1.75]|X[-1.5]|X[-1]|X[-0.7]|} \everyrow{\hline}
\hline
\rowfont\bfseries {Attribute} & \multicolumn{5}{|l|}{Value} \\
\tabucline[1.5pt]{}
BrowseName & \multicolumn{5}{|l|}{CornerRadiusClassType} \\
IsAbstract & \multicolumn{5}{|l|}{False} \\
\tabucline[1.5pt]{}
\rowfont \bfseries References & NodeClass & BrowseName & DataType & Type\-Definition & {Modeling\-Rule} \\
\multicolumn{6}{|l|}{Subtype of CuttingItemClassType (See section \ref{type:CuttingItemClassType})} \\
Has\-Property & Variable & Code & String & Property\-Type & Mandatory \\
Has\-Property & Variable & Units & String & Property\-Type & Mandatory \\
\end{tabu}
\end{table} 


\FloatBarrier
\subsubsection{Defintion of \texttt{ CuttingDiameterClassType}}
  \label{type:CuttingDiameterClassType}

\FloatBarrier
\begin{table}[ht]
\centering 
  \caption{\texttt{CuttingDiameterClassType} Definition}
  \label{table:CuttingDiameterClassType}
\fontsize{9pt}{11pt}\selectfont
\tabulinesep=3pt
\begin{tabu} to 6in {|X[-1.35]|X[-0.7]|X[-1.75]|X[-1.5]|X[-1]|X[-0.7]|} \everyrow{\hline}
\hline
\rowfont\bfseries {Attribute} & \multicolumn{5}{|l|}{Value} \\
\tabucline[1.5pt]{}
BrowseName & \multicolumn{5}{|l|}{CuttingDiameterClassType} \\
IsAbstract & \multicolumn{5}{|l|}{False} \\
\tabucline[1.5pt]{}
\rowfont \bfseries References & NodeClass & BrowseName & DataType & Type\-Definition & {Modeling\-Rule} \\
\multicolumn{6}{|l|}{Subtype of CuttingItemClassType (See section \ref{type:CuttingItemClassType})} \\
Has\-Property & Variable & Code & String & Property\-Type & Mandatory \\
Has\-Property & Variable & Units & String & Property\-Type & Mandatory \\
\end{tabu}
\end{table} 


\FloatBarrier
\subsubsection{Defintion of \texttt{ CuttingEdgeLengthClassType}}
  \label{type:CuttingEdgeLengthClassType}

\FloatBarrier
\begin{table}[ht]
\centering 
  \caption{\texttt{CuttingEdgeLengthClassType} Definition}
  \label{table:CuttingEdgeLengthClassType}
\fontsize{9pt}{11pt}\selectfont
\tabulinesep=3pt
\begin{tabu} to 6in {|X[-1.35]|X[-0.7]|X[-1.75]|X[-1.5]|X[-1]|X[-0.7]|} \everyrow{\hline}
\hline
\rowfont\bfseries {Attribute} & \multicolumn{5}{|l|}{Value} \\
\tabucline[1.5pt]{}
BrowseName & \multicolumn{5}{|l|}{CuttingEdgeLengthClassType} \\
IsAbstract & \multicolumn{5}{|l|}{False} \\
\tabucline[1.5pt]{}
\rowfont \bfseries References & NodeClass & BrowseName & DataType & Type\-Definition & {Modeling\-Rule} \\
\multicolumn{6}{|l|}{Subtype of CuttingItemClassType (See section \ref{type:CuttingItemClassType})} \\
Has\-Property & Variable & Code & String & Property\-Type & Mandatory \\
Has\-Property & Variable & Units & String & Property\-Type & Mandatory \\
\end{tabu}
\end{table} 


\FloatBarrier
\subsubsection{Defintion of \texttt{ CuttingHeightClassType}}
  \label{type:CuttingHeightClassType}

\FloatBarrier
\begin{table}[ht]
\centering 
  \caption{\texttt{CuttingHeightClassType} Definition}
  \label{table:CuttingHeightClassType}
\fontsize{9pt}{11pt}\selectfont
\tabulinesep=3pt
\begin{tabu} to 6in {|X[-1.35]|X[-0.7]|X[-1.75]|X[-1.5]|X[-1]|X[-0.7]|} \everyrow{\hline}
\hline
\rowfont\bfseries {Attribute} & \multicolumn{5}{|l|}{Value} \\
\tabucline[1.5pt]{}
BrowseName & \multicolumn{5}{|l|}{CuttingHeightClassType} \\
IsAbstract & \multicolumn{5}{|l|}{False} \\
\tabucline[1.5pt]{}
\rowfont \bfseries References & NodeClass & BrowseName & DataType & Type\-Definition & {Modeling\-Rule} \\
\multicolumn{6}{|l|}{Subtype of CuttingItemClassType (See section \ref{type:CuttingItemClassType})} \\
Has\-Property & Variable & Code & String & Property\-Type & Mandatory \\
Has\-Property & Variable & Units & String & Property\-Type & Mandatory \\
\end{tabu}
\end{table} 


\FloatBarrier
\subsubsection{Defintion of \texttt{ CuttingItemFunctionalLengthClassType}}
  \label{type:CuttingItemFunctionalLengthClassType}

\FloatBarrier
\begin{table}[ht]
\centering 
  \caption{\texttt{CuttingItemFunctionalLengthClassType} Definition}
  \label{table:CuttingItemFunctionalLengthClassType}
\fontsize{9pt}{11pt}\selectfont
\tabulinesep=3pt
\begin{tabu} to 6in {|X[-1.35]|X[-0.7]|X[-1.75]|X[-1.5]|X[-1]|X[-0.7]|} \everyrow{\hline}
\hline
\rowfont\bfseries {Attribute} & \multicolumn{5}{|l|}{Value} \\
\tabucline[1.5pt]{}
BrowseName & \multicolumn{5}{|l|}{CuttingItemFunctionalLengthClassType} \\
IsAbstract & \multicolumn{5}{|l|}{False} \\
\tabucline[1.5pt]{}
\rowfont \bfseries References & NodeClass & BrowseName & DataType & Type\-Definition & {Modeling\-Rule} \\
\multicolumn{6}{|l|}{Subtype of CuttingItemClassType (See section \ref{type:CuttingItemClassType})} \\
Has\-Property & Variable & Code & String & Property\-Type & Mandatory \\
Has\-Property & Variable & Units & String & Property\-Type & Mandatory \\
\end{tabu}
\end{table} 


\FloatBarrier
\subsubsection{Defintion of \texttt{ CuttingItemWeightClassType}}
  \label{type:CuttingItemWeightClassType}

\FloatBarrier
\begin{table}[ht]
\centering 
  \caption{\texttt{CuttingItemWeightClassType} Definition}
  \label{table:CuttingItemWeightClassType}
\fontsize{9pt}{11pt}\selectfont
\tabulinesep=3pt
\begin{tabu} to 6in {|X[-1.35]|X[-0.7]|X[-1.75]|X[-1.5]|X[-1]|X[-0.7]|} \everyrow{\hline}
\hline
\rowfont\bfseries {Attribute} & \multicolumn{5}{|l|}{Value} \\
\tabucline[1.5pt]{}
BrowseName & \multicolumn{5}{|l|}{CuttingItemWeightClassType} \\
IsAbstract & \multicolumn{5}{|l|}{False} \\
\tabucline[1.5pt]{}
\rowfont \bfseries References & NodeClass & BrowseName & DataType & Type\-Definition & {Modeling\-Rule} \\
\multicolumn{6}{|l|}{Subtype of CuttingItemClassType (See section \ref{type:CuttingItemClassType})} \\
Has\-Property & Variable & Code & String & Property\-Type & Mandatory \\
Has\-Property & Variable & Units & String & Property\-Type & Mandatory \\
\end{tabu}
\end{table} 


\FloatBarrier
\subsubsection{Defintion of \texttt{ DriveAngleClassType}}
  \label{type:DriveAngleClassType}

\FloatBarrier
\begin{table}[ht]
\centering 
  \caption{\texttt{DriveAngleClassType} Definition}
  \label{table:DriveAngleClassType}
\fontsize{9pt}{11pt}\selectfont
\tabulinesep=3pt
\begin{tabu} to 6in {|X[-1.35]|X[-0.7]|X[-1.75]|X[-1.5]|X[-1]|X[-0.7]|} \everyrow{\hline}
\hline
\rowfont\bfseries {Attribute} & \multicolumn{5}{|l|}{Value} \\
\tabucline[1.5pt]{}
BrowseName & \multicolumn{5}{|l|}{DriveAngleClassType} \\
IsAbstract & \multicolumn{5}{|l|}{False} \\
\tabucline[1.5pt]{}
\rowfont \bfseries References & NodeClass & BrowseName & DataType & Type\-Definition & {Modeling\-Rule} \\
\multicolumn{6}{|l|}{Subtype of CuttingItemClassType (See section \ref{type:CuttingItemClassType})} \\
Has\-Property & Variable & Code & String & Property\-Type & Mandatory \\
Has\-Property & Variable & Units & String & Property\-Type & Mandatory \\
\end{tabu}
\end{table} 


\FloatBarrier
\subsubsection{Defintion of \texttt{ FlangeDiameterClassType}}
  \label{type:FlangeDiameterClassType}

\FloatBarrier
\begin{table}[ht]
\centering 
  \caption{\texttt{FlangeDiameterClassType} Definition}
  \label{table:FlangeDiameterClassType}
\fontsize{9pt}{11pt}\selectfont
\tabulinesep=3pt
\begin{tabu} to 6in {|X[-1.35]|X[-0.7]|X[-1.75]|X[-1.5]|X[-1]|X[-0.7]|} \everyrow{\hline}
\hline
\rowfont\bfseries {Attribute} & \multicolumn{5}{|l|}{Value} \\
\tabucline[1.5pt]{}
BrowseName & \multicolumn{5}{|l|}{FlangeDiameterClassType} \\
IsAbstract & \multicolumn{5}{|l|}{False} \\
\tabucline[1.5pt]{}
\rowfont \bfseries References & NodeClass & BrowseName & DataType & Type\-Definition & {Modeling\-Rule} \\
\multicolumn{6}{|l|}{Subtype of CuttingItemClassType (See section \ref{type:CuttingItemClassType})} \\
Has\-Property & Variable & Code & String & Property\-Type & Mandatory \\
Has\-Property & Variable & Units & String & Property\-Type & Mandatory \\
\end{tabu}
\end{table} 


\FloatBarrier
\subsubsection{Defintion of \texttt{ FunctionalWidthClassType}}
  \label{type:FunctionalWidthClassType}

\FloatBarrier
\begin{table}[ht]
\centering 
  \caption{\texttt{FunctionalWidthClassType} Definition}
  \label{table:FunctionalWidthClassType}
\fontsize{9pt}{11pt}\selectfont
\tabulinesep=3pt
\begin{tabu} to 6in {|X[-1.35]|X[-0.7]|X[-1.75]|X[-1.5]|X[-1]|X[-0.7]|} \everyrow{\hline}
\hline
\rowfont\bfseries {Attribute} & \multicolumn{5}{|l|}{Value} \\
\tabucline[1.5pt]{}
BrowseName & \multicolumn{5}{|l|}{FunctionalWidthClassType} \\
IsAbstract & \multicolumn{5}{|l|}{False} \\
\tabucline[1.5pt]{}
\rowfont \bfseries References & NodeClass & BrowseName & DataType & Type\-Definition & {Modeling\-Rule} \\
\multicolumn{6}{|l|}{Subtype of CuttingItemClassType (See section \ref{type:CuttingItemClassType})} \\
Has\-Property & Variable & Code & String & Property\-Type & Mandatory \\
Has\-Property & Variable & Units & String & Property\-Type & Mandatory \\
\end{tabu}
\end{table} 


\FloatBarrier
\subsubsection{Defintion of \texttt{ IncribedCircleDiameterClassType}}
  \label{type:IncribedCircleDiameterClassType}

\FloatBarrier
\begin{table}[ht]
\centering 
  \caption{\texttt{IncribedCircleDiameterClassType} Definition}
  \label{table:IncribedCircleDiameterClassType}
\fontsize{9pt}{11pt}\selectfont
\tabulinesep=3pt
\begin{tabu} to 6in {|X[-1.35]|X[-0.7]|X[-1.75]|X[-1.5]|X[-1]|X[-0.7]|} \everyrow{\hline}
\hline
\rowfont\bfseries {Attribute} & \multicolumn{5}{|l|}{Value} \\
\tabucline[1.5pt]{}
BrowseName & \multicolumn{5}{|l|}{IncribedCircleDiameterClassType} \\
IsAbstract & \multicolumn{5}{|l|}{False} \\
\tabucline[1.5pt]{}
\rowfont \bfseries References & NodeClass & BrowseName & DataType & Type\-Definition & {Modeling\-Rule} \\
\multicolumn{6}{|l|}{Subtype of CuttingItemClassType (See section \ref{type:CuttingItemClassType})} \\
Has\-Property & Variable & Code & String & Property\-Type & Mandatory \\
Has\-Property & Variable & Units & String & Property\-Type & Mandatory \\
\end{tabu}
\end{table} 


\FloatBarrier
\subsubsection{Defintion of \texttt{ InsertWidthClassType}}
  \label{type:InsertWidthClassType}

\FloatBarrier
\begin{table}[ht]
\centering 
  \caption{\texttt{InsertWidthClassType} Definition}
  \label{table:InsertWidthClassType}
\fontsize{9pt}{11pt}\selectfont
\tabulinesep=3pt
\begin{tabu} to 6in {|X[-1.35]|X[-0.7]|X[-1.75]|X[-1.5]|X[-1]|X[-0.7]|} \everyrow{\hline}
\hline
\rowfont\bfseries {Attribute} & \multicolumn{5}{|l|}{Value} \\
\tabucline[1.5pt]{}
BrowseName & \multicolumn{5}{|l|}{InsertWidthClassType} \\
IsAbstract & \multicolumn{5}{|l|}{False} \\
\tabucline[1.5pt]{}
\rowfont \bfseries References & NodeClass & BrowseName & DataType & Type\-Definition & {Modeling\-Rule} \\
\multicolumn{6}{|l|}{Subtype of CuttingItemClassType (See section \ref{type:CuttingItemClassType})} \\
Has\-Property & Variable & Code & String & Property\-Type & Mandatory \\
Has\-Property & Variable & Units & String & Property\-Type & Mandatory \\
\end{tabu}
\end{table} 


\FloatBarrier
\subsubsection{Defintion of \texttt{ PointAngleClassType}}
  \label{type:PointAngleClassType}

\FloatBarrier
\begin{table}[ht]
\centering 
  \caption{\texttt{PointAngleClassType} Definition}
  \label{table:PointAngleClassType}
\fontsize{9pt}{11pt}\selectfont
\tabulinesep=3pt
\begin{tabu} to 6in {|X[-1.35]|X[-0.7]|X[-1.75]|X[-1.5]|X[-1]|X[-0.7]|} \everyrow{\hline}
\hline
\rowfont\bfseries {Attribute} & \multicolumn{5}{|l|}{Value} \\
\tabucline[1.5pt]{}
BrowseName & \multicolumn{5}{|l|}{PointAngleClassType} \\
IsAbstract & \multicolumn{5}{|l|}{False} \\
\tabucline[1.5pt]{}
\rowfont \bfseries References & NodeClass & BrowseName & DataType & Type\-Definition & {Modeling\-Rule} \\
\multicolumn{6}{|l|}{Subtype of CuttingItemClassType (See section \ref{type:CuttingItemClassType})} \\
Has\-Property & Variable & Code & String & Property\-Type & Mandatory \\
Has\-Property & Variable & Units & String & Property\-Type & Mandatory \\
\end{tabu}
\end{table} 


\FloatBarrier
\subsubsection{Defintion of \texttt{ StepDiameterLengthClassType}}
  \label{type:StepDiameterLengthClassType}

\FloatBarrier
\begin{table}[ht]
\centering 
  \caption{\texttt{StepDiameterLengthClassType} Definition}
  \label{table:StepDiameterLengthClassType}
\fontsize{9pt}{11pt}\selectfont
\tabulinesep=3pt
\begin{tabu} to 6in {|X[-1.35]|X[-0.7]|X[-1.75]|X[-1.5]|X[-1]|X[-0.7]|} \everyrow{\hline}
\hline
\rowfont\bfseries {Attribute} & \multicolumn{5}{|l|}{Value} \\
\tabucline[1.5pt]{}
BrowseName & \multicolumn{5}{|l|}{StepDiameterLengthClassType} \\
IsAbstract & \multicolumn{5}{|l|}{False} \\
\tabucline[1.5pt]{}
\rowfont \bfseries References & NodeClass & BrowseName & DataType & Type\-Definition & {Modeling\-Rule} \\
\multicolumn{6}{|l|}{Subtype of CuttingItemClassType (See section \ref{type:CuttingItemClassType})} \\
Has\-Property & Variable & Code & String & Property\-Type & Mandatory \\
Has\-Property & Variable & Units & String & Property\-Type & Mandatory \\
\end{tabu}
\end{table} 


\FloatBarrier
\subsubsection{Defintion of \texttt{ StepIncludedAngleClassType}}
  \label{type:StepIncludedAngleClassType}

\FloatBarrier
\begin{table}[ht]
\centering 
  \caption{\texttt{StepIncludedAngleClassType} Definition}
  \label{table:StepIncludedAngleClassType}
\fontsize{9pt}{11pt}\selectfont
\tabulinesep=3pt
\begin{tabu} to 6in {|X[-1.35]|X[-0.7]|X[-1.75]|X[-1.5]|X[-1]|X[-0.7]|} \everyrow{\hline}
\hline
\rowfont\bfseries {Attribute} & \multicolumn{5}{|l|}{Value} \\
\tabucline[1.5pt]{}
BrowseName & \multicolumn{5}{|l|}{StepIncludedAngleClassType} \\
IsAbstract & \multicolumn{5}{|l|}{False} \\
\tabucline[1.5pt]{}
\rowfont \bfseries References & NodeClass & BrowseName & DataType & Type\-Definition & {Modeling\-Rule} \\
\multicolumn{6}{|l|}{Subtype of CuttingItemClassType (See section \ref{type:CuttingItemClassType})} \\
Has\-Property & Variable & Code & String & Property\-Type & Mandatory \\
Has\-Property & Variable & Units & String & Property\-Type & Mandatory \\
\end{tabu}
\end{table} 


\FloatBarrier
\subsubsection{Defintion of \texttt{ ToolCuttingEdgeAngleClassType}}
  \label{type:ToolCuttingEdgeAngleClassType}

\FloatBarrier
\begin{table}[ht]
\centering 
  \caption{\texttt{ToolCuttingEdgeAngleClassType} Definition}
  \label{table:ToolCuttingEdgeAngleClassType}
\fontsize{9pt}{11pt}\selectfont
\tabulinesep=3pt
\begin{tabu} to 6in {|X[-1.35]|X[-0.7]|X[-1.75]|X[-1.5]|X[-1]|X[-0.7]|} \everyrow{\hline}
\hline
\rowfont\bfseries {Attribute} & \multicolumn{5}{|l|}{Value} \\
\tabucline[1.5pt]{}
BrowseName & \multicolumn{5}{|l|}{ToolCuttingEdgeAngleClassType} \\
IsAbstract & \multicolumn{5}{|l|}{False} \\
\tabucline[1.5pt]{}
\rowfont \bfseries References & NodeClass & BrowseName & DataType & Type\-Definition & {Modeling\-Rule} \\
\multicolumn{6}{|l|}{Subtype of CuttingItemClassType (See section \ref{type:CuttingItemClassType})} \\
Has\-Property & Variable & Code & String & Property\-Type & Mandatory \\
Has\-Property & Variable & Units & String & Property\-Type & Mandatory \\
\end{tabu}
\end{table} 


\FloatBarrier
\subsubsection{Defintion of \texttt{ ToolLeadAngleClassType}}
  \label{type:ToolLeadAngleClassType}

\FloatBarrier
\begin{table}[ht]
\centering 
  \caption{\texttt{ToolLeadAngleClassType} Definition}
  \label{table:ToolLeadAngleClassType}
\fontsize{9pt}{11pt}\selectfont
\tabulinesep=3pt
\begin{tabu} to 6in {|X[-1.35]|X[-0.7]|X[-1.75]|X[-1.5]|X[-1]|X[-0.7]|} \everyrow{\hline}
\hline
\rowfont\bfseries {Attribute} & \multicolumn{5}{|l|}{Value} \\
\tabucline[1.5pt]{}
BrowseName & \multicolumn{5}{|l|}{ToolLeadAngleClassType} \\
IsAbstract & \multicolumn{5}{|l|}{False} \\
\tabucline[1.5pt]{}
\rowfont \bfseries References & NodeClass & BrowseName & DataType & Type\-Definition & {Modeling\-Rule} \\
\multicolumn{6}{|l|}{Subtype of CuttingItemClassType (See section \ref{type:CuttingItemClassType})} \\
Has\-Property & Variable & Code & String & Property\-Type & Mandatory \\
Has\-Property & Variable & Units & String & Property\-Type & Mandatory \\
\end{tabu}
\end{table} 


\FloatBarrier
\subsubsection{Defintion of \texttt{ ToolOrientationClassType}}
  \label{type:ToolOrientationClassType}

\FloatBarrier
\begin{table}[ht]
\centering 
  \caption{\texttt{ToolOrientationClassType} Definition}
  \label{table:ToolOrientationClassType}
\fontsize{9pt}{11pt}\selectfont
\tabulinesep=3pt
\begin{tabu} to 6in {|X[-1.35]|X[-0.7]|X[-1.75]|X[-1.5]|X[-1]|X[-0.7]|} \everyrow{\hline}
\hline
\rowfont\bfseries {Attribute} & \multicolumn{5}{|l|}{Value} \\
\tabucline[1.5pt]{}
BrowseName & \multicolumn{5}{|l|}{ToolOrientationClassType} \\
IsAbstract & \multicolumn{5}{|l|}{False} \\
\tabucline[1.5pt]{}
\rowfont \bfseries References & NodeClass & BrowseName & DataType & Type\-Definition & {Modeling\-Rule} \\
\multicolumn{6}{|l|}{Subtype of CuttingItemClassType (See section \ref{type:CuttingItemClassType})} \\
Has\-Property & Variable & Code & String & Property\-Type & Mandatory \\
Has\-Property & Variable & Units & String & Property\-Type & Mandatory \\
\end{tabu}
\end{table} 


\FloatBarrier
\subsubsection{Defintion of \texttt{ WiperEdgeLengthClassType}}
  \label{type:WiperEdgeLengthClassType}

\FloatBarrier
\begin{table}[ht]
\centering 
  \caption{\texttt{WiperEdgeLengthClassType} Definition}
  \label{table:WiperEdgeLengthClassType}
\fontsize{9pt}{11pt}\selectfont
\tabulinesep=3pt
\begin{tabu} to 6in {|X[-1.35]|X[-0.7]|X[-1.75]|X[-1.5]|X[-1]|X[-0.7]|} \everyrow{\hline}
\hline
\rowfont\bfseries {Attribute} & \multicolumn{5}{|l|}{Value} \\
\tabucline[1.5pt]{}
BrowseName & \multicolumn{5}{|l|}{WiperEdgeLengthClassType} \\
IsAbstract & \multicolumn{5}{|l|}{False} \\
\tabucline[1.5pt]{}
\rowfont \bfseries References & NodeClass & BrowseName & DataType & Type\-Definition & {Modeling\-Rule} \\
\multicolumn{6}{|l|}{Subtype of CuttingItemClassType (See section \ref{type:CuttingItemClassType})} \\
Has\-Property & Variable & Code & String & Property\-Type & Mandatory \\
Has\-Property & Variable & Units & String & Property\-Type & Mandatory \\
\end{tabu}
\end{table} 


\FloatBarrier
\subsubsection{Defintion of \texttt{ DepthOfCutMaxClassType}}
  \label{type:DepthOfCutMaxClassType}

\FloatBarrier
\begin{table}[ht]
\centering 
  \caption{\texttt{DepthOfCutMaxClassType} Definition}
  \label{table:DepthOfCutMaxClassType}
\fontsize{9pt}{11pt}\selectfont
\tabulinesep=3pt
\begin{tabu} to 6in {|X[-1.35]|X[-0.7]|X[-1.75]|X[-1.5]|X[-1]|X[-0.7]|} \everyrow{\hline}
\hline
\rowfont\bfseries {Attribute} & \multicolumn{5}{|l|}{Value} \\
\tabucline[1.5pt]{}
BrowseName & \multicolumn{5}{|l|}{DepthOfCutMaxClassType} \\
IsAbstract & \multicolumn{5}{|l|}{False} \\
\tabucline[1.5pt]{}
\rowfont \bfseries References & NodeClass & BrowseName & DataType & Type\-Definition & {Modeling\-Rule} \\
\multicolumn{6}{|l|}{Subtype of CuttingToolClassType (See section \ref{type:CuttingToolClassType})} \\
Has\-Property & Variable & Code & String & Property\-Type & Mandatory \\
Has\-Property & Variable & Units & String & Property\-Type & Mandatory \\
\end{tabu}
\end{table} 


\FloatBarrier
\subsubsection{Defintion of \texttt{ FlangeDiameterMaxClassType}}
  \label{type:FlangeDiameterMaxClassType}

\FloatBarrier
\begin{table}[ht]
\centering 
  \caption{\texttt{FlangeDiameterMaxClassType} Definition}
  \label{table:FlangeDiameterMaxClassType}
\fontsize{9pt}{11pt}\selectfont
\tabulinesep=3pt
\begin{tabu} to 6in {|X[-1.35]|X[-0.7]|X[-1.75]|X[-1.5]|X[-1]|X[-0.7]|} \everyrow{\hline}
\hline
\rowfont\bfseries {Attribute} & \multicolumn{5}{|l|}{Value} \\
\tabucline[1.5pt]{}
BrowseName & \multicolumn{5}{|l|}{FlangeDiameterMaxClassType} \\
IsAbstract & \multicolumn{5}{|l|}{False} \\
\tabucline[1.5pt]{}
\rowfont \bfseries References & NodeClass & BrowseName & DataType & Type\-Definition & {Modeling\-Rule} \\
\multicolumn{6}{|l|}{Subtype of CuttingToolClassType (See section \ref{type:CuttingToolClassType})} \\
Has\-Property & Variable & Code & String & Property\-Type & Mandatory \\
Has\-Property & Variable & Units & String & Property\-Type & Mandatory \\
\end{tabu}
\end{table} 


\FloatBarrier
\subsubsection{Defintion of \texttt{ FunctionalLengthClassType}}
  \label{type:FunctionalLengthClassType}

\FloatBarrier
\begin{table}[ht]
\centering 
  \caption{\texttt{FunctionalLengthClassType} Definition}
  \label{table:FunctionalLengthClassType}
\fontsize{9pt}{11pt}\selectfont
\tabulinesep=3pt
\begin{tabu} to 6in {|X[-1.35]|X[-0.7]|X[-1.75]|X[-1.5]|X[-1]|X[-0.7]|} \everyrow{\hline}
\hline
\rowfont\bfseries {Attribute} & \multicolumn{5}{|l|}{Value} \\
\tabucline[1.5pt]{}
BrowseName & \multicolumn{5}{|l|}{FunctionalLengthClassType} \\
IsAbstract & \multicolumn{5}{|l|}{False} \\
\tabucline[1.5pt]{}
\rowfont \bfseries References & NodeClass & BrowseName & DataType & Type\-Definition & {Modeling\-Rule} \\
\multicolumn{6}{|l|}{Subtype of CuttingToolClassType (See section \ref{type:CuttingToolClassType})} \\
Has\-Property & Variable & Functional\-Length & String & Property\-Type & Mandatory \\
Has\-Property & Variable & Units & String & Property\-Type & Mandatory \\
\end{tabu}
\end{table} 


\FloatBarrier
\subsubsection{Defintion of \texttt{ OverallToolLengthClassType}}
  \label{type:OverallToolLengthClassType}

\FloatBarrier
\begin{table}[ht]
\centering 
  \caption{\texttt{OverallToolLengthClassType} Definition}
  \label{table:OverallToolLengthClassType}
\fontsize{9pt}{11pt}\selectfont
\tabulinesep=3pt
\begin{tabu} to 6in {|X[-1.35]|X[-0.7]|X[-1.75]|X[-1.5]|X[-1]|X[-0.7]|} \everyrow{\hline}
\hline
\rowfont\bfseries {Attribute} & \multicolumn{5}{|l|}{Value} \\
\tabucline[1.5pt]{}
BrowseName & \multicolumn{5}{|l|}{OverallToolLengthClassType} \\
IsAbstract & \multicolumn{5}{|l|}{False} \\
\tabucline[1.5pt]{}
\rowfont \bfseries References & NodeClass & BrowseName & DataType & Type\-Definition & {Modeling\-Rule} \\
\multicolumn{6}{|l|}{Subtype of CuttingToolClassType (See section \ref{type:CuttingToolClassType})} \\
Has\-Property & Variable & Code & String & Property\-Type & Mandatory \\
Has\-Property & Variable & Units & String & Property\-Type & Mandatory \\
\end{tabu}
\end{table} 


\FloatBarrier
\subsubsection{Defintion of \texttt{ ProtrudingLengthClassType}}
  \label{type:ProtrudingLengthClassType}

\FloatBarrier
\begin{table}[ht]
\centering 
  \caption{\texttt{ProtrudingLengthClassType} Definition}
  \label{table:ProtrudingLengthClassType}
\fontsize{9pt}{11pt}\selectfont
\tabulinesep=3pt
\begin{tabu} to 6in {|X[-1.35]|X[-0.7]|X[-1.75]|X[-1.5]|X[-1]|X[-0.7]|} \everyrow{\hline}
\hline
\rowfont\bfseries {Attribute} & \multicolumn{5}{|l|}{Value} \\
\tabucline[1.5pt]{}
BrowseName & \multicolumn{5}{|l|}{ProtrudingLengthClassType} \\
IsAbstract & \multicolumn{5}{|l|}{False} \\
\tabucline[1.5pt]{}
\rowfont \bfseries References & NodeClass & BrowseName & DataType & Type\-Definition & {Modeling\-Rule} \\
\multicolumn{6}{|l|}{Subtype of CuttingToolClassType (See section \ref{type:CuttingToolClassType})} \\
Has\-Property & Variable & Code & String & Property\-Type & Mandatory \\
Has\-Property & Variable & Units & String & Property\-Type & Mandatory \\
\end{tabu}
\end{table} 


\FloatBarrier
\subsubsection{Defintion of \texttt{ ShankDiameterClassType}}
  \label{type:ShankDiameterClassType}

\FloatBarrier
\begin{table}[ht]
\centering 
  \caption{\texttt{ShankDiameterClassType} Definition}
  \label{table:ShankDiameterClassType}
\fontsize{9pt}{11pt}\selectfont
\tabulinesep=3pt
\begin{tabu} to 6in {|X[-1.35]|X[-0.7]|X[-1.75]|X[-1.5]|X[-1]|X[-0.7]|} \everyrow{\hline}
\hline
\rowfont\bfseries {Attribute} & \multicolumn{5}{|l|}{Value} \\
\tabucline[1.5pt]{}
BrowseName & \multicolumn{5}{|l|}{ShankDiameterClassType} \\
IsAbstract & \multicolumn{5}{|l|}{False} \\
\tabucline[1.5pt]{}
\rowfont \bfseries References & NodeClass & BrowseName & DataType & Type\-Definition & {Modeling\-Rule} \\
\multicolumn{6}{|l|}{Subtype of CuttingToolClassType (See section \ref{type:CuttingToolClassType})} \\
Has\-Property & Variable & Code & String & Property\-Type & Mandatory \\
Has\-Property & Variable & Units & String & Property\-Type & Mandatory \\
\end{tabu}
\end{table} 


\FloatBarrier
\subsubsection{Defintion of \texttt{ ShankHeightClassType}}
  \label{type:ShankHeightClassType}

\FloatBarrier
\begin{table}[ht]
\centering 
  \caption{\texttt{ShankHeightClassType} Definition}
  \label{table:ShankHeightClassType}
\fontsize{9pt}{11pt}\selectfont
\tabulinesep=3pt
\begin{tabu} to 6in {|X[-1.35]|X[-0.7]|X[-1.75]|X[-1.5]|X[-1]|X[-0.7]|} \everyrow{\hline}
\hline
\rowfont\bfseries {Attribute} & \multicolumn{5}{|l|}{Value} \\
\tabucline[1.5pt]{}
BrowseName & \multicolumn{5}{|l|}{ShankHeightClassType} \\
IsAbstract & \multicolumn{5}{|l|}{False} \\
\tabucline[1.5pt]{}
\rowfont \bfseries References & NodeClass & BrowseName & DataType & Type\-Definition & {Modeling\-Rule} \\
\multicolumn{6}{|l|}{Subtype of CuttingToolClassType (See section \ref{type:CuttingToolClassType})} \\
Has\-Property & Variable & Code & String & Property\-Type & Mandatory \\
Has\-Property & Variable & Units & String & Property\-Type & Mandatory \\
\end{tabu}
\end{table} 


\FloatBarrier
\subsubsection{Defintion of \texttt{ ShankLengthClassType}}
  \label{type:ShankLengthClassType}

\FloatBarrier
\begin{table}[ht]
\centering 
  \caption{\texttt{ShankLengthClassType} Definition}
  \label{table:ShankLengthClassType}
\fontsize{9pt}{11pt}\selectfont
\tabulinesep=3pt
\begin{tabu} to 6in {|X[-1.35]|X[-0.7]|X[-1.75]|X[-1.5]|X[-1]|X[-0.7]|} \everyrow{\hline}
\hline
\rowfont\bfseries {Attribute} & \multicolumn{5}{|l|}{Value} \\
\tabucline[1.5pt]{}
BrowseName & \multicolumn{5}{|l|}{ShankLengthClassType} \\
IsAbstract & \multicolumn{5}{|l|}{False} \\
\tabucline[1.5pt]{}
\rowfont \bfseries References & NodeClass & BrowseName & DataType & Type\-Definition & {Modeling\-Rule} \\
\multicolumn{6}{|l|}{Subtype of CuttingToolClassType (See section \ref{type:CuttingToolClassType})} \\
Has\-Property & Variable & Code & String & Property\-Type & Mandatory \\
Has\-Property & Variable & Units & String & Property\-Type & Mandatory \\
\end{tabu}
\end{table} 


\FloatBarrier
\subsubsection{Defintion of \texttt{ UsableLengthMaxClassType}}
  \label{type:UsableLengthMaxClassType}

\FloatBarrier
\begin{table}[ht]
\centering 
  \caption{\texttt{UsableLengthMaxClassType} Definition}
  \label{table:UsableLengthMaxClassType}
\fontsize{9pt}{11pt}\selectfont
\tabulinesep=3pt
\begin{tabu} to 6in {|X[-1.35]|X[-0.7]|X[-1.75]|X[-1.5]|X[-1]|X[-0.7]|} \everyrow{\hline}
\hline
\rowfont\bfseries {Attribute} & \multicolumn{5}{|l|}{Value} \\
\tabucline[1.5pt]{}
BrowseName & \multicolumn{5}{|l|}{UsableLengthMaxClassType} \\
IsAbstract & \multicolumn{5}{|l|}{False} \\
\tabucline[1.5pt]{}
\rowfont \bfseries References & NodeClass & BrowseName & DataType & Type\-Definition & {Modeling\-Rule} \\
\multicolumn{6}{|l|}{Subtype of CuttingToolClassType (See section \ref{type:CuttingToolClassType})} \\
Has\-Property & Variable & Code & String & Property\-Type & Mandatory \\
Has\-Property & Variable & Units & String & Property\-Type & Mandatory \\
\end{tabu}
\end{table} 


\FloatBarrier
\subsubsection{Defintion of \texttt{ WeightClassType}}
  \label{type:WeightClassType}

\FloatBarrier
\begin{table}[ht]
\centering 
  \caption{\texttt{WeightClassType} Definition}
  \label{table:WeightClassType}
\fontsize{9pt}{11pt}\selectfont
\tabulinesep=3pt
\begin{tabu} to 6in {|X[-1.35]|X[-0.7]|X[-1.75]|X[-1.5]|X[-1]|X[-0.7]|} \everyrow{\hline}
\hline
\rowfont\bfseries {Attribute} & \multicolumn{5}{|l|}{Value} \\
\tabucline[1.5pt]{}
BrowseName & \multicolumn{5}{|l|}{WeightClassType} \\
IsAbstract & \multicolumn{5}{|l|}{False} \\
\tabucline[1.5pt]{}
\rowfont \bfseries References & NodeClass & BrowseName & DataType & Type\-Definition & {Modeling\-Rule} \\
\multicolumn{6}{|l|}{Subtype of CuttingToolClassType (See section \ref{type:CuttingToolClassType})} \\
Has\-Property & Variable & Code & String & Property\-Type & Mandatory \\
Has\-Property & Variable & Units & String & Property\-Type & Mandatory \\
\end{tabu}
\end{table} 


\FloatBarrier
