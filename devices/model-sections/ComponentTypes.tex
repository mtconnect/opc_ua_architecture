% Generated 2020-01-09 17:13:24 -0800
\subsection{Component Types} \label{model:ComponentTypes}

\begin{figure}[ht]
  \centering
    \includegraphics[width=1.0\textwidth]{./diagrams/types/ComponentTypes.png}
  \caption{Component Types Diagram}
  \label{fig:ComponentTypes}
\end{figure}

\FloatBarrier


All the sub types of components organized into top level organizational types.

\subsubsection{Defintion of \texttt{ ActuatorType}}
  \label{type:ActuatorType}

\FloatBarrier

the information for an apparatus for moving or controlling a mechanism or system

\begin{table}[ht]
\centering 
  \caption{\texttt{ActuatorType} Definition}
  \label{table:ActuatorType}
\fontsize{9pt}{11pt}\selectfont
\tabulinesep=3pt
\begin{tabu} to 6in {|X[-1.35]|X[-0.7]|X[-1.75]|X[-1.5]|X[-1]|X[-0.7]|} \everyrow{\hline}
\hline
\rowfont\bfseries {Attribute} & \multicolumn{5}{|l|}{Value} \\
\tabucline[1.5pt]{}
BrowseName & \multicolumn{5}{|l|}{ActuatorType} \\
IsAbstract & \multicolumn{5}{|l|}{False} \\
\tabucline[1.5pt]{}
\rowfont \bfseries References & NodeClass & BrowseName & DataType & Type\-Definition & {Modeling\-Rule} \\
\multicolumn{6}{|l|}{Subtype of MTComponentType (See Components Documentation)} \\
\end{tabu}
\end{table} 


\FloatBarrier
\subsubsection{Defintion of \texttt{ AuxiliariesType}}
  \label{type:AuxiliariesType}

\FloatBarrier

representing functional sub-systems that provide supplementary or extended capabilities for a piece of equipment, 
but they are not required for the basic operation of the equipment

\begin{table}[ht]
\centering 
  \caption{\texttt{AuxiliariesType} Definition}
  \label{table:AuxiliariesType}
\fontsize{9pt}{11pt}\selectfont
\tabulinesep=3pt
\begin{tabu} to 6in {|X[-1.35]|X[-0.7]|X[-1.75]|X[-1.5]|X[-1]|X[-0.7]|} \everyrow{\hline}
\hline
\rowfont\bfseries {Attribute} & \multicolumn{5}{|l|}{Value} \\
\tabucline[1.5pt]{}
BrowseName & \multicolumn{5}{|l|}{AuxiliariesType} \\
IsAbstract & \multicolumn{5}{|l|}{False} \\
\tabucline[1.5pt]{}
\rowfont \bfseries References & NodeClass & BrowseName & DataType & Type\-Definition & {Modeling\-Rule} \\
\multicolumn{6}{|l|}{Subtype of MTComponentType (See Components Documentation)} \\
HasSubtype & ObjectType & \multicolumn{2}{l}{BarFeederType} & \multicolumn{2}{|l|}{See section \ref{type:BarFeederType}} \\
HasSubtype & ObjectType & \multicolumn{2}{l}{EnvironmentalType} & \multicolumn{2}{|l|}{See section \ref{type:EnvironmentalType}} \\
HasSubtype & ObjectType & \multicolumn{2}{l}{LoaderType} & \multicolumn{2}{|l|}{See section \ref{type:LoaderType}} \\
HasSubtype & ObjectType & \multicolumn{2}{l}{SensorType} & \multicolumn{2}{|l|}{See section \ref{type:SensorType}} \\
HasSubtype & ObjectType & \multicolumn{2}{l}{ToolingDeliveryType} & \multicolumn{2}{|l|}{See section \ref{type:ToolingDeliveryType}} \\
HasSubtype & ObjectType & \multicolumn{2}{l}{WasteDisposalType} & \multicolumn{2}{|l|}{See section \ref{type:WasteDisposalType}} \\
\end{tabu}
\end{table} 


\FloatBarrier
\subsubsection{Defintion of \texttt{ BarFeederType}}
  \label{type:BarFeederType}

\FloatBarrier

a unit involved in delivering bar stock to a piece of equipment.

\begin{table}[ht]
\centering 
  \caption{\texttt{BarFeederType} Definition}
  \label{table:BarFeederType}
\fontsize{9pt}{11pt}\selectfont
\tabulinesep=3pt
\begin{tabu} to 6in {|X[-1.35]|X[-0.7]|X[-1.75]|X[-1.5]|X[-1]|X[-0.7]|} \everyrow{\hline}
\hline
\rowfont\bfseries {Attribute} & \multicolumn{5}{|l|}{Value} \\
\tabucline[1.5pt]{}
BrowseName & \multicolumn{5}{|l|}{BarFeederType} \\
IsAbstract & \multicolumn{5}{|l|}{False} \\
\tabucline[1.5pt]{}
\rowfont \bfseries References & NodeClass & BrowseName & DataType & Type\-Definition & {Modeling\-Rule} \\
\multicolumn{6}{|l|}{Subtype of AuxiliariesType (See section \ref{type:AuxiliariesType})} \\
\end{tabu}
\end{table} 


\FloatBarrier
\subsubsection{Defintion of \texttt{ EnvironmentalType}}
  \label{type:EnvironmentalType}

\FloatBarrier

the information for a unit or function involved in monitoring, managing, or conditioning 
the environment around or within a piece of equipment.

\begin{table}[ht]
\centering 
  \caption{\texttt{EnvironmentalType} Definition}
  \label{table:EnvironmentalType}
\fontsize{9pt}{11pt}\selectfont
\tabulinesep=3pt
\begin{tabu} to 6in {|X[-1.35]|X[-0.7]|X[-1.75]|X[-1.5]|X[-1]|X[-0.7]|} \everyrow{\hline}
\hline
\rowfont\bfseries {Attribute} & \multicolumn{5}{|l|}{Value} \\
\tabucline[1.5pt]{}
BrowseName & \multicolumn{5}{|l|}{EnvironmentalType} \\
IsAbstract & \multicolumn{5}{|l|}{False} \\
\tabucline[1.5pt]{}
\rowfont \bfseries References & NodeClass & BrowseName & DataType & Type\-Definition & {Modeling\-Rule} \\
\multicolumn{6}{|l|}{Subtype of AuxiliariesType (See section \ref{type:AuxiliariesType})} \\
\end{tabu}
\end{table} 


\FloatBarrier
\subsubsection{Defintion of \texttt{ LoaderType}}
  \label{type:LoaderType}

\FloatBarrier

the information for a unit comprised of all the parts involved in moving and distributing materials, 
parts, tooling, and other items to or from a piece of equipment

\begin{table}[ht]
\centering 
  \caption{\texttt{LoaderType} Definition}
  \label{table:LoaderType}
\fontsize{9pt}{11pt}\selectfont
\tabulinesep=3pt
\begin{tabu} to 6in {|X[-1.35]|X[-0.7]|X[-1.75]|X[-1.5]|X[-1]|X[-0.7]|} \everyrow{\hline}
\hline
\rowfont\bfseries {Attribute} & \multicolumn{5}{|l|}{Value} \\
\tabucline[1.5pt]{}
BrowseName & \multicolumn{5}{|l|}{LoaderType} \\
IsAbstract & \multicolumn{5}{|l|}{False} \\
\tabucline[1.5pt]{}
\rowfont \bfseries References & NodeClass & BrowseName & DataType & Type\-Definition & {Modeling\-Rule} \\
\multicolumn{6}{|l|}{Subtype of AuxiliariesType (See section \ref{type:AuxiliariesType})} \\
\end{tabu}
\end{table} 


\FloatBarrier
\subsubsection{Defintion of \texttt{ SensorType}}
  \label{type:SensorType}

\FloatBarrier

the information for a piece of equipment that responds to a physical stimulus and transmits a resulting impulse or value from a sensing unit

\begin{table}[ht]
\centering 
  \caption{\texttt{SensorType} Definition}
  \label{table:SensorType}
\fontsize{9pt}{11pt}\selectfont
\tabulinesep=3pt
\begin{tabu} to 6in {|X[-1.35]|X[-0.7]|X[-1.75]|X[-1.5]|X[-1]|X[-0.7]|} \everyrow{\hline}
\hline
\rowfont\bfseries {Attribute} & \multicolumn{5}{|l|}{Value} \\
\tabucline[1.5pt]{}
BrowseName & \multicolumn{5}{|l|}{SensorType} \\
IsAbstract & \multicolumn{5}{|l|}{False} \\
\tabucline[1.5pt]{}
\rowfont \bfseries References & NodeClass & BrowseName & DataType & Type\-Definition & {Modeling\-Rule} \\
\multicolumn{6}{|l|}{Subtype of AuxiliariesType (See section \ref{type:AuxiliariesType})} \\
\end{tabu}
\end{table} 


\FloatBarrier
\subsubsection{Defintion of \texttt{ ToolingDeliveryType}}
  \label{type:ToolingDeliveryType}

\FloatBarrier

a unit involved in managing, positioning, storing, and delivering tooling within a piece of equipment.

\begin{table}[ht]
\centering 
  \caption{\texttt{ToolingDeliveryType} Definition}
  \label{table:ToolingDeliveryType}
\fontsize{9pt}{11pt}\selectfont
\tabulinesep=3pt
\begin{tabu} to 6in {|X[-1.35]|X[-0.7]|X[-1.75]|X[-1.5]|X[-1]|X[-0.7]|} \everyrow{\hline}
\hline
\rowfont\bfseries {Attribute} & \multicolumn{5}{|l|}{Value} \\
\tabucline[1.5pt]{}
BrowseName & \multicolumn{5}{|l|}{ToolingDeliveryType} \\
IsAbstract & \multicolumn{5}{|l|}{False} \\
\tabucline[1.5pt]{}
\rowfont \bfseries References & NodeClass & BrowseName & DataType & Type\-Definition & {Modeling\-Rule} \\
\multicolumn{6}{|l|}{Subtype of AuxiliariesType (See section \ref{type:AuxiliariesType})} \\
\end{tabu}
\end{table} 


\FloatBarrier
\subsubsection{Defintion of \texttt{ WasteDisposalType}}
  \label{type:WasteDisposalType}

\FloatBarrier

the information for a unit comprised of all the parts involved in removing manufacturing byproducts from a piece of equipment

\begin{table}[ht]
\centering 
  \caption{\texttt{WasteDisposalType} Definition}
  \label{table:WasteDisposalType}
\fontsize{9pt}{11pt}\selectfont
\tabulinesep=3pt
\begin{tabu} to 6in {|X[-1.35]|X[-0.7]|X[-1.75]|X[-1.5]|X[-1]|X[-0.7]|} \everyrow{\hline}
\hline
\rowfont\bfseries {Attribute} & \multicolumn{5}{|l|}{Value} \\
\tabucline[1.5pt]{}
BrowseName & \multicolumn{5}{|l|}{WasteDisposalType} \\
IsAbstract & \multicolumn{5}{|l|}{False} \\
\tabucline[1.5pt]{}
\rowfont \bfseries References & NodeClass & BrowseName & DataType & Type\-Definition & {Modeling\-Rule} \\
\multicolumn{6}{|l|}{Subtype of AuxiliariesType (See section \ref{type:AuxiliariesType})} \\
\end{tabu}
\end{table} 


\FloatBarrier
\subsubsection{Defintion of \texttt{ AxesType}}
  \label{type:AxesType}

\FloatBarrier

Organizes parts of the device that perform linear or rotational motion

\begin{table}[ht]
\centering 
  \caption{\texttt{AxesType} Definition}
  \label{table:AxesType}
\fontsize{9pt}{11pt}\selectfont
\tabulinesep=3pt
\begin{tabu} to 6in {|X[-1.35]|X[-0.7]|X[-1.75]|X[-1.5]|X[-1]|X[-0.7]|} \everyrow{\hline}
\hline
\rowfont\bfseries {Attribute} & \multicolumn{5}{|l|}{Value} \\
\tabucline[1.5pt]{}
BrowseName & \multicolumn{5}{|l|}{AxesType} \\
IsAbstract & \multicolumn{5}{|l|}{False} \\
\tabucline[1.5pt]{}
\rowfont \bfseries References & NodeClass & BrowseName & DataType & Type\-Definition & {Modeling\-Rule} \\
\multicolumn{6}{|l|}{Subtype of MTComponentType (See Components Documentation)} \\
HasSubtype & ObjectType & \multicolumn{2}{l}{LinearType} & \multicolumn{2}{|l|}{See section \ref{type:LinearType}} \\
HasSubtype & ObjectType & \multicolumn{2}{l}{RotaryType} & \multicolumn{2}{|l|}{See section \ref{type:RotaryType}} \\
\end{tabu}
\end{table} 


\FloatBarrier
\subsubsection{Defintion of \texttt{ LinearType}}
  \label{type:LinearType}

\FloatBarrier

the movement of a physical piece of equipment, or a portion of the equipment, in a straight line.

\begin{table}[ht]
\centering 
  \caption{\texttt{LinearType} Definition}
  \label{table:LinearType}
\fontsize{9pt}{11pt}\selectfont
\tabulinesep=3pt
\begin{tabu} to 6in {|X[-1.35]|X[-0.7]|X[-1.75]|X[-1.5]|X[-1]|X[-0.7]|} \everyrow{\hline}
\hline
\rowfont\bfseries {Attribute} & \multicolumn{5}{|l|}{Value} \\
\tabucline[1.5pt]{}
BrowseName & \multicolumn{5}{|l|}{LinearType} \\
IsAbstract & \multicolumn{5}{|l|}{False} \\
\tabucline[1.5pt]{}
\rowfont \bfseries References & NodeClass & BrowseName & DataType & Type\-Definition & {Modeling\-Rule} \\
\multicolumn{6}{|l|}{Subtype of AxesType (See section \ref{type:AxesType})} \\
\end{tabu}
\end{table} 


\FloatBarrier
\subsubsection{Defintion of \texttt{ RotaryType}}
  \label{type:RotaryType}

\FloatBarrier

rotary movement of a physical piece of equipment or a portion of the equipment.

\begin{table}[ht]
\centering 
  \caption{\texttt{RotaryType} Definition}
  \label{table:RotaryType}
\fontsize{9pt}{11pt}\selectfont
\tabulinesep=3pt
\begin{tabu} to 6in {|X[-1.35]|X[-0.7]|X[-1.75]|X[-1.5]|X[-1]|X[-0.7]|} \everyrow{\hline}
\hline
\rowfont\bfseries {Attribute} & \multicolumn{5}{|l|}{Value} \\
\tabucline[1.5pt]{}
BrowseName & \multicolumn{5}{|l|}{RotaryType} \\
IsAbstract & \multicolumn{5}{|l|}{False} \\
\tabucline[1.5pt]{}
\rowfont \bfseries References & NodeClass & BrowseName & DataType & Type\-Definition & {Modeling\-Rule} \\
\multicolumn{6}{|l|}{Subtype of AxesType (See section \ref{type:AxesType})} \\
HasSubtype & ObjectType & \multicolumn{2}{l}{ChuckType} & \multicolumn{2}{|l|}{See section \ref{type:ChuckType}} \\
\end{tabu}
\end{table} 


\FloatBarrier
\subsubsection{Defintion of \texttt{ ChuckType}}
  \label{type:ChuckType}

\FloatBarrier

provides the information about a mechanism that holds a part or stock material in place

\begin{table}[ht]
\centering 
  \caption{\texttt{ChuckType} Definition}
  \label{table:ChuckType}
\fontsize{9pt}{11pt}\selectfont
\tabulinesep=3pt
\begin{tabu} to 6in {|X[-1.35]|X[-0.7]|X[-1.75]|X[-1.5]|X[-1]|X[-0.7]|} \everyrow{\hline}
\hline
\rowfont\bfseries {Attribute} & \multicolumn{5}{|l|}{Value} \\
\tabucline[1.5pt]{}
BrowseName & \multicolumn{5}{|l|}{ChuckType} \\
IsAbstract & \multicolumn{5}{|l|}{False} \\
\tabucline[1.5pt]{}
\rowfont \bfseries References & NodeClass & BrowseName & DataType & Type\-Definition & {Modeling\-Rule} \\
\multicolumn{6}{|l|}{Subtype of RotaryType (See section \ref{type:RotaryType})} \\
\end{tabu}
\end{table} 


\FloatBarrier
\subsubsection{Defintion of \texttt{ ControllerType}}
  \label{type:ControllerType}

\FloatBarrier

intelligent or computational function within a piece of equipment

\begin{table}[ht]
\centering 
  \caption{\texttt{ControllerType} Definition}
  \label{table:ControllerType}
\fontsize{9pt}{11pt}\selectfont
\tabulinesep=3pt
\begin{tabu} to 6in {|X[-1.35]|X[-0.7]|X[-1.75]|X[-1.5]|X[-1]|X[-0.7]|} \everyrow{\hline}
\hline
\rowfont\bfseries {Attribute} & \multicolumn{5}{|l|}{Value} \\
\tabucline[1.5pt]{}
BrowseName & \multicolumn{5}{|l|}{ControllerType} \\
IsAbstract & \multicolumn{5}{|l|}{False} \\
\tabucline[1.5pt]{}
\rowfont \bfseries References & NodeClass & BrowseName & DataType & Type\-Definition & {Modeling\-Rule} \\
\multicolumn{6}{|l|}{Subtype of MTComponentType (See Components Documentation)} \\
HasSubtype & ObjectType & \multicolumn{2}{l}{PathType} & \multicolumn{2}{|l|}{See section \ref{type:PathType}} \\
\end{tabu}
\end{table} 


\FloatBarrier
\subsubsection{Defintion of \texttt{ PathType}}
  \label{type:PathType}

\FloatBarrier

information for an independent operation or function within a \mtuatype{ControllerType}

\begin{table}[ht]
\centering 
  \caption{\texttt{PathType} Definition}
  \label{table:PathType}
\fontsize{9pt}{11pt}\selectfont
\tabulinesep=3pt
\begin{tabu} to 6in {|X[-1.35]|X[-0.7]|X[-1.75]|X[-1.5]|X[-1]|X[-0.7]|} \everyrow{\hline}
\hline
\rowfont\bfseries {Attribute} & \multicolumn{5}{|l|}{Value} \\
\tabucline[1.5pt]{}
BrowseName & \multicolumn{5}{|l|}{PathType} \\
IsAbstract & \multicolumn{5}{|l|}{False} \\
\tabucline[1.5pt]{}
\rowfont \bfseries References & NodeClass & BrowseName & DataType & Type\-Definition & {Modeling\-Rule} \\
\multicolumn{6}{|l|}{Subtype of ControllerType (See section \ref{type:ControllerType})} \\
\end{tabu}
\end{table} 


\FloatBarrier
\subsubsection{Defintion of \texttt{ DoorType}}
  \label{type:DoorType}

\FloatBarrier

the information for a mechanical mechanism or closure that can cover, for example, a physical access portal into a piece of equipment

\begin{table}[ht]
\centering 
  \caption{\texttt{DoorType} Definition}
  \label{table:DoorType}
\fontsize{9pt}{11pt}\selectfont
\tabulinesep=3pt
\begin{tabu} to 6in {|X[-1.35]|X[-0.7]|X[-1.75]|X[-1.5]|X[-1]|X[-0.7]|} \everyrow{\hline}
\hline
\rowfont\bfseries {Attribute} & \multicolumn{5}{|l|}{Value} \\
\tabucline[1.5pt]{}
BrowseName & \multicolumn{5}{|l|}{DoorType} \\
IsAbstract & \multicolumn{5}{|l|}{False} \\
\tabucline[1.5pt]{}
\rowfont \bfseries References & NodeClass & BrowseName & DataType & Type\-Definition & {Modeling\-Rule} \\
\multicolumn{6}{|l|}{Subtype of MTComponentType (See Components Documentation)} \\
\end{tabu}
\end{table} 


\FloatBarrier
\subsubsection{Defintion of \texttt{ InterfacesType}}
  \label{type:InterfacesType}

\FloatBarrier
\begin{table}[ht]
\centering 
  \caption{\texttt{InterfacesType} Definition}
  \label{table:InterfacesType}
\fontsize{9pt}{11pt}\selectfont
\tabulinesep=3pt
\begin{tabu} to 6in {|X[-1.35]|X[-0.7]|X[-1.75]|X[-1.5]|X[-1]|X[-0.7]|} \everyrow{\hline}
\hline
\rowfont\bfseries {Attribute} & \multicolumn{5}{|l|}{Value} \\
\tabucline[1.5pt]{}
BrowseName & \multicolumn{5}{|l|}{InterfacesType} \\
IsAbstract & \multicolumn{5}{|l|}{False} \\
\tabucline[1.5pt]{}
\rowfont \bfseries References & NodeClass & BrowseName & DataType & Type\-Definition & {Modeling\-Rule} \\
\multicolumn{6}{|l|}{Subtype of MTComponentType (See Components Documentation)} \\
HasSubtype & ObjectType & \multicolumn{2}{l}{BarFeederInterfaceType} & \multicolumn{2}{|l|}{See section \ref{type:BarFeederInterfaceType}} \\
HasSubtype & ObjectType & \multicolumn{2}{l}{ChuckInterfaceType} & \multicolumn{2}{|l|}{See section \ref{type:ChuckInterfaceType}} \\
HasSubtype & ObjectType & \multicolumn{2}{l}{DoorInterfaceType} & \multicolumn{2}{|l|}{See section \ref{type:DoorInterfaceType}} \\
HasSubtype & ObjectType & \multicolumn{2}{l}{MaterialHandlerInterfaceType} & \multicolumn{2}{|l|}{See section \ref{type:MaterialHandlerInterfaceType}} \\
\end{tabu}
\end{table} 


\FloatBarrier
\subsubsection{Defintion of \texttt{ BarFeederInterfaceType}}
  \label{type:BarFeederInterfaceType}

\FloatBarrier

information used to coordinate the operations between a Bar Feeder and another piece of equipment

\begin{table}[ht]
\centering 
  \caption{\texttt{BarFeederInterfaceType} Definition}
  \label{table:BarFeederInterfaceType}
\fontsize{9pt}{11pt}\selectfont
\tabulinesep=3pt
\begin{tabu} to 6in {|X[-1.35]|X[-0.7]|X[-1.75]|X[-1.5]|X[-1]|X[-0.7]|} \everyrow{\hline}
\hline
\rowfont\bfseries {Attribute} & \multicolumn{5}{|l|}{Value} \\
\tabucline[1.5pt]{}
BrowseName & \multicolumn{5}{|l|}{BarFeederInterfaceType} \\
IsAbstract & \multicolumn{5}{|l|}{False} \\
\tabucline[1.5pt]{}
\rowfont \bfseries References & NodeClass & BrowseName & DataType & Type\-Definition & {Modeling\-Rule} \\
\multicolumn{6}{|l|}{Subtype of InterfacesType (See section \ref{type:InterfacesType})} \\
\end{tabu}
\end{table} 


\FloatBarrier
\subsubsection{Defintion of \texttt{ ChuckInterfaceType}}
  \label{type:ChuckInterfaceType}

\FloatBarrier

information used to coordinate the operations between two pieces of equipment, one of which controls the operation of a chuck

\begin{table}[ht]
\centering 
  \caption{\texttt{ChuckInterfaceType} Definition}
  \label{table:ChuckInterfaceType}
\fontsize{9pt}{11pt}\selectfont
\tabulinesep=3pt
\begin{tabu} to 6in {|X[-1.35]|X[-0.7]|X[-1.75]|X[-1.5]|X[-1]|X[-0.7]|} \everyrow{\hline}
\hline
\rowfont\bfseries {Attribute} & \multicolumn{5}{|l|}{Value} \\
\tabucline[1.5pt]{}
BrowseName & \multicolumn{5}{|l|}{ChuckInterfaceType} \\
IsAbstract & \multicolumn{5}{|l|}{False} \\
\tabucline[1.5pt]{}
\rowfont \bfseries References & NodeClass & BrowseName & DataType & Type\-Definition & {Modeling\-Rule} \\
\multicolumn{6}{|l|}{Subtype of InterfacesType (See section \ref{type:InterfacesType})} \\
\end{tabu}
\end{table} 


\FloatBarrier
\subsubsection{Defintion of \texttt{ DoorInterfaceType}}
  \label{type:DoorInterfaceType}

\FloatBarrier

information used to coordinate the operations between two pieces of equipment, one of which controls the operation of a door

\begin{table}[ht]
\centering 
  \caption{\texttt{DoorInterfaceType} Definition}
  \label{table:DoorInterfaceType}
\fontsize{9pt}{11pt}\selectfont
\tabulinesep=3pt
\begin{tabu} to 6in {|X[-1.35]|X[-0.7]|X[-1.75]|X[-1.5]|X[-1]|X[-0.7]|} \everyrow{\hline}
\hline
\rowfont\bfseries {Attribute} & \multicolumn{5}{|l|}{Value} \\
\tabucline[1.5pt]{}
BrowseName & \multicolumn{5}{|l|}{DoorInterfaceType} \\
IsAbstract & \multicolumn{5}{|l|}{False} \\
\tabucline[1.5pt]{}
\rowfont \bfseries References & NodeClass & BrowseName & DataType & Type\-Definition & {Modeling\-Rule} \\
\multicolumn{6}{|l|}{Subtype of InterfacesType (See section \ref{type:InterfacesType})} \\
\end{tabu}
\end{table} 


\FloatBarrier
\subsubsection{Defintion of \texttt{ MaterialHandlerInterfaceType}}
  \label{type:MaterialHandlerInterfaceType}

\FloatBarrier

set of information used to coordinate the operations between a piece of equipment
and another associated piece of equipment used to automatically handle various types of 
materials or services associated with the original piece of equipment

\begin{table}[ht]
\centering 
  \caption{\texttt{MaterialHandlerInterfaceType} Definition}
  \label{table:MaterialHandlerInterfaceType}
\fontsize{9pt}{11pt}\selectfont
\tabulinesep=3pt
\begin{tabu} to 6in {|X[-1.35]|X[-0.7]|X[-1.75]|X[-1.5]|X[-1]|X[-0.7]|} \everyrow{\hline}
\hline
\rowfont\bfseries {Attribute} & \multicolumn{5}{|l|}{Value} \\
\tabucline[1.5pt]{}
BrowseName & \multicolumn{5}{|l|}{MaterialHandlerInterfaceType} \\
IsAbstract & \multicolumn{5}{|l|}{False} \\
\tabucline[1.5pt]{}
\rowfont \bfseries References & NodeClass & BrowseName & DataType & Type\-Definition & {Modeling\-Rule} \\
\multicolumn{6}{|l|}{Subtype of InterfacesType (See section \ref{type:InterfacesType})} \\
\end{tabu}
\end{table} 


\FloatBarrier
\subsubsection{Defintion of \texttt{ ResourcesType}}
  \label{type:ResourcesType}

\FloatBarrier
\begin{table}[ht]
\centering 
  \caption{\texttt{ResourcesType} Definition}
  \label{table:ResourcesType}
\fontsize{9pt}{11pt}\selectfont
\tabulinesep=3pt
\begin{tabu} to 6in {|X[-1.35]|X[-0.7]|X[-1.75]|X[-1.5]|X[-1]|X[-0.7]|} \everyrow{\hline}
\hline
\rowfont\bfseries {Attribute} & \multicolumn{5}{|l|}{Value} \\
\tabucline[1.5pt]{}
BrowseName & \multicolumn{5}{|l|}{ResourcesType} \\
IsAbstract & \multicolumn{5}{|l|}{False} \\
\tabucline[1.5pt]{}
\rowfont \bfseries References & NodeClass & BrowseName & DataType & Type\-Definition & {Modeling\-Rule} \\
\multicolumn{6}{|l|}{Subtype of MTComponentType (See Components Documentation)} \\
HasSubtype & ObjectType & \multicolumn{2}{l}{MaterialsType} & \multicolumn{2}{|l|}{See section \ref{type:MaterialsType}} \\
HasSubtype & ObjectType & \multicolumn{2}{l}{PersonnelType} & \multicolumn{2}{|l|}{See section \ref{type:PersonnelType}} \\
\end{tabu}
\end{table} 


\FloatBarrier
\subsubsection{Defintion of \texttt{ MaterialsType}}
  \label{type:MaterialsType}

\FloatBarrier

information about materials or other items consumed or used by the piece of equipment for 
production of parts, materials, or other types of goods

\begin{table}[ht]
\centering 
  \caption{\texttt{MaterialsType} Definition}
  \label{table:MaterialsType}
\fontsize{9pt}{11pt}\selectfont
\tabulinesep=3pt
\begin{tabu} to 6in {|X[-1.35]|X[-0.7]|X[-1.75]|X[-1.5]|X[-1]|X[-0.7]|} \everyrow{\hline}
\hline
\rowfont\bfseries {Attribute} & \multicolumn{5}{|l|}{Value} \\
\tabucline[1.5pt]{}
BrowseName & \multicolumn{5}{|l|}{MaterialsType} \\
IsAbstract & \multicolumn{5}{|l|}{False} \\
\tabucline[1.5pt]{}
\rowfont \bfseries References & NodeClass & BrowseName & DataType & Type\-Definition & {Modeling\-Rule} \\
\multicolumn{6}{|l|}{Subtype of ResourcesType (See section \ref{type:ResourcesType})} \\
HasSubtype & ObjectType & \multicolumn{2}{l}{StockType} & \multicolumn{2}{|l|}{See section \ref{type:StockType}} \\
\end{tabu}
\end{table} 


\FloatBarrier
\subsubsection{Defintion of \texttt{ StockType}}
  \label{type:StockType}

\FloatBarrier

the information for the material that is used in a manufacturing process and to which 
work is applied in a machine or piece of equipment to produce parts.

\begin{table}[ht]
\centering 
  \caption{\texttt{StockType} Definition}
  \label{table:StockType}
\fontsize{9pt}{11pt}\selectfont
\tabulinesep=3pt
\begin{tabu} to 6in {|X[-1.35]|X[-0.7]|X[-1.75]|X[-1.5]|X[-1]|X[-0.7]|} \everyrow{\hline}
\hline
\rowfont\bfseries {Attribute} & \multicolumn{5}{|l|}{Value} \\
\tabucline[1.5pt]{}
BrowseName & \multicolumn{5}{|l|}{StockType} \\
IsAbstract & \multicolumn{5}{|l|}{False} \\
\tabucline[1.5pt]{}
\rowfont \bfseries References & NodeClass & BrowseName & DataType & Type\-Definition & {Modeling\-Rule} \\
\multicolumn{6}{|l|}{Subtype of MaterialsType (See section \ref{type:MaterialsType})} \\
\end{tabu}
\end{table} 


\FloatBarrier
\subsubsection{Defintion of \texttt{ PersonnelType}}
  \label{type:PersonnelType}

\FloatBarrier
\begin{table}[ht]
\centering 
  \caption{\texttt{PersonnelType} Definition}
  \label{table:PersonnelType}
\fontsize{9pt}{11pt}\selectfont
\tabulinesep=3pt
\begin{tabu} to 6in {|X[-1.35]|X[-0.7]|X[-1.75]|X[-1.5]|X[-1]|X[-0.7]|} \everyrow{\hline}
\hline
\rowfont\bfseries {Attribute} & \multicolumn{5}{|l|}{Value} \\
\tabucline[1.5pt]{}
BrowseName & \multicolumn{5}{|l|}{PersonnelType} \\
IsAbstract & \multicolumn{5}{|l|}{False} \\
\tabucline[1.5pt]{}
\rowfont \bfseries References & NodeClass & BrowseName & DataType & Type\-Definition & {Modeling\-Rule} \\
\multicolumn{6}{|l|}{Subtype of ResourcesType (See section \ref{type:ResourcesType})} \\
\end{tabu}
\end{table} 


\FloatBarrier
\subsubsection{Defintion of \texttt{ SystemsType}}
  \label{type:SystemsType}

\FloatBarrier

major sub-systems that are permanently integrated into a piece of equipment

\begin{table}[ht]
\centering 
  \caption{\texttt{SystemsType} Definition}
  \label{table:SystemsType}
\fontsize{9pt}{11pt}\selectfont
\tabulinesep=3pt
\begin{tabu} to 6in {|X[-1.35]|X[-0.7]|X[-1.75]|X[-1.5]|X[-1]|X[-0.7]|} \everyrow{\hline}
\hline
\rowfont\bfseries {Attribute} & \multicolumn{5}{|l|}{Value} \\
\tabucline[1.5pt]{}
BrowseName & \multicolumn{5}{|l|}{SystemsType} \\
IsAbstract & \multicolumn{5}{|l|}{False} \\
\tabucline[1.5pt]{}
\rowfont \bfseries References & NodeClass & BrowseName & DataType & Type\-Definition & {Modeling\-Rule} \\
\multicolumn{6}{|l|}{Subtype of MTComponentType (See Components Documentation)} \\
HasSubtype & ObjectType & \multicolumn{2}{l}{CoolantType} & \multicolumn{2}{|l|}{See section \ref{type:CoolantType}} \\
HasSubtype & ObjectType & \multicolumn{2}{l}{DielectricType} & \multicolumn{2}{|l|}{See section \ref{type:DielectricType}} \\
HasSubtype & ObjectType & \multicolumn{2}{l}{ElectricType} & \multicolumn{2}{|l|}{See section \ref{type:ElectricType}} \\
HasSubtype & ObjectType & \multicolumn{2}{l}{EnclosureType} & \multicolumn{2}{|l|}{See section \ref{type:EnclosureType}} \\
HasSubtype & ObjectType & \multicolumn{2}{l}{FeederType} & \multicolumn{2}{|l|}{See section \ref{type:FeederType}} \\
HasSubtype & ObjectType & \multicolumn{2}{l}{HydraulicType} & \multicolumn{2}{|l|}{See section \ref{type:HydraulicType}} \\
HasSubtype & ObjectType & \multicolumn{2}{l}{LubricationType} & \multicolumn{2}{|l|}{See section \ref{type:LubricationType}} \\
HasSubtype & ObjectType & \multicolumn{2}{l}{PneumaticType} & \multicolumn{2}{|l|}{See section \ref{type:PneumaticType}} \\
HasSubtype & ObjectType & \multicolumn{2}{l}{ProcessPowerType} & \multicolumn{2}{|l|}{See section \ref{type:ProcessPowerType}} \\
HasSubtype & ObjectType & \multicolumn{2}{l}{ProtectiveType} & \multicolumn{2}{|l|}{See section \ref{type:ProtectiveType}} \\
\end{tabu}
\end{table} 


\FloatBarrier
\subsubsection{Defintion of \texttt{ CoolantType}}
  \label{type:CoolantType}

\FloatBarrier

a system comprised of all the parts involved in distribution and management of fluids that remove heat from a piece of equipment.

\begin{table}[ht]
\centering 
  \caption{\texttt{CoolantType} Definition}
  \label{table:CoolantType}
\fontsize{9pt}{11pt}\selectfont
\tabulinesep=3pt
\begin{tabu} to 6in {|X[-1.35]|X[-0.7]|X[-1.75]|X[-1.5]|X[-1]|X[-0.7]|} \everyrow{\hline}
\hline
\rowfont\bfseries {Attribute} & \multicolumn{5}{|l|}{Value} \\
\tabucline[1.5pt]{}
BrowseName & \multicolumn{5}{|l|}{CoolantType} \\
IsAbstract & \multicolumn{5}{|l|}{False} \\
\tabucline[1.5pt]{}
\rowfont \bfseries References & NodeClass & BrowseName & DataType & Type\-Definition & {Modeling\-Rule} \\
\multicolumn{6}{|l|}{Subtype of SystemsType (See section \ref{type:SystemsType})} \\
\end{tabu}
\end{table} 


\FloatBarrier
\subsubsection{Defintion of \texttt{ DielectricType}}
  \label{type:DielectricType}

\FloatBarrier

a system that manages a chemical mixture used in a manufacturing process being performed at that piece of equipment.

\begin{table}[ht]
\centering 
  \caption{\texttt{DielectricType} Definition}
  \label{table:DielectricType}
\fontsize{9pt}{11pt}\selectfont
\tabulinesep=3pt
\begin{tabu} to 6in {|X[-1.35]|X[-0.7]|X[-1.75]|X[-1.5]|X[-1]|X[-0.7]|} \everyrow{\hline}
\hline
\rowfont\bfseries {Attribute} & \multicolumn{5}{|l|}{Value} \\
\tabucline[1.5pt]{}
BrowseName & \multicolumn{5}{|l|}{DielectricType} \\
IsAbstract & \multicolumn{5}{|l|}{False} \\
\tabucline[1.5pt]{}
\rowfont \bfseries References & NodeClass & BrowseName & DataType & Type\-Definition & {Modeling\-Rule} \\
\multicolumn{6}{|l|}{Subtype of SystemsType (See section \ref{type:SystemsType})} \\
\end{tabu}
\end{table} 


\FloatBarrier
\subsubsection{Defintion of \texttt{ ElectricType}}
  \label{type:ElectricType}

\FloatBarrier

represents the information for the main power supply for device piece of equipment and the distribution of that power throughout the equipment.

\begin{table}[ht]
\centering 
  \caption{\texttt{ElectricType} Definition}
  \label{table:ElectricType}
\fontsize{9pt}{11pt}\selectfont
\tabulinesep=3pt
\begin{tabu} to 6in {|X[-1.35]|X[-0.7]|X[-1.75]|X[-1.5]|X[-1]|X[-0.7]|} \everyrow{\hline}
\hline
\rowfont\bfseries {Attribute} & \multicolumn{5}{|l|}{Value} \\
\tabucline[1.5pt]{}
BrowseName & \multicolumn{5}{|l|}{ElectricType} \\
IsAbstract & \multicolumn{5}{|l|}{False} \\
\tabucline[1.5pt]{}
\rowfont \bfseries References & NodeClass & BrowseName & DataType & Type\-Definition & {Modeling\-Rule} \\
\multicolumn{6}{|l|}{Subtype of SystemsType (See section \ref{type:SystemsType})} \\
\end{tabu}
\end{table} 


\FloatBarrier
\subsubsection{Defintion of \texttt{ EnclosureType}}
  \label{type:EnclosureType}

\FloatBarrier

a structure used to contain or isolate a piece of equipment or area.

\begin{table}[ht]
\centering 
  \caption{\texttt{EnclosureType} Definition}
  \label{table:EnclosureType}
\fontsize{9pt}{11pt}\selectfont
\tabulinesep=3pt
\begin{tabu} to 6in {|X[-1.35]|X[-0.7]|X[-1.75]|X[-1.5]|X[-1]|X[-0.7]|} \everyrow{\hline}
\hline
\rowfont\bfseries {Attribute} & \multicolumn{5}{|l|}{Value} \\
\tabucline[1.5pt]{}
BrowseName & \multicolumn{5}{|l|}{EnclosureType} \\
IsAbstract & \multicolumn{5}{|l|}{False} \\
\tabucline[1.5pt]{}
\rowfont \bfseries References & NodeClass & BrowseName & DataType & Type\-Definition & {Modeling\-Rule} \\
\multicolumn{6}{|l|}{Subtype of SystemsType (See section \ref{type:SystemsType})} \\
\end{tabu}
\end{table} 


\FloatBarrier
\subsubsection{Defintion of \texttt{ FeederType}}
  \label{type:FeederType}

\FloatBarrier

the information for a system that manages the delivery of materials within a piece of equipment.

\begin{table}[ht]
\centering 
  \caption{\texttt{FeederType} Definition}
  \label{table:FeederType}
\fontsize{9pt}{11pt}\selectfont
\tabulinesep=3pt
\begin{tabu} to 6in {|X[-1.35]|X[-0.7]|X[-1.75]|X[-1.5]|X[-1]|X[-0.7]|} \everyrow{\hline}
\hline
\rowfont\bfseries {Attribute} & \multicolumn{5}{|l|}{Value} \\
\tabucline[1.5pt]{}
BrowseName & \multicolumn{5}{|l|}{FeederType} \\
IsAbstract & \multicolumn{5}{|l|}{False} \\
\tabucline[1.5pt]{}
\rowfont \bfseries References & NodeClass & BrowseName & DataType & Type\-Definition & {Modeling\-Rule} \\
\multicolumn{6}{|l|}{Subtype of SystemsType (See section \ref{type:SystemsType})} \\
\end{tabu}
\end{table} 


\FloatBarrier
\subsubsection{Defintion of \texttt{ HydraulicType}}
  \label{type:HydraulicType}

\FloatBarrier

system comprised of all the parts involved in moving and distributing pressurized liquid throughout the piece of equipment.

\begin{table}[ht]
\centering 
  \caption{\texttt{HydraulicType} Definition}
  \label{table:HydraulicType}
\fontsize{9pt}{11pt}\selectfont
\tabulinesep=3pt
\begin{tabu} to 6in {|X[-1.35]|X[-0.7]|X[-1.75]|X[-1.5]|X[-1]|X[-0.7]|} \everyrow{\hline}
\hline
\rowfont\bfseries {Attribute} & \multicolumn{5}{|l|}{Value} \\
\tabucline[1.5pt]{}
BrowseName & \multicolumn{5}{|l|}{HydraulicType} \\
IsAbstract & \multicolumn{5}{|l|}{False} \\
\tabucline[1.5pt]{}
\rowfont \bfseries References & NodeClass & BrowseName & DataType & Type\-Definition & {Modeling\-Rule} \\
\multicolumn{6}{|l|}{Subtype of SystemsType (See section \ref{type:SystemsType})} \\
\end{tabu}
\end{table} 


\FloatBarrier
\subsubsection{Defintion of \texttt{ LubricationType}}
  \label{type:LubricationType}

\FloatBarrier

a system comprised of all the parts involved in distribution and management of fluids used to 
lubricate portions of the piece of equipment.

\begin{table}[ht]
\centering 
  \caption{\texttt{LubricationType} Definition}
  \label{table:LubricationType}
\fontsize{9pt}{11pt}\selectfont
\tabulinesep=3pt
\begin{tabu} to 6in {|X[-1.35]|X[-0.7]|X[-1.75]|X[-1.5]|X[-1]|X[-0.7]|} \everyrow{\hline}
\hline
\rowfont\bfseries {Attribute} & \multicolumn{5}{|l|}{Value} \\
\tabucline[1.5pt]{}
BrowseName & \multicolumn{5}{|l|}{LubricationType} \\
IsAbstract & \multicolumn{5}{|l|}{False} \\
\tabucline[1.5pt]{}
\rowfont \bfseries References & NodeClass & BrowseName & DataType & Type\-Definition & {Modeling\-Rule} \\
\multicolumn{6}{|l|}{Subtype of SystemsType (See section \ref{type:SystemsType})} \\
\end{tabu}
\end{table} 


\FloatBarrier
\subsubsection{Defintion of \texttt{ PneumaticType}}
  \label{type:PneumaticType}

\FloatBarrier

a system comprised of all the parts involved in moving and distributing pressurized gas throughout the piece of equipment.

\begin{table}[ht]
\centering 
  \caption{\texttt{PneumaticType} Definition}
  \label{table:PneumaticType}
\fontsize{9pt}{11pt}\selectfont
\tabulinesep=3pt
\begin{tabu} to 6in {|X[-1.35]|X[-0.7]|X[-1.75]|X[-1.5]|X[-1]|X[-0.7]|} \everyrow{\hline}
\hline
\rowfont\bfseries {Attribute} & \multicolumn{5}{|l|}{Value} \\
\tabucline[1.5pt]{}
BrowseName & \multicolumn{5}{|l|}{PneumaticType} \\
IsAbstract & \multicolumn{5}{|l|}{False} \\
\tabucline[1.5pt]{}
\rowfont \bfseries References & NodeClass & BrowseName & DataType & Type\-Definition & {Modeling\-Rule} \\
\multicolumn{6}{|l|}{Subtype of SystemsType (See section \ref{type:SystemsType})} \\
\end{tabu}
\end{table} 


\FloatBarrier
\subsubsection{Defintion of \texttt{ ProcessPowerType}}
  \label{type:ProcessPowerType}

\FloatBarrier

the information for a power source associated with a piece of equipment that supplies 
energy to the manufacturing process separate from the Electric system

\begin{table}[ht]
\centering 
  \caption{\texttt{ProcessPowerType} Definition}
  \label{table:ProcessPowerType}
\fontsize{9pt}{11pt}\selectfont
\tabulinesep=3pt
\begin{tabu} to 6in {|X[-1.35]|X[-0.7]|X[-1.75]|X[-1.5]|X[-1]|X[-0.7]|} \everyrow{\hline}
\hline
\rowfont\bfseries {Attribute} & \multicolumn{5}{|l|}{Value} \\
\tabucline[1.5pt]{}
BrowseName & \multicolumn{5}{|l|}{ProcessPowerType} \\
IsAbstract & \multicolumn{5}{|l|}{False} \\
\tabucline[1.5pt]{}
\rowfont \bfseries References & NodeClass & BrowseName & DataType & Type\-Definition & {Modeling\-Rule} \\
\multicolumn{6}{|l|}{Subtype of SystemsType (See section \ref{type:SystemsType})} \\
\end{tabu}
\end{table} 


\FloatBarrier
\subsubsection{Defintion of \texttt{ ProtectiveType}}
  \label{type:ProtectiveType}

\FloatBarrier

the information for those functions that detect or prevent harm or damage to equipment or personnel.

\begin{table}[ht]
\centering 
  \caption{\texttt{ProtectiveType} Definition}
  \label{table:ProtectiveType}
\fontsize{9pt}{11pt}\selectfont
\tabulinesep=3pt
\begin{tabu} to 6in {|X[-1.35]|X[-0.7]|X[-1.75]|X[-1.5]|X[-1]|X[-0.7]|} \everyrow{\hline}
\hline
\rowfont\bfseries {Attribute} & \multicolumn{5}{|l|}{Value} \\
\tabucline[1.5pt]{}
BrowseName & \multicolumn{5}{|l|}{ProtectiveType} \\
IsAbstract & \multicolumn{5}{|l|}{False} \\
\tabucline[1.5pt]{}
\rowfont \bfseries References & NodeClass & BrowseName & DataType & Type\-Definition & {Modeling\-Rule} \\
\multicolumn{6}{|l|}{Subtype of SystemsType (See section \ref{type:SystemsType})} \\
\end{tabu}
\end{table} 


\FloatBarrier
