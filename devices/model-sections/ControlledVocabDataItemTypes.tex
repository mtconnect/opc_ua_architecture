% Generated 2020-01-09 18:51:19 -0800
\subsection{Controlled Vocab Data Item Types} \label{model:ControlledVocabDataItemTypes}

The MTConnect Standard has the definitions of \glspl{MTEvent} in 
Section 8.2 of the MTConnect Standard Part 2 \cite{MTCPart2} and the definition of the 
values can be found in Section 5.5 of MTConnect Standard Part 3 \cite{MTCPart3}. 

The description of each type was copied from the MTConnect Standard,
but this not the definitive text for each type. For authoritative normative text, 
please refer to the sources \cite{MTCPart2} and \cite{MTCPart3}.

\subsubsection{Defintion of \texttt{ MTControlledVocabEventClassType}}
  \label{type:MTControlledVocabEventClassType}

\FloatBarrier

The abstract base type for controlled events that represent states that are provided
in related enumerations. These data items will be represented in an object of
type \mtuatype{MTControlledVocabEventType} derived from the OPC UA type
\uamodel{MultiStateValueDiscreteType}

\begin{table}[ht]
\centering 
  \caption{\texttt{MTControlledVocabEventClassType} Definition}
  \label{table:MTControlledVocabEventClassType}
\fontsize{9pt}{11pt}\selectfont
\tabulinesep=3pt
\begin{tabu} to 6in {|X[-1.35]|X[-0.7]|X[-1.75]|X[-1.5]|X[-1]|X[-0.7]|} \everyrow{\hline}
\hline
\rowfont\bfseries {Attribute} & \multicolumn{5}{|l|}{Value} \\
\tabucline[1.5pt]{}
BrowseName & \multicolumn{5}{|l|}{MTControlledVocabEventClassType} \\
IsAbstract & \multicolumn{5}{|l|}{True} \\
\tabucline[1.5pt]{}
\rowfont \bfseries References & NodeClass & BrowseName & DataType & Type\-Definition & {Modeling\-Rule} \\
\multicolumn{6}{|l|}{Subtype of MTEventClassType (See Data Items Documentation)} \\
HasSubtype & ObjectType & \multicolumn{2}{l}{ControllerModeOverrideClassType} & \multicolumn{2}{|l|}{See section \ref{type:ControllerModeOverrideClassType}} \\
HasSubtype & ObjectType & \multicolumn{2}{l}{DirectionClassType} & \multicolumn{2}{|l|}{See section \ref{type:DirectionClassType}} \\
HasSubtype & ObjectType & \multicolumn{2}{l}{DoorStateClassType} & \multicolumn{2}{|l|}{See section \ref{type:DoorStateClassType}} \\
HasSubtype & ObjectType & \multicolumn{2}{l}{EmergencyStopClassType} & \multicolumn{2}{|l|}{See section \ref{type:EmergencyStopClassType}} \\
HasSubtype & ObjectType & \multicolumn{2}{l}{EndOfBarClassType} & \multicolumn{2}{|l|}{See section \ref{type:EndOfBarClassType}} \\
HasSubtype & ObjectType & \multicolumn{2}{l}{EquipmentModeClassType} & \multicolumn{2}{|l|}{See section \ref{type:EquipmentModeClassType}} \\
HasSubtype & ObjectType & \multicolumn{2}{l}{ExecutionClassType} & \multicolumn{2}{|l|}{See section \ref{type:ExecutionClassType}} \\
HasSubtype & ObjectType & \multicolumn{2}{l}{FunctionalModeClassType} & \multicolumn{2}{|l|}{See section \ref{type:FunctionalModeClassType}} \\
HasSubtype & ObjectType & \multicolumn{2}{l}{InterfaceStateClassType} & \multicolumn{2}{|l|}{See section \ref{type:InterfaceStateClassType}} \\
HasSubtype & ObjectType & \multicolumn{2}{l}{MaterialChangeClassType} & \multicolumn{2}{|l|}{See section \ref{type:MaterialChangeClassType}} \\
HasSubtype & ObjectType & \multicolumn{2}{l}{MaterialFeedClassType} & \multicolumn{2}{|l|}{See section \ref{type:MaterialFeedClassType}} \\
HasSubtype & ObjectType & \multicolumn{2}{l}{MaterialLoadClassType} & \multicolumn{2}{|l|}{See section \ref{type:MaterialLoadClassType}} \\
HasSubtype & ObjectType & \multicolumn{2}{l}{MaterialRetractClassType} & \multicolumn{2}{|l|}{See section \ref{type:MaterialRetractClassType}} \\
HasSubtype & ObjectType & \multicolumn{2}{l}{MaterialUnloadClassType} & \multicolumn{2}{|l|}{See section \ref{type:MaterialUnloadClassType}} \\
HasSubtype & ObjectType & \multicolumn{2}{l}{OpenChuckClassType} & \multicolumn{2}{|l|}{See section \ref{type:OpenChuckClassType}} \\
HasSubtype & ObjectType & \multicolumn{2}{l}{OpenDoorClassType} & \multicolumn{2}{|l|}{See section \ref{type:OpenDoorClassType}} \\
HasSubtype & ObjectType & \multicolumn{2}{l}{PartChangeClassType} & \multicolumn{2}{|l|}{See section \ref{type:PartChangeClassType}} \\
HasSubtype & ObjectType & \multicolumn{2}{l}{PathModeClassType} & \multicolumn{2}{|l|}{See section \ref{type:PathModeClassType}} \\
HasSubtype & ObjectType & \multicolumn{2}{l}{PowerStateClassType} & \multicolumn{2}{|l|}{See section \ref{type:PowerStateClassType}} \\
HasSubtype & ObjectType & \multicolumn{2}{l}{ProgramEditClassType} & \multicolumn{2}{|l|}{See section \ref{type:ProgramEditClassType}} \\
HasSubtype & ObjectType & \multicolumn{2}{l}{RotaryModeClassType} & \multicolumn{2}{|l|}{See section \ref{type:RotaryModeClassType}} \\
HasSubtype & ObjectType & \multicolumn{2}{l}{SpindleInterlockClassType} & \multicolumn{2}{|l|}{See section \ref{type:SpindleInterlockClassType}} \\
\multicolumn{6}{|l|}{Continued...} \\
\end{tabu}
\end{table}
\begin{table}[ht]
\fontsize{9pt}{11pt}\selectfont
\tabulinesep=3pt
\begin{tabu} to 6in {|X[-1.35]|X[-0.7]|X[-1.75]|X[-1.5]|X[-1]|X[-0.7]|} \everyrow{\hline}
\hline
\rowfont \bfseries References & NodeClass & BrowseName & DataType & Type\-Definition & {Modeling\-Rule} \\
HasSubtype & ObjectType & \multicolumn{2}{l}{ActuatorStateClassType} & \multicolumn{2}{|l|}{See section \ref{type:ActuatorStateClassType}} \\
HasSubtype & ObjectType & \multicolumn{2}{l}{AvailabilityClassType} & \multicolumn{2}{|l|}{See section \ref{type:AvailabilityClassType}} \\
HasSubtype & ObjectType & \multicolumn{2}{l}{AxisCouplingClassType} & \multicolumn{2}{|l|}{See section \ref{type:AxisCouplingClassType}} \\
HasSubtype & ObjectType & \multicolumn{2}{l}{AxisInterlockClassType} & \multicolumn{2}{|l|}{See section \ref{type:AxisInterlockClassType}} \\
HasSubtype & ObjectType & \multicolumn{2}{l}{AxisStateClassType} & \multicolumn{2}{|l|}{See section \ref{type:AxisStateClassType}} \\
HasSubtype & ObjectType & \multicolumn{2}{l}{ChuckInterlockClassType} & \multicolumn{2}{|l|}{See section \ref{type:ChuckInterlockClassType}} \\
HasSubtype & ObjectType & \multicolumn{2}{l}{ChuckStateClassType} & \multicolumn{2}{|l|}{See section \ref{type:ChuckStateClassType}} \\
HasSubtype & ObjectType & \multicolumn{2}{l}{CloseChuckClassType} & \multicolumn{2}{|l|}{See section \ref{type:CloseChuckClassType}} \\
HasSubtype & ObjectType & \multicolumn{2}{l}{CloseDoorClassType} & \multicolumn{2}{|l|}{See section \ref{type:CloseDoorClassType}} \\
HasSubtype & ObjectType & \multicolumn{2}{l}{CompositionStateClassType} & \multicolumn{2}{|l|}{See section \ref{type:CompositionStateClassType}} \\
HasSubtype & ObjectType & \multicolumn{2}{l}{ControllerModeClassType} & \multicolumn{2}{|l|}{See section \ref{type:ControllerModeClassType}} \\
\end{tabu}
\end{table} 


\FloatBarrier
\subsubsection{Defintion of \texttt{ ActuatorStateClassType}}
  \label{type:ActuatorStateClassType}

\FloatBarrier

Represents the operational state of an apparatus for moving or controlling.

Represents the operational state of an apparatus for moving or controlling a mechanism or system.

\begin{table}[ht]
\centering 
  \caption{\texttt{ActuatorStateClassType} Definition}
  \label{table:ActuatorStateClassType}
\fontsize{9pt}{11pt}\selectfont
\tabulinesep=3pt
\begin{tabu} to 6in {|X[-1.35]|X[-0.7]|X[-1.75]|X[-1.5]|X[-1]|X[-0.7]|} \everyrow{\hline}
\hline
\rowfont\bfseries {Attribute} & \multicolumn{5}{|l|}{Value} \\
\tabucline[1.5pt]{}
BrowseName & \multicolumn{5}{|l|}{ActuatorStateClassType} \\
IsAbstract & \multicolumn{5}{|l|}{False} \\
\tabucline[1.5pt]{}
\rowfont \bfseries References & NodeClass & BrowseName & DataType & Type\-Definition & {Modeling\-Rule} \\
\multicolumn{6}{|l|}{Subtype of MTControlledVocabEventClassType (See section \ref{type:MTControlledVocabEventClassType})} \\
Has\-Property & Variable & Enum\-Strings & Active\-State\-Data\-Type & Active\-State\-Data\-Type & Mandatory \\
\end{tabu}
\end{table} 


\FloatBarrier
\paragraph{Referenced Properties and Objects}

\begin{itemize}
\item \textbf{Allowable Values} for \texttt{ActiveStateDataType}
\FloatBarrier



\begin{table}[ht]
\centering 
  \caption{\texttt{ActiveStateDataType} Enumeration}
  \label{enum:ActiveStateDataType}
\tabulinesep=3pt
\begin{tabu} to 6in {|l|r|} \everyrow{\hline}
\hline
\rowfont\bfseries {Name} & {Index} \\
\tabucline[1.5pt]{}
\texttt{ACTIVE} & \texttt{0} \\
\texttt{INACTIVE} & \texttt{1} \\
\end{tabu}
\end{table} 
\FloatBarrier
\end{itemize}
\FloatBarrier
\subsubsection{Defintion of \texttt{ AvailabilityClassType}}
  \label{type:AvailabilityClassType}

\FloatBarrier

Represents the Agent's ability to communicate with the data source. This MUST be provided for a 
Device Element and MAY be provided for any other Structural Element.

Represents the agent's ability to communicate with the data source.

\begin{table}[ht]
\centering 
  \caption{\texttt{AvailabilityClassType} Definition}
  \label{table:AvailabilityClassType}
\fontsize{9pt}{11pt}\selectfont
\tabulinesep=3pt
\begin{tabu} to 6in {|X[-1.35]|X[-0.7]|X[-1.75]|X[-1.5]|X[-1]|X[-0.7]|} \everyrow{\hline}
\hline
\rowfont\bfseries {Attribute} & \multicolumn{5}{|l|}{Value} \\
\tabucline[1.5pt]{}
BrowseName & \multicolumn{5}{|l|}{AvailabilityClassType} \\
IsAbstract & \multicolumn{5}{|l|}{False} \\
\tabucline[1.5pt]{}
\rowfont \bfseries References & NodeClass & BrowseName & DataType & Type\-Definition & {Modeling\-Rule} \\
\multicolumn{6}{|l|}{Subtype of MTControlledVocabEventClassType (See section \ref{type:MTControlledVocabEventClassType})} \\
Has\-Property & Variable & Enum\-Strings & Availability\-Data\-Type & Availability\-Data\-Type & Mandatory \\
\end{tabu}
\end{table} 


\FloatBarrier
\paragraph{Referenced Properties and Objects}

\begin{itemize}
\item \textbf{Allowable Values} for \texttt{AvailabilityDataType}
\FloatBarrier



\begin{table}[ht]
\centering 
  \caption{\texttt{AvailabilityDataType} Enumeration}
  \label{enum:AvailabilityDataType}
\tabulinesep=3pt
\begin{tabu} to 6in {|l|r|} \everyrow{\hline}
\hline
\rowfont\bfseries {Name} & {Index} \\
\tabucline[1.5pt]{}
\texttt{AVAILABLE} & \texttt{0} \\
\texttt{UNAVAILABLE} & \texttt{1} \\
\end{tabu}
\end{table} 
\FloatBarrier
\end{itemize}
\FloatBarrier
\subsubsection{Defintion of \texttt{ AxisCouplingClassType}}
  \label{type:AxisCouplingClassType}

\FloatBarrier

Describes the way the axes will be associated to each other.
This is used in conjunction with \mtmodel{COUPLED_AXES} to indicate the way they are interacting.

The coupling MUST be viewed from the perspective of a specific axis. Therefore, a \mtmodel{MASTER} coupling 
indicates that this axis is the master for the \mtmodel{COUPLED_AXES}.

Describes the way the axes will be associated to each other. 
  
 This is used in conjunction with coupledaxes event to indicate the way they are interacting.

\begin{table}[ht]
\centering 
  \caption{\texttt{AxisCouplingClassType} Definition}
  \label{table:AxisCouplingClassType}
\fontsize{9pt}{11pt}\selectfont
\tabulinesep=3pt
\begin{tabu} to 6in {|X[-1.35]|X[-0.7]|X[-1.75]|X[-1.5]|X[-1]|X[-0.7]|} \everyrow{\hline}
\hline
\rowfont\bfseries {Attribute} & \multicolumn{5}{|l|}{Value} \\
\tabucline[1.5pt]{}
BrowseName & \multicolumn{5}{|l|}{AxisCouplingClassType} \\
IsAbstract & \multicolumn{5}{|l|}{False} \\
\tabucline[1.5pt]{}
\rowfont \bfseries References & NodeClass & BrowseName & DataType & Type\-Definition & {Modeling\-Rule} \\
\multicolumn{6}{|l|}{Subtype of MTControlledVocabEventClassType (See section \ref{type:MTControlledVocabEventClassType})} \\
Has\-Property & Variable & Enum\-Strings & Axis\-Coupling\-Data\-Type & Axis\-Coupling\-Data\-Type & Mandatory \\
\end{tabu}
\end{table} 


\FloatBarrier
\paragraph{Referenced Properties and Objects}

\begin{itemize}
\item \textbf{Allowable Values} for \texttt{AxisCouplingDataType}
\FloatBarrier



\begin{table}[ht]
\centering 
  \caption{\texttt{AxisCouplingDataType} Enumeration}
  \label{enum:AxisCouplingDataType}
\tabulinesep=3pt
\begin{tabu} to 6in {|l|r|} \everyrow{\hline}
\hline
\rowfont\bfseries {Name} & {Index} \\
\tabucline[1.5pt]{}
\texttt{MASTER} & \texttt{0} \\
\texttt{SLAVE} & \texttt{1} \\
\texttt{SYNCHRONOUS} & \texttt{2} \\
\texttt{TANDEM} & \texttt{3} \\
\end{tabu}
\end{table} 
\FloatBarrier
\end{itemize}
\FloatBarrier
\subsubsection{Defintion of \texttt{ AxisInterlockClassType}}
  \label{type:AxisInterlockClassType}

\FloatBarrier

An indicator of the state of the axis lockout function when power has been removed and the axis is allowed to move freely.

An indicator of the state of the axis lockout function when power has been removed and the axis is allowed to move freely.

\begin{table}[ht]
\centering 
  \caption{\texttt{AxisInterlockClassType} Definition}
  \label{table:AxisInterlockClassType}
\fontsize{9pt}{11pt}\selectfont
\tabulinesep=3pt
\begin{tabu} to 6in {|X[-1.35]|X[-0.7]|X[-1.75]|X[-1.5]|X[-1]|X[-0.7]|} \everyrow{\hline}
\hline
\rowfont\bfseries {Attribute} & \multicolumn{5}{|l|}{Value} \\
\tabucline[1.5pt]{}
BrowseName & \multicolumn{5}{|l|}{AxisInterlockClassType} \\
IsAbstract & \multicolumn{5}{|l|}{False} \\
\tabucline[1.5pt]{}
\rowfont \bfseries References & NodeClass & BrowseName & DataType & Type\-Definition & {Modeling\-Rule} \\
\multicolumn{6}{|l|}{Subtype of MTControlledVocabEventClassType (See section \ref{type:MTControlledVocabEventClassType})} \\
Has\-Property & Variable & Enum\-Strings & Active\-State\-Data\-Type & Active\-State\-Data\-Type & Mandatory \\
\end{tabu}
\end{table} 


\FloatBarrier
\paragraph{Referenced Properties and Objects}

\begin{itemize}
\item \textbf{Allowable Values} for \texttt{ActiveStateDataType}
\FloatBarrier



\begin{table}[ht]
\centering 
  \caption{\texttt{ActiveStateDataType} Enumeration}
\tabulinesep=3pt
\begin{tabu} to 6in {|l|r|} \everyrow{\hline}
\hline
\rowfont\bfseries {Name} & {Index} \\
\tabucline[1.5pt]{}
\texttt{ACTIVE} & \texttt{0} \\
\texttt{INACTIVE} & \texttt{1} \\
\end{tabu}
\end{table} 
\FloatBarrier
\end{itemize}
\FloatBarrier
\subsubsection{Defintion of \texttt{ AxisStateClassType}}
  \label{type:AxisStateClassType}

\FloatBarrier

An indicator of the controlled state of a \mtmodel{LINEAR} or \mtmodel{ROTARY} component representing an axis.

An indicator of the controlled state of a linear or rotary component representing an axis.

\begin{table}[ht]
\centering 
  \caption{\texttt{AxisStateClassType} Definition}
  \label{table:AxisStateClassType}
\fontsize{9pt}{11pt}\selectfont
\tabulinesep=3pt
\begin{tabu} to 6in {|X[-1.35]|X[-0.7]|X[-1.75]|X[-1.5]|X[-1]|X[-0.7]|} \everyrow{\hline}
\hline
\rowfont\bfseries {Attribute} & \multicolumn{5}{|l|}{Value} \\
\tabucline[1.5pt]{}
BrowseName & \multicolumn{5}{|l|}{AxisStateClassType} \\
IsAbstract & \multicolumn{5}{|l|}{False} \\
\tabucline[1.5pt]{}
\rowfont \bfseries References & NodeClass & BrowseName & DataType & Type\-Definition & {Modeling\-Rule} \\
\multicolumn{6}{|l|}{Subtype of MTControlledVocabEventClassType (See section \ref{type:MTControlledVocabEventClassType})} \\
Has\-Property & Variable & Enum\-Strings & Axis\-State\-Data\-Type & Axis\-State\-Data\-Type & Mandatory \\
\end{tabu}
\end{table} 


\FloatBarrier
\paragraph{Referenced Properties and Objects}

\begin{itemize}
\item \textbf{Allowable Values} for \texttt{AxisStateDataType}
\FloatBarrier



\begin{table}[ht]
\centering 
  \caption{\texttt{AxisStateDataType} Enumeration}
  \label{enum:AxisStateDataType}
\tabulinesep=3pt
\begin{tabu} to 6in {|l|r|} \everyrow{\hline}
\hline
\rowfont\bfseries {Name} & {Index} \\
\tabucline[1.5pt]{}
\texttt{HOME} & \texttt{0} \\
\texttt{PARKED} & \texttt{1} \\
\texttt{STOPPED} & \texttt{2} \\
\texttt{TRAVEL} & \texttt{3} \\
\end{tabu}
\end{table} 
\FloatBarrier
\end{itemize}
\FloatBarrier
\subsubsection{Defintion of \texttt{ ChuckInterlockClassType}}
  \label{type:ChuckInterlockClassType}

\FloatBarrier

An indication of the state of an interlock function or control logic state intended to prevent the 
associated \mtmodel{Chuck} composition or function from being operated.

An indication of the state of an interlock function or control logic state intended to prevent the associated chuck component from being operated.

\begin{table}[ht]
\centering 
  \caption{\texttt{ChuckInterlockClassType} Definition}
  \label{table:ChuckInterlockClassType}
\fontsize{9pt}{11pt}\selectfont
\tabulinesep=3pt
\begin{tabu} to 6in {|X[-1.35]|X[-0.7]|X[-1.75]|X[-1.5]|X[-1]|X[-0.7]|} \everyrow{\hline}
\hline
\rowfont\bfseries {Attribute} & \multicolumn{5}{|l|}{Value} \\
\tabucline[1.5pt]{}
BrowseName & \multicolumn{5}{|l|}{ChuckInterlockClassType} \\
IsAbstract & \multicolumn{5}{|l|}{False} \\
\tabucline[1.5pt]{}
\rowfont \bfseries References & NodeClass & BrowseName & DataType & Type\-Definition & {Modeling\-Rule} \\
\multicolumn{6}{|l|}{Subtype of MTControlledVocabEventClassType (See section \ref{type:MTControlledVocabEventClassType})} \\
Has\-Property & Variable & Enum\-Strings & Active\-State\-Data\-Type & Active\-State\-Data\-Type & Mandatory \\
\end{tabu}
\end{table} 


\FloatBarrier
\paragraph{Referenced Properties and Objects}

\begin{itemize}
\item \textbf{Allowable Values} for \texttt{ActiveStateDataType}
\FloatBarrier



\begin{table}[ht]
\centering 
  \caption{\texttt{ActiveStateDataType} Enumeration}
\tabulinesep=3pt
\begin{tabu} to 6in {|l|r|} \everyrow{\hline}
\hline
\rowfont\bfseries {Name} & {Index} \\
\tabucline[1.5pt]{}
\texttt{ACTIVE} & \texttt{0} \\
\texttt{INACTIVE} & \texttt{1} \\
\end{tabu}
\end{table} 
\FloatBarrier
\end{itemize}
\FloatBarrier
\subsubsection{Defintion of \texttt{ ChuckStateClassType}}
  \label{type:ChuckStateClassType}

\FloatBarrier

An indication of the operating state of a mechanism that holds a part or stock material during a 
manufacturing process. It may also represent a mechanism that holds any other mechanism 
in place within a piece of equipment.

An indication of the operating state of a mechanism that holds a part or stock material during a manufacturing process. It may also represent a mechanism that holds any other mechanism in place within a piece of equipment.

\begin{table}[ht]
\centering 
  \caption{\texttt{ChuckStateClassType} Definition}
  \label{table:ChuckStateClassType}
\fontsize{9pt}{11pt}\selectfont
\tabulinesep=3pt
\begin{tabu} to 6in {|X[-1.35]|X[-0.7]|X[-1.75]|X[-1.5]|X[-1]|X[-0.7]|} \everyrow{\hline}
\hline
\rowfont\bfseries {Attribute} & \multicolumn{5}{|l|}{Value} \\
\tabucline[1.5pt]{}
BrowseName & \multicolumn{5}{|l|}{ChuckStateClassType} \\
IsAbstract & \multicolumn{5}{|l|}{False} \\
\tabucline[1.5pt]{}
\rowfont \bfseries References & NodeClass & BrowseName & DataType & Type\-Definition & {Modeling\-Rule} \\
\multicolumn{6}{|l|}{Subtype of MTControlledVocabEventClassType (See section \ref{type:MTControlledVocabEventClassType})} \\
Has\-Property & Variable & Enum\-Strings & Open\-State\-Data\-Type & Open\-State\-Data\-Type & Mandatory \\
\end{tabu}
\end{table} 


\FloatBarrier
\paragraph{Referenced Properties and Objects}

\begin{itemize}
\item \textbf{Allowable Values} for \texttt{OpenStateDataType}
\FloatBarrier



\begin{table}[ht]
\centering 
  \caption{\texttt{OpenStateDataType} Enumeration}
  \label{enum:OpenStateDataType}
\tabulinesep=3pt
\begin{tabu} to 6in {|l|r|} \everyrow{\hline}
\hline
\rowfont\bfseries {Name} & {Index} \\
\tabucline[1.5pt]{}
\texttt{CLOSED} & \texttt{0} \\
\texttt{OPEN} & \texttt{1} \\
\texttt{UNLATCHED} & \texttt{2} \\
\end{tabu}
\end{table} 
\FloatBarrier
\end{itemize}
\FloatBarrier
\subsubsection{Defintion of \texttt{ CloseChuckClassType}}
  \label{type:CloseChuckClassType}

\FloatBarrier

Service to close a chuck.

Service to close a chuck.

\begin{table}[ht]
\centering 
  \caption{\texttt{CloseChuckClassType} Definition}
  \label{table:CloseChuckClassType}
\fontsize{9pt}{11pt}\selectfont
\tabulinesep=3pt
\begin{tabu} to 6in {|X[-1.35]|X[-0.7]|X[-1.75]|X[-1.5]|X[-1]|X[-0.7]|} \everyrow{\hline}
\hline
\rowfont\bfseries {Attribute} & \multicolumn{5}{|l|}{Value} \\
\tabucline[1.5pt]{}
BrowseName & \multicolumn{5}{|l|}{CloseChuckClassType} \\
IsAbstract & \multicolumn{5}{|l|}{False} \\
\tabucline[1.5pt]{}
\rowfont \bfseries References & NodeClass & BrowseName & DataType & Type\-Definition & {Modeling\-Rule} \\
\multicolumn{6}{|l|}{Subtype of MTControlledVocabEventClassType (See section \ref{type:MTControlledVocabEventClassType})} \\
Has\-Property & Variable & Enum\-Strings & Interface\-State\-Data\-Type & Interface\-State\-Data\-Type & Mandatory \\
\end{tabu}
\end{table} 


\FloatBarrier
\paragraph{Referenced Properties and Objects}

\begin{itemize}
\item \textbf{Allowable Values} for \texttt{InterfaceStateDataType}
\FloatBarrier



\begin{table}[ht]
\centering 
  \caption{\texttt{InterfaceStateDataType} Enumeration}
  \label{enum:InterfaceStateDataType}
\tabulinesep=3pt
\begin{tabu} to 6in {|l|r|} \everyrow{\hline}
\hline
\rowfont\bfseries {Name} & {Index} \\
\tabucline[1.5pt]{}
\texttt{ACTIVE} & \texttt{0} \\
\texttt{COMPLETE} & \texttt{1} \\
\texttt{FAIL} & \texttt{2} \\
\texttt{NOT_READY} & \texttt{4} \\
\texttt{READY} & \texttt{5} \\
\end{tabu}
\end{table} 
\FloatBarrier
\end{itemize}
\FloatBarrier
\subsubsection{Defintion of \texttt{ CloseDoorClassType}}
  \label{type:CloseDoorClassType}

\FloatBarrier

Service to close a door.

Service to close a door.

\begin{table}[ht]
\centering 
  \caption{\texttt{CloseDoorClassType} Definition}
  \label{table:CloseDoorClassType}
\fontsize{9pt}{11pt}\selectfont
\tabulinesep=3pt
\begin{tabu} to 6in {|X[-1.35]|X[-0.7]|X[-1.75]|X[-1.5]|X[-1]|X[-0.7]|} \everyrow{\hline}
\hline
\rowfont\bfseries {Attribute} & \multicolumn{5}{|l|}{Value} \\
\tabucline[1.5pt]{}
BrowseName & \multicolumn{5}{|l|}{CloseDoorClassType} \\
IsAbstract & \multicolumn{5}{|l|}{False} \\
\tabucline[1.5pt]{}
\rowfont \bfseries References & NodeClass & BrowseName & DataType & Type\-Definition & {Modeling\-Rule} \\
\multicolumn{6}{|l|}{Subtype of MTControlledVocabEventClassType (See section \ref{type:MTControlledVocabEventClassType})} \\
Has\-Property & Variable & Enum\-Strings & Interface\-State\-Data\-Type & Interface\-State\-Data\-Type & Mandatory \\
\end{tabu}
\end{table} 


\FloatBarrier
\paragraph{Referenced Properties and Objects}

\begin{itemize}
\item \textbf{Allowable Values} for \texttt{InterfaceStateDataType}
\FloatBarrier



\begin{table}[ht]
\centering 
  \caption{\texttt{InterfaceStateDataType} Enumeration}
\tabulinesep=3pt
\begin{tabu} to 6in {|l|r|} \everyrow{\hline}
\hline
\rowfont\bfseries {Name} & {Index} \\
\tabucline[1.5pt]{}
\texttt{ACTIVE} & \texttt{0} \\
\texttt{COMPLETE} & \texttt{1} \\
\texttt{FAIL} & \texttt{2} \\
\texttt{NOT_READY} & \texttt{4} \\
\texttt{READY} & \texttt{5} \\
\end{tabu}
\end{table} 
\FloatBarrier
\end{itemize}
\FloatBarrier
\subsubsection{Defintion of \texttt{ CompositionStateClassType}}
  \label{type:CompositionStateClassType}

\FloatBarrier

An indication of the operating condition of a mechanism represented by a \mtmodel{Composition} type element.

A \gls{subType} MUST always be specified.

A \mtmodel{compositionId} MUST always be specified.

An indication of the operating condition of a mechanism represented by a composition type element.

\begin{table}[ht]
\centering 
  \caption{\texttt{CompositionStateClassType} Definition}
  \label{table:CompositionStateClassType}
\fontsize{9pt}{11pt}\selectfont
\tabulinesep=3pt
\begin{tabu} to 6in {|X[-1.35]|X[-0.7]|X[-1.75]|X[-1.5]|X[-1]|X[-0.7]|} \everyrow{\hline}
\hline
\rowfont\bfseries {Attribute} & \multicolumn{5}{|l|}{Value} \\
\tabucline[1.5pt]{}
BrowseName & \multicolumn{5}{|l|}{CompositionStateClassType} \\
IsAbstract & \multicolumn{5}{|l|}{False} \\
\tabucline[1.5pt]{}
\rowfont \bfseries References & NodeClass & BrowseName & DataType & Type\-Definition & {Modeling\-Rule} \\
\multicolumn{6}{|l|}{Subtype of MTControlledVocabEventClassType (See section \ref{type:MTControlledVocabEventClassType})} \\
Has\-Property & Variable & Enum\-Strings & Composition\-State\-Data\-Type & Composition\-State\-Data\-Type & Mandatory \\
\end{tabu}
\end{table} 


\FloatBarrier
\paragraph{Referenced Properties and Objects}

\begin{itemize}
\item \textbf{Allowable Values} for \texttt{CompositionStateDataType}
\FloatBarrier



\begin{table}[ht]
\centering 
  \caption{\texttt{CompositionStateDataType} Enumeration}
  \label{enum:CompositionStateDataType}
\tabulinesep=3pt
\begin{tabu} to 6in {|l|r|} \everyrow{\hline}
\hline
\rowfont\bfseries {Name} & {Index} \\
\tabucline[1.5pt]{}
\texttt{ACTIVE} & \texttt{0} \\
\texttt{CLOSED} & \texttt{1} \\
\texttt{DOWN} & \texttt{2} \\
\texttt{INACTIVE} & \texttt{3} \\
\texttt{LEFT} & \texttt{4} \\
\texttt{OFF} & \texttt{5} \\
\texttt{ON} & \texttt{6} \\
\texttt{OPEN} & \texttt{7} \\
\texttt{RIGHT} & \texttt{8} \\
\texttt{TRANSITIONING} & \texttt{9} \\
\texttt{UNLATCHED} & \texttt{10} \\
\texttt{UP} & \texttt{11} \\
\end{tabu}
\end{table} 
\FloatBarrier
\end{itemize}
\FloatBarrier
\subsubsection{Defintion of \texttt{ ControllerModeClassType}}
  \label{type:ControllerModeClassType}

\FloatBarrier

The current mode of the \mtmodel{Controller} component.

The current operating mode of the controller component.

\begin{table}[ht]
\centering 
  \caption{\texttt{ControllerModeClassType} Definition}
  \label{table:ControllerModeClassType}
\fontsize{9pt}{11pt}\selectfont
\tabulinesep=3pt
\begin{tabu} to 6in {|X[-1.35]|X[-0.7]|X[-1.75]|X[-1.5]|X[-1]|X[-0.7]|} \everyrow{\hline}
\hline
\rowfont\bfseries {Attribute} & \multicolumn{5}{|l|}{Value} \\
\tabucline[1.5pt]{}
BrowseName & \multicolumn{5}{|l|}{ControllerModeClassType} \\
IsAbstract & \multicolumn{5}{|l|}{False} \\
\tabucline[1.5pt]{}
\rowfont \bfseries References & NodeClass & BrowseName & DataType & Type\-Definition & {Modeling\-Rule} \\
\multicolumn{6}{|l|}{Subtype of MTControlledVocabEventClassType (See section \ref{type:MTControlledVocabEventClassType})} \\
Has\-Property & Variable & Enum\-Strings & Controller\-Mode\-Data\-Type & Controller\-Mode\-Data\-Type & Mandatory \\
\end{tabu}
\end{table} 


\FloatBarrier
\paragraph{Referenced Properties and Objects}

\begin{itemize}
\item \textbf{Allowable Values} for \texttt{ControllerModeDataType}
\FloatBarrier



\begin{table}[ht]
\centering 
  \caption{\texttt{ControllerModeDataType} Enumeration}
  \label{enum:ControllerModeDataType}
\tabulinesep=3pt
\begin{tabu} to 6in {|l|r|} \everyrow{\hline}
\hline
\rowfont\bfseries {Name} & {Index} \\
\tabucline[1.5pt]{}
\texttt{AUTOMATIC} & \texttt{0} \\
\texttt{EDIT} & \texttt{1} \\
\texttt{MANUAL} & \texttt{2} \\
\texttt{MANUAL_DATA_INPUT} & \texttt{3} \\
\texttt{SEMI_AUTOMATIC} & \texttt{4} \\
\end{tabu}
\end{table} 
\FloatBarrier
\end{itemize}
\FloatBarrier
\subsubsection{Defintion of \texttt{ ControllerModeOverrideClassType}}
  \label{type:ControllerModeOverrideClassType}

\FloatBarrier

A setting or operator selection that changes the behavior of a piece of equipment.

A setting or operator selection that changes the behavior of a piece of equipment.

\begin{table}[ht]
\centering 
  \caption{\texttt{ControllerModeOverrideClassType} Definition}
  \label{table:ControllerModeOverrideClassType}
\fontsize{9pt}{11pt}\selectfont
\tabulinesep=3pt
\begin{tabu} to 6in {|X[-1.35]|X[-0.7]|X[-1.75]|X[-1.5]|X[-1]|X[-0.7]|} \everyrow{\hline}
\hline
\rowfont\bfseries {Attribute} & \multicolumn{5}{|l|}{Value} \\
\tabucline[1.5pt]{}
BrowseName & \multicolumn{5}{|l|}{ControllerModeOverrideClassType} \\
IsAbstract & \multicolumn{5}{|l|}{False} \\
\tabucline[1.5pt]{}
\rowfont \bfseries References & NodeClass & BrowseName & DataType & Type\-Definition & {Modeling\-Rule} \\
\multicolumn{6}{|l|}{Subtype of MTControlledVocabEventClassType (See section \ref{type:MTControlledVocabEventClassType})} \\
Has\-Property & Variable & Enum\-Strings & On\-Off\-Data\-Type & On\-Off\-Data\-Type & Mandatory \\
\end{tabu}
\end{table} 


\FloatBarrier
\paragraph{Referenced Properties and Objects}

\begin{itemize}
\item \textbf{Allowable Values} for \texttt{OnOffDataType}
\FloatBarrier



\begin{table}[ht]
\centering 
  \caption{\texttt{OnOffDataType} Enumeration}
  \label{enum:OnOffDataType}
\tabulinesep=3pt
\begin{tabu} to 6in {|l|r|} \everyrow{\hline}
\hline
\rowfont\bfseries {Name} & {Index} \\
\tabucline[1.5pt]{}
\texttt{OFF} & \texttt{0} \\
\texttt{ON} & \texttt{1} \\
\end{tabu}
\end{table} 
\FloatBarrier
\end{itemize}
\FloatBarrier
\subsubsection{Defintion of \texttt{ DirectionClassType}}
  \label{type:DirectionClassType}

\FloatBarrier

The direction of motion. A \gls{subType} MUST always be specified.

The direction of motion.

\begin{table}[ht]
\centering 
  \caption{\texttt{DirectionClassType} Definition}
  \label{table:DirectionClassType}
\fontsize{9pt}{11pt}\selectfont
\tabulinesep=3pt
\begin{tabu} to 6in {|X[-1.35]|X[-0.7]|X[-1.75]|X[-1.5]|X[-1]|X[-0.7]|} \everyrow{\hline}
\hline
\rowfont\bfseries {Attribute} & \multicolumn{5}{|l|}{Value} \\
\tabucline[1.5pt]{}
BrowseName & \multicolumn{5}{|l|}{DirectionClassType} \\
IsAbstract & \multicolumn{5}{|l|}{False} \\
\tabucline[1.5pt]{}
\rowfont \bfseries References & NodeClass & BrowseName & DataType & Type\-Definition & {Modeling\-Rule} \\
\multicolumn{6}{|l|}{Subtype of MTControlledVocabEventClassType (See section \ref{type:MTControlledVocabEventClassType})} \\
Has\-Property & Variable & Enum\-Strings & Direction\-Data\-Type & Direction\-Data\-Type & Mandatory \\
\end{tabu}
\end{table} 


\FloatBarrier
\paragraph{Referenced Properties and Objects}

\begin{itemize}
\item \textbf{Allowable Values} for \texttt{DirectionDataType}
\FloatBarrier



\begin{table}[ht]
\centering 
  \caption{\texttt{DirectionDataType} Enumeration}
  \label{enum:DirectionDataType}
\tabulinesep=3pt
\begin{tabu} to 6in {|l|r|} \everyrow{\hline}
\hline
\rowfont\bfseries {Name} & {Index} \\
\tabucline[1.5pt]{}
\texttt{CLOCKWISE} & \texttt{0} \\
\texttt{COUNTER_CLOCKWISE} & \texttt{1} \\
\texttt{NEGATIVE} & \texttt{2} \\
\texttt{POSITIVE} & \texttt{3} \\
\end{tabu}
\end{table} 
\FloatBarrier
\end{itemize}
\FloatBarrier
\subsubsection{Defintion of \texttt{ DoorStateClassType}}
  \label{type:DoorStateClassType}

\FloatBarrier

The opened or closed state of the door.

The operational state of a door type component or composition element.

\begin{table}[ht]
\centering 
  \caption{\texttt{DoorStateClassType} Definition}
  \label{table:DoorStateClassType}
\fontsize{9pt}{11pt}\selectfont
\tabulinesep=3pt
\begin{tabu} to 6in {|X[-1.35]|X[-0.7]|X[-1.75]|X[-1.5]|X[-1]|X[-0.7]|} \everyrow{\hline}
\hline
\rowfont\bfseries {Attribute} & \multicolumn{5}{|l|}{Value} \\
\tabucline[1.5pt]{}
BrowseName & \multicolumn{5}{|l|}{DoorStateClassType} \\
IsAbstract & \multicolumn{5}{|l|}{False} \\
\tabucline[1.5pt]{}
\rowfont \bfseries References & NodeClass & BrowseName & DataType & Type\-Definition & {Modeling\-Rule} \\
\multicolumn{6}{|l|}{Subtype of MTControlledVocabEventClassType (See section \ref{type:MTControlledVocabEventClassType})} \\
Has\-Property & Variable & Enum\-Strings & Open\-State\-Data\-Type & Open\-State\-Data\-Type & Mandatory \\
\end{tabu}
\end{table} 


\FloatBarrier
\paragraph{Referenced Properties and Objects}

\begin{itemize}
\item \textbf{Allowable Values} for \texttt{OpenStateDataType}
\FloatBarrier



\begin{table}[ht]
\centering 
  \caption{\texttt{OpenStateDataType} Enumeration}
\tabulinesep=3pt
\begin{tabu} to 6in {|l|r|} \everyrow{\hline}
\hline
\rowfont\bfseries {Name} & {Index} \\
\tabucline[1.5pt]{}
\texttt{CLOSED} & \texttt{0} \\
\texttt{OPEN} & \texttt{1} \\
\texttt{UNLATCHED} & \texttt{2} \\
\end{tabu}
\end{table} 
\FloatBarrier
\end{itemize}
\FloatBarrier
\subsubsection{Defintion of \texttt{ EmergencyStopClassType}}
  \label{type:EmergencyStopClassType}

\FloatBarrier

The current state of the emergency stop signal.

The current state of the emergency stop signal for a piece of equipment, controller path, or any other component or subsystem of a piece of equipment.

\begin{table}[ht]
\centering 
  \caption{\texttt{EmergencyStopClassType} Definition}
  \label{table:EmergencyStopClassType}
\fontsize{9pt}{11pt}\selectfont
\tabulinesep=3pt
\begin{tabu} to 6in {|X[-1.35]|X[-0.7]|X[-1.75]|X[-1.5]|X[-1]|X[-0.7]|} \everyrow{\hline}
\hline
\rowfont\bfseries {Attribute} & \multicolumn{5}{|l|}{Value} \\
\tabucline[1.5pt]{}
BrowseName & \multicolumn{5}{|l|}{EmergencyStopClassType} \\
IsAbstract & \multicolumn{5}{|l|}{False} \\
\tabucline[1.5pt]{}
\rowfont \bfseries References & NodeClass & BrowseName & DataType & Type\-Definition & {Modeling\-Rule} \\
\multicolumn{6}{|l|}{Subtype of MTControlledVocabEventClassType (See section \ref{type:MTControlledVocabEventClassType})} \\
Has\-Property & Variable & Enum\-Strings & Emergency\-Stop\-Data\-Type & Emergency\-Stop\-Data\-Type & Mandatory \\
\end{tabu}
\end{table} 


\FloatBarrier
\paragraph{Referenced Properties and Objects}

\begin{itemize}
\item \textbf{Allowable Values} for \texttt{EmergencyStopDataType}
\FloatBarrier



\begin{table}[ht]
\centering 
  \caption{\texttt{EmergencyStopDataType} Enumeration}
  \label{enum:EmergencyStopDataType}
\tabulinesep=3pt
\begin{tabu} to 6in {|l|r|} \everyrow{\hline}
\hline
\rowfont\bfseries {Name} & {Index} \\
\tabucline[1.5pt]{}
\texttt{ARMED} & \texttt{0} \\
\texttt{TRIGGERED} & \texttt{1} \\
\end{tabu}
\end{table} 
\FloatBarrier
\end{itemize}
\FloatBarrier
\subsubsection{Defintion of \texttt{ EndOfBarClassType}}
  \label{type:EndOfBarClassType}

\FloatBarrier

An indication of whether the end of a piece of bar stock being feed by a bar feeder has been reached.

An indication of whether the end of a piece of bar stock being feed by a bar feeder has been reached.

\begin{table}[ht]
\centering 
  \caption{\texttt{EndOfBarClassType} Definition}
  \label{table:EndOfBarClassType}
\fontsize{9pt}{11pt}\selectfont
\tabulinesep=3pt
\begin{tabu} to 6in {|X[-1.35]|X[-0.7]|X[-1.75]|X[-1.5]|X[-1]|X[-0.7]|} \everyrow{\hline}
\hline
\rowfont\bfseries {Attribute} & \multicolumn{5}{|l|}{Value} \\
\tabucline[1.5pt]{}
BrowseName & \multicolumn{5}{|l|}{EndOfBarClassType} \\
IsAbstract & \multicolumn{5}{|l|}{False} \\
\tabucline[1.5pt]{}
\rowfont \bfseries References & NodeClass & BrowseName & DataType & Type\-Definition & {Modeling\-Rule} \\
\multicolumn{6}{|l|}{Subtype of MTControlledVocabEventClassType (See section \ref{type:MTControlledVocabEventClassType})} \\
Has\-Property & Variable & Enum\-Strings & Yes\-No\-Data\-Type & Yes\-No\-Data\-Type & Mandatory \\
\end{tabu}
\end{table} 


\FloatBarrier
\paragraph{Referenced Properties and Objects}

\begin{itemize}
\item \textbf{Allowable Values} for \texttt{YesNoDataType}
\FloatBarrier



\begin{table}[ht]
\centering 
  \caption{\texttt{YesNoDataType} Enumeration}
  \label{enum:YesNoDataType}
\tabulinesep=3pt
\begin{tabu} to 6in {|l|r|} \everyrow{\hline}
\hline
\rowfont\bfseries {Name} & {Index} \\
\tabucline[1.5pt]{}
\texttt{NO} & \texttt{0} \\
\texttt{YES} & \texttt{1} \\
\end{tabu}
\end{table} 
\FloatBarrier
\end{itemize}
\FloatBarrier
\subsubsection{Defintion of \texttt{ EquipmentModeClassType}}
  \label{type:EquipmentModeClassType}

\FloatBarrier

An indication that a piece of equipment, or a sub-part of a piece of
equipment, is performing specific types of activities.

A \gls{subType} MUST always be specified.

An indication that a piece of equipment, or a sub-part of a piece of equipment, is performing specific types of activities.

\begin{table}[ht]
\centering 
  \caption{\texttt{EquipmentModeClassType} Definition}
  \label{table:EquipmentModeClassType}
\fontsize{9pt}{11pt}\selectfont
\tabulinesep=3pt
\begin{tabu} to 6in {|X[-1.35]|X[-0.7]|X[-1.75]|X[-1.5]|X[-1]|X[-0.7]|} \everyrow{\hline}
\hline
\rowfont\bfseries {Attribute} & \multicolumn{5}{|l|}{Value} \\
\tabucline[1.5pt]{}
BrowseName & \multicolumn{5}{|l|}{EquipmentModeClassType} \\
IsAbstract & \multicolumn{5}{|l|}{False} \\
\tabucline[1.5pt]{}
\rowfont \bfseries References & NodeClass & BrowseName & DataType & Type\-Definition & {Modeling\-Rule} \\
\multicolumn{6}{|l|}{Subtype of MTControlledVocabEventClassType (See section \ref{type:MTControlledVocabEventClassType})} \\
Has\-Property & Variable & Enum\-Strings & On\-Off\-Data\-Type & On\-Off\-Data\-Type & Mandatory \\
\end{tabu}
\end{table} 


\FloatBarrier
\paragraph{Referenced Properties and Objects}

\begin{itemize}
\item \textbf{Allowable Values} for \texttt{OnOffDataType}
\FloatBarrier



\begin{table}[ht]
\centering 
  \caption{\texttt{OnOffDataType} Enumeration}
\tabulinesep=3pt
\begin{tabu} to 6in {|l|r|} \everyrow{\hline}
\hline
\rowfont\bfseries {Name} & {Index} \\
\tabucline[1.5pt]{}
\texttt{OFF} & \texttt{0} \\
\texttt{ON} & \texttt{1} \\
\end{tabu}
\end{table} 
\FloatBarrier
\end{itemize}
\FloatBarrier
\subsubsection{Defintion of \texttt{ ExecutionClassType}}
  \label{type:ExecutionClassType}

\FloatBarrier

The execution status of the \mtmodel{Controller} or \mtmodel{Path}.

The execution status of the controller.

\begin{table}[ht]
\centering 
  \caption{\texttt{ExecutionClassType} Definition}
  \label{table:ExecutionClassType}
\fontsize{9pt}{11pt}\selectfont
\tabulinesep=3pt
\begin{tabu} to 6in {|X[-1.35]|X[-0.7]|X[-1.75]|X[-1.5]|X[-1]|X[-0.7]|} \everyrow{\hline}
\hline
\rowfont\bfseries {Attribute} & \multicolumn{5}{|l|}{Value} \\
\tabucline[1.5pt]{}
BrowseName & \multicolumn{5}{|l|}{ExecutionClassType} \\
IsAbstract & \multicolumn{5}{|l|}{False} \\
\tabucline[1.5pt]{}
\rowfont \bfseries References & NodeClass & BrowseName & DataType & Type\-Definition & {Modeling\-Rule} \\
\multicolumn{6}{|l|}{Subtype of MTControlledVocabEventClassType (See section \ref{type:MTControlledVocabEventClassType})} \\
Has\-Property & Variable & Enum\-Strings & Execution\-Data\-Type & Execution\-Data\-Type & Mandatory \\
\end{tabu}
\end{table} 


\FloatBarrier
\paragraph{Referenced Properties and Objects}

\begin{itemize}
\item \textbf{Allowable Values} for \texttt{ExecutionDataType}
\FloatBarrier



\begin{table}[ht]
\centering 
  \caption{\texttt{ExecutionDataType} Enumeration}
  \label{enum:ExecutionDataType}
\tabulinesep=3pt
\begin{tabu} to 6in {|l|r|} \everyrow{\hline}
\hline
\rowfont\bfseries {Name} & {Index} \\
\tabucline[1.5pt]{}
\texttt{ACTIVE} & \texttt{0} \\
\texttt{FEED_HOLD} & \texttt{1} \\
\texttt{INTERRUPTED} & \texttt{2} \\
\texttt{OPTIONAL_STOP} & \texttt{3} \\
\texttt{READY} & \texttt{4} \\
\texttt{PROGRAM_COMPLETED} & \texttt{5} \\
\texttt{PROGRAM_STOPPED} & \texttt{6} \\
\texttt{STOPPED} & \texttt{7} \\
\end{tabu}
\end{table} 
\FloatBarrier
\end{itemize}
\FloatBarrier
\subsubsection{Defintion of \texttt{ FunctionalModeClassType}}
  \label{type:FunctionalModeClassType}

\FloatBarrier

The current intended production status of the device or component.

Typically, the \texttt{FUNCTIONAL_MODE} SHOULD be modeled as a data item for the Device element, but 
MAY be modeled for any Structural Element in the XML document.

The current intended production status of the device or component.

\begin{table}[ht]
\centering 
  \caption{\texttt{FunctionalModeClassType} Definition}
  \label{table:FunctionalModeClassType}
\fontsize{9pt}{11pt}\selectfont
\tabulinesep=3pt
\begin{tabu} to 6in {|X[-1.35]|X[-0.7]|X[-1.75]|X[-1.5]|X[-1]|X[-0.7]|} \everyrow{\hline}
\hline
\rowfont\bfseries {Attribute} & \multicolumn{5}{|l|}{Value} \\
\tabucline[1.5pt]{}
BrowseName & \multicolumn{5}{|l|}{FunctionalModeClassType} \\
IsAbstract & \multicolumn{5}{|l|}{False} \\
\tabucline[1.5pt]{}
\rowfont \bfseries References & NodeClass & BrowseName & DataType & Type\-Definition & {Modeling\-Rule} \\
\multicolumn{6}{|l|}{Subtype of MTControlledVocabEventClassType (See section \ref{type:MTControlledVocabEventClassType})} \\
Has\-Property & Variable & Enum\-Strings & Functional\-Mode\-Data\-Type & Functional\-Mode\-Data\-Type & Mandatory \\
\end{tabu}
\end{table} 


\FloatBarrier
\paragraph{Referenced Properties and Objects}

\begin{itemize}
\item \textbf{Allowable Values} for \texttt{FunctionalModeDataType}
\FloatBarrier



\begin{table}[ht]
\centering 
  \caption{\texttt{FunctionalModeDataType} Enumeration}
  \label{enum:FunctionalModeDataType}
\tabulinesep=3pt
\begin{tabu} to 6in {|l|r|} \everyrow{\hline}
\hline
\rowfont\bfseries {Name} & {Index} \\
\tabucline[1.5pt]{}
\texttt{MAINTENANCE} & \texttt{0} \\
\texttt{PRODUCTION} & \texttt{1} \\
\texttt{PROCESS_DEVELOPMENT} & \texttt{2} \\
\texttt{SETUP} & \texttt{3} \\
\texttt{TEARDOWN} & \texttt{4} \\
\end{tabu}
\end{table} 
\FloatBarrier
\end{itemize}
\FloatBarrier
\subsubsection{Defintion of \texttt{ InterfaceStateClassType}}
  \label{type:InterfaceStateClassType}

\FloatBarrier

The current functional or operational state of an Interface type element indicating whether the Interface is active or not currently functioning.

An indication of the operational state of an interface component component.

\begin{table}[ht]
\centering 
  \caption{\texttt{InterfaceStateClassType} Definition}
  \label{table:InterfaceStateClassType}
\fontsize{9pt}{11pt}\selectfont
\tabulinesep=3pt
\begin{tabu} to 6in {|X[-1.35]|X[-0.7]|X[-1.75]|X[-1.5]|X[-1]|X[-0.7]|} \everyrow{\hline}
\hline
\rowfont\bfseries {Attribute} & \multicolumn{5}{|l|}{Value} \\
\tabucline[1.5pt]{}
BrowseName & \multicolumn{5}{|l|}{InterfaceStateClassType} \\
IsAbstract & \multicolumn{5}{|l|}{False} \\
\tabucline[1.5pt]{}
\rowfont \bfseries References & NodeClass & BrowseName & DataType & Type\-Definition & {Modeling\-Rule} \\
\multicolumn{6}{|l|}{Subtype of MTControlledVocabEventClassType (See section \ref{type:MTControlledVocabEventClassType})} \\
Has\-Property & Variable & Enum\-Strings & Interface\-Status\-Data\-Type & Interface\-Status\-Data\-Type & Mandatory \\
\end{tabu}
\end{table} 


\FloatBarrier
\paragraph{Referenced Properties and Objects}

\begin{itemize}
\item \textbf{Allowable Values} for \texttt{InterfaceStatusDataType}
\FloatBarrier



\begin{table}[ht]
\centering 
  \caption{\texttt{InterfaceStatusDataType} Enumeration}
  \label{enum:InterfaceStatusDataType}
\tabulinesep=3pt
\begin{tabu} to 6in {|l|r|} \everyrow{\hline}
\hline
\rowfont\bfseries {Name} & {Index} \\
\tabucline[1.5pt]{}
\texttt{DISABLED} & \texttt{0} \\
\texttt{ENABLED} & \texttt{1} \\
\end{tabu}
\end{table} 
\FloatBarrier
\end{itemize}
\FloatBarrier
\subsubsection{Defintion of \texttt{ MaterialChangeClassType}}
  \label{type:MaterialChangeClassType}

\FloatBarrier

Service to change the type of material or product being loaded or fed to a piece of equipment.

Service to change the type of material or product being loaded or fed to a piece of equipment.

\begin{table}[ht]
\centering 
  \caption{\texttt{MaterialChangeClassType} Definition}
  \label{table:MaterialChangeClassType}
\fontsize{9pt}{11pt}\selectfont
\tabulinesep=3pt
\begin{tabu} to 6in {|X[-1.35]|X[-0.7]|X[-1.75]|X[-1.5]|X[-1]|X[-0.7]|} \everyrow{\hline}
\hline
\rowfont\bfseries {Attribute} & \multicolumn{5}{|l|}{Value} \\
\tabucline[1.5pt]{}
BrowseName & \multicolumn{5}{|l|}{MaterialChangeClassType} \\
IsAbstract & \multicolumn{5}{|l|}{False} \\
\tabucline[1.5pt]{}
\rowfont \bfseries References & NodeClass & BrowseName & DataType & Type\-Definition & {Modeling\-Rule} \\
\multicolumn{6}{|l|}{Subtype of MTControlledVocabEventClassType (See section \ref{type:MTControlledVocabEventClassType})} \\
Has\-Property & Variable & Enum\-Strings & Interface\-State\-Data\-Type & Interface\-State\-Data\-Type & Mandatory \\
\end{tabu}
\end{table} 


\FloatBarrier
\paragraph{Referenced Properties and Objects}

\begin{itemize}
\item \textbf{Allowable Values} for \texttt{InterfaceStateDataType}
\FloatBarrier



\begin{table}[ht]
\centering 
  \caption{\texttt{InterfaceStateDataType} Enumeration}
\tabulinesep=3pt
\begin{tabu} to 6in {|l|r|} \everyrow{\hline}
\hline
\rowfont\bfseries {Name} & {Index} \\
\tabucline[1.5pt]{}
\texttt{ACTIVE} & \texttt{0} \\
\texttt{COMPLETE} & \texttt{1} \\
\texttt{FAIL} & \texttt{2} \\
\texttt{NOT_READY} & \texttt{4} \\
\texttt{READY} & \texttt{5} \\
\end{tabu}
\end{table} 
\FloatBarrier
\end{itemize}
\FloatBarrier
\subsubsection{Defintion of \texttt{ MaterialFeedClassType}}
  \label{type:MaterialFeedClassType}

\FloatBarrier

Service to advance material or feed product to a piece of equipment from a continuous or bulk source.

Service to advance material or feed product to a piece of equipment from a continuous or bulk source. 

\begin{table}[ht]
\centering 
  \caption{\texttt{MaterialFeedClassType} Definition}
  \label{table:MaterialFeedClassType}
\fontsize{9pt}{11pt}\selectfont
\tabulinesep=3pt
\begin{tabu} to 6in {|X[-1.35]|X[-0.7]|X[-1.75]|X[-1.5]|X[-1]|X[-0.7]|} \everyrow{\hline}
\hline
\rowfont\bfseries {Attribute} & \multicolumn{5}{|l|}{Value} \\
\tabucline[1.5pt]{}
BrowseName & \multicolumn{5}{|l|}{MaterialFeedClassType} \\
IsAbstract & \multicolumn{5}{|l|}{False} \\
\tabucline[1.5pt]{}
\rowfont \bfseries References & NodeClass & BrowseName & DataType & Type\-Definition & {Modeling\-Rule} \\
\multicolumn{6}{|l|}{Subtype of MTControlledVocabEventClassType (See section \ref{type:MTControlledVocabEventClassType})} \\
Has\-Property & Variable & Enum\-Strings & Interface\-State\-Data\-Type & Interface\-State\-Data\-Type & Mandatory \\
\end{tabu}
\end{table} 


\FloatBarrier
\paragraph{Referenced Properties and Objects}

\begin{itemize}
\item \textbf{Allowable Values} for \texttt{InterfaceStateDataType}
\FloatBarrier



\begin{table}[ht]
\centering 
  \caption{\texttt{InterfaceStateDataType} Enumeration}
\tabulinesep=3pt
\begin{tabu} to 6in {|l|r|} \everyrow{\hline}
\hline
\rowfont\bfseries {Name} & {Index} \\
\tabucline[1.5pt]{}
\texttt{ACTIVE} & \texttt{0} \\
\texttt{COMPLETE} & \texttt{1} \\
\texttt{FAIL} & \texttt{2} \\
\texttt{NOT_READY} & \texttt{4} \\
\texttt{READY} & \texttt{5} \\
\end{tabu}
\end{table} 
\FloatBarrier
\end{itemize}
\FloatBarrier
\subsubsection{Defintion of \texttt{ MaterialLoadClassType}}
  \label{type:MaterialLoadClassType}

\FloatBarrier

Service to load a piece of material or product.

Service to load a piece of material or product.

\begin{table}[ht]
\centering 
  \caption{\texttt{MaterialLoadClassType} Definition}
  \label{table:MaterialLoadClassType}
\fontsize{9pt}{11pt}\selectfont
\tabulinesep=3pt
\begin{tabu} to 6in {|X[-1.35]|X[-0.7]|X[-1.75]|X[-1.5]|X[-1]|X[-0.7]|} \everyrow{\hline}
\hline
\rowfont\bfseries {Attribute} & \multicolumn{5}{|l|}{Value} \\
\tabucline[1.5pt]{}
BrowseName & \multicolumn{5}{|l|}{MaterialLoadClassType} \\
IsAbstract & \multicolumn{5}{|l|}{False} \\
\tabucline[1.5pt]{}
\rowfont \bfseries References & NodeClass & BrowseName & DataType & Type\-Definition & {Modeling\-Rule} \\
\multicolumn{6}{|l|}{Subtype of MTControlledVocabEventClassType (See section \ref{type:MTControlledVocabEventClassType})} \\
Has\-Property & Variable & Enum\-Strings & Interface\-State\-Data\-Type & Interface\-State\-Data\-Type & Mandatory \\
\end{tabu}
\end{table} 


\FloatBarrier
\paragraph{Referenced Properties and Objects}

\begin{itemize}
\item \textbf{Allowable Values} for \texttt{InterfaceStateDataType}
\FloatBarrier



\begin{table}[ht]
\centering 
  \caption{\texttt{InterfaceStateDataType} Enumeration}
\tabulinesep=3pt
\begin{tabu} to 6in {|l|r|} \everyrow{\hline}
\hline
\rowfont\bfseries {Name} & {Index} \\
\tabucline[1.5pt]{}
\texttt{ACTIVE} & \texttt{0} \\
\texttt{COMPLETE} & \texttt{1} \\
\texttt{FAIL} & \texttt{2} \\
\texttt{NOT_READY} & \texttt{4} \\
\texttt{READY} & \texttt{5} \\
\end{tabu}
\end{table} 
\FloatBarrier
\end{itemize}
\FloatBarrier
\subsubsection{Defintion of \texttt{ MaterialRetractClassType}}
  \label{type:MaterialRetractClassType}

\FloatBarrier

Service to remove or retract material or product.

Service to remove or retract material or product.

\begin{table}[ht]
\centering 
  \caption{\texttt{MaterialRetractClassType} Definition}
  \label{table:MaterialRetractClassType}
\fontsize{9pt}{11pt}\selectfont
\tabulinesep=3pt
\begin{tabu} to 6in {|X[-1.35]|X[-0.7]|X[-1.75]|X[-1.5]|X[-1]|X[-0.7]|} \everyrow{\hline}
\hline
\rowfont\bfseries {Attribute} & \multicolumn{5}{|l|}{Value} \\
\tabucline[1.5pt]{}
BrowseName & \multicolumn{5}{|l|}{MaterialRetractClassType} \\
IsAbstract & \multicolumn{5}{|l|}{False} \\
\tabucline[1.5pt]{}
\rowfont \bfseries References & NodeClass & BrowseName & DataType & Type\-Definition & {Modeling\-Rule} \\
\multicolumn{6}{|l|}{Subtype of MTControlledVocabEventClassType (See section \ref{type:MTControlledVocabEventClassType})} \\
Has\-Property & Variable & Enum\-Strings & Interface\-State\-Data\-Type & Interface\-State\-Data\-Type & Mandatory \\
\end{tabu}
\end{table} 


\FloatBarrier
\paragraph{Referenced Properties and Objects}

\begin{itemize}
\item \textbf{Allowable Values} for \texttt{InterfaceStateDataType}
\FloatBarrier



\begin{table}[ht]
\centering 
  \caption{\texttt{InterfaceStateDataType} Enumeration}
\tabulinesep=3pt
\begin{tabu} to 6in {|l|r|} \everyrow{\hline}
\hline
\rowfont\bfseries {Name} & {Index} \\
\tabucline[1.5pt]{}
\texttt{ACTIVE} & \texttt{0} \\
\texttt{COMPLETE} & \texttt{1} \\
\texttt{FAIL} & \texttt{2} \\
\texttt{NOT_READY} & \texttt{4} \\
\texttt{READY} & \texttt{5} \\
\end{tabu}
\end{table} 
\FloatBarrier
\end{itemize}
\FloatBarrier
\subsubsection{Defintion of \texttt{ MaterialUnloadClassType}}
  \label{type:MaterialUnloadClassType}

\FloatBarrier

Service to unload a piece of material or product.

Service to unload a piece of material or product.

\begin{table}[ht]
\centering 
  \caption{\texttt{MaterialUnloadClassType} Definition}
  \label{table:MaterialUnloadClassType}
\fontsize{9pt}{11pt}\selectfont
\tabulinesep=3pt
\begin{tabu} to 6in {|X[-1.35]|X[-0.7]|X[-1.75]|X[-1.5]|X[-1]|X[-0.7]|} \everyrow{\hline}
\hline
\rowfont\bfseries {Attribute} & \multicolumn{5}{|l|}{Value} \\
\tabucline[1.5pt]{}
BrowseName & \multicolumn{5}{|l|}{MaterialUnloadClassType} \\
IsAbstract & \multicolumn{5}{|l|}{False} \\
\tabucline[1.5pt]{}
\rowfont \bfseries References & NodeClass & BrowseName & DataType & Type\-Definition & {Modeling\-Rule} \\
\multicolumn{6}{|l|}{Subtype of MTControlledVocabEventClassType (See section \ref{type:MTControlledVocabEventClassType})} \\
Has\-Property & Variable & Enum\-Strings & Interface\-State\-Data\-Type & Interface\-State\-Data\-Type & Mandatory \\
\end{tabu}
\end{table} 


\FloatBarrier
\paragraph{Referenced Properties and Objects}

\begin{itemize}
\item \textbf{Allowable Values} for \texttt{InterfaceStateDataType}
\FloatBarrier



\begin{table}[ht]
\centering 
  \caption{\texttt{InterfaceStateDataType} Enumeration}
\tabulinesep=3pt
\begin{tabu} to 6in {|l|r|} \everyrow{\hline}
\hline
\rowfont\bfseries {Name} & {Index} \\
\tabucline[1.5pt]{}
\texttt{ACTIVE} & \texttt{0} \\
\texttt{COMPLETE} & \texttt{1} \\
\texttt{FAIL} & \texttt{2} \\
\texttt{NOT_READY} & \texttt{4} \\
\texttt{READY} & \texttt{5} \\
\end{tabu}
\end{table} 
\FloatBarrier
\end{itemize}
\FloatBarrier
\subsubsection{Defintion of \texttt{ OpenChuckClassType}}
  \label{type:OpenChuckClassType}

\FloatBarrier

Service to open a chuck.

Service to open a chuck. 

\begin{table}[ht]
\centering 
  \caption{\texttt{OpenChuckClassType} Definition}
  \label{table:OpenChuckClassType}
\fontsize{9pt}{11pt}\selectfont
\tabulinesep=3pt
\begin{tabu} to 6in {|X[-1.35]|X[-0.7]|X[-1.75]|X[-1.5]|X[-1]|X[-0.7]|} \everyrow{\hline}
\hline
\rowfont\bfseries {Attribute} & \multicolumn{5}{|l|}{Value} \\
\tabucline[1.5pt]{}
BrowseName & \multicolumn{5}{|l|}{OpenChuckClassType} \\
IsAbstract & \multicolumn{5}{|l|}{False} \\
\tabucline[1.5pt]{}
\rowfont \bfseries References & NodeClass & BrowseName & DataType & Type\-Definition & {Modeling\-Rule} \\
\multicolumn{6}{|l|}{Subtype of MTControlledVocabEventClassType (See section \ref{type:MTControlledVocabEventClassType})} \\
Has\-Property & Variable & Enum\-Strings & Interface\-State\-Data\-Type & Interface\-State\-Data\-Type & Mandatory \\
\end{tabu}
\end{table} 


\FloatBarrier
\paragraph{Referenced Properties and Objects}

\begin{itemize}
\item \textbf{Allowable Values} for \texttt{InterfaceStateDataType}
\FloatBarrier



\begin{table}[ht]
\centering 
  \caption{\texttt{InterfaceStateDataType} Enumeration}
\tabulinesep=3pt
\begin{tabu} to 6in {|l|r|} \everyrow{\hline}
\hline
\rowfont\bfseries {Name} & {Index} \\
\tabucline[1.5pt]{}
\texttt{ACTIVE} & \texttt{0} \\
\texttt{COMPLETE} & \texttt{1} \\
\texttt{FAIL} & \texttt{2} \\
\texttt{NOT_READY} & \texttt{4} \\
\texttt{READY} & \texttt{5} \\
\end{tabu}
\end{table} 
\FloatBarrier
\end{itemize}
\FloatBarrier
\subsubsection{Defintion of \texttt{ OpenDoorClassType}}
  \label{type:OpenDoorClassType}

\FloatBarrier

Service to open a door.

Service to open a door. 

\begin{table}[ht]
\centering 
  \caption{\texttt{OpenDoorClassType} Definition}
  \label{table:OpenDoorClassType}
\fontsize{9pt}{11pt}\selectfont
\tabulinesep=3pt
\begin{tabu} to 6in {|X[-1.35]|X[-0.7]|X[-1.75]|X[-1.5]|X[-1]|X[-0.7]|} \everyrow{\hline}
\hline
\rowfont\bfseries {Attribute} & \multicolumn{5}{|l|}{Value} \\
\tabucline[1.5pt]{}
BrowseName & \multicolumn{5}{|l|}{OpenDoorClassType} \\
IsAbstract & \multicolumn{5}{|l|}{False} \\
\tabucline[1.5pt]{}
\rowfont \bfseries References & NodeClass & BrowseName & DataType & Type\-Definition & {Modeling\-Rule} \\
\multicolumn{6}{|l|}{Subtype of MTControlledVocabEventClassType (See section \ref{type:MTControlledVocabEventClassType})} \\
Has\-Property & Variable & Enum\-Strings & Interface\-State\-Data\-Type & Interface\-State\-Data\-Type & Mandatory \\
\end{tabu}
\end{table} 


\FloatBarrier
\paragraph{Referenced Properties and Objects}

\begin{itemize}
\item \textbf{Allowable Values} for \texttt{InterfaceStateDataType}
\FloatBarrier



\begin{table}[ht]
\centering 
  \caption{\texttt{InterfaceStateDataType} Enumeration}
\tabulinesep=3pt
\begin{tabu} to 6in {|l|r|} \everyrow{\hline}
\hline
\rowfont\bfseries {Name} & {Index} \\
\tabucline[1.5pt]{}
\texttt{ACTIVE} & \texttt{0} \\
\texttt{COMPLETE} & \texttt{1} \\
\texttt{FAIL} & \texttt{2} \\
\texttt{NOT_READY} & \texttt{4} \\
\texttt{READY} & \texttt{5} \\
\end{tabu}
\end{table} 
\FloatBarrier
\end{itemize}
\FloatBarrier
\subsubsection{Defintion of \texttt{ PartChangeClassType}}
  \label{type:PartChangeClassType}

\FloatBarrier

Service to change the part or product associated with a piece of equipment to a different part or product.

Service to change the part or product associated with a piece of equipment to a different part or product.  

\begin{table}[ht]
\centering 
  \caption{\texttt{PartChangeClassType} Definition}
  \label{table:PartChangeClassType}
\fontsize{9pt}{11pt}\selectfont
\tabulinesep=3pt
\begin{tabu} to 6in {|X[-1.35]|X[-0.7]|X[-1.75]|X[-1.5]|X[-1]|X[-0.7]|} \everyrow{\hline}
\hline
\rowfont\bfseries {Attribute} & \multicolumn{5}{|l|}{Value} \\
\tabucline[1.5pt]{}
BrowseName & \multicolumn{5}{|l|}{PartChangeClassType} \\
IsAbstract & \multicolumn{5}{|l|}{False} \\
\tabucline[1.5pt]{}
\rowfont \bfseries References & NodeClass & BrowseName & DataType & Type\-Definition & {Modeling\-Rule} \\
\multicolumn{6}{|l|}{Subtype of MTControlledVocabEventClassType (See section \ref{type:MTControlledVocabEventClassType})} \\
Has\-Property & Variable & Enum\-Strings & Interface\-State\-Data\-Type & Interface\-State\-Data\-Type & Mandatory \\
\end{tabu}
\end{table} 


\FloatBarrier
\paragraph{Referenced Properties and Objects}

\begin{itemize}
\item \textbf{Allowable Values} for \texttt{InterfaceStateDataType}
\FloatBarrier



\begin{table}[ht]
\centering 
  \caption{\texttt{InterfaceStateDataType} Enumeration}
\tabulinesep=3pt
\begin{tabu} to 6in {|l|r|} \everyrow{\hline}
\hline
\rowfont\bfseries {Name} & {Index} \\
\tabucline[1.5pt]{}
\texttt{ACTIVE} & \texttt{0} \\
\texttt{COMPLETE} & \texttt{1} \\
\texttt{FAIL} & \texttt{2} \\
\texttt{NOT_READY} & \texttt{4} \\
\texttt{READY} & \texttt{5} \\
\end{tabu}
\end{table} 
\FloatBarrier
\end{itemize}
\FloatBarrier
\subsubsection{Defintion of \texttt{ PathModeClassType}}
  \label{type:PathModeClassType}

\FloatBarrier

Describes the operational relationship between a \mtmodel{PATH} \textit{Structural Element} and another \mtmodel{PATH} 
\textit{Structural Element} for pieces of equipment comprised of multiple logical groupings of controlled axes or other logical operations.

Describes the operational relationship between a path structural element and another path structural element for pieces of equipment comprised of multiple logical groupings of controlled axes or other logical operations.

\begin{table}[ht]
\centering 
  \caption{\texttt{PathModeClassType} Definition}
  \label{table:PathModeClassType}
\fontsize{9pt}{11pt}\selectfont
\tabulinesep=3pt
\begin{tabu} to 6in {|X[-1.35]|X[-0.7]|X[-1.75]|X[-1.5]|X[-1]|X[-0.7]|} \everyrow{\hline}
\hline
\rowfont\bfseries {Attribute} & \multicolumn{5}{|l|}{Value} \\
\tabucline[1.5pt]{}
BrowseName & \multicolumn{5}{|l|}{PathModeClassType} \\
IsAbstract & \multicolumn{5}{|l|}{False} \\
\tabucline[1.5pt]{}
\rowfont \bfseries References & NodeClass & BrowseName & DataType & Type\-Definition & {Modeling\-Rule} \\
\multicolumn{6}{|l|}{Subtype of MTControlledVocabEventClassType (See section \ref{type:MTControlledVocabEventClassType})} \\
Has\-Property & Variable & Enum\-Strings & Path\-Mode\-Data\-Type & Path\-Mode\-Data\-Type & Mandatory \\
\end{tabu}
\end{table} 


\FloatBarrier
\paragraph{Referenced Properties and Objects}

\begin{itemize}
\item \textbf{Allowable Values} for \texttt{PathModeDataType}
\FloatBarrier



\begin{table}[ht]
\centering 
  \caption{\texttt{PathModeDataType} Enumeration}
  \label{enum:PathModeDataType}
\tabulinesep=3pt
\begin{tabu} to 6in {|l|r|} \everyrow{\hline}
\hline
\rowfont\bfseries {Name} & {Index} \\
\tabucline[1.5pt]{}
\texttt{INDEPENDENT} & \texttt{0} \\
\texttt{MASTER} & \texttt{1} \\
\texttt{MIRROR} & \texttt{2} \\
\texttt{SYNCHRONOUS} & \texttt{3} \\
\end{tabu}
\end{table} 
\FloatBarrier
\end{itemize}
\FloatBarrier
\subsubsection{Defintion of \texttt{ PowerStateClassType}}
  \label{type:PowerStateClassType}

\FloatBarrier

The indication of the status of the source of energy for a \textit{Structural Element} to allow it to perform its 
intended function or the state of an enabling signal providing permission for the \textit{Structural Element} to
perform its functions.

 DEPRECATION WARNING: \texttt{PowerState} may be deprecated in the future.

The indication of the status of the source of energy for a structural element to allow it to perform its intended function or the state of an enabling signal providing permission for the structural element to perform its functions.

\begin{table}[ht]
\centering 
  \caption{\texttt{PowerStateClassType} Definition}
  \label{table:PowerStateClassType}
\fontsize{9pt}{11pt}\selectfont
\tabulinesep=3pt
\begin{tabu} to 6in {|X[-1.35]|X[-0.7]|X[-1.75]|X[-1.5]|X[-1]|X[-0.7]|} \everyrow{\hline}
\hline
\rowfont\bfseries {Attribute} & \multicolumn{5}{|l|}{Value} \\
\tabucline[1.5pt]{}
BrowseName & \multicolumn{5}{|l|}{PowerStateClassType} \\
IsAbstract & \multicolumn{5}{|l|}{False} \\
\tabucline[1.5pt]{}
\rowfont \bfseries References & NodeClass & BrowseName & DataType & Type\-Definition & {Modeling\-Rule} \\
\multicolumn{6}{|l|}{Subtype of MTControlledVocabEventClassType (See section \ref{type:MTControlledVocabEventClassType})} \\
Has\-Property & Variable & Enum\-Strings & On\-Off\-Data\-Type & On\-Off\-Data\-Type & Mandatory \\
\end{tabu}
\end{table} 


\FloatBarrier
\paragraph{Referenced Properties and Objects}

\begin{itemize}
\item \textbf{Allowable Values} for \texttt{OnOffDataType}
\FloatBarrier



\begin{table}[ht]
\centering 
  \caption{\texttt{OnOffDataType} Enumeration}
\tabulinesep=3pt
\begin{tabu} to 6in {|l|r|} \everyrow{\hline}
\hline
\rowfont\bfseries {Name} & {Index} \\
\tabucline[1.5pt]{}
\texttt{OFF} & \texttt{0} \\
\texttt{ON} & \texttt{1} \\
\end{tabu}
\end{table} 
\FloatBarrier
\end{itemize}
\FloatBarrier
\subsubsection{Defintion of \texttt{ ProgramEditClassType}}
  \label{type:ProgramEditClassType}

\FloatBarrier

An indication of the status of the \mtmodel{Controller} component’s program editing mode.

On many controls, a program can be edited while another program is currently being executed.

An indication of the status of the controller components program editing mode. 
 On many controls, a program can be edited while another program is currently being executed.

\begin{table}[ht]
\centering 
  \caption{\texttt{ProgramEditClassType} Definition}
  \label{table:ProgramEditClassType}
\fontsize{9pt}{11pt}\selectfont
\tabulinesep=3pt
\begin{tabu} to 6in {|X[-1.35]|X[-0.7]|X[-1.75]|X[-1.5]|X[-1]|X[-0.7]|} \everyrow{\hline}
\hline
\rowfont\bfseries {Attribute} & \multicolumn{5}{|l|}{Value} \\
\tabucline[1.5pt]{}
BrowseName & \multicolumn{5}{|l|}{ProgramEditClassType} \\
IsAbstract & \multicolumn{5}{|l|}{False} \\
\tabucline[1.5pt]{}
\rowfont \bfseries References & NodeClass & BrowseName & DataType & Type\-Definition & {Modeling\-Rule} \\
\multicolumn{6}{|l|}{Subtype of MTControlledVocabEventClassType (See section \ref{type:MTControlledVocabEventClassType})} \\
Has\-Property & Variable & Enum\-Strings & Program\-Edit\-Data\-Type & Program\-Edit\-Data\-Type & Mandatory \\
\end{tabu}
\end{table} 


\FloatBarrier
\paragraph{Referenced Properties and Objects}

\begin{itemize}
\item \textbf{Allowable Values} for \texttt{ProgramEditDataType}
\FloatBarrier



\begin{table}[ht]
\centering 
  \caption{\texttt{ProgramEditDataType} Enumeration}
  \label{enum:ProgramEditDataType}
\tabulinesep=3pt
\begin{tabu} to 6in {|l|r|} \everyrow{\hline}
\hline
\rowfont\bfseries {Name} & {Index} \\
\tabucline[1.5pt]{}
\texttt{ACTIVE} & \texttt{0} \\
\texttt{NOT_READY} & \texttt{1} \\
\texttt{READY} & \texttt{2} \\
\end{tabu}
\end{table} 
\FloatBarrier
\end{itemize}
\FloatBarrier
\subsubsection{Defintion of \texttt{ RotaryModeClassType}}
  \label{type:RotaryModeClassType}

\FloatBarrier

The current operating mode for a \mtmodel{Rotary} type axis.

The current operating mode for a rotary type axis.

\begin{table}[ht]
\centering 
  \caption{\texttt{RotaryModeClassType} Definition}
  \label{table:RotaryModeClassType}
\fontsize{9pt}{11pt}\selectfont
\tabulinesep=3pt
\begin{tabu} to 6in {|X[-1.35]|X[-0.7]|X[-1.75]|X[-1.5]|X[-1]|X[-0.7]|} \everyrow{\hline}
\hline
\rowfont\bfseries {Attribute} & \multicolumn{5}{|l|}{Value} \\
\tabucline[1.5pt]{}
BrowseName & \multicolumn{5}{|l|}{RotaryModeClassType} \\
IsAbstract & \multicolumn{5}{|l|}{False} \\
\tabucline[1.5pt]{}
\rowfont \bfseries References & NodeClass & BrowseName & DataType & Type\-Definition & {Modeling\-Rule} \\
\multicolumn{6}{|l|}{Subtype of MTControlledVocabEventClassType (See section \ref{type:MTControlledVocabEventClassType})} \\
Has\-Property & Variable & Enum\-Strings & Rotary\-Mode\-Data\-Type & Rotary\-Mode\-Data\-Type & Mandatory \\
\end{tabu}
\end{table} 


\FloatBarrier
\paragraph{Referenced Properties and Objects}

\begin{itemize}
\item \textbf{Allowable Values} for \texttt{RotaryModeDataType}
\FloatBarrier



\begin{table}[ht]
\centering 
  \caption{\texttt{RotaryModeDataType} Enumeration}
  \label{enum:RotaryModeDataType}
\tabulinesep=3pt
\begin{tabu} to 6in {|l|r|} \everyrow{\hline}
\hline
\rowfont\bfseries {Name} & {Index} \\
\tabucline[1.5pt]{}
\texttt{CONTOUR} & \texttt{0} \\
\texttt{INDEX} & \texttt{1} \\
\texttt{SPINDLE} & \texttt{2} \\
\end{tabu}
\end{table} 
\FloatBarrier
\end{itemize}
\FloatBarrier
\subsubsection{Defintion of \texttt{ SpindleInterlockClassType}}
  \label{type:SpindleInterlockClassType}

\FloatBarrier

An indication of the status of the spindle for a piece of equipment when power has been removed and it is free to rotate.

An indication of the status of the spindle for a piece of equipment when power has been removed and it is free to rotate.

\begin{table}[ht]
\centering 
  \caption{\texttt{SpindleInterlockClassType} Definition}
  \label{table:SpindleInterlockClassType}
\fontsize{9pt}{11pt}\selectfont
\tabulinesep=3pt
\begin{tabu} to 6in {|X[-1.35]|X[-0.7]|X[-1.75]|X[-1.5]|X[-1]|X[-0.7]|} \everyrow{\hline}
\hline
\rowfont\bfseries {Attribute} & \multicolumn{5}{|l|}{Value} \\
\tabucline[1.5pt]{}
BrowseName & \multicolumn{5}{|l|}{SpindleInterlockClassType} \\
IsAbstract & \multicolumn{5}{|l|}{False} \\
\tabucline[1.5pt]{}
\rowfont \bfseries References & NodeClass & BrowseName & DataType & Type\-Definition & {Modeling\-Rule} \\
\multicolumn{6}{|l|}{Subtype of MTControlledVocabEventClassType (See section \ref{type:MTControlledVocabEventClassType})} \\
Has\-Property & Variable & Enum\-Strings & Active\-State\-Data\-Type & Active\-State\-Data\-Type & Mandatory \\
\end{tabu}
\end{table} 


\FloatBarrier
\paragraph{Referenced Properties and Objects}

\begin{itemize}
\item \textbf{Allowable Values} for \texttt{ActiveStateDataType}
\FloatBarrier



\begin{table}[ht]
\centering 
  \caption{\texttt{ActiveStateDataType} Enumeration}
\tabulinesep=3pt
\begin{tabu} to 6in {|l|r|} \everyrow{\hline}
\hline
\rowfont\bfseries {Name} & {Index} \\
\tabucline[1.5pt]{}
\texttt{ACTIVE} & \texttt{0} \\
\texttt{INACTIVE} & \texttt{1} \\
\end{tabu}
\end{table} 
\FloatBarrier
\end{itemize}
\FloatBarrier
