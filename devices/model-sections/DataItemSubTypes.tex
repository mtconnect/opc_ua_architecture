% Generated 2020-01-10 14:19:52 -0800
\subsection{Data Item Sub Types} \label{model:DataItemSubTypes}

he \glspl{subType} are related to the various \glspl{Variable} using the \uamodel{HasMTSubClassType}. 
These provide the the second level type classifiaction of the MTConnect \glspl{MTDataItem} and 
also the \uamodel{ConditionSubClassId} for the \glspl{MTCondition}.

\subsubsection{Defintion of \texttt{ MTDataItemSubClassType}}
  \label{type:MTDataItemSubClassType}

\FloatBarrier



data entity describing a piece of information reported about a piece of equipment.

\begin{table}[ht]
\centering 
  \caption{\texttt{MTDataItemSubClassType} Definition}
  \label{table:MTDataItemSubClassType}
\fontsize{9pt}{11pt}\selectfont
\tabulinesep=3pt
\begin{tabu} to 6in {|X[-1.35]|X[-0.7]|X[-1.75]|X[-1.5]|X[-1]|X[-0.7]|} \everyrow{\hline}
\hline
\rowfont\bfseries {Attribute} & \multicolumn{5}{|l|}{Value} \\
\tabucline[1.5pt]{}
BrowseName & \multicolumn{5}{|l|}{MTDataItemSubClassType} \\
IsAbstract & \multicolumn{5}{|l|}{True} \\
\tabucline[1.5pt]{}
\rowfont \bfseries References & NodeClass & BrowseName & DataType & Type\-Definition & {Modeling\-Rule} \\
\multicolumn{6}{|l|}{Subtype of BaseConditionClassType (See \cite{UAPart9} Documentation)} \\
HasSubtype & ObjectType & \multicolumn{2}{l}{RelativeSubClassType} & \multicolumn{2}{|l|}{See section \ref{type:RelativeSubClassType}} \\
HasSubtype & ObjectType & \multicolumn{2}{l}{RemainingSubClassType} & \multicolumn{2}{|l|}{See section \ref{type:RemainingSubClassType}} \\
HasSubtype & ObjectType & \multicolumn{2}{l}{RequestSubClassType} & \multicolumn{2}{|l|}{See section \ref{type:RequestSubClassType}} \\
HasSubtype & ObjectType & \multicolumn{2}{l}{ResponseSubClassType} & \multicolumn{2}{|l|}{See section \ref{type:ResponseSubClassType}} \\
HasSubtype & ObjectType & \multicolumn{2}{l}{RockwellSubClassType} & \multicolumn{2}{|l|}{See section \ref{type:RockwellSubClassType}} \\
HasSubtype & ObjectType & \multicolumn{2}{l}{RotarySubClassType} & \multicolumn{2}{|l|}{See section \ref{type:RotarySubClassType}} \\
HasSubtype & ObjectType & \multicolumn{2}{l}{SetUpSubClassType} & \multicolumn{2}{|l|}{See section \ref{type:SetUpSubClassType}} \\
HasSubtype & ObjectType & \multicolumn{2}{l}{ShoreSubClassType} & \multicolumn{2}{|l|}{See section \ref{type:ShoreSubClassType}} \\
HasSubtype & ObjectType & \multicolumn{2}{l}{StandardSubClassType} & \multicolumn{2}{|l|}{See section \ref{type:StandardSubClassType}} \\
HasSubtype & ObjectType & \multicolumn{2}{l}{SwitchedSubClassType} & \multicolumn{2}{|l|}{See section \ref{type:SwitchedSubClassType}} \\
HasSubtype & ObjectType & \multicolumn{2}{l}{TargetSubClassType} & \multicolumn{2}{|l|}{See section \ref{type:TargetSubClassType}} \\
HasSubtype & ObjectType & \multicolumn{2}{l}{ToolChangeStopSubClassType} & \multicolumn{2}{|l|}{See section \ref{type:ToolChangeStopSubClassType}} \\
HasSubtype & ObjectType & \multicolumn{2}{l}{ToolEdgeSubClassType} & \multicolumn{2}{|l|}{See section \ref{type:ToolEdgeSubClassType}} \\
HasSubtype & ObjectType & \multicolumn{2}{l}{ToolGroupSubClassType} & \multicolumn{2}{|l|}{See section \ref{type:ToolGroupSubClassType}} \\
HasSubtype & ObjectType & \multicolumn{2}{l}{ToolSubClassType} & \multicolumn{2}{|l|}{See section \ref{type:ToolSubClassType}} \\
HasSubtype & ObjectType & \multicolumn{2}{l}{UasbleSubClassType} & \multicolumn{2}{|l|}{See section \ref{type:UasbleSubClassType}} \\
HasSubtype & ObjectType & \multicolumn{2}{l}{VerticalSubClassType} & \multicolumn{2}{|l|}{See section \ref{type:VerticalSubClassType}} \\
HasSubtype & ObjectType & \multicolumn{2}{l}{VickersSubClassType} & \multicolumn{2}{|l|}{See section \ref{type:VickersSubClassType}} \\
HasSubtype & ObjectType & \multicolumn{2}{l}{VolumeSubClassType} & \multicolumn{2}{|l|}{See section \ref{type:VolumeSubClassType}} \\
HasSubtype & ObjectType & \multicolumn{2}{l}{WeightSubClassType} & \multicolumn{2}{|l|}{See section \ref{type:WeightSubClassType}} \\
HasSubtype & ObjectType & \multicolumn{2}{l}{WorkingSubClassType} & \multicolumn{2}{|l|}{See section \ref{type:WorkingSubClassType}} \\
HasSubtype & ObjectType & \multicolumn{2}{l}{WorkpieceSubClassType} & \multicolumn{2}{|l|}{See section \ref{type:WorkpieceSubClassType}} \\
\multicolumn{6}{|l|}{Continued...} \\
\end{tabu}
\end{table}
\begin{table}[ht]
\fontsize{9pt}{11pt}\selectfont
\tabulinesep=3pt
\begin{tabu} to 6in {|X[-1.35]|X[-0.7]|X[-1.75]|X[-1.5]|X[-1]|X[-0.7]|} \everyrow{\hline}
\hline
\rowfont \bfseries References & NodeClass & BrowseName & DataType & Type\-Definition & {Modeling\-Rule} \\
HasSubtype & ObjectType & \multicolumn{2}{l}{LineSubClassType} & \multicolumn{2}{|l|}{See section \ref{type:LineSubClassType}} \\
HasSubtype & ObjectType & \multicolumn{2}{l}{LoadedSubClassType} & \multicolumn{2}{|l|}{See section \ref{type:LoadedSubClassType}} \\
HasSubtype & ObjectType & \multicolumn{2}{l}{MachineAxisLockSubClassType} & \multicolumn{2}{|l|}{See section \ref{type:MachineAxisLockSubClassType}} \\
HasSubtype & ObjectType & \multicolumn{2}{l}{MaintenanceSubClassType} & \multicolumn{2}{|l|}{See section \ref{type:MaintenanceSubClassType}} \\
HasSubtype & ObjectType & \multicolumn{2}{l}{ManualUnclampSubClassType} & \multicolumn{2}{|l|}{See section \ref{type:ManualUnclampSubClassType}} \\
HasSubtype & ObjectType & \multicolumn{2}{l}{MaximumSubClassType} & \multicolumn{2}{|l|}{See section \ref{type:MaximumSubClassType}} \\
HasSubtype & ObjectType & \multicolumn{2}{l}{MinimumSubClassType} & \multicolumn{2}{|l|}{See section \ref{type:MinimumSubClassType}} \\
HasSubtype & ObjectType & \multicolumn{2}{l}{MohsSubClassType} & \multicolumn{2}{|l|}{See section \ref{type:MohsSubClassType}} \\
HasSubtype & ObjectType & \multicolumn{2}{l}{MoleSubClassType} & \multicolumn{2}{|l|}{See section \ref{type:MoleSubClassType}} \\
HasSubtype & ObjectType & \multicolumn{2}{l}{MotionSubClassType} & \multicolumn{2}{|l|}{See section \ref{type:MotionSubClassType}} \\
HasSubtype & ObjectType & \multicolumn{2}{l}{NoScaleSubClassType} & \multicolumn{2}{|l|}{See section \ref{type:NoScaleSubClassType}} \\
HasSubtype & ObjectType & \multicolumn{2}{l}{OperatingSubClassType} & \multicolumn{2}{|l|}{See section \ref{type:OperatingSubClassType}} \\
HasSubtype & ObjectType & \multicolumn{2}{l}{OperatorSubClassType} & \multicolumn{2}{|l|}{See section \ref{type:OperatorSubClassType}} \\
HasSubtype & ObjectType & \multicolumn{2}{l}{OptionalStopSubClassType} & \multicolumn{2}{|l|}{See section \ref{type:OptionalStopSubClassType}} \\
HasSubtype & ObjectType & \multicolumn{2}{l}{OverrideSubClassType} & \multicolumn{2}{|l|}{See section \ref{type:OverrideSubClassType}} \\
HasSubtype & ObjectType & \multicolumn{2}{l}{PoweredSubClassType} & \multicolumn{2}{|l|}{See section \ref{type:PoweredSubClassType}} \\
HasSubtype & ObjectType & \multicolumn{2}{l}{PrimarySubClassType} & \multicolumn{2}{|l|}{See section \ref{type:PrimarySubClassType}} \\
HasSubtype & ObjectType & \multicolumn{2}{l}{ProbeSubClassType} & \multicolumn{2}{|l|}{See section \ref{type:ProbeSubClassType}} \\
HasSubtype & ObjectType & \multicolumn{2}{l}{ProcessSubClassType} & \multicolumn{2}{|l|}{See section \ref{type:ProcessSubClassType}} \\
HasSubtype & ObjectType & \multicolumn{2}{l}{ProgrammedSubClassType} & \multicolumn{2}{|l|}{See section \ref{type:ProgrammedSubClassType}} \\
HasSubtype & ObjectType & \multicolumn{2}{l}{RadialSubClassType} & \multicolumn{2}{|l|}{See section \ref{type:RadialSubClassType}} \\
HasSubtype & ObjectType & \multicolumn{2}{l}{RapidSubClassType} & \multicolumn{2}{|l|}{See section \ref{type:RapidSubClassType}} \\
\multicolumn{6}{|l|}{Continued...} \\
\end{tabu}
\end{table}
\begin{table}[ht]
\fontsize{9pt}{11pt}\selectfont
\tabulinesep=3pt
\begin{tabu} to 6in {|X[-1.35]|X[-0.7]|X[-1.75]|X[-1.5]|X[-1]|X[-0.7]|} \everyrow{\hline}
\hline
\rowfont \bfseries References & NodeClass & BrowseName & DataType & Type\-Definition & {Modeling\-Rule} \\
HasSubtype & ObjectType & \multicolumn{2}{l}{AlternatingSubClassType} & \multicolumn{2}{|l|}{See section \ref{type:AlternatingSubClassType}} \\
HasSubtype & ObjectType & \multicolumn{2}{l}{AScaleSubClassType} & \multicolumn{2}{|l|}{See section \ref{type:AScaleSubClassType}} \\
HasSubtype & ObjectType & \multicolumn{2}{l}{AuxiliarySubClassType} & \multicolumn{2}{|l|}{See section \ref{type:AuxiliarySubClassType}} \\
HasSubtype & ObjectType & \multicolumn{2}{l}{BadSubClassType} & \multicolumn{2}{|l|}{See section \ref{type:BadSubClassType}} \\
HasSubtype & ObjectType & \multicolumn{2}{l}{BrinellSubClassType} & \multicolumn{2}{|l|}{See section \ref{type:BrinellSubClassType}} \\
HasSubtype & ObjectType & \multicolumn{2}{l}{BScaleSubClassType} & \multicolumn{2}{|l|}{See section \ref{type:BScaleSubClassType}} \\
HasSubtype & ObjectType & \multicolumn{2}{l}{CommandedSubClassType} & \multicolumn{2}{|l|}{See section \ref{type:CommandedSubClassType}} \\
HasSubtype & ObjectType & \multicolumn{2}{l}{ControlSubClassType} & \multicolumn{2}{|l|}{See section \ref{type:ControlSubClassType}} \\
HasSubtype & ObjectType & \multicolumn{2}{l}{CScaleSubClassType} & \multicolumn{2}{|l|}{See section \ref{type:CScaleSubClassType}} \\
HasSubtype & ObjectType & \multicolumn{2}{l}{DelaySubClassType} & \multicolumn{2}{|l|}{See section \ref{type:DelaySubClassType}} \\
HasSubtype & ObjectType & \multicolumn{2}{l}{DirectSubClassType} & \multicolumn{2}{|l|}{See section \ref{type:DirectSubClassType}} \\
HasSubtype & ObjectType & \multicolumn{2}{l}{DryRunSubClassType} & \multicolumn{2}{|l|}{See section \ref{type:DryRunSubClassType}} \\
HasSubtype & ObjectType & \multicolumn{2}{l}{DScaleSubClassType} & \multicolumn{2}{|l|}{See section \ref{type:DScaleSubClassType}} \\
HasSubtype & ObjectType & \multicolumn{2}{l}{FixtureSubClassType} & \multicolumn{2}{|l|}{See section \ref{type:FixtureSubClassType}} \\
HasSubtype & ObjectType & \multicolumn{2}{l}{GoodSubClassType} & \multicolumn{2}{|l|}{See section \ref{type:GoodSubClassType}} \\
HasSubtype & ObjectType & \multicolumn{2}{l}{IncrementalSubClassType} & \multicolumn{2}{|l|}{See section \ref{type:IncrementalSubClassType}} \\
HasSubtype & ObjectType & \multicolumn{2}{l}{JobSubClassType} & \multicolumn{2}{|l|}{See section \ref{type:JobSubClassType}} \\
HasSubtype & ObjectType & \multicolumn{2}{l}{KineticSubClassType} & \multicolumn{2}{|l|}{See section \ref{type:KineticSubClassType}} \\
HasSubtype & ObjectType & \multicolumn{2}{l}{LateralSubClassType} & \multicolumn{2}{|l|}{See section \ref{type:LateralSubClassType}} \\
HasSubtype & ObjectType & \multicolumn{2}{l}{LeebSubClassType} & \multicolumn{2}{|l|}{See section \ref{type:LeebSubClassType}} \\
HasSubtype & ObjectType & \multicolumn{2}{l}{LengthSubClassType} & \multicolumn{2}{|l|}{See section \ref{type:LengthSubClassType}} \\
HasSubtype & ObjectType & \multicolumn{2}{l}{LinearSubClassType} & \multicolumn{2}{|l|}{See section \ref{type:LinearSubClassType}} \\
\multicolumn{6}{|l|}{Continued...} \\
\end{tabu}
\end{table}
\begin{table}[ht]
\fontsize{9pt}{11pt}\selectfont
\tabulinesep=3pt
\begin{tabu} to 6in {|X[-1.35]|X[-0.7]|X[-1.75]|X[-1.5]|X[-1]|X[-0.7]|} \everyrow{\hline}
\hline
\rowfont \bfseries References & NodeClass & BrowseName & DataType & Type\-Definition & {Modeling\-Rule} \\
HasSubtype & ObjectType & \multicolumn{2}{l}{AbsoluteSubClassType} & \multicolumn{2}{|l|}{See section \ref{type:AbsoluteSubClassType}} \\
HasSubtype & ObjectType & \multicolumn{2}{l}{ActionSubClassType} & \multicolumn{2}{|l|}{See section \ref{type:ActionSubClassType}} \\
HasSubtype & ObjectType & \multicolumn{2}{l}{ActualSubClassType} & \multicolumn{2}{|l|}{See section \ref{type:ActualSubClassType}} \\
HasSubtype & ObjectType & \multicolumn{2}{l}{AllSubClassType} & \multicolumn{2}{|l|}{See section \ref{type:AllSubClassType}} \\
\end{tabu}
\end{table} 


\FloatBarrier
\subsubsection{Defintion of \texttt{ AbsoluteSubClassType}}
  \label{type:AbsoluteSubClassType}

\FloatBarrier

The magnitude or measurement of a type irrespective of its relation to other values.

The position of a block of program code relative to the beginning of the control program.

\begin{table}[ht]
\centering 
  \caption{\texttt{AbsoluteSubClassType} Definition}
  \label{table:AbsoluteSubClassType}
\fontsize{9pt}{11pt}\selectfont
\tabulinesep=3pt
\begin{tabu} to 6in {|X[-1.35]|X[-0.7]|X[-1.75]|X[-1.5]|X[-1]|X[-0.7]|} \everyrow{\hline}
\hline
\rowfont\bfseries {Attribute} & \multicolumn{5}{|l|}{Value} \\
\tabucline[1.5pt]{}
BrowseName & \multicolumn{5}{|l|}{AbsoluteSubClassType} \\
IsAbstract & \multicolumn{5}{|l|}{False} \\
\tabucline[1.5pt]{}
\rowfont \bfseries References & NodeClass & BrowseName & DataType & Type\-Definition & {Modeling\-Rule} \\
\multicolumn{6}{|l|}{Subtype of MTDataItemSubClassType (See section \ref{type:MTDataItemSubClassType})} \\
\end{tabu}
\end{table} 


\FloatBarrier
\subsubsection{Defintion of \texttt{ ActionSubClassType}}
  \label{type:ActionSubClassType}

\FloatBarrier

An indication of the operating state or value of a type.

An indication of the operating state of a mechanism represented by a composition type component.
 The operating state indicates whether the composition element is activated or disabled. 
 The valid data value must be active value or inactive value.

\begin{table}[ht]
\centering 
  \caption{\texttt{ActionSubClassType} Definition}
  \label{table:ActionSubClassType}
\fontsize{9pt}{11pt}\selectfont
\tabulinesep=3pt
\begin{tabu} to 6in {|X[-1.35]|X[-0.7]|X[-1.75]|X[-1.5]|X[-1]|X[-0.7]|} \everyrow{\hline}
\hline
\rowfont\bfseries {Attribute} & \multicolumn{5}{|l|}{Value} \\
\tabucline[1.5pt]{}
BrowseName & \multicolumn{5}{|l|}{ActionSubClassType} \\
IsAbstract & \multicolumn{5}{|l|}{False} \\
\tabucline[1.5pt]{}
\rowfont \bfseries References & NodeClass & BrowseName & DataType & Type\-Definition & {Modeling\-Rule} \\
\multicolumn{6}{|l|}{Subtype of MTDataItemSubClassType (See section \ref{type:MTDataItemSubClassType})} \\
\end{tabu}
\end{table} 


\FloatBarrier
\subsubsection{Defintion of \texttt{ ActualSubClassType}}
  \label{type:ActualSubClassType}

\FloatBarrier

The measured value of the a type.

The measured value of the data item type given by a sensor or encoder.

\begin{table}[ht]
\centering 
  \caption{\texttt{ActualSubClassType} Definition}
  \label{table:ActualSubClassType}
\fontsize{9pt}{11pt}\selectfont
\tabulinesep=3pt
\begin{tabu} to 6in {|X[-1.35]|X[-0.7]|X[-1.75]|X[-1.5]|X[-1]|X[-0.7]|} \everyrow{\hline}
\hline
\rowfont\bfseries {Attribute} & \multicolumn{5}{|l|}{Value} \\
\tabucline[1.5pt]{}
BrowseName & \multicolumn{5}{|l|}{ActualSubClassType} \\
IsAbstract & \multicolumn{5}{|l|}{False} \\
\tabucline[1.5pt]{}
\rowfont \bfseries References & NodeClass & BrowseName & DataType & Type\-Definition & {Modeling\-Rule} \\
\multicolumn{6}{|l|}{Subtype of MTDataItemSubClassType (See section \ref{type:MTDataItemSubClassType})} \\
\end{tabu}
\end{table} 


\FloatBarrier
\subsubsection{Defintion of \texttt{ AllSubClassType}}
  \label{type:AllSubClassType}

\FloatBarrier



The count of all the parts produced.  If the subtype is not given, this is the default.

\begin{table}[ht]
\centering 
  \caption{\texttt{AllSubClassType} Definition}
  \label{table:AllSubClassType}
\fontsize{9pt}{11pt}\selectfont
\tabulinesep=3pt
\begin{tabu} to 6in {|X[-1.35]|X[-0.7]|X[-1.75]|X[-1.5]|X[-1]|X[-0.7]|} \everyrow{\hline}
\hline
\rowfont\bfseries {Attribute} & \multicolumn{5}{|l|}{Value} \\
\tabucline[1.5pt]{}
BrowseName & \multicolumn{5}{|l|}{AllSubClassType} \\
IsAbstract & \multicolumn{5}{|l|}{False} \\
\tabucline[1.5pt]{}
\rowfont \bfseries References & NodeClass & BrowseName & DataType & Type\-Definition & {Modeling\-Rule} \\
\multicolumn{6}{|l|}{Subtype of MTDataItemSubClassType (See section \ref{type:MTDataItemSubClassType})} \\
\end{tabu}
\end{table} 


\FloatBarrier
\subsubsection{Defintion of \texttt{ AlternatingSubClassType}}
  \label{type:AlternatingSubClassType}

\FloatBarrier

The measurement of a type occurring in turn repeatedly.

The measurement of alternating voltage or current.   If not specified further in statistic, defaults to RMS voltage. 

\begin{table}[ht]
\centering 
  \caption{\texttt{AlternatingSubClassType} Definition}
  \label{table:AlternatingSubClassType}
\fontsize{9pt}{11pt}\selectfont
\tabulinesep=3pt
\begin{tabu} to 6in {|X[-1.35]|X[-0.7]|X[-1.75]|X[-1.5]|X[-1]|X[-0.7]|} \everyrow{\hline}
\hline
\rowfont\bfseries {Attribute} & \multicolumn{5}{|l|}{Value} \\
\tabucline[1.5pt]{}
BrowseName & \multicolumn{5}{|l|}{AlternatingSubClassType} \\
IsAbstract & \multicolumn{5}{|l|}{False} \\
\tabucline[1.5pt]{}
\rowfont \bfseries References & NodeClass & BrowseName & DataType & Type\-Definition & {Modeling\-Rule} \\
\multicolumn{6}{|l|}{Subtype of MTDataItemSubClassType (See section \ref{type:MTDataItemSubClassType})} \\
\end{tabu}
\end{table} 


\FloatBarrier
\subsubsection{Defintion of \texttt{ AScaleSubClassType}}
  \label{type:AScaleSubClassType}

\FloatBarrier

A Scale weighting factor for the measurement of sound level.

\begin{table}[ht]
\centering 
  \caption{\texttt{AScaleSubClassType} Definition}
  \label{table:AScaleSubClassType}
\fontsize{9pt}{11pt}\selectfont
\tabulinesep=3pt
\begin{tabu} to 6in {|X[-1.35]|X[-0.7]|X[-1.75]|X[-1.5]|X[-1]|X[-0.7]|} \everyrow{\hline}
\hline
\rowfont\bfseries {Attribute} & \multicolumn{5}{|l|}{Value} \\
\tabucline[1.5pt]{}
BrowseName & \multicolumn{5}{|l|}{AScaleSubClassType} \\
IsAbstract & \multicolumn{5}{|l|}{False} \\
\tabucline[1.5pt]{}
\rowfont \bfseries References & NodeClass & BrowseName & DataType & Type\-Definition & {Modeling\-Rule} \\
\multicolumn{6}{|l|}{Subtype of MTDataItemSubClassType (See section \ref{type:MTDataItemSubClassType})} \\
\end{tabu}
\end{table} 


\FloatBarrier
\subsubsection{Defintion of \texttt{ AuxiliarySubClassType}}
  \label{type:AuxiliarySubClassType}

\FloatBarrier

Example: When multiple locations on a piece of bar stock are referenced as the indication for the \mtmodel{END_OF_BAR}, 
the additional location(s) MUST be designated as \mtmodel{AUXILIARY} indication(s) for the \mtmodel{END_OF_BAR}.

When multiple locations on a piece of bar stock are referenced as the indication for the endofbar event, the additional location(s) must be designated as auxiliary subtype indication(s) for the endofbar event.  

\begin{table}[ht]
\centering 
  \caption{\texttt{AuxiliarySubClassType} Definition}
  \label{table:AuxiliarySubClassType}
\fontsize{9pt}{11pt}\selectfont
\tabulinesep=3pt
\begin{tabu} to 6in {|X[-1.35]|X[-0.7]|X[-1.75]|X[-1.5]|X[-1]|X[-0.7]|} \everyrow{\hline}
\hline
\rowfont\bfseries {Attribute} & \multicolumn{5}{|l|}{Value} \\
\tabucline[1.5pt]{}
BrowseName & \multicolumn{5}{|l|}{AuxiliarySubClassType} \\
IsAbstract & \multicolumn{5}{|l|}{False} \\
\tabucline[1.5pt]{}
\rowfont \bfseries References & NodeClass & BrowseName & DataType & Type\-Definition & {Modeling\-Rule} \\
\multicolumn{6}{|l|}{Subtype of MTDataItemSubClassType (See section \ref{type:MTDataItemSubClassType})} \\
\end{tabu}
\end{table} 


\FloatBarrier
\subsubsection{Defintion of \texttt{ BadSubClassType}}
  \label{type:BadSubClassType}

\FloatBarrier

Indicates the count of incorrect parts produced.

Indicates the count of incorrect parts produced.

\begin{table}[ht]
\centering 
  \caption{\texttt{BadSubClassType} Definition}
  \label{table:BadSubClassType}
\fontsize{9pt}{11pt}\selectfont
\tabulinesep=3pt
\begin{tabu} to 6in {|X[-1.35]|X[-0.7]|X[-1.75]|X[-1.5]|X[-1]|X[-0.7]|} \everyrow{\hline}
\hline
\rowfont\bfseries {Attribute} & \multicolumn{5}{|l|}{Value} \\
\tabucline[1.5pt]{}
BrowseName & \multicolumn{5}{|l|}{BadSubClassType} \\
IsAbstract & \multicolumn{5}{|l|}{False} \\
\tabucline[1.5pt]{}
\rowfont \bfseries References & NodeClass & BrowseName & DataType & Type\-Definition & {Modeling\-Rule} \\
\multicolumn{6}{|l|}{Subtype of MTDataItemSubClassType (See section \ref{type:MTDataItemSubClassType})} \\
\end{tabu}
\end{table} 


\FloatBarrier
\subsubsection{Defintion of \texttt{ BrinellSubClassType}}
  \label{type:BrinellSubClassType}

\FloatBarrier

A scale to measure the resistance to deformation of a surface.

A scale to measure the resistance to deformation of a surface.

\begin{table}[ht]
\centering 
  \caption{\texttt{BrinellSubClassType} Definition}
  \label{table:BrinellSubClassType}
\fontsize{9pt}{11pt}\selectfont
\tabulinesep=3pt
\begin{tabu} to 6in {|X[-1.35]|X[-0.7]|X[-1.75]|X[-1.5]|X[-1]|X[-0.7]|} \everyrow{\hline}
\hline
\rowfont\bfseries {Attribute} & \multicolumn{5}{|l|}{Value} \\
\tabucline[1.5pt]{}
BrowseName & \multicolumn{5}{|l|}{BrinellSubClassType} \\
IsAbstract & \multicolumn{5}{|l|}{False} \\
\tabucline[1.5pt]{}
\rowfont \bfseries References & NodeClass & BrowseName & DataType & Type\-Definition & {Modeling\-Rule} \\
\multicolumn{6}{|l|}{Subtype of MTDataItemSubClassType (See section \ref{type:MTDataItemSubClassType})} \\
\end{tabu}
\end{table} 


\FloatBarrier
\subsubsection{Defintion of \texttt{ BScaleSubClassType}}
  \label{type:BScaleSubClassType}

\FloatBarrier

B Scale weighting factor for the measurement of sound level.

\begin{table}[ht]
\centering 
  \caption{\texttt{BScaleSubClassType} Definition}
  \label{table:BScaleSubClassType}
\fontsize{9pt}{11pt}\selectfont
\tabulinesep=3pt
\begin{tabu} to 6in {|X[-1.35]|X[-0.7]|X[-1.75]|X[-1.5]|X[-1]|X[-0.7]|} \everyrow{\hline}
\hline
\rowfont\bfseries {Attribute} & \multicolumn{5}{|l|}{Value} \\
\tabucline[1.5pt]{}
BrowseName & \multicolumn{5}{|l|}{BScaleSubClassType} \\
IsAbstract & \multicolumn{5}{|l|}{False} \\
\tabucline[1.5pt]{}
\rowfont \bfseries References & NodeClass & BrowseName & DataType & Type\-Definition & {Modeling\-Rule} \\
\multicolumn{6}{|l|}{Subtype of MTDataItemSubClassType (See section \ref{type:MTDataItemSubClassType})} \\
\end{tabu}
\end{table} 


\FloatBarrier
\subsubsection{Defintion of \texttt{ CommandedSubClassType}}
  \label{type:CommandedSubClassType}

\FloatBarrier

The value as specified by the Controller type component.

A value specified by the controller type component.

\begin{table}[ht]
\centering 
  \caption{\texttt{CommandedSubClassType} Definition}
  \label{table:CommandedSubClassType}
\fontsize{9pt}{11pt}\selectfont
\tabulinesep=3pt
\begin{tabu} to 6in {|X[-1.35]|X[-0.7]|X[-1.75]|X[-1.5]|X[-1]|X[-0.7]|} \everyrow{\hline}
\hline
\rowfont\bfseries {Attribute} & \multicolumn{5}{|l|}{Value} \\
\tabucline[1.5pt]{}
BrowseName & \multicolumn{5}{|l|}{CommandedSubClassType} \\
IsAbstract & \multicolumn{5}{|l|}{False} \\
\tabucline[1.5pt]{}
\rowfont \bfseries References & NodeClass & BrowseName & DataType & Type\-Definition & {Modeling\-Rule} \\
\multicolumn{6}{|l|}{Subtype of MTDataItemSubClassType (See section \ref{type:MTDataItemSubClassType})} \\
\end{tabu}
\end{table} 


\FloatBarrier
\subsubsection{Defintion of \texttt{ ControlSubClassType}}
  \label{type:ControlSubClassType}

\FloatBarrier

The state of the enabling signal or control logic that enables or disables the function or operation of the \textit{Structural Element}.

The state of the enabling signal or control logic that enables or disables the function or operation of the structural element.

\begin{table}[ht]
\centering 
  \caption{\texttt{ControlSubClassType} Definition}
  \label{table:ControlSubClassType}
\fontsize{9pt}{11pt}\selectfont
\tabulinesep=3pt
\begin{tabu} to 6in {|X[-1.35]|X[-0.7]|X[-1.75]|X[-1.5]|X[-1]|X[-0.7]|} \everyrow{\hline}
\hline
\rowfont\bfseries {Attribute} & \multicolumn{5}{|l|}{Value} \\
\tabucline[1.5pt]{}
BrowseName & \multicolumn{5}{|l|}{ControlSubClassType} \\
IsAbstract & \multicolumn{5}{|l|}{False} \\
\tabucline[1.5pt]{}
\rowfont \bfseries References & NodeClass & BrowseName & DataType & Type\-Definition & {Modeling\-Rule} \\
\multicolumn{6}{|l|}{Subtype of MTDataItemSubClassType (See section \ref{type:MTDataItemSubClassType})} \\
\end{tabu}
\end{table} 


\FloatBarrier
\subsubsection{Defintion of \texttt{ CScaleSubClassType}}
  \label{type:CScaleSubClassType}

\FloatBarrier

C Scale weighting factor for the measurement of sound level.

\begin{table}[ht]
\centering 
  \caption{\texttt{CScaleSubClassType} Definition}
  \label{table:CScaleSubClassType}
\fontsize{9pt}{11pt}\selectfont
\tabulinesep=3pt
\begin{tabu} to 6in {|X[-1.35]|X[-0.7]|X[-1.75]|X[-1.5]|X[-1]|X[-0.7]|} \everyrow{\hline}
\hline
\rowfont\bfseries {Attribute} & \multicolumn{5}{|l|}{Value} \\
\tabucline[1.5pt]{}
BrowseName & \multicolumn{5}{|l|}{CScaleSubClassType} \\
IsAbstract & \multicolumn{5}{|l|}{False} \\
\tabucline[1.5pt]{}
\rowfont \bfseries References & NodeClass & BrowseName & DataType & Type\-Definition & {Modeling\-Rule} \\
\multicolumn{6}{|l|}{Subtype of MTDataItemSubClassType (See section \ref{type:MTDataItemSubClassType})} \\
\end{tabu}
\end{table} 


\FloatBarrier
\subsubsection{Defintion of \texttt{ DelaySubClassType}}
  \label{type:DelaySubClassType}

\FloatBarrier

Measurement of the time that a piece of equipment is waiting for an event or an action to occur.

A piece of equipment waiting for an event or an action to occur.

\begin{table}[ht]
\centering 
  \caption{\texttt{DelaySubClassType} Definition}
  \label{table:DelaySubClassType}
\fontsize{9pt}{11pt}\selectfont
\tabulinesep=3pt
\begin{tabu} to 6in {|X[-1.35]|X[-0.7]|X[-1.75]|X[-1.5]|X[-1]|X[-0.7]|} \everyrow{\hline}
\hline
\rowfont\bfseries {Attribute} & \multicolumn{5}{|l|}{Value} \\
\tabucline[1.5pt]{}
BrowseName & \multicolumn{5}{|l|}{DelaySubClassType} \\
IsAbstract & \multicolumn{5}{|l|}{False} \\
\tabucline[1.5pt]{}
\rowfont \bfseries References & NodeClass & BrowseName & DataType & Type\-Definition & {Modeling\-Rule} \\
\multicolumn{6}{|l|}{Subtype of MTDataItemSubClassType (See section \ref{type:MTDataItemSubClassType})} \\
\end{tabu}
\end{table} 


\FloatBarrier
\subsubsection{Defintion of \texttt{ DirectSubClassType}}
  \label{type:DirectSubClassType}

\FloatBarrier

Measurement of DC current or voltage.

The measurement of DC current or voltage.

\begin{table}[ht]
\centering 
  \caption{\texttt{DirectSubClassType} Definition}
  \label{table:DirectSubClassType}
\fontsize{9pt}{11pt}\selectfont
\tabulinesep=3pt
\begin{tabu} to 6in {|X[-1.35]|X[-0.7]|X[-1.75]|X[-1.5]|X[-1]|X[-0.7]|} \everyrow{\hline}
\hline
\rowfont\bfseries {Attribute} & \multicolumn{5}{|l|}{Value} \\
\tabucline[1.5pt]{}
BrowseName & \multicolumn{5}{|l|}{DirectSubClassType} \\
IsAbstract & \multicolumn{5}{|l|}{False} \\
\tabucline[1.5pt]{}
\rowfont \bfseries References & NodeClass & BrowseName & DataType & Type\-Definition & {Modeling\-Rule} \\
\multicolumn{6}{|l|}{Subtype of MTDataItemSubClassType (See section \ref{type:MTDataItemSubClassType})} \\
\end{tabu}
\end{table} 


\FloatBarrier
\subsubsection{Defintion of \texttt{ DryRunSubClassType}}
  \label{type:DryRunSubClassType}

\FloatBarrier

A setting or operator selection used to execute a test mode to confirm the execution of machine functions.

\begin{table}[ht]
\centering 
  \caption{\texttt{DryRunSubClassType} Definition}
  \label{table:DryRunSubClassType}
\fontsize{9pt}{11pt}\selectfont
\tabulinesep=3pt
\begin{tabu} to 6in {|X[-1.35]|X[-0.7]|X[-1.75]|X[-1.5]|X[-1]|X[-0.7]|} \everyrow{\hline}
\hline
\rowfont\bfseries {Attribute} & \multicolumn{5}{|l|}{Value} \\
\tabucline[1.5pt]{}
BrowseName & \multicolumn{5}{|l|}{DryRunSubClassType} \\
IsAbstract & \multicolumn{5}{|l|}{False} \\
\tabucline[1.5pt]{}
\rowfont \bfseries References & NodeClass & BrowseName & DataType & Type\-Definition & {Modeling\-Rule} \\
\multicolumn{6}{|l|}{Subtype of MTDataItemSubClassType (See section \ref{type:MTDataItemSubClassType})} \\
\end{tabu}
\end{table} 


\FloatBarrier
\subsubsection{Defintion of \texttt{ DScaleSubClassType}}
  \label{type:DScaleSubClassType}

\FloatBarrier

D Scale weighting factor for the measurement of sound level.

\begin{table}[ht]
\centering 
  \caption{\texttt{DScaleSubClassType} Definition}
  \label{table:DScaleSubClassType}
\fontsize{9pt}{11pt}\selectfont
\tabulinesep=3pt
\begin{tabu} to 6in {|X[-1.35]|X[-0.7]|X[-1.75]|X[-1.5]|X[-1]|X[-0.7]|} \everyrow{\hline}
\hline
\rowfont\bfseries {Attribute} & \multicolumn{5}{|l|}{Value} \\
\tabucline[1.5pt]{}
BrowseName & \multicolumn{5}{|l|}{DScaleSubClassType} \\
IsAbstract & \multicolumn{5}{|l|}{False} \\
\tabucline[1.5pt]{}
\rowfont \bfseries References & NodeClass & BrowseName & DataType & Type\-Definition & {Modeling\-Rule} \\
\multicolumn{6}{|l|}{Subtype of MTDataItemSubClassType (See section \ref{type:MTDataItemSubClassType})} \\
\end{tabu}
\end{table} 


\FloatBarrier
\subsubsection{Defintion of \texttt{ FixtureSubClassType}}
  \label{type:FixtureSubClassType}

\FloatBarrier

Fixture denotes a specifc type of a piece of equipment.

\begin{table}[ht]
\centering 
  \caption{\texttt{FixtureSubClassType} Definition}
  \label{table:FixtureSubClassType}
\fontsize{9pt}{11pt}\selectfont
\tabulinesep=3pt
\begin{tabu} to 6in {|X[-1.35]|X[-0.7]|X[-1.75]|X[-1.5]|X[-1]|X[-0.7]|} \everyrow{\hline}
\hline
\rowfont\bfseries {Attribute} & \multicolumn{5}{|l|}{Value} \\
\tabucline[1.5pt]{}
BrowseName & \multicolumn{5}{|l|}{FixtureSubClassType} \\
IsAbstract & \multicolumn{5}{|l|}{False} \\
\tabucline[1.5pt]{}
\rowfont \bfseries References & NodeClass & BrowseName & DataType & Type\-Definition & {Modeling\-Rule} \\
\multicolumn{6}{|l|}{Subtype of MTDataItemSubClassType (See section \ref{type:MTDataItemSubClassType})} \\
\end{tabu}
\end{table} 


\FloatBarrier
\subsubsection{Defintion of \texttt{ GoodSubClassType}}
  \label{type:GoodSubClassType}

\FloatBarrier

Indicates the count of correct parts made.

Indicates the count of correct parts made.

\begin{table}[ht]
\centering 
  \caption{\texttt{GoodSubClassType} Definition}
  \label{table:GoodSubClassType}
\fontsize{9pt}{11pt}\selectfont
\tabulinesep=3pt
\begin{tabu} to 6in {|X[-1.35]|X[-0.7]|X[-1.75]|X[-1.5]|X[-1]|X[-0.7]|} \everyrow{\hline}
\hline
\rowfont\bfseries {Attribute} & \multicolumn{5}{|l|}{Value} \\
\tabucline[1.5pt]{}
BrowseName & \multicolumn{5}{|l|}{GoodSubClassType} \\
IsAbstract & \multicolumn{5}{|l|}{False} \\
\tabucline[1.5pt]{}
\rowfont \bfseries References & NodeClass & BrowseName & DataType & Type\-Definition & {Modeling\-Rule} \\
\multicolumn{6}{|l|}{Subtype of MTDataItemSubClassType (See section \ref{type:MTDataItemSubClassType})} \\
\end{tabu}
\end{table} 


\FloatBarrier
\subsubsection{Defintion of \texttt{ IncrementalSubClassType}}
  \label{type:IncrementalSubClassType}

\FloatBarrier

A small change which could be either positive or negative in a Type's value or function.

Example: The position of a block of program code relative to the occurrence of the last \mtmodel{LINE_LABEL} encountered in the control program.

The position of a block of program code relative to the occurrence of the last linelabel event encountered in the control program.

\begin{table}[ht]
\centering 
  \caption{\texttt{IncrementalSubClassType} Definition}
  \label{table:IncrementalSubClassType}
\fontsize{9pt}{11pt}\selectfont
\tabulinesep=3pt
\begin{tabu} to 6in {|X[-1.35]|X[-0.7]|X[-1.75]|X[-1.5]|X[-1]|X[-0.7]|} \everyrow{\hline}
\hline
\rowfont\bfseries {Attribute} & \multicolumn{5}{|l|}{Value} \\
\tabucline[1.5pt]{}
BrowseName & \multicolumn{5}{|l|}{IncrementalSubClassType} \\
IsAbstract & \multicolumn{5}{|l|}{False} \\
\tabucline[1.5pt]{}
\rowfont \bfseries References & NodeClass & BrowseName & DataType & Type\-Definition & {Modeling\-Rule} \\
\multicolumn{6}{|l|}{Subtype of MTDataItemSubClassType (See section \ref{type:MTDataItemSubClassType})} \\
\end{tabu}
\end{table} 


\FloatBarrier
\subsubsection{Defintion of \texttt{ JobSubClassType}}
  \label{type:JobSubClassType}

\FloatBarrier

The value of a signal or calculation issued to adjust the feedrate of the axes associated with a Path component when the axes, 
or a single axis, are being operated in a manual mode or method (jogging).

\begin{table}[ht]
\centering 
  \caption{\texttt{JobSubClassType} Definition}
  \label{table:JobSubClassType}
\fontsize{9pt}{11pt}\selectfont
\tabulinesep=3pt
\begin{tabu} to 6in {|X[-1.35]|X[-0.7]|X[-1.75]|X[-1.5]|X[-1]|X[-0.7]|} \everyrow{\hline}
\hline
\rowfont\bfseries {Attribute} & \multicolumn{5}{|l|}{Value} \\
\tabucline[1.5pt]{}
BrowseName & \multicolumn{5}{|l|}{JobSubClassType} \\
IsAbstract & \multicolumn{5}{|l|}{False} \\
\tabucline[1.5pt]{}
\rowfont \bfseries References & NodeClass & BrowseName & DataType & Type\-Definition & {Modeling\-Rule} \\
\multicolumn{6}{|l|}{Subtype of MTDataItemSubClassType (See section \ref{type:MTDataItemSubClassType})} \\
\end{tabu}
\end{table} 


\FloatBarrier
\subsubsection{Defintion of \texttt{ KineticSubClassType}}
  \label{type:KineticSubClassType}

\FloatBarrier



\begin{table}[ht]
\centering 
  \caption{\texttt{KineticSubClassType} Definition}
  \label{table:KineticSubClassType}
\fontsize{9pt}{11pt}\selectfont
\tabulinesep=3pt
\begin{tabu} to 6in {|X[-1.35]|X[-0.7]|X[-1.75]|X[-1.5]|X[-1]|X[-0.7]|} \everyrow{\hline}
\hline
\rowfont\bfseries {Attribute} & \multicolumn{5}{|l|}{Value} \\
\tabucline[1.5pt]{}
BrowseName & \multicolumn{5}{|l|}{KineticSubClassType} \\
IsAbstract & \multicolumn{5}{|l|}{False} \\
\tabucline[1.5pt]{}
\rowfont \bfseries References & NodeClass & BrowseName & DataType & Type\-Definition & {Modeling\-Rule} \\
\multicolumn{6}{|l|}{Subtype of MTDataItemSubClassType (See section \ref{type:MTDataItemSubClassType})} \\
\end{tabu}
\end{table} 


\FloatBarrier
\subsubsection{Defintion of \texttt{ LateralSubClassType}}
  \label{type:LateralSubClassType}

\FloatBarrier

An indication of the position of a mechanism that may move in a lateral direction. The mechanism is represented by 
a \mtmodel{Composition} type component.

An indication of the position of a mechanism that may move in a lateral direction.   The mechanism is represented by a composition type component. 
 The position information indicates whether the composition element is positioned to the right, to the left, or is in transition.  
 The valid data value must be right value, left value, or transitioning value.

\begin{table}[ht]
\centering 
  \caption{\texttt{LateralSubClassType} Definition}
  \label{table:LateralSubClassType}
\fontsize{9pt}{11pt}\selectfont
\tabulinesep=3pt
\begin{tabu} to 6in {|X[-1.35]|X[-0.7]|X[-1.75]|X[-1.5]|X[-1]|X[-0.7]|} \everyrow{\hline}
\hline
\rowfont\bfseries {Attribute} & \multicolumn{5}{|l|}{Value} \\
\tabucline[1.5pt]{}
BrowseName & \multicolumn{5}{|l|}{LateralSubClassType} \\
IsAbstract & \multicolumn{5}{|l|}{False} \\
\tabucline[1.5pt]{}
\rowfont \bfseries References & NodeClass & BrowseName & DataType & Type\-Definition & {Modeling\-Rule} \\
\multicolumn{6}{|l|}{Subtype of MTDataItemSubClassType (See section \ref{type:MTDataItemSubClassType})} \\
\end{tabu}
\end{table} 


\FloatBarrier
\subsubsection{Defintion of \texttt{ LeebSubClassType}}
  \label{type:LeebSubClassType}

\FloatBarrier

A scale to measure the elasticity of a surface.

A scale to measure the elasticity of a surface.

\begin{table}[ht]
\centering 
  \caption{\texttt{LeebSubClassType} Definition}
  \label{table:LeebSubClassType}
\fontsize{9pt}{11pt}\selectfont
\tabulinesep=3pt
\begin{tabu} to 6in {|X[-1.35]|X[-0.7]|X[-1.75]|X[-1.5]|X[-1]|X[-0.7]|} \everyrow{\hline}
\hline
\rowfont\bfseries {Attribute} & \multicolumn{5}{|l|}{Value} \\
\tabucline[1.5pt]{}
BrowseName & \multicolumn{5}{|l|}{LeebSubClassType} \\
IsAbstract & \multicolumn{5}{|l|}{False} \\
\tabucline[1.5pt]{}
\rowfont \bfseries References & NodeClass & BrowseName & DataType & Type\-Definition & {Modeling\-Rule} \\
\multicolumn{6}{|l|}{Subtype of MTDataItemSubClassType (See section \ref{type:MTDataItemSubClassType})} \\
\end{tabu}
\end{table} 


\FloatBarrier
\subsubsection{Defintion of \texttt{ LengthSubClassType}}
  \label{type:LengthSubClassType}

\FloatBarrier

The measurement or extent of something from end to end.

The measurement of the length of an object.

\begin{table}[ht]
\centering 
  \caption{\texttt{LengthSubClassType} Definition}
  \label{table:LengthSubClassType}
\fontsize{9pt}{11pt}\selectfont
\tabulinesep=3pt
\begin{tabu} to 6in {|X[-1.35]|X[-0.7]|X[-1.75]|X[-1.5]|X[-1]|X[-0.7]|} \everyrow{\hline}
\hline
\rowfont\bfseries {Attribute} & \multicolumn{5}{|l|}{Value} \\
\tabucline[1.5pt]{}
BrowseName & \multicolumn{5}{|l|}{LengthSubClassType} \\
IsAbstract & \multicolumn{5}{|l|}{False} \\
\tabucline[1.5pt]{}
\rowfont \bfseries References & NodeClass & BrowseName & DataType & Type\-Definition & {Modeling\-Rule} \\
\multicolumn{6}{|l|}{Subtype of MTDataItemSubClassType (See section \ref{type:MTDataItemSubClassType})} \\
\end{tabu}
\end{table} 


\FloatBarrier
\subsubsection{Defintion of \texttt{ LinearSubClassType}}
  \label{type:LinearSubClassType}

\FloatBarrier

The direction of motion.

A linear axis represents the movement of a physical piece of equipment, or a portion of the equipment, in a straight line. 

\begin{table}[ht]
\centering 
  \caption{\texttt{LinearSubClassType} Definition}
  \label{table:LinearSubClassType}
\fontsize{9pt}{11pt}\selectfont
\tabulinesep=3pt
\begin{tabu} to 6in {|X[-1.35]|X[-0.7]|X[-1.75]|X[-1.5]|X[-1]|X[-0.7]|} \everyrow{\hline}
\hline
\rowfont\bfseries {Attribute} & \multicolumn{5}{|l|}{Value} \\
\tabucline[1.5pt]{}
BrowseName & \multicolumn{5}{|l|}{LinearSubClassType} \\
IsAbstract & \multicolumn{5}{|l|}{False} \\
\tabucline[1.5pt]{}
\rowfont \bfseries References & NodeClass & BrowseName & DataType & Type\-Definition & {Modeling\-Rule} \\
\multicolumn{6}{|l|}{Subtype of MTDataItemSubClassType (See section \ref{type:MTDataItemSubClassType})} \\
\end{tabu}
\end{table} 


\FloatBarrier
\subsubsection{Defintion of \texttt{ LineSubClassType}}
  \label{type:LineSubClassType}

\FloatBarrier

The state of the power source for the \textit{Structural Element}.

DEPRECATED in Version 1.4.0.

\begin{table}[ht]
\centering 
  \caption{\texttt{LineSubClassType} Definition}
  \label{table:LineSubClassType}
\fontsize{9pt}{11pt}\selectfont
\tabulinesep=3pt
\begin{tabu} to 6in {|X[-1.35]|X[-0.7]|X[-1.75]|X[-1.5]|X[-1]|X[-0.7]|} \everyrow{\hline}
\hline
\rowfont\bfseries {Attribute} & \multicolumn{5}{|l|}{Value} \\
\tabucline[1.5pt]{}
BrowseName & \multicolumn{5}{|l|}{LineSubClassType} \\
IsAbstract & \multicolumn{5}{|l|}{False} \\
\tabucline[1.5pt]{}
\rowfont \bfseries References & NodeClass & BrowseName & DataType & Type\-Definition & {Modeling\-Rule} \\
\multicolumn{6}{|l|}{Subtype of MTDataItemSubClassType (See section \ref{type:MTDataItemSubClassType})} \\
\end{tabu}
\end{table} 


\FloatBarrier
\subsubsection{Defintion of \texttt{ LoadedSubClassType}}
  \label{type:LoadedSubClassType}

\FloatBarrier

An indication that the sub-parts of a piece of equipment are under load.

Subparts of a piece of equipment are under load.

\begin{table}[ht]
\centering 
  \caption{\texttt{LoadedSubClassType} Definition}
  \label{table:LoadedSubClassType}
\fontsize{9pt}{11pt}\selectfont
\tabulinesep=3pt
\begin{tabu} to 6in {|X[-1.35]|X[-0.7]|X[-1.75]|X[-1.5]|X[-1]|X[-0.7]|} \everyrow{\hline}
\hline
\rowfont\bfseries {Attribute} & \multicolumn{5}{|l|}{Value} \\
\tabucline[1.5pt]{}
BrowseName & \multicolumn{5}{|l|}{LoadedSubClassType} \\
IsAbstract & \multicolumn{5}{|l|}{False} \\
\tabucline[1.5pt]{}
\rowfont \bfseries References & NodeClass & BrowseName & DataType & Type\-Definition & {Modeling\-Rule} \\
\multicolumn{6}{|l|}{Subtype of MTDataItemSubClassType (See section \ref{type:MTDataItemSubClassType})} \\
\end{tabu}
\end{table} 


\FloatBarrier
\subsubsection{Defintion of \texttt{ MachineAxisLockSubClassType}}
  \label{type:MachineAxisLockSubClassType}

\FloatBarrier

A setting or operator selection that changes the behavior of the controller on a piece of equipment.

\begin{table}[ht]
\centering 
  \caption{\texttt{MachineAxisLockSubClassType} Definition}
  \label{table:MachineAxisLockSubClassType}
\fontsize{9pt}{11pt}\selectfont
\tabulinesep=3pt
\begin{tabu} to 6in {|X[-1.35]|X[-0.7]|X[-1.75]|X[-1.5]|X[-1]|X[-0.7]|} \everyrow{\hline}
\hline
\rowfont\bfseries {Attribute} & \multicolumn{5}{|l|}{Value} \\
\tabucline[1.5pt]{}
BrowseName & \multicolumn{5}{|l|}{MachineAxisLockSubClassType} \\
IsAbstract & \multicolumn{5}{|l|}{False} \\
\tabucline[1.5pt]{}
\rowfont \bfseries References & NodeClass & BrowseName & DataType & Type\-Definition & {Modeling\-Rule} \\
\multicolumn{6}{|l|}{Subtype of MTDataItemSubClassType (See section \ref{type:MTDataItemSubClassType})} \\
\end{tabu}
\end{table} 


\FloatBarrier
\subsubsection{Defintion of \texttt{ MaintenanceSubClassType}}
  \label{type:MaintenanceSubClassType}

\FloatBarrier

The identifier of the person currently responsible for performing maintenance on the piece of equipment.

Action related to maintenance on the piece of equipment.

\begin{table}[ht]
\centering 
  \caption{\texttt{MaintenanceSubClassType} Definition}
  \label{table:MaintenanceSubClassType}
\fontsize{9pt}{11pt}\selectfont
\tabulinesep=3pt
\begin{tabu} to 6in {|X[-1.35]|X[-0.7]|X[-1.75]|X[-1.5]|X[-1]|X[-0.7]|} \everyrow{\hline}
\hline
\rowfont\bfseries {Attribute} & \multicolumn{5}{|l|}{Value} \\
\tabucline[1.5pt]{}
BrowseName & \multicolumn{5}{|l|}{MaintenanceSubClassType} \\
IsAbstract & \multicolumn{5}{|l|}{False} \\
\tabucline[1.5pt]{}
\rowfont \bfseries References & NodeClass & BrowseName & DataType & Type\-Definition & {Modeling\-Rule} \\
\multicolumn{6}{|l|}{Subtype of MTDataItemSubClassType (See section \ref{type:MTDataItemSubClassType})} \\
\end{tabu}
\end{table} 


\FloatBarrier
\subsubsection{Defintion of \texttt{ ManualUnclampSubClassType}}
  \label{type:ManualUnclampSubClassType}

\FloatBarrier

An indication of the state of an operator controlled interlock that can inhibit the ability to 
initiate an unclamp action of an electronically controlled chuck.

\begin{table}[ht]
\centering 
  \caption{\texttt{ManualUnclampSubClassType} Definition}
  \label{table:ManualUnclampSubClassType}
\fontsize{9pt}{11pt}\selectfont
\tabulinesep=3pt
\begin{tabu} to 6in {|X[-1.35]|X[-0.7]|X[-1.75]|X[-1.5]|X[-1]|X[-0.7]|} \everyrow{\hline}
\hline
\rowfont\bfseries {Attribute} & \multicolumn{5}{|l|}{Value} \\
\tabucline[1.5pt]{}
BrowseName & \multicolumn{5}{|l|}{ManualUnclampSubClassType} \\
IsAbstract & \multicolumn{5}{|l|}{False} \\
\tabucline[1.5pt]{}
\rowfont \bfseries References & NodeClass & BrowseName & DataType & Type\-Definition & {Modeling\-Rule} \\
\multicolumn{6}{|l|}{Subtype of MTDataItemSubClassType (See section \ref{type:MTDataItemSubClassType})} \\
\end{tabu}
\end{table} 


\FloatBarrier
\subsubsection{Defintion of \texttt{ MaximumSubClassType}}
  \label{type:MaximumSubClassType}

\FloatBarrier

Maximum or peak value recorded for the data item during the calculation period.

The upper limit of data reported for a data item.

\begin{table}[ht]
\centering 
  \caption{\texttt{MaximumSubClassType} Definition}
  \label{table:MaximumSubClassType}
\fontsize{9pt}{11pt}\selectfont
\tabulinesep=3pt
\begin{tabu} to 6in {|X[-1.35]|X[-0.7]|X[-1.75]|X[-1.5]|X[-1]|X[-0.7]|} \everyrow{\hline}
\hline
\rowfont\bfseries {Attribute} & \multicolumn{5}{|l|}{Value} \\
\tabucline[1.5pt]{}
BrowseName & \multicolumn{5}{|l|}{MaximumSubClassType} \\
IsAbstract & \multicolumn{5}{|l|}{False} \\
\tabucline[1.5pt]{}
\rowfont \bfseries References & NodeClass & BrowseName & DataType & Type\-Definition & {Modeling\-Rule} \\
\multicolumn{6}{|l|}{Subtype of MTDataItemSubClassType (See section \ref{type:MTDataItemSubClassType})} \\
\end{tabu}
\end{table} 


\FloatBarrier
\subsubsection{Defintion of \texttt{ MinimumSubClassType}}
  \label{type:MinimumSubClassType}

\FloatBarrier

Minimum value recorded for the data item during the calculation period.

The lower limit of data reported for a data item.

\begin{table}[ht]
\centering 
  \caption{\texttt{MinimumSubClassType} Definition}
  \label{table:MinimumSubClassType}
\fontsize{9pt}{11pt}\selectfont
\tabulinesep=3pt
\begin{tabu} to 6in {|X[-1.35]|X[-0.7]|X[-1.75]|X[-1.5]|X[-1]|X[-0.7]|} \everyrow{\hline}
\hline
\rowfont\bfseries {Attribute} & \multicolumn{5}{|l|}{Value} \\
\tabucline[1.5pt]{}
BrowseName & \multicolumn{5}{|l|}{MinimumSubClassType} \\
IsAbstract & \multicolumn{5}{|l|}{False} \\
\tabucline[1.5pt]{}
\rowfont \bfseries References & NodeClass & BrowseName & DataType & Type\-Definition & {Modeling\-Rule} \\
\multicolumn{6}{|l|}{Subtype of MTDataItemSubClassType (See section \ref{type:MTDataItemSubClassType})} \\
\end{tabu}
\end{table} 


\FloatBarrier
\subsubsection{Defintion of \texttt{ MohsSubClassType}}
  \label{type:MohsSubClassType}

\FloatBarrier

A scale to measure the resistance to scratching of a surface.

A scale to measure the resistance to scratching of a surface.

\begin{table}[ht]
\centering 
  \caption{\texttt{MohsSubClassType} Definition}
  \label{table:MohsSubClassType}
\fontsize{9pt}{11pt}\selectfont
\tabulinesep=3pt
\begin{tabu} to 6in {|X[-1.35]|X[-0.7]|X[-1.75]|X[-1.5]|X[-1]|X[-0.7]|} \everyrow{\hline}
\hline
\rowfont\bfseries {Attribute} & \multicolumn{5}{|l|}{Value} \\
\tabucline[1.5pt]{}
BrowseName & \multicolumn{5}{|l|}{MohsSubClassType} \\
IsAbstract & \multicolumn{5}{|l|}{False} \\
\tabucline[1.5pt]{}
\rowfont \bfseries References & NodeClass & BrowseName & DataType & Type\-Definition & {Modeling\-Rule} \\
\multicolumn{6}{|l|}{Subtype of MTDataItemSubClassType (See section \ref{type:MTDataItemSubClassType})} \\
\end{tabu}
\end{table} 


\FloatBarrier
\subsubsection{Defintion of \texttt{ MoleSubClassType}}
  \label{type:MoleSubClassType}

\FloatBarrier



\begin{table}[ht]
\centering 
  \caption{\texttt{MoleSubClassType} Definition}
  \label{table:MoleSubClassType}
\fontsize{9pt}{11pt}\selectfont
\tabulinesep=3pt
\begin{tabu} to 6in {|X[-1.35]|X[-0.7]|X[-1.75]|X[-1.5]|X[-1]|X[-0.7]|} \everyrow{\hline}
\hline
\rowfont\bfseries {Attribute} & \multicolumn{5}{|l|}{Value} \\
\tabucline[1.5pt]{}
BrowseName & \multicolumn{5}{|l|}{MoleSubClassType} \\
IsAbstract & \multicolumn{5}{|l|}{False} \\
\tabucline[1.5pt]{}
\rowfont \bfseries References & NodeClass & BrowseName & DataType & Type\-Definition & {Modeling\-Rule} \\
\multicolumn{6}{|l|}{Subtype of MTDataItemSubClassType (See section \ref{type:MTDataItemSubClassType})} \\
\end{tabu}
\end{table} 


\FloatBarrier
\subsubsection{Defintion of \texttt{ MotionSubClassType}}
  \label{type:MotionSubClassType}

\FloatBarrier

An indication of the open or closed state of a mechanism. The mechanism is represented by a \mtmodel{Composition} type component.

An indication of the open or closed state of a mechanism.   The mechanism is represented by a composition type component. 
 The operating state indicates whether the state of the composition element is open, closed, or unlatched.   
 The valid data value must be open value, unlatched value, or closed value.

\begin{table}[ht]
\centering 
  \caption{\texttt{MotionSubClassType} Definition}
  \label{table:MotionSubClassType}
\fontsize{9pt}{11pt}\selectfont
\tabulinesep=3pt
\begin{tabu} to 6in {|X[-1.35]|X[-0.7]|X[-1.75]|X[-1.5]|X[-1]|X[-0.7]|} \everyrow{\hline}
\hline
\rowfont\bfseries {Attribute} & \multicolumn{5}{|l|}{Value} \\
\tabucline[1.5pt]{}
BrowseName & \multicolumn{5}{|l|}{MotionSubClassType} \\
IsAbstract & \multicolumn{5}{|l|}{False} \\
\tabucline[1.5pt]{}
\rowfont \bfseries References & NodeClass & BrowseName & DataType & Type\-Definition & {Modeling\-Rule} \\
\multicolumn{6}{|l|}{Subtype of MTDataItemSubClassType (See section \ref{type:MTDataItemSubClassType})} \\
\end{tabu}
\end{table} 


\FloatBarrier
\subsubsection{Defintion of \texttt{ NoScaleSubClassType}}
  \label{type:NoScaleSubClassType}

\FloatBarrier

No weighting factor on the frequency scale for the measurement of sound level.

\begin{table}[ht]
\centering 
  \caption{\texttt{NoScaleSubClassType} Definition}
  \label{table:NoScaleSubClassType}
\fontsize{9pt}{11pt}\selectfont
\tabulinesep=3pt
\begin{tabu} to 6in {|X[-1.35]|X[-0.7]|X[-1.75]|X[-1.5]|X[-1]|X[-0.7]|} \everyrow{\hline}
\hline
\rowfont\bfseries {Attribute} & \multicolumn{5}{|l|}{Value} \\
\tabucline[1.5pt]{}
BrowseName & \multicolumn{5}{|l|}{NoScaleSubClassType} \\
IsAbstract & \multicolumn{5}{|l|}{False} \\
\tabucline[1.5pt]{}
\rowfont \bfseries References & NodeClass & BrowseName & DataType & Type\-Definition & {Modeling\-Rule} \\
\multicolumn{6}{|l|}{Subtype of MTDataItemSubClassType (See section \ref{type:MTDataItemSubClassType})} \\
\end{tabu}
\end{table} 


\FloatBarrier
\subsubsection{Defintion of \texttt{ OperatingSubClassType}}
  \label{type:OperatingSubClassType}

\FloatBarrier

An indication that the major sub-parts of a piece of equipment are powered or performing any activity whether producing a part or product or not. 

Example: For traditional machine tools, this includes when the piece of equipment is \mtmodel{WORKING} or it is idle.

A piece of equipment are powered or performing any activity.

\begin{table}[ht]
\centering 
  \caption{\texttt{OperatingSubClassType} Definition}
  \label{table:OperatingSubClassType}
\fontsize{9pt}{11pt}\selectfont
\tabulinesep=3pt
\begin{tabu} to 6in {|X[-1.35]|X[-0.7]|X[-1.75]|X[-1.5]|X[-1]|X[-0.7]|} \everyrow{\hline}
\hline
\rowfont\bfseries {Attribute} & \multicolumn{5}{|l|}{Value} \\
\tabucline[1.5pt]{}
BrowseName & \multicolumn{5}{|l|}{OperatingSubClassType} \\
IsAbstract & \multicolumn{5}{|l|}{False} \\
\tabucline[1.5pt]{}
\rowfont \bfseries References & NodeClass & BrowseName & DataType & Type\-Definition & {Modeling\-Rule} \\
\multicolumn{6}{|l|}{Subtype of MTDataItemSubClassType (See section \ref{type:MTDataItemSubClassType})} \\
\end{tabu}
\end{table} 


\FloatBarrier
\subsubsection{Defintion of \texttt{ OperatorSubClassType}}
  \label{type:OperatorSubClassType}

\FloatBarrier

The identifier of the person currently responsible for operating the piece of equipment.

The identifier of the person currently responsible for operating the piece of equipment.

\begin{table}[ht]
\centering 
  \caption{\texttt{OperatorSubClassType} Definition}
  \label{table:OperatorSubClassType}
\fontsize{9pt}{11pt}\selectfont
\tabulinesep=3pt
\begin{tabu} to 6in {|X[-1.35]|X[-0.7]|X[-1.75]|X[-1.5]|X[-1]|X[-0.7]|} \everyrow{\hline}
\hline
\rowfont\bfseries {Attribute} & \multicolumn{5}{|l|}{Value} \\
\tabucline[1.5pt]{}
BrowseName & \multicolumn{5}{|l|}{OperatorSubClassType} \\
IsAbstract & \multicolumn{5}{|l|}{False} \\
\tabucline[1.5pt]{}
\rowfont \bfseries References & NodeClass & BrowseName & DataType & Type\-Definition & {Modeling\-Rule} \\
\multicolumn{6}{|l|}{Subtype of MTDataItemSubClassType (See section \ref{type:MTDataItemSubClassType})} \\
\end{tabu}
\end{table} 


\FloatBarrier
\subsubsection{Defintion of \texttt{ OptionalStopSubClassType}}
  \label{type:OptionalStopSubClassType}

\FloatBarrier

A setting or operator selection that changes the behavior of the controller on a piece of equipment.

\begin{table}[ht]
\centering 
  \caption{\texttt{OptionalStopSubClassType} Definition}
  \label{table:OptionalStopSubClassType}
\fontsize{9pt}{11pt}\selectfont
\tabulinesep=3pt
\begin{tabu} to 6in {|X[-1.35]|X[-0.7]|X[-1.75]|X[-1.5]|X[-1]|X[-0.7]|} \everyrow{\hline}
\hline
\rowfont\bfseries {Attribute} & \multicolumn{5}{|l|}{Value} \\
\tabucline[1.5pt]{}
BrowseName & \multicolumn{5}{|l|}{OptionalStopSubClassType} \\
IsAbstract & \multicolumn{5}{|l|}{False} \\
\tabucline[1.5pt]{}
\rowfont \bfseries References & NodeClass & BrowseName & DataType & Type\-Definition & {Modeling\-Rule} \\
\multicolumn{6}{|l|}{Subtype of MTDataItemSubClassType (See section \ref{type:MTDataItemSubClassType})} \\
\end{tabu}
\end{table} 


\FloatBarrier
\subsubsection{Defintion of \texttt{ OverrideSubClassType}}
  \label{type:OverrideSubClassType}

\FloatBarrier

The operator's overridden value.

DEPRECATED: The operators overridden value.

\begin{table}[ht]
\centering 
  \caption{\texttt{OverrideSubClassType} Definition}
  \label{table:OverrideSubClassType}
\fontsize{9pt}{11pt}\selectfont
\tabulinesep=3pt
\begin{tabu} to 6in {|X[-1.35]|X[-0.7]|X[-1.75]|X[-1.5]|X[-1]|X[-0.7]|} \everyrow{\hline}
\hline
\rowfont\bfseries {Attribute} & \multicolumn{5}{|l|}{Value} \\
\tabucline[1.5pt]{}
BrowseName & \multicolumn{5}{|l|}{OverrideSubClassType} \\
IsAbstract & \multicolumn{5}{|l|}{False} \\
\tabucline[1.5pt]{}
\rowfont \bfseries References & NodeClass & BrowseName & DataType & Type\-Definition & {Modeling\-Rule} \\
\multicolumn{6}{|l|}{Subtype of MTDataItemSubClassType (See section \ref{type:MTDataItemSubClassType})} \\
\end{tabu}
\end{table} 


\FloatBarrier
\subsubsection{Defintion of \texttt{ PoweredSubClassType}}
  \label{type:PoweredSubClassType}

\FloatBarrier

An indication that primary power is applied to the piece of equipment and, as a minimum, the controller 
or logic portion of the piece of equipment is powered and functioning or components that are required to remain on are powered.

Primary  power is  applied  to the  piece  of  equipment and,  as  a minimum, the controller or logic portion of the piece of equipment is powered and functioning or components that are required to remain on are powered.

\begin{table}[ht]
\centering 
  \caption{\texttt{PoweredSubClassType} Definition}
  \label{table:PoweredSubClassType}
\fontsize{9pt}{11pt}\selectfont
\tabulinesep=3pt
\begin{tabu} to 6in {|X[-1.35]|X[-0.7]|X[-1.75]|X[-1.5]|X[-1]|X[-0.7]|} \everyrow{\hline}
\hline
\rowfont\bfseries {Attribute} & \multicolumn{5}{|l|}{Value} \\
\tabucline[1.5pt]{}
BrowseName & \multicolumn{5}{|l|}{PoweredSubClassType} \\
IsAbstract & \multicolumn{5}{|l|}{False} \\
\tabucline[1.5pt]{}
\rowfont \bfseries References & NodeClass & BrowseName & DataType & Type\-Definition & {Modeling\-Rule} \\
\multicolumn{6}{|l|}{Subtype of MTDataItemSubClassType (See section \ref{type:MTDataItemSubClassType})} \\
\end{tabu}
\end{table} 


\FloatBarrier
\subsubsection{Defintion of \texttt{ PrimarySubClassType}}
  \label{type:PrimarySubClassType}

\FloatBarrier

Specific applications MAY reference one or more locations on a piece of bar stock as the indication 
for the \mtmodel{END_OF_BAR}. The main or most important location MUST be 
designated as the \mtmodel{PRIMARY} indication for the \mtmodel{END_OF_BAR}.

Specific applications MAY reference one or more locations on a piece of bar stock as the indication for the endofbar event. The main or most important location must be designated as the primary subtype indication for the endofbar event.   
 If no subtype is specified, primary subtype must be the default endofbar event indication.

\begin{table}[ht]
\centering 
  \caption{\texttt{PrimarySubClassType} Definition}
  \label{table:PrimarySubClassType}
\fontsize{9pt}{11pt}\selectfont
\tabulinesep=3pt
\begin{tabu} to 6in {|X[-1.35]|X[-0.7]|X[-1.75]|X[-1.5]|X[-1]|X[-0.7]|} \everyrow{\hline}
\hline
\rowfont\bfseries {Attribute} & \multicolumn{5}{|l|}{Value} \\
\tabucline[1.5pt]{}
BrowseName & \multicolumn{5}{|l|}{PrimarySubClassType} \\
IsAbstract & \multicolumn{5}{|l|}{False} \\
\tabucline[1.5pt]{}
\rowfont \bfseries References & NodeClass & BrowseName & DataType & Type\-Definition & {Modeling\-Rule} \\
\multicolumn{6}{|l|}{Subtype of MTDataItemSubClassType (See section \ref{type:MTDataItemSubClassType})} \\
\end{tabu}
\end{table} 


\FloatBarrier
\subsubsection{Defintion of \texttt{ ProbeSubClassType}}
  \label{type:ProbeSubClassType}

\FloatBarrier

The position provided by a measurement probe.

The position provided by a measurement probe.

\begin{table}[ht]
\centering 
  \caption{\texttt{ProbeSubClassType} Definition}
  \label{table:ProbeSubClassType}
\fontsize{9pt}{11pt}\selectfont
\tabulinesep=3pt
\begin{tabu} to 6in {|X[-1.35]|X[-0.7]|X[-1.75]|X[-1.5]|X[-1]|X[-0.7]|} \everyrow{\hline}
\hline
\rowfont\bfseries {Attribute} & \multicolumn{5}{|l|}{Value} \\
\tabucline[1.5pt]{}
BrowseName & \multicolumn{5}{|l|}{ProbeSubClassType} \\
IsAbstract & \multicolumn{5}{|l|}{False} \\
\tabucline[1.5pt]{}
\rowfont \bfseries References & NodeClass & BrowseName & DataType & Type\-Definition & {Modeling\-Rule} \\
\multicolumn{6}{|l|}{Subtype of MTDataItemSubClassType (See section \ref{type:MTDataItemSubClassType})} \\
\end{tabu}
\end{table} 


\FloatBarrier
\subsubsection{Defintion of \texttt{ ProcessSubClassType}}
  \label{type:ProcessSubClassType}

\FloatBarrier

The measurement of the time from the beginning of production of a part or product on a piece of equipment 
until the time that production is complete for that part or product on that piece of equipment. 
This includes the time that the piece of equipment is running, producing parts or products, or in the process of producing parts.

The measurement of the time from the beginning of production of a part or product on a piece of equipment until the time that production is complete for that part or product on that piece of equipment.  This includes the time that the piece of equipment is running, producing parts or products, or in the process of producing parts.

\begin{table}[ht]
\centering 
  \caption{\texttt{ProcessSubClassType} Definition}
  \label{table:ProcessSubClassType}
\fontsize{9pt}{11pt}\selectfont
\tabulinesep=3pt
\begin{tabu} to 6in {|X[-1.35]|X[-0.7]|X[-1.75]|X[-1.5]|X[-1]|X[-0.7]|} \everyrow{\hline}
\hline
\rowfont\bfseries {Attribute} & \multicolumn{5}{|l|}{Value} \\
\tabucline[1.5pt]{}
BrowseName & \multicolumn{5}{|l|}{ProcessSubClassType} \\
IsAbstract & \multicolumn{5}{|l|}{False} \\
\tabucline[1.5pt]{}
\rowfont \bfseries References & NodeClass & BrowseName & DataType & Type\-Definition & {Modeling\-Rule} \\
\multicolumn{6}{|l|}{Subtype of MTDataItemSubClassType (See section \ref{type:MTDataItemSubClassType})} \\
\end{tabu}
\end{table} 


\FloatBarrier
\subsubsection{Defintion of \texttt{ ProgrammedSubClassType}}
  \label{type:ProgrammedSubClassType}

\FloatBarrier

The value of a signal or calculation issued to adjust the feedrate of the axes associated with a \mtmodel{Path} 
component when the axes, or a single axis, are operating as specified by a logic or motion program or set by a switch.

The value of a signal or calculation specified by a logic or motion program or set by a switch.

\begin{table}[ht]
\centering 
  \caption{\texttt{ProgrammedSubClassType} Definition}
  \label{table:ProgrammedSubClassType}
\fontsize{9pt}{11pt}\selectfont
\tabulinesep=3pt
\begin{tabu} to 6in {|X[-1.35]|X[-0.7]|X[-1.75]|X[-1.5]|X[-1]|X[-0.7]|} \everyrow{\hline}
\hline
\rowfont\bfseries {Attribute} & \multicolumn{5}{|l|}{Value} \\
\tabucline[1.5pt]{}
BrowseName & \multicolumn{5}{|l|}{ProgrammedSubClassType} \\
IsAbstract & \multicolumn{5}{|l|}{False} \\
\tabucline[1.5pt]{}
\rowfont \bfseries References & NodeClass & BrowseName & DataType & Type\-Definition & {Modeling\-Rule} \\
\multicolumn{6}{|l|}{Subtype of MTDataItemSubClassType (See section \ref{type:MTDataItemSubClassType})} \\
\end{tabu}
\end{table} 


\FloatBarrier
\subsubsection{Defintion of \texttt{ RadialSubClassType}}
  \label{type:RadialSubClassType}

\FloatBarrier

A reference to a radial type tool offset variable.

A reference to a radial type tool offset variable.

\begin{table}[ht]
\centering 
  \caption{\texttt{RadialSubClassType} Definition}
  \label{table:RadialSubClassType}
\fontsize{9pt}{11pt}\selectfont
\tabulinesep=3pt
\begin{tabu} to 6in {|X[-1.35]|X[-0.7]|X[-1.75]|X[-1.5]|X[-1]|X[-0.7]|} \everyrow{\hline}
\hline
\rowfont\bfseries {Attribute} & \multicolumn{5}{|l|}{Value} \\
\tabucline[1.5pt]{}
BrowseName & \multicolumn{5}{|l|}{RadialSubClassType} \\
IsAbstract & \multicolumn{5}{|l|}{False} \\
\tabucline[1.5pt]{}
\rowfont \bfseries References & NodeClass & BrowseName & DataType & Type\-Definition & {Modeling\-Rule} \\
\multicolumn{6}{|l|}{Subtype of MTDataItemSubClassType (See section \ref{type:MTDataItemSubClassType})} \\
\end{tabu}
\end{table} 


\FloatBarrier
\subsubsection{Defintion of \texttt{ RapidSubClassType}}
  \label{type:RapidSubClassType}

\FloatBarrier

The value of a signal or calculation issued to adjust the feedrate of the axes associated with a \mtmodel{Path} 
component when the axes, or a single axis, are being operated in a rapid positioning mode or method (rapid).

The value of a signal or calculation issued to adjust the feedrate of a component or composition that is operating in a rapid positioning mode.

\begin{table}[ht]
\centering 
  \caption{\texttt{RapidSubClassType} Definition}
  \label{table:RapidSubClassType}
\fontsize{9pt}{11pt}\selectfont
\tabulinesep=3pt
\begin{tabu} to 6in {|X[-1.35]|X[-0.7]|X[-1.75]|X[-1.5]|X[-1]|X[-0.7]|} \everyrow{\hline}
\hline
\rowfont\bfseries {Attribute} & \multicolumn{5}{|l|}{Value} \\
\tabucline[1.5pt]{}
BrowseName & \multicolumn{5}{|l|}{RapidSubClassType} \\
IsAbstract & \multicolumn{5}{|l|}{False} \\
\tabucline[1.5pt]{}
\rowfont \bfseries References & NodeClass & BrowseName & DataType & Type\-Definition & {Modeling\-Rule} \\
\multicolumn{6}{|l|}{Subtype of MTDataItemSubClassType (See section \ref{type:MTDataItemSubClassType})} \\
\end{tabu}
\end{table} 


\FloatBarrier
\subsubsection{Defintion of \texttt{ RelativeSubClassType}}
  \label{type:RelativeSubClassType}

\FloatBarrier



\begin{table}[ht]
\centering 
  \caption{\texttt{RelativeSubClassType} Definition}
  \label{table:RelativeSubClassType}
\fontsize{9pt}{11pt}\selectfont
\tabulinesep=3pt
\begin{tabu} to 6in {|X[-1.35]|X[-0.7]|X[-1.75]|X[-1.5]|X[-1]|X[-0.7]|} \everyrow{\hline}
\hline
\rowfont\bfseries {Attribute} & \multicolumn{5}{|l|}{Value} \\
\tabucline[1.5pt]{}
BrowseName & \multicolumn{5}{|l|}{RelativeSubClassType} \\
IsAbstract & \multicolumn{5}{|l|}{False} \\
\tabucline[1.5pt]{}
\rowfont \bfseries References & NodeClass & BrowseName & DataType & Type\-Definition & {Modeling\-Rule} \\
\multicolumn{6}{|l|}{Subtype of MTDataItemSubClassType (See section \ref{type:MTDataItemSubClassType})} \\
\end{tabu}
\end{table} 


\FloatBarrier
\subsubsection{Defintion of \texttt{ RemainingSubClassType}}
  \label{type:RemainingSubClassType}

\FloatBarrier

The remaining amount of the type specified.

Remaining measure of an object or an action.

\begin{table}[ht]
\centering 
  \caption{\texttt{RemainingSubClassType} Definition}
  \label{table:RemainingSubClassType}
\fontsize{9pt}{11pt}\selectfont
\tabulinesep=3pt
\begin{tabu} to 6in {|X[-1.35]|X[-0.7]|X[-1.75]|X[-1.5]|X[-1]|X[-0.7]|} \everyrow{\hline}
\hline
\rowfont\bfseries {Attribute} & \multicolumn{5}{|l|}{Value} \\
\tabucline[1.5pt]{}
BrowseName & \multicolumn{5}{|l|}{RemainingSubClassType} \\
IsAbstract & \multicolumn{5}{|l|}{False} \\
\tabucline[1.5pt]{}
\rowfont \bfseries References & NodeClass & BrowseName & DataType & Type\-Definition & {Modeling\-Rule} \\
\multicolumn{6}{|l|}{Subtype of MTDataItemSubClassType (See section \ref{type:MTDataItemSubClassType})} \\
\end{tabu}
\end{table} 


\FloatBarrier
\subsubsection{Defintion of \texttt{ RequestSubClassType}}
  \label{type:RequestSubClassType}

\FloatBarrier

\mtmodel{Request} subtype identifies if the data item defined for MTConnect Interaction Model \cite{MTCPart5} represents a request.

A subtype of an interface dataitem type to communicate a request. 

\begin{table}[ht]
\centering 
  \caption{\texttt{RequestSubClassType} Definition}
  \label{table:RequestSubClassType}
\fontsize{9pt}{11pt}\selectfont
\tabulinesep=3pt
\begin{tabu} to 6in {|X[-1.35]|X[-0.7]|X[-1.75]|X[-1.5]|X[-1]|X[-0.7]|} \everyrow{\hline}
\hline
\rowfont\bfseries {Attribute} & \multicolumn{5}{|l|}{Value} \\
\tabucline[1.5pt]{}
BrowseName & \multicolumn{5}{|l|}{RequestSubClassType} \\
IsAbstract & \multicolumn{5}{|l|}{False} \\
\tabucline[1.5pt]{}
\rowfont \bfseries References & NodeClass & BrowseName & DataType & Type\-Definition & {Modeling\-Rule} \\
\multicolumn{6}{|l|}{Subtype of MTDataItemSubClassType (See section \ref{type:MTDataItemSubClassType})} \\
\end{tabu}
\end{table} 


\FloatBarrier
\subsubsection{Defintion of \texttt{ ResponseSubClassType}}
  \label{type:ResponseSubClassType}

\FloatBarrier

\mtmodel{Response} subtype identifies if the data item defined for MTConnect Interaction Model \cite{MTCPart5} represents a response.

A subtype of an interface dataitem type to communicate a response.

\begin{table}[ht]
\centering 
  \caption{\texttt{ResponseSubClassType} Definition}
  \label{table:ResponseSubClassType}
\fontsize{9pt}{11pt}\selectfont
\tabulinesep=3pt
\begin{tabu} to 6in {|X[-1.35]|X[-0.7]|X[-1.75]|X[-1.5]|X[-1]|X[-0.7]|} \everyrow{\hline}
\hline
\rowfont\bfseries {Attribute} & \multicolumn{5}{|l|}{Value} \\
\tabucline[1.5pt]{}
BrowseName & \multicolumn{5}{|l|}{ResponseSubClassType} \\
IsAbstract & \multicolumn{5}{|l|}{False} \\
\tabucline[1.5pt]{}
\rowfont \bfseries References & NodeClass & BrowseName & DataType & Type\-Definition & {Modeling\-Rule} \\
\multicolumn{6}{|l|}{Subtype of MTDataItemSubClassType (See section \ref{type:MTDataItemSubClassType})} \\
\end{tabu}
\end{table} 


\FloatBarrier
\subsubsection{Defintion of \texttt{ RockwellSubClassType}}
  \label{type:RockwellSubClassType}

\FloatBarrier

A scale to measure the resistance to deformation of a surface.

A scale to measure the resistance to deformation of a surface.

\begin{table}[ht]
\centering 
  \caption{\texttt{RockwellSubClassType} Definition}
  \label{table:RockwellSubClassType}
\fontsize{9pt}{11pt}\selectfont
\tabulinesep=3pt
\begin{tabu} to 6in {|X[-1.35]|X[-0.7]|X[-1.75]|X[-1.5]|X[-1]|X[-0.7]|} \everyrow{\hline}
\hline
\rowfont\bfseries {Attribute} & \multicolumn{5}{|l|}{Value} \\
\tabucline[1.5pt]{}
BrowseName & \multicolumn{5}{|l|}{RockwellSubClassType} \\
IsAbstract & \multicolumn{5}{|l|}{False} \\
\tabucline[1.5pt]{}
\rowfont \bfseries References & NodeClass & BrowseName & DataType & Type\-Definition & {Modeling\-Rule} \\
\multicolumn{6}{|l|}{Subtype of MTDataItemSubClassType (See section \ref{type:MTDataItemSubClassType})} \\
\end{tabu}
\end{table} 


\FloatBarrier
\subsubsection{Defintion of \texttt{ RotarySubClassType}}
  \label{type:RotarySubClassType}

\FloatBarrier

The rotational direction of a rotary motion using the right hand rule convention.

A rotary axis represents any non-linear or rotary movement of a physical piece of equipment or a portion of the equipment. 

\begin{table}[ht]
\centering 
  \caption{\texttt{RotarySubClassType} Definition}
  \label{table:RotarySubClassType}
\fontsize{9pt}{11pt}\selectfont
\tabulinesep=3pt
\begin{tabu} to 6in {|X[-1.35]|X[-0.7]|X[-1.75]|X[-1.5]|X[-1]|X[-0.7]|} \everyrow{\hline}
\hline
\rowfont\bfseries {Attribute} & \multicolumn{5}{|l|}{Value} \\
\tabucline[1.5pt]{}
BrowseName & \multicolumn{5}{|l|}{RotarySubClassType} \\
IsAbstract & \multicolumn{5}{|l|}{False} \\
\tabucline[1.5pt]{}
\rowfont \bfseries References & NodeClass & BrowseName & DataType & Type\-Definition & {Modeling\-Rule} \\
\multicolumn{6}{|l|}{Subtype of MTDataItemSubClassType (See section \ref{type:MTDataItemSubClassType})} \\
\end{tabu}
\end{table} 


\FloatBarrier
\subsubsection{Defintion of \texttt{ SetUpSubClassType}}
  \label{type:SetUpSubClassType}

\FloatBarrier

The identifier of the person currently responsible for operating the piece of equipment.

A structural element is being prepared or modified to begin production of product.

\begin{table}[ht]
\centering 
  \caption{\texttt{SetUpSubClassType} Definition}
  \label{table:SetUpSubClassType}
\fontsize{9pt}{11pt}\selectfont
\tabulinesep=3pt
\begin{tabu} to 6in {|X[-1.35]|X[-0.7]|X[-1.75]|X[-1.5]|X[-1]|X[-0.7]|} \everyrow{\hline}
\hline
\rowfont\bfseries {Attribute} & \multicolumn{5}{|l|}{Value} \\
\tabucline[1.5pt]{}
BrowseName & \multicolumn{5}{|l|}{SetUpSubClassType} \\
IsAbstract & \multicolumn{5}{|l|}{False} \\
\tabucline[1.5pt]{}
\rowfont \bfseries References & NodeClass & BrowseName & DataType & Type\-Definition & {Modeling\-Rule} \\
\multicolumn{6}{|l|}{Subtype of MTDataItemSubClassType (See section \ref{type:MTDataItemSubClassType})} \\
\end{tabu}
\end{table} 


\FloatBarrier
\subsubsection{Defintion of \texttt{ ShoreSubClassType}}
  \label{type:ShoreSubClassType}

\FloatBarrier

A scale to measure the resistance to deformation of a surface.

A scale to measure the resistance to deformation of a surface.

\begin{table}[ht]
\centering 
  \caption{\texttt{ShoreSubClassType} Definition}
  \label{table:ShoreSubClassType}
\fontsize{9pt}{11pt}\selectfont
\tabulinesep=3pt
\begin{tabu} to 6in {|X[-1.35]|X[-0.7]|X[-1.75]|X[-1.5]|X[-1]|X[-0.7]|} \everyrow{\hline}
\hline
\rowfont\bfseries {Attribute} & \multicolumn{5}{|l|}{Value} \\
\tabucline[1.5pt]{}
BrowseName & \multicolumn{5}{|l|}{ShoreSubClassType} \\
IsAbstract & \multicolumn{5}{|l|}{False} \\
\tabucline[1.5pt]{}
\rowfont \bfseries References & NodeClass & BrowseName & DataType & Type\-Definition & {Modeling\-Rule} \\
\multicolumn{6}{|l|}{Subtype of MTDataItemSubClassType (See section \ref{type:MTDataItemSubClassType})} \\
\end{tabu}
\end{table} 


\FloatBarrier
\subsubsection{Defintion of \texttt{ StandardSubClassType}}
  \label{type:StandardSubClassType}

\FloatBarrier

The standard or original value of an object.

The standard or original length of an object.

\begin{table}[ht]
\centering 
  \caption{\texttt{StandardSubClassType} Definition}
  \label{table:StandardSubClassType}
\fontsize{9pt}{11pt}\selectfont
\tabulinesep=3pt
\begin{tabu} to 6in {|X[-1.35]|X[-0.7]|X[-1.75]|X[-1.5]|X[-1]|X[-0.7]|} \everyrow{\hline}
\hline
\rowfont\bfseries {Attribute} & \multicolumn{5}{|l|}{Value} \\
\tabucline[1.5pt]{}
BrowseName & \multicolumn{5}{|l|}{StandardSubClassType} \\
IsAbstract & \multicolumn{5}{|l|}{False} \\
\tabucline[1.5pt]{}
\rowfont \bfseries References & NodeClass & BrowseName & DataType & Type\-Definition & {Modeling\-Rule} \\
\multicolumn{6}{|l|}{Subtype of MTDataItemSubClassType (See section \ref{type:MTDataItemSubClassType})} \\
\end{tabu}
\end{table} 


\FloatBarrier
\subsubsection{Defintion of \texttt{ SwitchedSubClassType}}
  \label{type:SwitchedSubClassType}

\FloatBarrier

An indication of the activation state of a mechanism represented by a \mtmodel{Composition} type component.

The activation state indicates whether the \mtmodel{Composition} element is activated or not.

An indication of the activation state of a mechanism represented by a composition type component.
 The activation state indicates whether the composition element is activated or not.
 The valid data value must be on value or off value.

\begin{table}[ht]
\centering 
  \caption{\texttt{SwitchedSubClassType} Definition}
  \label{table:SwitchedSubClassType}
\fontsize{9pt}{11pt}\selectfont
\tabulinesep=3pt
\begin{tabu} to 6in {|X[-1.35]|X[-0.7]|X[-1.75]|X[-1.5]|X[-1]|X[-0.7]|} \everyrow{\hline}
\hline
\rowfont\bfseries {Attribute} & \multicolumn{5}{|l|}{Value} \\
\tabucline[1.5pt]{}
BrowseName & \multicolumn{5}{|l|}{SwitchedSubClassType} \\
IsAbstract & \multicolumn{5}{|l|}{False} \\
\tabucline[1.5pt]{}
\rowfont \bfseries References & NodeClass & BrowseName & DataType & Type\-Definition & {Modeling\-Rule} \\
\multicolumn{6}{|l|}{Subtype of MTDataItemSubClassType (See section \ref{type:MTDataItemSubClassType})} \\
\end{tabu}
\end{table} 


\FloatBarrier
\subsubsection{Defintion of \texttt{ TargetSubClassType}}
  \label{type:TargetSubClassType}

\FloatBarrier

Indicates the number of parts that are projected or planned to be produced.

The desired measure or count for a data item value.

\begin{table}[ht]
\centering 
  \caption{\texttt{TargetSubClassType} Definition}
  \label{table:TargetSubClassType}
\fontsize{9pt}{11pt}\selectfont
\tabulinesep=3pt
\begin{tabu} to 6in {|X[-1.35]|X[-0.7]|X[-1.75]|X[-1.5]|X[-1]|X[-0.7]|} \everyrow{\hline}
\hline
\rowfont\bfseries {Attribute} & \multicolumn{5}{|l|}{Value} \\
\tabucline[1.5pt]{}
BrowseName & \multicolumn{5}{|l|}{TargetSubClassType} \\
IsAbstract & \multicolumn{5}{|l|}{False} \\
\tabucline[1.5pt]{}
\rowfont \bfseries References & NodeClass & BrowseName & DataType & Type\-Definition & {Modeling\-Rule} \\
\multicolumn{6}{|l|}{Subtype of MTDataItemSubClassType (See section \ref{type:MTDataItemSubClassType})} \\
\end{tabu}
\end{table} 


\FloatBarrier
\subsubsection{Defintion of \texttt{ ToolChangeStopSubClassType}}
  \label{type:ToolChangeStopSubClassType}

\FloatBarrier

A setting or operator selection that changes the behavior of the controller on a piece of equipment.

\begin{table}[ht]
\centering 
  \caption{\texttt{ToolChangeStopSubClassType} Definition}
  \label{table:ToolChangeStopSubClassType}
\fontsize{9pt}{11pt}\selectfont
\tabulinesep=3pt
\begin{tabu} to 6in {|X[-1.35]|X[-0.7]|X[-1.75]|X[-1.5]|X[-1]|X[-0.7]|} \everyrow{\hline}
\hline
\rowfont\bfseries {Attribute} & \multicolumn{5}{|l|}{Value} \\
\tabucline[1.5pt]{}
BrowseName & \multicolumn{5}{|l|}{ToolChangeStopSubClassType} \\
IsAbstract & \multicolumn{5}{|l|}{False} \\
\tabucline[1.5pt]{}
\rowfont \bfseries References & NodeClass & BrowseName & DataType & Type\-Definition & {Modeling\-Rule} \\
\multicolumn{6}{|l|}{Subtype of MTDataItemSubClassType (See section \ref{type:MTDataItemSubClassType})} \\
\end{tabu}
\end{table} 


\FloatBarrier
\subsubsection{Defintion of \texttt{ ToolEdgeSubClassType}}
  \label{type:ToolEdgeSubClassType}

\FloatBarrier



\begin{table}[ht]
\centering 
  \caption{\texttt{ToolEdgeSubClassType} Definition}
  \label{table:ToolEdgeSubClassType}
\fontsize{9pt}{11pt}\selectfont
\tabulinesep=3pt
\begin{tabu} to 6in {|X[-1.35]|X[-0.7]|X[-1.75]|X[-1.5]|X[-1]|X[-0.7]|} \everyrow{\hline}
\hline
\rowfont\bfseries {Attribute} & \multicolumn{5}{|l|}{Value} \\
\tabucline[1.5pt]{}
BrowseName & \multicolumn{5}{|l|}{ToolEdgeSubClassType} \\
IsAbstract & \multicolumn{5}{|l|}{False} \\
\tabucline[1.5pt]{}
\rowfont \bfseries References & NodeClass & BrowseName & DataType & Type\-Definition & {Modeling\-Rule} \\
\multicolumn{6}{|l|}{Subtype of MTDataItemSubClassType (See section \ref{type:MTDataItemSubClassType})} \\
\end{tabu}
\end{table} 


\FloatBarrier
\subsubsection{Defintion of \texttt{ ToolGroupSubClassType}}
  \label{type:ToolGroupSubClassType}

\FloatBarrier

The tool group a specific tool is assigned to in the part program.

An identifier for the tool group associated with a specific tool. Commonly used to designate spare tools.

\begin{table}[ht]
\centering 
  \caption{\texttt{ToolGroupSubClassType} Definition}
  \label{table:ToolGroupSubClassType}
\fontsize{9pt}{11pt}\selectfont
\tabulinesep=3pt
\begin{tabu} to 6in {|X[-1.35]|X[-0.7]|X[-1.75]|X[-1.5]|X[-1]|X[-0.7]|} \everyrow{\hline}
\hline
\rowfont\bfseries {Attribute} & \multicolumn{5}{|l|}{Value} \\
\tabucline[1.5pt]{}
BrowseName & \multicolumn{5}{|l|}{ToolGroupSubClassType} \\
IsAbstract & \multicolumn{5}{|l|}{False} \\
\tabucline[1.5pt]{}
\rowfont \bfseries References & NodeClass & BrowseName & DataType & Type\-Definition & {Modeling\-Rule} \\
\multicolumn{6}{|l|}{Subtype of MTDataItemSubClassType (See section \ref{type:MTDataItemSubClassType})} \\
\end{tabu}
\end{table} 


\FloatBarrier
\subsubsection{Defintion of \texttt{ ToolSubClassType}}
  \label{type:ToolSubClassType}

\FloatBarrier



\begin{table}[ht]
\centering 
  \caption{\texttt{ToolSubClassType} Definition}
  \label{table:ToolSubClassType}
\fontsize{9pt}{11pt}\selectfont
\tabulinesep=3pt
\begin{tabu} to 6in {|X[-1.35]|X[-0.7]|X[-1.75]|X[-1.5]|X[-1]|X[-0.7]|} \everyrow{\hline}
\hline
\rowfont\bfseries {Attribute} & \multicolumn{5}{|l|}{Value} \\
\tabucline[1.5pt]{}
BrowseName & \multicolumn{5}{|l|}{ToolSubClassType} \\
IsAbstract & \multicolumn{5}{|l|}{False} \\
\tabucline[1.5pt]{}
\rowfont \bfseries References & NodeClass & BrowseName & DataType & Type\-Definition & {Modeling\-Rule} \\
\multicolumn{6}{|l|}{Subtype of MTDataItemSubClassType (See section \ref{type:MTDataItemSubClassType})} \\
\end{tabu}
\end{table} 


\FloatBarrier
\subsubsection{Defintion of \texttt{ UasbleSubClassType}}
  \label{type:UasbleSubClassType}

\FloatBarrier

The remaining useable value of an object.

\begin{table}[ht]
\centering 
  \caption{\texttt{UasbleSubClassType} Definition}
  \label{table:UasbleSubClassType}
\fontsize{9pt}{11pt}\selectfont
\tabulinesep=3pt
\begin{tabu} to 6in {|X[-1.35]|X[-0.7]|X[-1.75]|X[-1.5]|X[-1]|X[-0.7]|} \everyrow{\hline}
\hline
\rowfont\bfseries {Attribute} & \multicolumn{5}{|l|}{Value} \\
\tabucline[1.5pt]{}
BrowseName & \multicolumn{5}{|l|}{UasbleSubClassType} \\
IsAbstract & \multicolumn{5}{|l|}{False} \\
\tabucline[1.5pt]{}
\rowfont \bfseries References & NodeClass & BrowseName & DataType & Type\-Definition & {Modeling\-Rule} \\
\multicolumn{6}{|l|}{Subtype of MTDataItemSubClassType (See section \ref{type:MTDataItemSubClassType})} \\
\end{tabu}
\end{table} 


\FloatBarrier
\subsubsection{Defintion of \texttt{ VerticalSubClassType}}
  \label{type:VerticalSubClassType}

\FloatBarrier

An indication of the position of a mechanism that may move in a vertical direction.

An indication of the position of a mechanism that may move in a vertical direction. The mechanism is represented by a composition type component. 
 The position information indicates whether the composition element is positioned to the top, to the bottom, or is in transition.  
 The valid data value must be up value, down value, or transitioning value.

\begin{table}[ht]
\centering 
  \caption{\texttt{VerticalSubClassType} Definition}
  \label{table:VerticalSubClassType}
\fontsize{9pt}{11pt}\selectfont
\tabulinesep=3pt
\begin{tabu} to 6in {|X[-1.35]|X[-0.7]|X[-1.75]|X[-1.5]|X[-1]|X[-0.7]|} \everyrow{\hline}
\hline
\rowfont\bfseries {Attribute} & \multicolumn{5}{|l|}{Value} \\
\tabucline[1.5pt]{}
BrowseName & \multicolumn{5}{|l|}{VerticalSubClassType} \\
IsAbstract & \multicolumn{5}{|l|}{False} \\
\tabucline[1.5pt]{}
\rowfont \bfseries References & NodeClass & BrowseName & DataType & Type\-Definition & {Modeling\-Rule} \\
\multicolumn{6}{|l|}{Subtype of MTDataItemSubClassType (See section \ref{type:MTDataItemSubClassType})} \\
\end{tabu}
\end{table} 


\FloatBarrier
\subsubsection{Defintion of \texttt{ VickersSubClassType}}
  \label{type:VickersSubClassType}

\FloatBarrier

A scale to measure the resistance to deformation of a surface.

A scale to measure the resistance to deformation of a surface.

\begin{table}[ht]
\centering 
  \caption{\texttt{VickersSubClassType} Definition}
  \label{table:VickersSubClassType}
\fontsize{9pt}{11pt}\selectfont
\tabulinesep=3pt
\begin{tabu} to 6in {|X[-1.35]|X[-0.7]|X[-1.75]|X[-1.5]|X[-1]|X[-0.7]|} \everyrow{\hline}
\hline
\rowfont\bfseries {Attribute} & \multicolumn{5}{|l|}{Value} \\
\tabucline[1.5pt]{}
BrowseName & \multicolumn{5}{|l|}{VickersSubClassType} \\
IsAbstract & \multicolumn{5}{|l|}{False} \\
\tabucline[1.5pt]{}
\rowfont \bfseries References & NodeClass & BrowseName & DataType & Type\-Definition & {Modeling\-Rule} \\
\multicolumn{6}{|l|}{Subtype of MTDataItemSubClassType (See section \ref{type:MTDataItemSubClassType})} \\
\end{tabu}
\end{table} 


\FloatBarrier
\subsubsection{Defintion of \texttt{ VolumeSubClassType}}
  \label{type:VolumeSubClassType}

\FloatBarrier

A measurement of space accupied by a physical object.

\begin{table}[ht]
\centering 
  \caption{\texttt{VolumeSubClassType} Definition}
  \label{table:VolumeSubClassType}
\fontsize{9pt}{11pt}\selectfont
\tabulinesep=3pt
\begin{tabu} to 6in {|X[-1.35]|X[-0.7]|X[-1.75]|X[-1.5]|X[-1]|X[-0.7]|} \everyrow{\hline}
\hline
\rowfont\bfseries {Attribute} & \multicolumn{5}{|l|}{Value} \\
\tabucline[1.5pt]{}
BrowseName & \multicolumn{5}{|l|}{VolumeSubClassType} \\
IsAbstract & \multicolumn{5}{|l|}{False} \\
\tabucline[1.5pt]{}
\rowfont \bfseries References & NodeClass & BrowseName & DataType & Type\-Definition & {Modeling\-Rule} \\
\multicolumn{6}{|l|}{Subtype of MTDataItemSubClassType (See section \ref{type:MTDataItemSubClassType})} \\
\end{tabu}
\end{table} 


\FloatBarrier
\subsubsection{Defintion of \texttt{ WeightSubClassType}}
  \label{type:WeightSubClassType}

\FloatBarrier

A physical object's relative mass.

The total weight of the Cutting Tool in grams. The force exerted by the mass of the Cutting Tool. 

\begin{table}[ht]
\centering 
  \caption{\texttt{WeightSubClassType} Definition}
  \label{table:WeightSubClassType}
\fontsize{9pt}{11pt}\selectfont
\tabulinesep=3pt
\begin{tabu} to 6in {|X[-1.35]|X[-0.7]|X[-1.75]|X[-1.5]|X[-1]|X[-0.7]|} \everyrow{\hline}
\hline
\rowfont\bfseries {Attribute} & \multicolumn{5}{|l|}{Value} \\
\tabucline[1.5pt]{}
BrowseName & \multicolumn{5}{|l|}{WeightSubClassType} \\
IsAbstract & \multicolumn{5}{|l|}{False} \\
\tabucline[1.5pt]{}
\rowfont \bfseries References & NodeClass & BrowseName & DataType & Type\-Definition & {Modeling\-Rule} \\
\multicolumn{6}{|l|}{Subtype of MTDataItemSubClassType (See section \ref{type:MTDataItemSubClassType})} \\
\end{tabu}
\end{table} 


\FloatBarrier
\subsubsection{Defintion of \texttt{ WorkingSubClassType}}
  \label{type:WorkingSubClassType}

\FloatBarrier

An indication that a piece of equipment is performing any activity.

A piece of equipment performing any activity, the equipment is active and performing a function under load or not.

\begin{table}[ht]
\centering 
  \caption{\texttt{WorkingSubClassType} Definition}
  \label{table:WorkingSubClassType}
\fontsize{9pt}{11pt}\selectfont
\tabulinesep=3pt
\begin{tabu} to 6in {|X[-1.35]|X[-0.7]|X[-1.75]|X[-1.5]|X[-1]|X[-0.7]|} \everyrow{\hline}
\hline
\rowfont\bfseries {Attribute} & \multicolumn{5}{|l|}{Value} \\
\tabucline[1.5pt]{}
BrowseName & \multicolumn{5}{|l|}{WorkingSubClassType} \\
IsAbstract & \multicolumn{5}{|l|}{False} \\
\tabucline[1.5pt]{}
\rowfont \bfseries References & NodeClass & BrowseName & DataType & Type\-Definition & {Modeling\-Rule} \\
\multicolumn{6}{|l|}{Subtype of MTDataItemSubClassType (See section \ref{type:MTDataItemSubClassType})} \\
\end{tabu}
\end{table} 


\FloatBarrier
\subsubsection{Defintion of \texttt{ WorkpieceSubClassType}}
  \label{type:WorkpieceSubClassType}

\FloatBarrier

A physical object being or to be worked on with a tool or machine.

\begin{table}[ht]
\centering 
  \caption{\texttt{WorkpieceSubClassType} Definition}
  \label{table:WorkpieceSubClassType}
\fontsize{9pt}{11pt}\selectfont
\tabulinesep=3pt
\begin{tabu} to 6in {|X[-1.35]|X[-0.7]|X[-1.75]|X[-1.5]|X[-1]|X[-0.7]|} \everyrow{\hline}
\hline
\rowfont\bfseries {Attribute} & \multicolumn{5}{|l|}{Value} \\
\tabucline[1.5pt]{}
BrowseName & \multicolumn{5}{|l|}{WorkpieceSubClassType} \\
IsAbstract & \multicolumn{5}{|l|}{False} \\
\tabucline[1.5pt]{}
\rowfont \bfseries References & NodeClass & BrowseName & DataType & Type\-Definition & {Modeling\-Rule} \\
\multicolumn{6}{|l|}{Subtype of MTDataItemSubClassType (See section \ref{type:MTDataItemSubClassType})} \\
\end{tabu}
\end{table} 


\FloatBarrier
