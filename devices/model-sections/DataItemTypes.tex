% Generated 2020-01-09 18:51:19 -0800
\subsection{Data Item Types} \label{model:DataItemTypes}

The data item types represent the MTConnect types as defined in Section 8 of the 
MTConnect Standard Part 2 \cite{MTCPart2} and are represented using the XML \gls{type} 
\mtmodel{attribute}. The type and sub-type relationships are given in the standard and are 
not documented in the companion specifiction. 

The model of the types is similar to the OPC UA condition classes as given in OPC UA Part 9
\cite{UAPart9}. Since MTConnect conditions use the same type system as the data items, 
the representation of the data item type will be derived from the \texttt{BaseConditionClassType}. 
The MTConnect Types will not use the \mtmodel{...ClassType}, but will be the name as specified
in the MTConnect standard with \mtmodel{Type} appended.

For example: The \mtmodel{CONTOLLER_MODE} will be given as the \mtmodel{ControllerModeType}. 
The relationship to the \gls{MTDataItem} OPC UA types are presented so that it will be
easier to map the MTConnect types to the correct super class as given in the OPC UA model.



\subsubsection{Defintion of \texttt{ MTDataItemClassType}}
  \label{type:MTDataItemClassType}

\FloatBarrier

Abstract base class for all the data item class types. The names are created by pascal typing the names
and then generating appending \mtmodel{Type}.

data entity describing a piece of information reported about a piece of equipment.

\begin{table}[ht]
\centering 
  \caption{\texttt{MTDataItemClassType} Definition}
  \label{table:MTDataItemClassType}
\fontsize{9pt}{11pt}\selectfont
\tabulinesep=3pt
\begin{tabu} to 6in {|X[-1.35]|X[-0.7]|X[-1.75]|X[-1.5]|X[-1]|X[-0.7]|} \everyrow{\hline}
\hline
\rowfont\bfseries {Attribute} & \multicolumn{5}{|l|}{Value} \\
\tabucline[1.5pt]{}
BrowseName & \multicolumn{5}{|l|}{MTDataItemClassType} \\
IsAbstract & \multicolumn{5}{|l|}{True} \\
\tabucline[1.5pt]{}
\rowfont \bfseries References & NodeClass & BrowseName & DataType & Type\-Definition & {Modeling\-Rule} \\
\multicolumn{6}{|l|}{Subtype of BaseConditionClassType (See \cite{UAPart9} Documentation)} \\
HasSubtype & ObjectType & \multicolumn{2}{l}{MTConditionClassType} & \multicolumn{2}{|l|}{See section \ref{type:MTConditionClassType}} \\
HasSubtype & ObjectType & \multicolumn{2}{l}{MTEventClassType} & \multicolumn{2}{|l|}{See section \ref{type:MTEventClassType}} \\
HasSubtype & ObjectType & \multicolumn{2}{l}{MTSampleClassType} & \multicolumn{2}{|l|}{See section \ref{type:MTSampleClassType}} \\
\end{tabu}
\end{table} 


\FloatBarrier
\subsubsection{Defintion of \texttt{ MTMessageClassType}}
  \label{type:MTMessageClassType}

\FloatBarrier



Any text string of information to be transferred from a piece of equipment to a client software application.

\begin{table}[ht]
\centering 
  \caption{\texttt{MTMessageClassType} Definition}
  \label{table:MTMessageClassType}
\fontsize{9pt}{11pt}\selectfont
\tabulinesep=3pt
\begin{tabu} to 6in {|X[-1.35]|X[-0.7]|X[-1.75]|X[-1.5]|X[-1]|X[-0.7]|} \everyrow{\hline}
\hline
\rowfont\bfseries {Attribute} & \multicolumn{5}{|l|}{Value} \\
\tabucline[1.5pt]{}
BrowseName & \multicolumn{5}{|l|}{MTMessageClassType} \\
IsAbstract & \multicolumn{5}{|l|}{False} \\
\tabucline[1.5pt]{}
\rowfont \bfseries References & NodeClass & BrowseName & DataType & Type\-Definition & {Modeling\-Rule} \\
\multicolumn{6}{|l|}{Subtype of MTEventClassType (See Data Items Documentation)} \\
\end{tabu}
\end{table} 


\FloatBarrier
