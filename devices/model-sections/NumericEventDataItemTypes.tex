% Generated 2020-01-09 18:51:19 -0800
\subsection{Numeric Event Data Item Types} \label{model:NumericEventDataItemTypes}
\subsubsection{Defintion of \texttt{ MTNumericEventClassType}}
  \label{type:MTNumericEventClassType}

\FloatBarrier

The root type for all of the event types that have numeric \gls{CDATA}.

\begin{table}[ht]
\centering 
  \caption{\texttt{MTNumericEventClassType} Definition}
  \label{table:MTNumericEventClassType}
\fontsize{9pt}{11pt}\selectfont
\tabulinesep=3pt
\begin{tabu} to 6in {|X[-1.35]|X[-0.7]|X[-1.75]|X[-1.5]|X[-1]|X[-0.7]|} \everyrow{\hline}
\hline
\rowfont\bfseries {Attribute} & \multicolumn{5}{|l|}{Value} \\
\tabucline[1.5pt]{}
BrowseName & \multicolumn{5}{|l|}{MTNumericEventClassType} \\
IsAbstract & \multicolumn{5}{|l|}{True} \\
\tabucline[1.5pt]{}
\rowfont \bfseries References & NodeClass & BrowseName & DataType & Type\-Definition & {Modeling\-Rule} \\
\multicolumn{6}{|l|}{Subtype of MTEventClassType (See Data Items Documentation)} \\
HasSubtype & ObjectType & \multicolumn{2}{l}{AxisFeedrateOverrideClassType} & \multicolumn{2}{|l|}{See section \ref{type:AxisFeedrateOverrideClassType}} \\
HasSubtype & ObjectType & \multicolumn{2}{l}{BlockCountClassType} & \multicolumn{2}{|l|}{See section \ref{type:BlockCountClassType}} \\
HasSubtype & ObjectType & \multicolumn{2}{l}{HardnessClassType} & \multicolumn{2}{|l|}{See section \ref{type:HardnessClassType}} \\
HasSubtype & ObjectType & \multicolumn{2}{l}{LineNumberClassType} & \multicolumn{2}{|l|}{See section \ref{type:LineNumberClassType}} \\
HasSubtype & ObjectType & \multicolumn{2}{l}{PartCountClassType} & \multicolumn{2}{|l|}{See section \ref{type:PartCountClassType}} \\
HasSubtype & ObjectType & \multicolumn{2}{l}{PathFeedrateOverrideClassType} & \multicolumn{2}{|l|}{See section \ref{type:PathFeedrateOverrideClassType}} \\
HasSubtype & ObjectType & \multicolumn{2}{l}{RotaryVelocityOverrideClassType} & \multicolumn{2}{|l|}{See section \ref{type:RotaryVelocityOverrideClassType}} \\
\end{tabu}
\end{table} 


\FloatBarrier
\subsubsection{Defintion of \texttt{ AxisFeedrateOverrideClassType}}
  \label{type:AxisFeedrateOverrideClassType}

\FloatBarrier

The value of a signal or calculation issued to adjust the feedrate of an individual linear type axis.

The value provided for \mtmodel{AXIS_FEEDRATE_OVERRIDE} is expressed as a percentage of the designated feedrate for the axis.

When \mtmodel{AXIS_FEEDRATE_OVERRIDE} is applied, the resulting commanded feedrate for the axis is limited to the 
value of the original feedrate multiplied by the value of the \mtmodel{AXIS_FEEDRATE_OVERRIDE}.

There MAY be different subtypes of \mtmodel{AXIS_FEEDRATE_OVERRIDE}; each representing an override value for a 
designated subtype of feedrate depending on the state of operation of the axis. The subtypes of operation 
of an axis are currently defined as \mtmodel{PROGRAMMED}, \mtmodel{JOG}, and \mtmodel{RAPID}.

The value of a signal or calculation issued to adjust the feedrate of an individual linear type axis.

\begin{table}[ht]
\centering 
  \caption{\texttt{AxisFeedrateOverrideClassType} Definition}
  \label{table:AxisFeedrateOverrideClassType}
\fontsize{9pt}{11pt}\selectfont
\tabulinesep=3pt
\begin{tabu} to 6in {|X[-1.35]|X[-0.7]|X[-1.75]|X[-1.5]|X[-1]|X[-0.7]|} \everyrow{\hline}
\hline
\rowfont\bfseries {Attribute} & \multicolumn{5}{|l|}{Value} \\
\tabucline[1.5pt]{}
BrowseName & \multicolumn{5}{|l|}{AxisFeedrateOverrideClassType} \\
IsAbstract & \multicolumn{5}{|l|}{False} \\
\tabucline[1.5pt]{}
\rowfont \bfseries References & NodeClass & BrowseName & DataType & Type\-Definition & {Modeling\-Rule} \\
\multicolumn{6}{|l|}{Subtype of MTNumericEventClassType (See section \ref{type:MTNumericEventClassType})} \\
\end{tabu}
\end{table} 


\FloatBarrier
\subsubsection{Defintion of \texttt{ BlockCountClassType}}
  \label{type:BlockCountClassType}

\FloatBarrier

The total count of the number of blocks of program code that have been executed since execution started.

\mtmodel{BLOCK_COUNT} counts blocks of program code executed regardless of program structure 
(e.g., looping or branching within the program).

The starting value for \mtmodel{BLOCK_COUNT} MAY be established by an initial value provided in 
the Constraint element defined for the data item.

The total count of the number of blocks of program code that have been executed since execution started.

\begin{table}[ht]
\centering 
  \caption{\texttt{BlockCountClassType} Definition}
  \label{table:BlockCountClassType}
\fontsize{9pt}{11pt}\selectfont
\tabulinesep=3pt
\begin{tabu} to 6in {|X[-1.35]|X[-0.7]|X[-1.75]|X[-1.5]|X[-1]|X[-0.7]|} \everyrow{\hline}
\hline
\rowfont\bfseries {Attribute} & \multicolumn{5}{|l|}{Value} \\
\tabucline[1.5pt]{}
BrowseName & \multicolumn{5}{|l|}{BlockCountClassType} \\
IsAbstract & \multicolumn{5}{|l|}{False} \\
\tabucline[1.5pt]{}
\rowfont \bfseries References & NodeClass & BrowseName & DataType & Type\-Definition & {Modeling\-Rule} \\
\multicolumn{6}{|l|}{Subtype of MTNumericEventClassType (See section \ref{type:MTNumericEventClassType})} \\
\end{tabu}
\end{table} 


\FloatBarrier
\subsubsection{Defintion of \texttt{ HardnessClassType}}
  \label{type:HardnessClassType}

\FloatBarrier

The measurement of the hardness of a material. 

The measurement does not provide a unit.

A \gls{subType} MUST always be specified to designate the hardness scale associated with the measurement.

The measurement of the hardness of a material.

\begin{table}[ht]
\centering 
  \caption{\texttt{HardnessClassType} Definition}
  \label{table:HardnessClassType}
\fontsize{9pt}{11pt}\selectfont
\tabulinesep=3pt
\begin{tabu} to 6in {|X[-1.35]|X[-0.7]|X[-1.75]|X[-1.5]|X[-1]|X[-0.7]|} \everyrow{\hline}
\hline
\rowfont\bfseries {Attribute} & \multicolumn{5}{|l|}{Value} \\
\tabucline[1.5pt]{}
BrowseName & \multicolumn{5}{|l|}{HardnessClassType} \\
IsAbstract & \multicolumn{5}{|l|}{False} \\
\tabucline[1.5pt]{}
\rowfont \bfseries References & NodeClass & BrowseName & DataType & Type\-Definition & {Modeling\-Rule} \\
\multicolumn{6}{|l|}{Subtype of MTNumericEventClassType (See section \ref{type:MTNumericEventClassType})} \\
\end{tabu}
\end{table} 


\FloatBarrier
\subsubsection{Defintion of \texttt{ LineNumberClassType}}
  \label{type:LineNumberClassType}

\FloatBarrier

A reference to the position of a block of program code within a control program. 
The line number MAY represent either an absolute position starting with the first line of 
the program or an incremental position relative to the occurrence of the last \mtmodel{LINE_LABEL}.
\mtmodel{LINE_NUMBER} does not change subject to any looping or branching in a control program.

A \gls{subType} MUST be defined.

A reference to the position of a block of program code within a control program.

\begin{table}[ht]
\centering 
  \caption{\texttt{LineNumberClassType} Definition}
  \label{table:LineNumberClassType}
\fontsize{9pt}{11pt}\selectfont
\tabulinesep=3pt
\begin{tabu} to 6in {|X[-1.35]|X[-0.7]|X[-1.75]|X[-1.5]|X[-1]|X[-0.7]|} \everyrow{\hline}
\hline
\rowfont\bfseries {Attribute} & \multicolumn{5}{|l|}{Value} \\
\tabucline[1.5pt]{}
BrowseName & \multicolumn{5}{|l|}{LineNumberClassType} \\
IsAbstract & \multicolumn{5}{|l|}{False} \\
\tabucline[1.5pt]{}
\rowfont \bfseries References & NodeClass & BrowseName & DataType & Type\-Definition & {Modeling\-Rule} \\
\multicolumn{6}{|l|}{Subtype of MTNumericEventClassType (See section \ref{type:MTNumericEventClassType})} \\
\end{tabu}
\end{table} 


\FloatBarrier
\subsubsection{Defintion of \texttt{ PartCountClassType}}
  \label{type:PartCountClassType}

\FloatBarrier

The current count of parts produced as represented by the Controller. The valid data value MUST be an integer value.

The count of parts produced.

\begin{table}[ht]
\centering 
  \caption{\texttt{PartCountClassType} Definition}
  \label{table:PartCountClassType}
\fontsize{9pt}{11pt}\selectfont
\tabulinesep=3pt
\begin{tabu} to 6in {|X[-1.35]|X[-0.7]|X[-1.75]|X[-1.5]|X[-1]|X[-0.7]|} \everyrow{\hline}
\hline
\rowfont\bfseries {Attribute} & \multicolumn{5}{|l|}{Value} \\
\tabucline[1.5pt]{}
BrowseName & \multicolumn{5}{|l|}{PartCountClassType} \\
IsAbstract & \multicolumn{5}{|l|}{False} \\
\tabucline[1.5pt]{}
\rowfont \bfseries References & NodeClass & BrowseName & DataType & Type\-Definition & {Modeling\-Rule} \\
\multicolumn{6}{|l|}{Subtype of MTNumericEventClassType (See section \ref{type:MTNumericEventClassType})} \\
\end{tabu}
\end{table} 


\FloatBarrier
\subsubsection{Defintion of \texttt{ PathFeedrateOverrideClassType}}
  \label{type:PathFeedrateOverrideClassType}

\FloatBarrier



The value of a signal or calculation issued to adjust the feedrate for the axes associated with a path component that may represent a single axis or the coordinated movement of multiple axes.

\begin{table}[ht]
\centering 
  \caption{\texttt{PathFeedrateOverrideClassType} Definition}
  \label{table:PathFeedrateOverrideClassType}
\fontsize{9pt}{11pt}\selectfont
\tabulinesep=3pt
\begin{tabu} to 6in {|X[-1.35]|X[-0.7]|X[-1.75]|X[-1.5]|X[-1]|X[-0.7]|} \everyrow{\hline}
\hline
\rowfont\bfseries {Attribute} & \multicolumn{5}{|l|}{Value} \\
\tabucline[1.5pt]{}
BrowseName & \multicolumn{5}{|l|}{PathFeedrateOverrideClassType} \\
IsAbstract & \multicolumn{5}{|l|}{False} \\
\tabucline[1.5pt]{}
\rowfont \bfseries References & NodeClass & BrowseName & DataType & Type\-Definition & {Modeling\-Rule} \\
\multicolumn{6}{|l|}{Subtype of MTNumericEventClassType (See section \ref{type:MTNumericEventClassType})} \\
\end{tabu}
\end{table} 


\FloatBarrier
\subsubsection{Defintion of \texttt{ RotaryVelocityOverrideClassType}}
  \label{type:RotaryVelocityOverrideClassType}

\FloatBarrier

A command issued to adjust the programmed velocity for a Rotary type axis.

This command represents a percentage change to the velocity calculated by a logic or
motion program or set by a switch for a Rotary type axis.
\mtmodel{ROTARY_VELOCITY_OVERRIDE} is expressed as a percentage of the programmed \mtmodel{ROTARY_VELOCITY}.

The value of a command issued to adjust the programmed velocity for a rotary type axis.
 This command represents a percentage change to the velocity calculated by a logic or motion program or set by a switch for a rotary type axis.

\begin{table}[ht]
\centering 
  \caption{\texttt{RotaryVelocityOverrideClassType} Definition}
  \label{table:RotaryVelocityOverrideClassType}
\fontsize{9pt}{11pt}\selectfont
\tabulinesep=3pt
\begin{tabu} to 6in {|X[-1.35]|X[-0.7]|X[-1.75]|X[-1.5]|X[-1]|X[-0.7]|} \everyrow{\hline}
\hline
\rowfont\bfseries {Attribute} & \multicolumn{5}{|l|}{Value} \\
\tabucline[1.5pt]{}
BrowseName & \multicolumn{5}{|l|}{RotaryVelocityOverrideClassType} \\
IsAbstract & \multicolumn{5}{|l|}{False} \\
\tabucline[1.5pt]{}
\rowfont \bfseries References & NodeClass & BrowseName & DataType & Type\-Definition & {Modeling\-Rule} \\
\multicolumn{6}{|l|}{Subtype of MTNumericEventClassType (See section \ref{type:MTNumericEventClassType})} \\
\end{tabu}
\end{table} 


\FloatBarrier
