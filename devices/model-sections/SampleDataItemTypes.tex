% Generated 2020-01-10 13:32:54 -0800
\subsection{Sample Data Item Types} \label{model:SampleDataItemTypes}

The MTConnect Standard has the definitions of \glspl{Sample} in 
Section 8.1 of the MTConnect Standard Part 2 \cite{MTCPart2} and the definition of the 
values can be found in Section 5.3 of MTConnect Standard Part 3 \cite{MTCPart3}. 

The description of each type was copied from the MTConnect Standard,
but this not the definitive text for each type. For authoritative normative text, 
please refer to the sources \cite{MTCPart2} and \cite{MTCPart3}.

\subsubsection{Defintion of \texttt{ MTSampleClassType}}
  \label{type:MTSampleClassType}

\FloatBarrier

The base type class for all data items with a \gls{category} of \mtmodel{SAMPLE}.

An XML element that provides the information and data reported from a piece of equipment for those dataitem elements defined with a category attribute of sample category in the mtconnectdevices document. 

\begin{table}[ht]
\centering 
  \caption{\texttt{MTSampleClassType} Definition}
  \label{table:MTSampleClassType}
\fontsize{9pt}{11pt}\selectfont
\tabulinesep=3pt
\begin{tabu} to 6in {|X[-1.35]|X[-0.7]|X[-1.75]|X[-1.5]|X[-1]|X[-0.7]|} \everyrow{\hline}
\hline
\rowfont\bfseries {Attribute} & \multicolumn{5}{|l|}{Value} \\
\tabucline[1.5pt]{}
BrowseName & \multicolumn{5}{|l|}{MTSampleClassType} \\
IsAbstract & \multicolumn{5}{|l|}{True} \\
\tabucline[1.5pt]{}
\rowfont \bfseries References & NodeClass & BrowseName & DataType & Type\-Definition & {Modeling\-Rule} \\
\multicolumn{6}{|l|}{Subtype of MTDataItemClassType (See Data Item Types Documentation)} \\
HasSubtype & ObjectType & \multicolumn{2}{l}{MassClassType} & \multicolumn{2}{|l|}{See section \ref{type:MassClassType}} \\
HasSubtype & ObjectType & \multicolumn{2}{l}{PathFeedrateClassType} & \multicolumn{2}{|l|}{See section \ref{type:PathFeedrateClassType}} \\
HasSubtype & ObjectType & \multicolumn{2}{l}{PathPositionClassType} & \multicolumn{2}{|l|}{See section \ref{type:PathPositionClassType}} \\
HasSubtype & ObjectType & \multicolumn{2}{l}{PHClassType} & \multicolumn{2}{|l|}{See section \ref{type:PHClassType}} \\
HasSubtype & ObjectType & \multicolumn{2}{l}{PositionClassType} & \multicolumn{2}{|l|}{See section \ref{type:PositionClassType}} \\
HasSubtype & ObjectType & \multicolumn{2}{l}{PowerFactorClassType} & \multicolumn{2}{|l|}{See section \ref{type:PowerFactorClassType}} \\
HasSubtype & ObjectType & \multicolumn{2}{l}{PressureClassType} & \multicolumn{2}{|l|}{See section \ref{type:PressureClassType}} \\
HasSubtype & ObjectType & \multicolumn{2}{l}{ProcessTimerClassType} & \multicolumn{2}{|l|}{See section \ref{type:ProcessTimerClassType}} \\
HasSubtype & ObjectType & \multicolumn{2}{l}{ResistenceClassType} & \multicolumn{2}{|l|}{See section \ref{type:ResistenceClassType}} \\
HasSubtype & ObjectType & \multicolumn{2}{l}{RotaryVelocityClassType} & \multicolumn{2}{|l|}{See section \ref{type:RotaryVelocityClassType}} \\
HasSubtype & ObjectType & \multicolumn{2}{l}{SoundLevelClassType} & \multicolumn{2}{|l|}{See section \ref{type:SoundLevelClassType}} \\
HasSubtype & ObjectType & \multicolumn{2}{l}{StrainClassType} & \multicolumn{2}{|l|}{See section \ref{type:StrainClassType}} \\
HasSubtype & ObjectType & \multicolumn{2}{l}{TemperatureClassType} & \multicolumn{2}{|l|}{See section \ref{type:TemperatureClassType}} \\
HasSubtype & ObjectType & \multicolumn{2}{l}{TensionClassType} & \multicolumn{2}{|l|}{See section \ref{type:TensionClassType}} \\
HasSubtype & ObjectType & \multicolumn{2}{l}{TiltClassType} & \multicolumn{2}{|l|}{See section \ref{type:TiltClassType}} \\
HasSubtype & ObjectType & \multicolumn{2}{l}{TorqueClassType} & \multicolumn{2}{|l|}{See section \ref{type:TorqueClassType}} \\
HasSubtype & ObjectType & \multicolumn{2}{l}{VelocityClassType} & \multicolumn{2}{|l|}{See section \ref{type:VelocityClassType}} \\
HasSubtype & ObjectType & \multicolumn{2}{l}{ViscosityClassType} & \multicolumn{2}{|l|}{See section \ref{type:ViscosityClassType}} \\
HasSubtype & ObjectType & \multicolumn{2}{l}{VoltageClassType} & \multicolumn{2}{|l|}{See section \ref{type:VoltageClassType}} \\
HasSubtype & ObjectType & \multicolumn{2}{l}{VoltAmpereClassType} & \multicolumn{2}{|l|}{See section \ref{type:VoltAmpereClassType}} \\
HasSubtype & ObjectType & \multicolumn{2}{l}{VoltAmpereReactiveClassType} & \multicolumn{2}{|l|}{See section \ref{type:VoltAmpereReactiveClassType}} \\
HasSubtype & ObjectType & \multicolumn{2}{l}{WattageClassType} & \multicolumn{2}{|l|}{See section \ref{type:WattageClassType}} \\
\multicolumn{6}{|l|}{Continued...} \\
\end{tabu}
\end{table}
\begin{table}[ht]
\fontsize{9pt}{11pt}\selectfont
\tabulinesep=3pt
\begin{tabu} to 6in {|X[-1.35]|X[-0.7]|X[-1.75]|X[-1.5]|X[-1]|X[-0.7]|} \everyrow{\hline}
\hline
\rowfont \bfseries References & NodeClass & BrowseName & DataType & Type\-Definition & {Modeling\-Rule} \\
HasSubtype & ObjectType & \multicolumn{2}{l}{AccelerationClassType} & \multicolumn{2}{|l|}{See section \ref{type:AccelerationClassType}} \\
HasSubtype & ObjectType & \multicolumn{2}{l}{AccumulatedTimeClassType} & \multicolumn{2}{|l|}{See section \ref{type:AccumulatedTimeClassType}} \\
HasSubtype & ObjectType & \multicolumn{2}{l}{AmperageClassType} & \multicolumn{2}{|l|}{See section \ref{type:AmperageClassType}} \\
HasSubtype & ObjectType & \multicolumn{2}{l}{AngleClassType} & \multicolumn{2}{|l|}{See section \ref{type:AngleClassType}} \\
HasSubtype & ObjectType & \multicolumn{2}{l}{AngularAccelerationClassType} & \multicolumn{2}{|l|}{See section \ref{type:AngularAccelerationClassType}} \\
HasSubtype & ObjectType & \multicolumn{2}{l}{AngularVelocityClassType} & \multicolumn{2}{|l|}{See section \ref{type:AngularVelocityClassType}} \\
HasSubtype & ObjectType & \multicolumn{2}{l}{AxisFeedrateClassType} & \multicolumn{2}{|l|}{See section \ref{type:AxisFeedrateClassType}} \\
HasSubtype & ObjectType & \multicolumn{2}{l}{ClockTimeClassType} & \multicolumn{2}{|l|}{See section \ref{type:ClockTimeClassType}} \\
HasSubtype & ObjectType & \multicolumn{2}{l}{ConcentrationClassType} & \multicolumn{2}{|l|}{See section \ref{type:ConcentrationClassType}} \\
HasSubtype & ObjectType & \multicolumn{2}{l}{ConductivityClassType} & \multicolumn{2}{|l|}{See section \ref{type:ConductivityClassType}} \\
HasSubtype & ObjectType & \multicolumn{2}{l}{DisplacementClassType} & \multicolumn{2}{|l|}{See section \ref{type:DisplacementClassType}} \\
HasSubtype & ObjectType & \multicolumn{2}{l}{ElectricalEnergyClassType} & \multicolumn{2}{|l|}{See section \ref{type:ElectricalEnergyClassType}} \\
HasSubtype & ObjectType & \multicolumn{2}{l}{EquipmentTimerClassType} & \multicolumn{2}{|l|}{See section \ref{type:EquipmentTimerClassType}} \\
HasSubtype & ObjectType & \multicolumn{2}{l}{FillLevelClassType} & \multicolumn{2}{|l|}{See section \ref{type:FillLevelClassType}} \\
HasSubtype & ObjectType & \multicolumn{2}{l}{FlowClassType} & \multicolumn{2}{|l|}{See section \ref{type:FlowClassType}} \\
HasSubtype & ObjectType & \multicolumn{2}{l}{FrequencyClassType} & \multicolumn{2}{|l|}{See section \ref{type:FrequencyClassType}} \\
HasSubtype & ObjectType & \multicolumn{2}{l}{LengthClassType} & \multicolumn{2}{|l|}{See section \ref{type:LengthClassType}} \\
HasSubtype & ObjectType & \multicolumn{2}{l}{LinearForceClassType} & \multicolumn{2}{|l|}{See section \ref{type:LinearForceClassType}} \\
HasSubtype & ObjectType & \multicolumn{2}{l}{LoadClassType} & \multicolumn{2}{|l|}{See section \ref{type:LoadClassType}} \\
\end{tabu}
\end{table} 


\FloatBarrier
\subsubsection{Defintion of \texttt{ AccelerationClassType}}
  \label{type:AccelerationClassType}

\FloatBarrier

Rate of change of velocity. $\frac{MILLIMETER}{SECOND^{2}}$

The measurement of the rate of change of velocity.

\begin{table}[ht]
\centering 
  \caption{\texttt{AccelerationClassType} Definition}
  \label{table:AccelerationClassType}
\fontsize{9pt}{11pt}\selectfont
\tabulinesep=3pt
\begin{tabu} to 6in {|X[-1.35]|X[-0.7]|X[-1.75]|X[-1.5]|X[-1]|X[-0.7]|} \everyrow{\hline}
\hline
\rowfont\bfseries {Attribute} & \multicolumn{5}{|l|}{Value} \\
\tabucline[1.5pt]{}
BrowseName & \multicolumn{5}{|l|}{AccelerationClassType} \\
IsAbstract & \multicolumn{5}{|l|}{False} \\
\tabucline[1.5pt]{}
\rowfont \bfseries References & NodeClass & BrowseName & DataType & Type\-Definition & {Modeling\-Rule} \\
\multicolumn{6}{|l|}{Subtype of MTSampleClassType (See section \ref{type:MTSampleClassType})} \\
\end{tabu}
\end{table} 


\FloatBarrier
\subsubsection{Defintion of \texttt{ AccumulatedTimeClassType}}
  \label{type:AccumulatedTimeClassType}

\FloatBarrier

The measurement of accumulated time for an activity or event. $SECOND$

The measurement of accumulated time for an activity or event.

\begin{table}[ht]
\centering 
  \caption{\texttt{AccumulatedTimeClassType} Definition}
  \label{table:AccumulatedTimeClassType}
\fontsize{9pt}{11pt}\selectfont
\tabulinesep=3pt
\begin{tabu} to 6in {|X[-1.35]|X[-0.7]|X[-1.75]|X[-1.5]|X[-1]|X[-0.7]|} \everyrow{\hline}
\hline
\rowfont\bfseries {Attribute} & \multicolumn{5}{|l|}{Value} \\
\tabucline[1.5pt]{}
BrowseName & \multicolumn{5}{|l|}{AccumulatedTimeClassType} \\
IsAbstract & \multicolumn{5}{|l|}{False} \\
\tabucline[1.5pt]{}
\rowfont \bfseries References & NodeClass & BrowseName & DataType & Type\-Definition & {Modeling\-Rule} \\
\multicolumn{6}{|l|}{Subtype of MTSampleClassType (See section \ref{type:MTSampleClassType})} \\
\end{tabu}
\end{table} 


\FloatBarrier
\subsubsection{Defintion of \texttt{ AmperageClassType}}
  \label{type:AmperageClassType}

\FloatBarrier

The measurement of electrical current. $AMPERE$

The measurement of electrical current.

\begin{table}[ht]
\centering 
  \caption{\texttt{AmperageClassType} Definition}
  \label{table:AmperageClassType}
\fontsize{9pt}{11pt}\selectfont
\tabulinesep=3pt
\begin{tabu} to 6in {|X[-1.35]|X[-0.7]|X[-1.75]|X[-1.5]|X[-1]|X[-0.7]|} \everyrow{\hline}
\hline
\rowfont\bfseries {Attribute} & \multicolumn{5}{|l|}{Value} \\
\tabucline[1.5pt]{}
BrowseName & \multicolumn{5}{|l|}{AmperageClassType} \\
IsAbstract & \multicolumn{5}{|l|}{False} \\
\tabucline[1.5pt]{}
\rowfont \bfseries References & NodeClass & BrowseName & DataType & Type\-Definition & {Modeling\-Rule} \\
\multicolumn{6}{|l|}{Subtype of MTSampleClassType (See section \ref{type:MTSampleClassType})} \\
\end{tabu}
\end{table} 


\FloatBarrier
\subsubsection{Defintion of \texttt{ AngleClassType}}
  \label{type:AngleClassType}

\FloatBarrier

The measurement of angular position. $DEGREE$

The measurement of angular position.

\begin{table}[ht]
\centering 
  \caption{\texttt{AngleClassType} Definition}
  \label{table:AngleClassType}
\fontsize{9pt}{11pt}\selectfont
\tabulinesep=3pt
\begin{tabu} to 6in {|X[-1.35]|X[-0.7]|X[-1.75]|X[-1.5]|X[-1]|X[-0.7]|} \everyrow{\hline}
\hline
\rowfont\bfseries {Attribute} & \multicolumn{5}{|l|}{Value} \\
\tabucline[1.5pt]{}
BrowseName & \multicolumn{5}{|l|}{AngleClassType} \\
IsAbstract & \multicolumn{5}{|l|}{False} \\
\tabucline[1.5pt]{}
\rowfont \bfseries References & NodeClass & BrowseName & DataType & Type\-Definition & {Modeling\-Rule} \\
\multicolumn{6}{|l|}{Subtype of MTSampleClassType (See section \ref{type:MTSampleClassType})} \\
\end{tabu}
\end{table} 


\FloatBarrier
\subsubsection{Defintion of \texttt{ AngularAccelerationClassType}}
  \label{type:AngularAccelerationClassType}

\FloatBarrier

Rate of change of angular velocity.  $\frac{DEGREE}{SECOND^{2}}$

The measurement rate of change of angular velocity.

\begin{table}[ht]
\centering 
  \caption{\texttt{AngularAccelerationClassType} Definition}
  \label{table:AngularAccelerationClassType}
\fontsize{9pt}{11pt}\selectfont
\tabulinesep=3pt
\begin{tabu} to 6in {|X[-1.35]|X[-0.7]|X[-1.75]|X[-1.5]|X[-1]|X[-0.7]|} \everyrow{\hline}
\hline
\rowfont\bfseries {Attribute} & \multicolumn{5}{|l|}{Value} \\
\tabucline[1.5pt]{}
BrowseName & \multicolumn{5}{|l|}{AngularAccelerationClassType} \\
IsAbstract & \multicolumn{5}{|l|}{False} \\
\tabucline[1.5pt]{}
\rowfont \bfseries References & NodeClass & BrowseName & DataType & Type\-Definition & {Modeling\-Rule} \\
\multicolumn{6}{|l|}{Subtype of MTSampleClassType (See section \ref{type:MTSampleClassType})} \\
\end{tabu}
\end{table} 


\FloatBarrier
\subsubsection{Defintion of \texttt{ AngularVelocityClassType}}
  \label{type:AngularVelocityClassType}

\FloatBarrier

Rate of change of angular position. $\frac{DEGREE}{SECOND}$

The measurement of the rate of change of angular position.

\begin{table}[ht]
\centering 
  \caption{\texttt{AngularVelocityClassType} Definition}
  \label{table:AngularVelocityClassType}
\fontsize{9pt}{11pt}\selectfont
\tabulinesep=3pt
\begin{tabu} to 6in {|X[-1.35]|X[-0.7]|X[-1.75]|X[-1.5]|X[-1]|X[-0.7]|} \everyrow{\hline}
\hline
\rowfont\bfseries {Attribute} & \multicolumn{5}{|l|}{Value} \\
\tabucline[1.5pt]{}
BrowseName & \multicolumn{5}{|l|}{AngularVelocityClassType} \\
IsAbstract & \multicolumn{5}{|l|}{False} \\
\tabucline[1.5pt]{}
\rowfont \bfseries References & NodeClass & BrowseName & DataType & Type\-Definition & {Modeling\-Rule} \\
\multicolumn{6}{|l|}{Subtype of MTSampleClassType (See section \ref{type:MTSampleClassType})} \\
\end{tabu}
\end{table} 


\FloatBarrier
\subsubsection{Defintion of \texttt{ AxisFeedrateClassType}}
  \label{type:AxisFeedrateClassType}

\FloatBarrier

The feedrate of a linear axis. $\frac{MILLIMETER}{SECOND}$

The measurement of the feedrate of a linear axis.

\begin{table}[ht]
\centering 
  \caption{\texttt{AxisFeedrateClassType} Definition}
  \label{table:AxisFeedrateClassType}
\fontsize{9pt}{11pt}\selectfont
\tabulinesep=3pt
\begin{tabu} to 6in {|X[-1.35]|X[-0.7]|X[-1.75]|X[-1.5]|X[-1]|X[-0.7]|} \everyrow{\hline}
\hline
\rowfont\bfseries {Attribute} & \multicolumn{5}{|l|}{Value} \\
\tabucline[1.5pt]{}
BrowseName & \multicolumn{5}{|l|}{AxisFeedrateClassType} \\
IsAbstract & \multicolumn{5}{|l|}{False} \\
\tabucline[1.5pt]{}
\rowfont \bfseries References & NodeClass & BrowseName & DataType & Type\-Definition & {Modeling\-Rule} \\
\multicolumn{6}{|l|}{Subtype of MTSampleClassType (See section \ref{type:MTSampleClassType})} \\
\end{tabu}
\end{table} 


\FloatBarrier
\subsubsection{Defintion of \texttt{ ClockTimeClassType}}
  \label{type:ClockTimeClassType}

\FloatBarrier

The value provided by a timing device at a specific point in time. $TIMESTAMP$

The value provided by a timing device at a specific point in time.

\begin{table}[ht]
\centering 
  \caption{\texttt{ClockTimeClassType} Definition}
  \label{table:ClockTimeClassType}
\fontsize{9pt}{11pt}\selectfont
\tabulinesep=3pt
\begin{tabu} to 6in {|X[-1.35]|X[-0.7]|X[-1.75]|X[-1.5]|X[-1]|X[-0.7]|} \everyrow{\hline}
\hline
\rowfont\bfseries {Attribute} & \multicolumn{5}{|l|}{Value} \\
\tabucline[1.5pt]{}
BrowseName & \multicolumn{5}{|l|}{ClockTimeClassType} \\
IsAbstract & \multicolumn{5}{|l|}{False} \\
\tabucline[1.5pt]{}
\rowfont \bfseries References & NodeClass & BrowseName & DataType & Type\-Definition & {Modeling\-Rule} \\
\multicolumn{6}{|l|}{Subtype of MTSampleClassType (See section \ref{type:MTSampleClassType})} \\
\end{tabu}
\end{table} 


\FloatBarrier
\subsubsection{Defintion of \texttt{ ConcentrationClassType}}
  \label{type:ConcentrationClassType}

\FloatBarrier

Percentage of one component within a mixture of components. $PERCENT$

The measurement of the percentage of one component within a mixture of components

\begin{table}[ht]
\centering 
  \caption{\texttt{ConcentrationClassType} Definition}
  \label{table:ConcentrationClassType}
\fontsize{9pt}{11pt}\selectfont
\tabulinesep=3pt
\begin{tabu} to 6in {|X[-1.35]|X[-0.7]|X[-1.75]|X[-1.5]|X[-1]|X[-0.7]|} \everyrow{\hline}
\hline
\rowfont\bfseries {Attribute} & \multicolumn{5}{|l|}{Value} \\
\tabucline[1.5pt]{}
BrowseName & \multicolumn{5}{|l|}{ConcentrationClassType} \\
IsAbstract & \multicolumn{5}{|l|}{False} \\
\tabucline[1.5pt]{}
\rowfont \bfseries References & NodeClass & BrowseName & DataType & Type\-Definition & {Modeling\-Rule} \\
\multicolumn{6}{|l|}{Subtype of MTSampleClassType (See section \ref{type:MTSampleClassType})} \\
\end{tabu}
\end{table} 


\FloatBarrier
\subsubsection{Defintion of \texttt{ ConductivityClassType}}
  \label{type:ConductivityClassType}

\FloatBarrier

The ability of a material to conduct electricity. $\frac{SIEMENS}{METER}$

The measurement of the ability of a material to conduct electricity.

\begin{table}[ht]
\centering 
  \caption{\texttt{ConductivityClassType} Definition}
  \label{table:ConductivityClassType}
\fontsize{9pt}{11pt}\selectfont
\tabulinesep=3pt
\begin{tabu} to 6in {|X[-1.35]|X[-0.7]|X[-1.75]|X[-1.5]|X[-1]|X[-0.7]|} \everyrow{\hline}
\hline
\rowfont\bfseries {Attribute} & \multicolumn{5}{|l|}{Value} \\
\tabucline[1.5pt]{}
BrowseName & \multicolumn{5}{|l|}{ConductivityClassType} \\
IsAbstract & \multicolumn{5}{|l|}{False} \\
\tabucline[1.5pt]{}
\rowfont \bfseries References & NodeClass & BrowseName & DataType & Type\-Definition & {Modeling\-Rule} \\
\multicolumn{6}{|l|}{Subtype of MTSampleClassType (See section \ref{type:MTSampleClassType})} \\
\end{tabu}
\end{table} 


\FloatBarrier
\subsubsection{Defintion of \texttt{ DisplacementClassType}}
  \label{type:DisplacementClassType}

\FloatBarrier

The change in position of an object. $MILLIMETER$

The measurement of the change in position of an object.

\begin{table}[ht]
\centering 
  \caption{\texttt{DisplacementClassType} Definition}
  \label{table:DisplacementClassType}
\fontsize{9pt}{11pt}\selectfont
\tabulinesep=3pt
\begin{tabu} to 6in {|X[-1.35]|X[-0.7]|X[-1.75]|X[-1.5]|X[-1]|X[-0.7]|} \everyrow{\hline}
\hline
\rowfont\bfseries {Attribute} & \multicolumn{5}{|l|}{Value} \\
\tabucline[1.5pt]{}
BrowseName & \multicolumn{5}{|l|}{DisplacementClassType} \\
IsAbstract & \multicolumn{5}{|l|}{False} \\
\tabucline[1.5pt]{}
\rowfont \bfseries References & NodeClass & BrowseName & DataType & Type\-Definition & {Modeling\-Rule} \\
\multicolumn{6}{|l|}{Subtype of MTSampleClassType (See section \ref{type:MTSampleClassType})} \\
\end{tabu}
\end{table} 


\FloatBarrier
\subsubsection{Defintion of \texttt{ ElectricalEnergyClassType}}
  \label{type:ElectricalEnergyClassType}

\FloatBarrier

The measurement of electrical energy consumption by a component. $WATT \times SECOND$

The measurement of electrical energy consumption by a component.

\begin{table}[ht]
\centering 
  \caption{\texttt{ElectricalEnergyClassType} Definition}
  \label{table:ElectricalEnergyClassType}
\fontsize{9pt}{11pt}\selectfont
\tabulinesep=3pt
\begin{tabu} to 6in {|X[-1.35]|X[-0.7]|X[-1.75]|X[-1.5]|X[-1]|X[-0.7]|} \everyrow{\hline}
\hline
\rowfont\bfseries {Attribute} & \multicolumn{5}{|l|}{Value} \\
\tabucline[1.5pt]{}
BrowseName & \multicolumn{5}{|l|}{ElectricalEnergyClassType} \\
IsAbstract & \multicolumn{5}{|l|}{False} \\
\tabucline[1.5pt]{}
\rowfont \bfseries References & NodeClass & BrowseName & DataType & Type\-Definition & {Modeling\-Rule} \\
\multicolumn{6}{|l|}{Subtype of MTSampleClassType (See section \ref{type:MTSampleClassType})} \\
\end{tabu}
\end{table} 


\FloatBarrier
\subsubsection{Defintion of \texttt{ EquipmentTimerClassType}}
  \label{type:EquipmentTimerClassType}

\FloatBarrier

The measurement of the amount of time a \mtmodel{SECOND} piece of equipment or a sub-part of a 
piece of equipment has performed specific activities. 
Often used to determine when maintenance may be required for the equipment.
 
 
Multiple subTypes of \mtmodel{EQUIPMENT_TIMER} MAY be defined.
A subType MUST always be specified.

$SECOND$

The measurement of the amount of time a piece of equipment or a sub-part of a piece of equipment has performed specific activities.

\begin{table}[ht]
\centering 
  \caption{\texttt{EquipmentTimerClassType} Definition}
  \label{table:EquipmentTimerClassType}
\fontsize{9pt}{11pt}\selectfont
\tabulinesep=3pt
\begin{tabu} to 6in {|X[-1.35]|X[-0.7]|X[-1.75]|X[-1.5]|X[-1]|X[-0.7]|} \everyrow{\hline}
\hline
\rowfont\bfseries {Attribute} & \multicolumn{5}{|l|}{Value} \\
\tabucline[1.5pt]{}
BrowseName & \multicolumn{5}{|l|}{EquipmentTimerClassType} \\
IsAbstract & \multicolumn{5}{|l|}{False} \\
\tabucline[1.5pt]{}
\rowfont \bfseries References & NodeClass & BrowseName & DataType & Type\-Definition & {Modeling\-Rule} \\
\multicolumn{6}{|l|}{Subtype of MTSampleClassType (See section \ref{type:MTSampleClassType})} \\
\end{tabu}
\end{table} 


\FloatBarrier
\subsubsection{Defintion of \texttt{ FillLevelClassType}}
  \label{type:FillLevelClassType}

\FloatBarrier

The measurement of the amount of a substance remaining compared to the planned 
maximum amount of that substance. $PERCENT$

The measurement of the amount of a substance remaining compared to the planned maximum amount of that substance.

\begin{table}[ht]
\centering 
  \caption{\texttt{FillLevelClassType} Definition}
  \label{table:FillLevelClassType}
\fontsize{9pt}{11pt}\selectfont
\tabulinesep=3pt
\begin{tabu} to 6in {|X[-1.35]|X[-0.7]|X[-1.75]|X[-1.5]|X[-1]|X[-0.7]|} \everyrow{\hline}
\hline
\rowfont\bfseries {Attribute} & \multicolumn{5}{|l|}{Value} \\
\tabucline[1.5pt]{}
BrowseName & \multicolumn{5}{|l|}{FillLevelClassType} \\
IsAbstract & \multicolumn{5}{|l|}{False} \\
\tabucline[1.5pt]{}
\rowfont \bfseries References & NodeClass & BrowseName & DataType & Type\-Definition & {Modeling\-Rule} \\
\multicolumn{6}{|l|}{Subtype of MTSampleClassType (See section \ref{type:MTSampleClassType})} \\
\end{tabu}
\end{table} 


\FloatBarrier
\subsubsection{Defintion of \texttt{ FlowClassType}}
  \label{type:FlowClassType}

\FloatBarrier

The rate of flow of a fluid. $\frac{LITER}{SECOND}$

The measurement of the rate of flow of a fluid.

\begin{table}[ht]
\centering 
  \caption{\texttt{FlowClassType} Definition}
  \label{table:FlowClassType}
\fontsize{9pt}{11pt}\selectfont
\tabulinesep=3pt
\begin{tabu} to 6in {|X[-1.35]|X[-0.7]|X[-1.75]|X[-1.5]|X[-1]|X[-0.7]|} \everyrow{\hline}
\hline
\rowfont\bfseries {Attribute} & \multicolumn{5}{|l|}{Value} \\
\tabucline[1.5pt]{}
BrowseName & \multicolumn{5}{|l|}{FlowClassType} \\
IsAbstract & \multicolumn{5}{|l|}{False} \\
\tabucline[1.5pt]{}
\rowfont \bfseries References & NodeClass & BrowseName & DataType & Type\-Definition & {Modeling\-Rule} \\
\multicolumn{6}{|l|}{Subtype of MTSampleClassType (See section \ref{type:MTSampleClassType})} \\
\end{tabu}
\end{table} 


\FloatBarrier
\subsubsection{Defintion of \texttt{ FrequencyClassType}}
  \label{type:FrequencyClassType}

\FloatBarrier

The measurement of the number of occurrences of a repeating event per unit time. $HERTZ$

The measurement of the number of occurrences of a repeating event per unit time.

\begin{table}[ht]
\centering 
  \caption{\texttt{FrequencyClassType} Definition}
  \label{table:FrequencyClassType}
\fontsize{9pt}{11pt}\selectfont
\tabulinesep=3pt
\begin{tabu} to 6in {|X[-1.35]|X[-0.7]|X[-1.75]|X[-1.5]|X[-1]|X[-0.7]|} \everyrow{\hline}
\hline
\rowfont\bfseries {Attribute} & \multicolumn{5}{|l|}{Value} \\
\tabucline[1.5pt]{}
BrowseName & \multicolumn{5}{|l|}{FrequencyClassType} \\
IsAbstract & \multicolumn{5}{|l|}{False} \\
\tabucline[1.5pt]{}
\rowfont \bfseries References & NodeClass & BrowseName & DataType & Type\-Definition & {Modeling\-Rule} \\
\multicolumn{6}{|l|}{Subtype of MTSampleClassType (See section \ref{type:MTSampleClassType})} \\
\end{tabu}
\end{table} 


\FloatBarrier
\subsubsection{Defintion of \texttt{ LengthClassType}}
  \label{type:LengthClassType}

\FloatBarrier

The length of an object. $MILLIMETER$

The measurement of the length of an object.

\begin{table}[ht]
\centering 
  \caption{\texttt{LengthClassType} Definition}
  \label{table:LengthClassType}
\fontsize{9pt}{11pt}\selectfont
\tabulinesep=3pt
\begin{tabu} to 6in {|X[-1.35]|X[-0.7]|X[-1.75]|X[-1.5]|X[-1]|X[-0.7]|} \everyrow{\hline}
\hline
\rowfont\bfseries {Attribute} & \multicolumn{5}{|l|}{Value} \\
\tabucline[1.5pt]{}
BrowseName & \multicolumn{5}{|l|}{LengthClassType} \\
IsAbstract & \multicolumn{5}{|l|}{False} \\
\tabucline[1.5pt]{}
\rowfont \bfseries References & NodeClass & BrowseName & DataType & Type\-Definition & {Modeling\-Rule} \\
\multicolumn{6}{|l|}{Subtype of MTSampleClassType (See section \ref{type:MTSampleClassType})} \\
\end{tabu}
\end{table} 


\FloatBarrier
\subsubsection{Defintion of \texttt{ LinearForceClassType}}
  \label{type:LinearForceClassType}

\FloatBarrier

The measure of the push or pull introduced by an actuator or exerted on an object. $NEWTON$

The measurement of the push or pull introduced by an actuator or exerted on an object.

\begin{table}[ht]
\centering 
  \caption{\texttt{LinearForceClassType} Definition}
  \label{table:LinearForceClassType}
\fontsize{9pt}{11pt}\selectfont
\tabulinesep=3pt
\begin{tabu} to 6in {|X[-1.35]|X[-0.7]|X[-1.75]|X[-1.5]|X[-1]|X[-0.7]|} \everyrow{\hline}
\hline
\rowfont\bfseries {Attribute} & \multicolumn{5}{|l|}{Value} \\
\tabucline[1.5pt]{}
BrowseName & \multicolumn{5}{|l|}{LinearForceClassType} \\
IsAbstract & \multicolumn{5}{|l|}{False} \\
\tabucline[1.5pt]{}
\rowfont \bfseries References & NodeClass & BrowseName & DataType & Type\-Definition & {Modeling\-Rule} \\
\multicolumn{6}{|l|}{Subtype of MTSampleClassType (See section \ref{type:MTSampleClassType})} \\
\end{tabu}
\end{table} 


\FloatBarrier
\subsubsection{Defintion of \texttt{ LoadClassType}}
  \label{type:LoadClassType}

\FloatBarrier

The measurement of the actual versus the standard rating of a piece of equipment. $PERCENT$

The measurement of the actual versus the standard rating of a piece of equipment.

\begin{table}[ht]
\centering 
  \caption{\texttt{LoadClassType} Definition}
  \label{table:LoadClassType}
\fontsize{9pt}{11pt}\selectfont
\tabulinesep=3pt
\begin{tabu} to 6in {|X[-1.35]|X[-0.7]|X[-1.75]|X[-1.5]|X[-1]|X[-0.7]|} \everyrow{\hline}
\hline
\rowfont\bfseries {Attribute} & \multicolumn{5}{|l|}{Value} \\
\tabucline[1.5pt]{}
BrowseName & \multicolumn{5}{|l|}{LoadClassType} \\
IsAbstract & \multicolumn{5}{|l|}{False} \\
\tabucline[1.5pt]{}
\rowfont \bfseries References & NodeClass & BrowseName & DataType & Type\-Definition & {Modeling\-Rule} \\
\multicolumn{6}{|l|}{Subtype of MTSampleClassType (See section \ref{type:MTSampleClassType})} \\
\end{tabu}
\end{table} 


\FloatBarrier
\subsubsection{Defintion of \texttt{ MassClassType}}
  \label{type:MassClassType}

\FloatBarrier

The measurement of the mass of an object(s) or an amount of material. $KILOGRAM$

The measurement of the mass of an object(s) or an amount of material.

\begin{table}[ht]
\centering 
  \caption{\texttt{MassClassType} Definition}
  \label{table:MassClassType}
\fontsize{9pt}{11pt}\selectfont
\tabulinesep=3pt
\begin{tabu} to 6in {|X[-1.35]|X[-0.7]|X[-1.75]|X[-1.5]|X[-1]|X[-0.7]|} \everyrow{\hline}
\hline
\rowfont\bfseries {Attribute} & \multicolumn{5}{|l|}{Value} \\
\tabucline[1.5pt]{}
BrowseName & \multicolumn{5}{|l|}{MassClassType} \\
IsAbstract & \multicolumn{5}{|l|}{False} \\
\tabucline[1.5pt]{}
\rowfont \bfseries References & NodeClass & BrowseName & DataType & Type\-Definition & {Modeling\-Rule} \\
\multicolumn{6}{|l|}{Subtype of MTSampleClassType (See section \ref{type:MTSampleClassType})} \\
\end{tabu}
\end{table} 


\FloatBarrier
\subsubsection{Defintion of \texttt{ PathFeedrateClassType}}
  \label{type:PathFeedrateClassType}

\FloatBarrier

The feedrate for the axes, or a single axis, associated with a \mtmodel{Path} component 
a vector. $\frac{MILLIMETER}{SECOND}$

The measurement of the feedrate for the axes, or a single axis, associated with a path component-a vector.

\begin{table}[ht]
\centering 
  \caption{\texttt{PathFeedrateClassType} Definition}
  \label{table:PathFeedrateClassType}
\fontsize{9pt}{11pt}\selectfont
\tabulinesep=3pt
\begin{tabu} to 6in {|X[-1.35]|X[-0.7]|X[-1.75]|X[-1.5]|X[-1]|X[-0.7]|} \everyrow{\hline}
\hline
\rowfont\bfseries {Attribute} & \multicolumn{5}{|l|}{Value} \\
\tabucline[1.5pt]{}
BrowseName & \multicolumn{5}{|l|}{PathFeedrateClassType} \\
IsAbstract & \multicolumn{5}{|l|}{False} \\
\tabucline[1.5pt]{}
\rowfont \bfseries References & NodeClass & BrowseName & DataType & Type\-Definition & {Modeling\-Rule} \\
\multicolumn{6}{|l|}{Subtype of MTSampleClassType (See section \ref{type:MTSampleClassType})} \\
\end{tabu}
\end{table} 


\FloatBarrier
\subsubsection{Defintion of \texttt{ PathPositionClassType}}
  \label{type:PathPositionClassType}

\FloatBarrier

A measured or calculated position of a control point associated with a \mtmodel{Controller} element, 
or PATH element if provided, of a piece of equipment.

The control point MUST be reported as a set of space-delimited floating-point 
numbers representing a point in 3-D space. The position of the control point MUST 
be reported in units of \mtmodel{MILLIMETER} and listed in order of X, Y, and Z 
referenced to the coordinate system of the piece of equipment.

$MILLIMETER (\mathbb{R}^{3})$

A measured or calculated position of a control point associated with a controller element, or path element if provided, of a piece of equipment.

\begin{table}[ht]
\centering 
  \caption{\texttt{PathPositionClassType} Definition}
  \label{table:PathPositionClassType}
\fontsize{9pt}{11pt}\selectfont
\tabulinesep=3pt
\begin{tabu} to 6in {|X[-1.35]|X[-0.7]|X[-1.75]|X[-1.5]|X[-1]|X[-0.7]|} \everyrow{\hline}
\hline
\rowfont\bfseries {Attribute} & \multicolumn{5}{|l|}{Value} \\
\tabucline[1.5pt]{}
BrowseName & \multicolumn{5}{|l|}{PathPositionClassType} \\
IsAbstract & \multicolumn{5}{|l|}{False} \\
\tabucline[1.5pt]{}
\rowfont \bfseries References & NodeClass & BrowseName & DataType & Type\-Definition & {Modeling\-Rule} \\
\multicolumn{6}{|l|}{Subtype of MTSampleClassType (See section \ref{type:MTSampleClassType})} \\
\end{tabu}
\end{table} 


\FloatBarrier
\subsubsection{Defintion of \texttt{ PHClassType}}
  \label{type:PHClassType}

\FloatBarrier

The measure of the acidity or alkalinity. $PH$

A measure of the acidity or alkalinity of a solution.

\begin{table}[ht]
\centering 
  \caption{\texttt{PHClassType} Definition}
  \label{table:PHClassType}
\fontsize{9pt}{11pt}\selectfont
\tabulinesep=3pt
\begin{tabu} to 6in {|X[-1.35]|X[-0.7]|X[-1.75]|X[-1.5]|X[-1]|X[-0.7]|} \everyrow{\hline}
\hline
\rowfont\bfseries {Attribute} & \multicolumn{5}{|l|}{Value} \\
\tabucline[1.5pt]{}
BrowseName & \multicolumn{5}{|l|}{PHClassType} \\
IsAbstract & \multicolumn{5}{|l|}{False} \\
\tabucline[1.5pt]{}
\rowfont \bfseries References & NodeClass & BrowseName & DataType & Type\-Definition & {Modeling\-Rule} \\
\multicolumn{6}{|l|}{Subtype of MTSampleClassType (See section \ref{type:MTSampleClassType})} \\
\end{tabu}
\end{table} 


\FloatBarrier
\subsubsection{Defintion of \texttt{ PositionClassType}}
  \label{type:PositionClassType}

\FloatBarrier

A calculated or measured position related to a Component element.

\mtmodel{POSITION} SHOULD be further defined withacoordinateSytemattribute. 
If a coordinateSystem attribute is not specified, the position of the control point 
MUST be reported in \mtmodel{MACHINE} coordinates. 

$MILLIMETER$

A measured or calculated position of a component element as reported by a piece of equipment.

\begin{table}[ht]
\centering 
  \caption{\texttt{PositionClassType} Definition}
  \label{table:PositionClassType}
\fontsize{9pt}{11pt}\selectfont
\tabulinesep=3pt
\begin{tabu} to 6in {|X[-1.35]|X[-0.7]|X[-1.75]|X[-1.5]|X[-1]|X[-0.7]|} \everyrow{\hline}
\hline
\rowfont\bfseries {Attribute} & \multicolumn{5}{|l|}{Value} \\
\tabucline[1.5pt]{}
BrowseName & \multicolumn{5}{|l|}{PositionClassType} \\
IsAbstract & \multicolumn{5}{|l|}{False} \\
\tabucline[1.5pt]{}
\rowfont \bfseries References & NodeClass & BrowseName & DataType & Type\-Definition & {Modeling\-Rule} \\
\multicolumn{6}{|l|}{Subtype of MTSampleClassType (See section \ref{type:MTSampleClassType})} \\
\end{tabu}
\end{table} 


\FloatBarrier
\subsubsection{Defintion of \texttt{ PowerFactorClassType}}
  \label{type:PowerFactorClassType}

\FloatBarrier

The measurement of the ratio of real power flowing to a load to the apparent power in
that AC circuit. $PERCENT$

The measurement of the ratio of real power flowing to a load to the apparent power in that AC circuit.

\begin{table}[ht]
\centering 
  \caption{\texttt{PowerFactorClassType} Definition}
  \label{table:PowerFactorClassType}
\fontsize{9pt}{11pt}\selectfont
\tabulinesep=3pt
\begin{tabu} to 6in {|X[-1.35]|X[-0.7]|X[-1.75]|X[-1.5]|X[-1]|X[-0.7]|} \everyrow{\hline}
\hline
\rowfont\bfseries {Attribute} & \multicolumn{5}{|l|}{Value} \\
\tabucline[1.5pt]{}
BrowseName & \multicolumn{5}{|l|}{PowerFactorClassType} \\
IsAbstract & \multicolumn{5}{|l|}{False} \\
\tabucline[1.5pt]{}
\rowfont \bfseries References & NodeClass & BrowseName & DataType & Type\-Definition & {Modeling\-Rule} \\
\multicolumn{6}{|l|}{Subtype of MTSampleClassType (See section \ref{type:MTSampleClassType})} \\
\end{tabu}
\end{table} 


\FloatBarrier
\subsubsection{Defintion of \texttt{ PressureClassType}}
  \label{type:PressureClassType}

\FloatBarrier

The force per unit area exerted by a gas or liquid. $PASCAL$

The measurement of force per unit area exerted by a gas or liquid.

\begin{table}[ht]
\centering 
  \caption{\texttt{PressureClassType} Definition}
  \label{table:PressureClassType}
\fontsize{9pt}{11pt}\selectfont
\tabulinesep=3pt
\begin{tabu} to 6in {|X[-1.35]|X[-0.7]|X[-1.75]|X[-1.5]|X[-1]|X[-0.7]|} \everyrow{\hline}
\hline
\rowfont\bfseries {Attribute} & \multicolumn{5}{|l|}{Value} \\
\tabucline[1.5pt]{}
BrowseName & \multicolumn{5}{|l|}{PressureClassType} \\
IsAbstract & \multicolumn{5}{|l|}{False} \\
\tabucline[1.5pt]{}
\rowfont \bfseries References & NodeClass & BrowseName & DataType & Type\-Definition & {Modeling\-Rule} \\
\multicolumn{6}{|l|}{Subtype of MTSampleClassType (See section \ref{type:MTSampleClassType})} \\
\end{tabu}
\end{table} 


\FloatBarrier
\subsubsection{Defintion of \texttt{ ProcessTimerClassType}}
  \label{type:ProcessTimerClassType}

\FloatBarrier

The measurement of the amount of time a piece of equipment has performed different types 
of activities associated with the process being performed at that piece of equipment.
Multiple subtypes of \mtmodel{PROCESS_TIMER} may be defined.

Typically, \mtmodel{PROCESS_TIMER} SHOULD be modeled as a data item for the Device element, 
but MAY be modeled for either a Controller or Path Structural Element in the XML document.
A \gls{subType} MUST always be specified.

$SECOND$

The measurement of the amount of time a piece of equipment has performed different types of activities associated with the process being performed at that piece of equipment.

\begin{table}[ht]
\centering 
  \caption{\texttt{ProcessTimerClassType} Definition}
  \label{table:ProcessTimerClassType}
\fontsize{9pt}{11pt}\selectfont
\tabulinesep=3pt
\begin{tabu} to 6in {|X[-1.35]|X[-0.7]|X[-1.75]|X[-1.5]|X[-1]|X[-0.7]|} \everyrow{\hline}
\hline
\rowfont\bfseries {Attribute} & \multicolumn{5}{|l|}{Value} \\
\tabucline[1.5pt]{}
BrowseName & \multicolumn{5}{|l|}{ProcessTimerClassType} \\
IsAbstract & \multicolumn{5}{|l|}{False} \\
\tabucline[1.5pt]{}
\rowfont \bfseries References & NodeClass & BrowseName & DataType & Type\-Definition & {Modeling\-Rule} \\
\multicolumn{6}{|l|}{Subtype of MTSampleClassType (See section \ref{type:MTSampleClassType})} \\
\end{tabu}
\end{table} 


\FloatBarrier
\subsubsection{Defintion of \texttt{ ResistenceClassType}}
  \label{type:ResistenceClassType}

\FloatBarrier

The degree to which a substance opposes the passage of an electric current. $OHM$

\begin{table}[ht]
\centering 
  \caption{\texttt{ResistenceClassType} Definition}
  \label{table:ResistenceClassType}
\fontsize{9pt}{11pt}\selectfont
\tabulinesep=3pt
\begin{tabu} to 6in {|X[-1.35]|X[-0.7]|X[-1.75]|X[-1.5]|X[-1]|X[-0.7]|} \everyrow{\hline}
\hline
\rowfont\bfseries {Attribute} & \multicolumn{5}{|l|}{Value} \\
\tabucline[1.5pt]{}
BrowseName & \multicolumn{5}{|l|}{ResistenceClassType} \\
IsAbstract & \multicolumn{5}{|l|}{False} \\
\tabucline[1.5pt]{}
\rowfont \bfseries References & NodeClass & BrowseName & DataType & Type\-Definition & {Modeling\-Rule} \\
\multicolumn{6}{|l|}{Subtype of MTSampleClassType (See section \ref{type:MTSampleClassType})} \\
\end{tabu}
\end{table} 


\FloatBarrier
\subsubsection{Defintion of \texttt{ RotaryVelocityClassType}}
  \label{type:RotaryVelocityClassType}

\FloatBarrier

The rotational speed of a rotary axis. $\frac{REVOLUTION}{MINUTE}$

The measurement of the rotational speed of a rotary axis.

\begin{table}[ht]
\centering 
  \caption{\texttt{RotaryVelocityClassType} Definition}
  \label{table:RotaryVelocityClassType}
\fontsize{9pt}{11pt}\selectfont
\tabulinesep=3pt
\begin{tabu} to 6in {|X[-1.35]|X[-0.7]|X[-1.75]|X[-1.5]|X[-1]|X[-0.7]|} \everyrow{\hline}
\hline
\rowfont\bfseries {Attribute} & \multicolumn{5}{|l|}{Value} \\
\tabucline[1.5pt]{}
BrowseName & \multicolumn{5}{|l|}{RotaryVelocityClassType} \\
IsAbstract & \multicolumn{5}{|l|}{False} \\
\tabucline[1.5pt]{}
\rowfont \bfseries References & NodeClass & BrowseName & DataType & Type\-Definition & {Modeling\-Rule} \\
\multicolumn{6}{|l|}{Subtype of MTSampleClassType (See section \ref{type:MTSampleClassType})} \\
\end{tabu}
\end{table} 


\FloatBarrier
\subsubsection{Defintion of \texttt{ SoundLevelClassType}}
  \label{type:SoundLevelClassType}

\FloatBarrier

Measurement of a sound level or sound pressure level relative to atmospheric pressure. $DECIBEL$

The measurement of a sound level or sound pressure level relative to atmospheric pressure.

\begin{table}[ht]
\centering 
  \caption{\texttt{SoundLevelClassType} Definition}
  \label{table:SoundLevelClassType}
\fontsize{9pt}{11pt}\selectfont
\tabulinesep=3pt
\begin{tabu} to 6in {|X[-1.35]|X[-0.7]|X[-1.75]|X[-1.5]|X[-1]|X[-0.7]|} \everyrow{\hline}
\hline
\rowfont\bfseries {Attribute} & \multicolumn{5}{|l|}{Value} \\
\tabucline[1.5pt]{}
BrowseName & \multicolumn{5}{|l|}{SoundLevelClassType} \\
IsAbstract & \multicolumn{5}{|l|}{False} \\
\tabucline[1.5pt]{}
\rowfont \bfseries References & NodeClass & BrowseName & DataType & Type\-Definition & {Modeling\-Rule} \\
\multicolumn{6}{|l|}{Subtype of MTSampleClassType (See section \ref{type:MTSampleClassType})} \\
\end{tabu}
\end{table} 


\FloatBarrier
\subsubsection{Defintion of \texttt{ StrainClassType}}
  \label{type:StrainClassType}

\FloatBarrier

The amount of deformation per unit length of an object when a load is applied. $PERCENT$

The measurement of the amount of deformation per unit length of an object when a load is applied.

\begin{table}[ht]
\centering 
  \caption{\texttt{StrainClassType} Definition}
  \label{table:StrainClassType}
\fontsize{9pt}{11pt}\selectfont
\tabulinesep=3pt
\begin{tabu} to 6in {|X[-1.35]|X[-0.7]|X[-1.75]|X[-1.5]|X[-1]|X[-0.7]|} \everyrow{\hline}
\hline
\rowfont\bfseries {Attribute} & \multicolumn{5}{|l|}{Value} \\
\tabucline[1.5pt]{}
BrowseName & \multicolumn{5}{|l|}{StrainClassType} \\
IsAbstract & \multicolumn{5}{|l|}{False} \\
\tabucline[1.5pt]{}
\rowfont \bfseries References & NodeClass & BrowseName & DataType & Type\-Definition & {Modeling\-Rule} \\
\multicolumn{6}{|l|}{Subtype of MTSampleClassType (See section \ref{type:MTSampleClassType})} \\
\end{tabu}
\end{table} 


\FloatBarrier
\subsubsection{Defintion of \texttt{ TemperatureClassType}}
  \label{type:TemperatureClassType}

\FloatBarrier

The measurement of temperature. $CELSIUS$

The measurement of temperature.

\begin{table}[ht]
\centering 
  \caption{\texttt{TemperatureClassType} Definition}
  \label{table:TemperatureClassType}
\fontsize{9pt}{11pt}\selectfont
\tabulinesep=3pt
\begin{tabu} to 6in {|X[-1.35]|X[-0.7]|X[-1.75]|X[-1.5]|X[-1]|X[-0.7]|} \everyrow{\hline}
\hline
\rowfont\bfseries {Attribute} & \multicolumn{5}{|l|}{Value} \\
\tabucline[1.5pt]{}
BrowseName & \multicolumn{5}{|l|}{TemperatureClassType} \\
IsAbstract & \multicolumn{5}{|l|}{False} \\
\tabucline[1.5pt]{}
\rowfont \bfseries References & NodeClass & BrowseName & DataType & Type\-Definition & {Modeling\-Rule} \\
\multicolumn{6}{|l|}{Subtype of MTSampleClassType (See section \ref{type:MTSampleClassType})} \\
\end{tabu}
\end{table} 


\FloatBarrier
\subsubsection{Defintion of \texttt{ TensionClassType}}
  \label{type:TensionClassType}

\FloatBarrier

A measurement of a force that stretches or elongates an object. $NEWTON$

The measurement of a force that stretches or elongates an object.

\begin{table}[ht]
\centering 
  \caption{\texttt{TensionClassType} Definition}
  \label{table:TensionClassType}
\fontsize{9pt}{11pt}\selectfont
\tabulinesep=3pt
\begin{tabu} to 6in {|X[-1.35]|X[-0.7]|X[-1.75]|X[-1.5]|X[-1]|X[-0.7]|} \everyrow{\hline}
\hline
\rowfont\bfseries {Attribute} & \multicolumn{5}{|l|}{Value} \\
\tabucline[1.5pt]{}
BrowseName & \multicolumn{5}{|l|}{TensionClassType} \\
IsAbstract & \multicolumn{5}{|l|}{False} \\
\tabucline[1.5pt]{}
\rowfont \bfseries References & NodeClass & BrowseName & DataType & Type\-Definition & {Modeling\-Rule} \\
\multicolumn{6}{|l|}{Subtype of MTSampleClassType (See section \ref{type:MTSampleClassType})} \\
\end{tabu}
\end{table} 


\FloatBarrier
\subsubsection{Defintion of \texttt{ TiltClassType}}
  \label{type:TiltClassType}

\FloatBarrier

A measurement of angular displacement. $MICRO \cdot RADIAN$

The measurement of angular displacement. 

\begin{table}[ht]
\centering 
  \caption{\texttt{TiltClassType} Definition}
  \label{table:TiltClassType}
\fontsize{9pt}{11pt}\selectfont
\tabulinesep=3pt
\begin{tabu} to 6in {|X[-1.35]|X[-0.7]|X[-1.75]|X[-1.5]|X[-1]|X[-0.7]|} \everyrow{\hline}
\hline
\rowfont\bfseries {Attribute} & \multicolumn{5}{|l|}{Value} \\
\tabucline[1.5pt]{}
BrowseName & \multicolumn{5}{|l|}{TiltClassType} \\
IsAbstract & \multicolumn{5}{|l|}{False} \\
\tabucline[1.5pt]{}
\rowfont \bfseries References & NodeClass & BrowseName & DataType & Type\-Definition & {Modeling\-Rule} \\
\multicolumn{6}{|l|}{Subtype of MTSampleClassType (See section \ref{type:MTSampleClassType})} \\
\end{tabu}
\end{table} 


\FloatBarrier
\subsubsection{Defintion of \texttt{ TorqueClassType}}
  \label{type:TorqueClassType}

\FloatBarrier

The turning force exerted on an object or by an object. $NEWTON \times METER$

The measurement of the turning force exerted on an object or by an object.

\begin{table}[ht]
\centering 
  \caption{\texttt{TorqueClassType} Definition}
  \label{table:TorqueClassType}
\fontsize{9pt}{11pt}\selectfont
\tabulinesep=3pt
\begin{tabu} to 6in {|X[-1.35]|X[-0.7]|X[-1.75]|X[-1.5]|X[-1]|X[-0.7]|} \everyrow{\hline}
\hline
\rowfont\bfseries {Attribute} & \multicolumn{5}{|l|}{Value} \\
\tabucline[1.5pt]{}
BrowseName & \multicolumn{5}{|l|}{TorqueClassType} \\
IsAbstract & \multicolumn{5}{|l|}{False} \\
\tabucline[1.5pt]{}
\rowfont \bfseries References & NodeClass & BrowseName & DataType & Type\-Definition & {Modeling\-Rule} \\
\multicolumn{6}{|l|}{Subtype of MTSampleClassType (See section \ref{type:MTSampleClassType})} \\
\end{tabu}
\end{table} 


\FloatBarrier
\subsubsection{Defintion of \texttt{ VelocityClassType}}
  \label{type:VelocityClassType}

\FloatBarrier

The rate of change of position. $\frac{MILLIMETER}{SECOND}$

The measurement of the rate of change of position of a component.

\begin{table}[ht]
\centering 
  \caption{\texttt{VelocityClassType} Definition}
  \label{table:VelocityClassType}
\fontsize{9pt}{11pt}\selectfont
\tabulinesep=3pt
\begin{tabu} to 6in {|X[-1.35]|X[-0.7]|X[-1.75]|X[-1.5]|X[-1]|X[-0.7]|} \everyrow{\hline}
\hline
\rowfont\bfseries {Attribute} & \multicolumn{5}{|l|}{Value} \\
\tabucline[1.5pt]{}
BrowseName & \multicolumn{5}{|l|}{VelocityClassType} \\
IsAbstract & \multicolumn{5}{|l|}{False} \\
\tabucline[1.5pt]{}
\rowfont \bfseries References & NodeClass & BrowseName & DataType & Type\-Definition & {Modeling\-Rule} \\
\multicolumn{6}{|l|}{Subtype of MTSampleClassType (See section \ref{type:MTSampleClassType})} \\
\end{tabu}
\end{table} 


\FloatBarrier
\subsubsection{Defintion of \texttt{ ViscosityClassType}}
  \label{type:ViscosityClassType}

\FloatBarrier

A measurement of a fluid’s resistance to flow. $PASCAL \times SECOND$.

The measurement of a fluids resistance to flow.

\begin{table}[ht]
\centering 
  \caption{\texttt{ViscosityClassType} Definition}
  \label{table:ViscosityClassType}
\fontsize{9pt}{11pt}\selectfont
\tabulinesep=3pt
\begin{tabu} to 6in {|X[-1.35]|X[-0.7]|X[-1.75]|X[-1.5]|X[-1]|X[-0.7]|} \everyrow{\hline}
\hline
\rowfont\bfseries {Attribute} & \multicolumn{5}{|l|}{Value} \\
\tabucline[1.5pt]{}
BrowseName & \multicolumn{5}{|l|}{ViscosityClassType} \\
IsAbstract & \multicolumn{5}{|l|}{False} \\
\tabucline[1.5pt]{}
\rowfont \bfseries References & NodeClass & BrowseName & DataType & Type\-Definition & {Modeling\-Rule} \\
\multicolumn{6}{|l|}{Subtype of MTSampleClassType (See section \ref{type:MTSampleClassType})} \\
\end{tabu}
\end{table} 


\FloatBarrier
\subsubsection{Defintion of \texttt{ VoltageClassType}}
  \label{type:VoltageClassType}

\FloatBarrier

The measurement of electrical potential between two points. $VOLT$

The measurement of electrical potential between two points.

\begin{table}[ht]
\centering 
  \caption{\texttt{VoltageClassType} Definition}
  \label{table:VoltageClassType}
\fontsize{9pt}{11pt}\selectfont
\tabulinesep=3pt
\begin{tabu} to 6in {|X[-1.35]|X[-0.7]|X[-1.75]|X[-1.5]|X[-1]|X[-0.7]|} \everyrow{\hline}
\hline
\rowfont\bfseries {Attribute} & \multicolumn{5}{|l|}{Value} \\
\tabucline[1.5pt]{}
BrowseName & \multicolumn{5}{|l|}{VoltageClassType} \\
IsAbstract & \multicolumn{5}{|l|}{False} \\
\tabucline[1.5pt]{}
\rowfont \bfseries References & NodeClass & BrowseName & DataType & Type\-Definition & {Modeling\-Rule} \\
\multicolumn{6}{|l|}{Subtype of MTSampleClassType (See section \ref{type:MTSampleClassType})} \\
\end{tabu}
\end{table} 


\FloatBarrier
\subsubsection{Defintion of \texttt{ VoltAmpereClassType}}
  \label{type:VoltAmpereClassType}

\FloatBarrier

The measure of the apparent power in an electrical circuit, equal to the product of 
root-mean-square (RMS) voltage and RMS current (commonly referred to as VA). $VOLT \times AMPERE$

The measurement of the apparent power in an electrical circuit, equal to the product of root-mean-square (RMS) voltage and RMS current (commonly referred to as VA).

\begin{table}[ht]
\centering 
  \caption{\texttt{VoltAmpereClassType} Definition}
  \label{table:VoltAmpereClassType}
\fontsize{9pt}{11pt}\selectfont
\tabulinesep=3pt
\begin{tabu} to 6in {|X[-1.35]|X[-0.7]|X[-1.75]|X[-1.5]|X[-1]|X[-0.7]|} \everyrow{\hline}
\hline
\rowfont\bfseries {Attribute} & \multicolumn{5}{|l|}{Value} \\
\tabucline[1.5pt]{}
BrowseName & \multicolumn{5}{|l|}{VoltAmpereClassType} \\
IsAbstract & \multicolumn{5}{|l|}{False} \\
\tabucline[1.5pt]{}
\rowfont \bfseries References & NodeClass & BrowseName & DataType & Type\-Definition & {Modeling\-Rule} \\
\multicolumn{6}{|l|}{Subtype of MTSampleClassType (See section \ref{type:MTSampleClassType})} \\
\end{tabu}
\end{table} 


\FloatBarrier
\subsubsection{Defintion of \texttt{ VoltAmpereReactiveClassType}}
  \label{type:VoltAmpereReactiveClassType}

\FloatBarrier

The measurement of reactive power in an AC electrical circuit (commonly referred to as VAR). 
$VOLT \times AMPERE (Reactive)$

The measurement of reactive power in an AC electrical circuit (commonly referred to as VAR).

\begin{table}[ht]
\centering 
  \caption{\texttt{VoltAmpereReactiveClassType} Definition}
  \label{table:VoltAmpereReactiveClassType}
\fontsize{9pt}{11pt}\selectfont
\tabulinesep=3pt
\begin{tabu} to 6in {|X[-1.35]|X[-0.7]|X[-1.75]|X[-1.5]|X[-1]|X[-0.7]|} \everyrow{\hline}
\hline
\rowfont\bfseries {Attribute} & \multicolumn{5}{|l|}{Value} \\
\tabucline[1.5pt]{}
BrowseName & \multicolumn{5}{|l|}{VoltAmpereReactiveClassType} \\
IsAbstract & \multicolumn{5}{|l|}{False} \\
\tabucline[1.5pt]{}
\rowfont \bfseries References & NodeClass & BrowseName & DataType & Type\-Definition & {Modeling\-Rule} \\
\multicolumn{6}{|l|}{Subtype of MTSampleClassType (See section \ref{type:MTSampleClassType})} \\
\end{tabu}
\end{table} 


\FloatBarrier
\subsubsection{Defintion of \texttt{ WattageClassType}}
  \label{type:WattageClassType}

\FloatBarrier

The measurement of power flowing through or dissipated by an electrical circuit or 
piece of equipment. $WATT$

The measurement of power flowing through or dissipated by an electrical circuit or piece of equipment.

\begin{table}[ht]
\centering 
  \caption{\texttt{WattageClassType} Definition}
  \label{table:WattageClassType}
\fontsize{9pt}{11pt}\selectfont
\tabulinesep=3pt
\begin{tabu} to 6in {|X[-1.35]|X[-0.7]|X[-1.75]|X[-1.5]|X[-1]|X[-0.7]|} \everyrow{\hline}
\hline
\rowfont\bfseries {Attribute} & \multicolumn{5}{|l|}{Value} \\
\tabucline[1.5pt]{}
BrowseName & \multicolumn{5}{|l|}{WattageClassType} \\
IsAbstract & \multicolumn{5}{|l|}{False} \\
\tabucline[1.5pt]{}
\rowfont \bfseries References & NodeClass & BrowseName & DataType & Type\-Definition & {Modeling\-Rule} \\
\multicolumn{6}{|l|}{Subtype of MTSampleClassType (See section \ref{type:MTSampleClassType})} \\
\end{tabu}
\end{table} 


\FloatBarrier
