% Generated 2020-01-09 17:13:24 -0800
\subsection{String Event Data Item Types} \label{model:StringEventDataItemTypes}
\subsubsection{Defintion of \texttt{ MTStringEventClassType}}
  \label{type:MTStringEventClassType}

\FloatBarrier

The base UA \gls{Type} for all \glspl{MTDataItem} that have a non-specific text representation.

\begin{table}[ht]
\centering 
  \caption{\texttt{MTStringEventClassType} Definition}
  \label{table:MTStringEventClassType}
\fontsize{9pt}{11pt}\selectfont
\tabulinesep=3pt
\begin{tabu} to 6in {|X[-1.35]|X[-0.7]|X[-1.75]|X[-1.5]|X[-1]|X[-0.7]|} \everyrow{\hline}
\hline
\rowfont\bfseries {Attribute} & \multicolumn{5}{|l|}{Value} \\
\tabucline[1.5pt]{}
BrowseName & \multicolumn{5}{|l|}{MTStringEventClassType} \\
IsAbstract & \multicolumn{5}{|l|}{True} \\
\tabucline[1.5pt]{}
\rowfont \bfseries References & NodeClass & BrowseName & DataType & Type\-Definition & {Modeling\-Rule} \\
\multicolumn{6}{|l|}{Subtype of MTEventClassType (See Data Items Documentation)} \\
HasSubtype & ObjectType & \multicolumn{2}{l}{BlockClassType} & \multicolumn{2}{|l|}{See section \ref{type:BlockClassType}} \\
HasSubtype & ObjectType & \multicolumn{2}{l}{CoupledAxesClassType} & \multicolumn{2}{|l|}{See section \ref{type:CoupledAxesClassType}} \\
HasSubtype & ObjectType & \multicolumn{2}{l}{LineClassType} & \multicolumn{2}{|l|}{See section \ref{type:LineClassType}} \\
HasSubtype & ObjectType & \multicolumn{2}{l}{LineLabelClassType} & \multicolumn{2}{|l|}{See section \ref{type:LineLabelClassType}} \\
HasSubtype & ObjectType & \multicolumn{2}{l}{MaterialClassType} & \multicolumn{2}{|l|}{See section \ref{type:MaterialClassType}} \\
HasSubtype & ObjectType & \multicolumn{2}{l}{MessageClassType} & \multicolumn{2}{|l|}{See section \ref{type:MessageClassType}} \\
HasSubtype & ObjectType & \multicolumn{2}{l}{OperatorIdClassType} & \multicolumn{2}{|l|}{See section \ref{type:OperatorIdClassType}} \\
HasSubtype & ObjectType & \multicolumn{2}{l}{PalletIdClassType} & \multicolumn{2}{|l|}{See section \ref{type:PalletIdClassType}} \\
HasSubtype & ObjectType & \multicolumn{2}{l}{PartIdClassType} & \multicolumn{2}{|l|}{See section \ref{type:PartIdClassType}} \\
HasSubtype & ObjectType & \multicolumn{2}{l}{PartNumberClassType} & \multicolumn{2}{|l|}{See section \ref{type:PartNumberClassType}} \\
HasSubtype & ObjectType & \multicolumn{2}{l}{ProgramClassType} & \multicolumn{2}{|l|}{See section \ref{type:ProgramClassType}} \\
HasSubtype & ObjectType & \multicolumn{2}{l}{ProgramCommentClassType} & \multicolumn{2}{|l|}{See section \ref{type:ProgramCommentClassType}} \\
HasSubtype & ObjectType & \multicolumn{2}{l}{ProgramEditNameClassType} & \multicolumn{2}{|l|}{See section \ref{type:ProgramEditNameClassType}} \\
HasSubtype & ObjectType & \multicolumn{2}{l}{ProgramHeaderClassType} & \multicolumn{2}{|l|}{See section \ref{type:ProgramHeaderClassType}} \\
HasSubtype & ObjectType & \multicolumn{2}{l}{SerialNumberClassType} & \multicolumn{2}{|l|}{See section \ref{type:SerialNumberClassType}} \\
HasSubtype & ObjectType & \multicolumn{2}{l}{ToolAssetIdClassType} & \multicolumn{2}{|l|}{See section \ref{type:ToolAssetIdClassType}} \\
HasSubtype & ObjectType & \multicolumn{2}{l}{ToolNumberClassType} & \multicolumn{2}{|l|}{See section \ref{type:ToolNumberClassType}} \\
HasSubtype & ObjectType & \multicolumn{2}{l}{ToolOffsetClassType} & \multicolumn{2}{|l|}{See section \ref{type:ToolOffsetClassType}} \\
HasSubtype & ObjectType & \multicolumn{2}{l}{UserClassType} & \multicolumn{2}{|l|}{See section \ref{type:UserClassType}} \\
HasSubtype & ObjectType & \multicolumn{2}{l}{WireClassType} & \multicolumn{2}{|l|}{See section \ref{type:WireClassType}} \\
HasSubtype & ObjectType & \multicolumn{2}{l}{WorkholdingClassType} & \multicolumn{2}{|l|}{See section \ref{type:WorkholdingClassType}} \\
HasSubtype & ObjectType & \multicolumn{2}{l}{WorkOffsetClassType} & \multicolumn{2}{|l|}{See section \ref{type:WorkOffsetClassType}} \\
\multicolumn{6}{|l|}{Continued...} \\
\end{tabu}
\end{table}
\begin{table}[ht]
\fontsize{9pt}{11pt}\selectfont
\tabulinesep=3pt
\begin{tabu} to 6in {|X[-1.35]|X[-0.7]|X[-1.75]|X[-1.5]|X[-1]|X[-0.7]|} \everyrow{\hline}
\hline
\rowfont \bfseries References & NodeClass & BrowseName & DataType & Type\-Definition & {Modeling\-Rule} \\
HasSubtype & ObjectType & \multicolumn{2}{l}{AssetChangedClassType} & \multicolumn{2}{|l|}{See section \ref{type:AssetChangedClassType}} \\
HasSubtype & ObjectType & \multicolumn{2}{l}{AssetRemovedClassType} & \multicolumn{2}{|l|}{See section \ref{type:AssetRemovedClassType}} \\
\end{tabu}
\end{table} 


\FloatBarrier
\subsubsection{Defintion of \texttt{ AssetChangedClassType}}
  \label{type:AssetChangedClassType}

\FloatBarrier
\begin{table}[ht]
\centering 
  \caption{\texttt{AssetChangedClassType} Definition}
  \label{table:AssetChangedClassType}
\fontsize{9pt}{11pt}\selectfont
\tabulinesep=3pt
\begin{tabu} to 6in {|X[-1.35]|X[-0.7]|X[-1.75]|X[-1.5]|X[-1]|X[-0.7]|} \everyrow{\hline}
\hline
\rowfont\bfseries {Attribute} & \multicolumn{5}{|l|}{Value} \\
\tabucline[1.5pt]{}
BrowseName & \multicolumn{5}{|l|}{AssetChangedClassType} \\
IsAbstract & \multicolumn{5}{|l|}{False} \\
\tabucline[1.5pt]{}
\rowfont \bfseries References & NodeClass & BrowseName & DataType & Type\-Definition & {Modeling\-Rule} \\
\multicolumn{6}{|l|}{Subtype of MTStringEventClassType (See section \ref{type:MTStringEventClassType})} \\
\end{tabu}
\end{table} 


\FloatBarrier
\subsubsection{Defintion of \texttt{ AssetRemovedClassType}}
  \label{type:AssetRemovedClassType}

\FloatBarrier
\begin{table}[ht]
\centering 
  \caption{\texttt{AssetRemovedClassType} Definition}
  \label{table:AssetRemovedClassType}
\fontsize{9pt}{11pt}\selectfont
\tabulinesep=3pt
\begin{tabu} to 6in {|X[-1.35]|X[-0.7]|X[-1.75]|X[-1.5]|X[-1]|X[-0.7]|} \everyrow{\hline}
\hline
\rowfont\bfseries {Attribute} & \multicolumn{5}{|l|}{Value} \\
\tabucline[1.5pt]{}
BrowseName & \multicolumn{5}{|l|}{AssetRemovedClassType} \\
IsAbstract & \multicolumn{5}{|l|}{False} \\
\tabucline[1.5pt]{}
\rowfont \bfseries References & NodeClass & BrowseName & DataType & Type\-Definition & {Modeling\-Rule} \\
\multicolumn{6}{|l|}{Subtype of MTStringEventClassType (See section \ref{type:MTStringEventClassType})} \\
\end{tabu}
\end{table} 


\FloatBarrier
\subsubsection{Defintion of \texttt{ BlockClassType}}
  \label{type:BlockClassType}

\FloatBarrier

The line of code or command being executed by a \mtmodel{Controller} \mtterm{Structural Element}.

The value reported for \mtmodel{Block} MUST include the entire expression for a line of program code, including all parameters.

\begin{table}[ht]
\centering 
  \caption{\texttt{BlockClassType} Definition}
  \label{table:BlockClassType}
\fontsize{9pt}{11pt}\selectfont
\tabulinesep=3pt
\begin{tabu} to 6in {|X[-1.35]|X[-0.7]|X[-1.75]|X[-1.5]|X[-1]|X[-0.7]|} \everyrow{\hline}
\hline
\rowfont\bfseries {Attribute} & \multicolumn{5}{|l|}{Value} \\
\tabucline[1.5pt]{}
BrowseName & \multicolumn{5}{|l|}{BlockClassType} \\
IsAbstract & \multicolumn{5}{|l|}{False} \\
\tabucline[1.5pt]{}
\rowfont \bfseries References & NodeClass & BrowseName & DataType & Type\-Definition & {Modeling\-Rule} \\
\multicolumn{6}{|l|}{Subtype of MTStringEventClassType (See section \ref{type:MTStringEventClassType})} \\
\end{tabu}
\end{table} 


\FloatBarrier
\subsubsection{Defintion of \texttt{ CoupledAxesClassType}}
  \label{type:CoupledAxesClassType}

\FloatBarrier

Refers to the set of associated axes.

The valid data value for \mtmodel{COUPLED_AXES} SHOULD be a space-delimited set of 
axes reported as the value of the name attribute for each axis. If name is not available,
the piece of equipment MUST report the value of the nativeName attribute for each axis.

\begin{table}[ht]
\centering 
  \caption{\texttt{CoupledAxesClassType} Definition}
  \label{table:CoupledAxesClassType}
\fontsize{9pt}{11pt}\selectfont
\tabulinesep=3pt
\begin{tabu} to 6in {|X[-1.35]|X[-0.7]|X[-1.75]|X[-1.5]|X[-1]|X[-0.7]|} \everyrow{\hline}
\hline
\rowfont\bfseries {Attribute} & \multicolumn{5}{|l|}{Value} \\
\tabucline[1.5pt]{}
BrowseName & \multicolumn{5}{|l|}{CoupledAxesClassType} \\
IsAbstract & \multicolumn{5}{|l|}{False} \\
\tabucline[1.5pt]{}
\rowfont \bfseries References & NodeClass & BrowseName & DataType & Type\-Definition & {Modeling\-Rule} \\
\multicolumn{6}{|l|}{Subtype of MTStringEventClassType (See section \ref{type:MTStringEventClassType})} \\
\end{tabu}
\end{table} 


\FloatBarrier
\subsubsection{Defintion of \texttt{ LineClassType}}
  \label{type:LineClassType}

\FloatBarrier
\begin{table}[ht]
\centering 
  \caption{\texttt{LineClassType} Definition}
  \label{table:LineClassType}
\fontsize{9pt}{11pt}\selectfont
\tabulinesep=3pt
\begin{tabu} to 6in {|X[-1.35]|X[-0.7]|X[-1.75]|X[-1.5]|X[-1]|X[-0.7]|} \everyrow{\hline}
\hline
\rowfont\bfseries {Attribute} & \multicolumn{5}{|l|}{Value} \\
\tabucline[1.5pt]{}
BrowseName & \multicolumn{5}{|l|}{LineClassType} \\
IsAbstract & \multicolumn{5}{|l|}{False} \\
\tabucline[1.5pt]{}
\rowfont \bfseries References & NodeClass & BrowseName & DataType & Type\-Definition & {Modeling\-Rule} \\
\multicolumn{6}{|l|}{Subtype of MTStringEventClassType (See section \ref{type:MTStringEventClassType})} \\
\end{tabu}
\end{table} 


\FloatBarrier
\subsubsection{Defintion of \texttt{ LineLabelClassType}}
  \label{type:LineLabelClassType}

\FloatBarrier

An optional identifier for a \mtmodel{BLOCK} of code in a \mtmodel{PROGRAM}.

\begin{table}[ht]
\centering 
  \caption{\texttt{LineLabelClassType} Definition}
  \label{table:LineLabelClassType}
\fontsize{9pt}{11pt}\selectfont
\tabulinesep=3pt
\begin{tabu} to 6in {|X[-1.35]|X[-0.7]|X[-1.75]|X[-1.5]|X[-1]|X[-0.7]|} \everyrow{\hline}
\hline
\rowfont\bfseries {Attribute} & \multicolumn{5}{|l|}{Value} \\
\tabucline[1.5pt]{}
BrowseName & \multicolumn{5}{|l|}{LineLabelClassType} \\
IsAbstract & \multicolumn{5}{|l|}{False} \\
\tabucline[1.5pt]{}
\rowfont \bfseries References & NodeClass & BrowseName & DataType & Type\-Definition & {Modeling\-Rule} \\
\multicolumn{6}{|l|}{Subtype of MTStringEventClassType (See section \ref{type:MTStringEventClassType})} \\
\end{tabu}
\end{table} 


\FloatBarrier
\subsubsection{Defintion of \texttt{ MaterialClassType}}
  \label{type:MaterialClassType}

\FloatBarrier

The identifier of a material used or consumed in the manufacturing process.

\begin{table}[ht]
\centering 
  \caption{\texttt{MaterialClassType} Definition}
  \label{table:MaterialClassType}
\fontsize{9pt}{11pt}\selectfont
\tabulinesep=3pt
\begin{tabu} to 6in {|X[-1.35]|X[-0.7]|X[-1.75]|X[-1.5]|X[-1]|X[-0.7]|} \everyrow{\hline}
\hline
\rowfont\bfseries {Attribute} & \multicolumn{5}{|l|}{Value} \\
\tabucline[1.5pt]{}
BrowseName & \multicolumn{5}{|l|}{MaterialClassType} \\
IsAbstract & \multicolumn{5}{|l|}{False} \\
\tabucline[1.5pt]{}
\rowfont \bfseries References & NodeClass & BrowseName & DataType & Type\-Definition & {Modeling\-Rule} \\
\multicolumn{6}{|l|}{Subtype of MTStringEventClassType (See section \ref{type:MTStringEventClassType})} \\
\end{tabu}
\end{table} 


\FloatBarrier
\subsubsection{Defintion of \texttt{ MessageClassType}}
  \label{type:MessageClassType}

\FloatBarrier
\begin{table}[ht]
\centering 
  \caption{\texttt{MessageClassType} Definition}
  \label{table:MessageClassType}
\fontsize{9pt}{11pt}\selectfont
\tabulinesep=3pt
\begin{tabu} to 6in {|X[-1.35]|X[-0.7]|X[-1.75]|X[-1.5]|X[-1]|X[-0.7]|} \everyrow{\hline}
\hline
\rowfont\bfseries {Attribute} & \multicolumn{5}{|l|}{Value} \\
\tabucline[1.5pt]{}
BrowseName & \multicolumn{5}{|l|}{MessageClassType} \\
IsAbstract & \multicolumn{5}{|l|}{False} \\
\tabucline[1.5pt]{}
\rowfont \bfseries References & NodeClass & BrowseName & DataType & Type\-Definition & {Modeling\-Rule} \\
\multicolumn{6}{|l|}{Subtype of MTStringEventClassType (See section \ref{type:MTStringEventClassType})} \\
\end{tabu}
\end{table} 


\FloatBarrier
\subsubsection{Defintion of \texttt{ OperatorIdClassType}}
  \label{type:OperatorIdClassType}

\FloatBarrier

The identifier of the person currently responsible for operating the piece of equipment.

DEPRECATION WARNING: May be deprecated in the future. See \mtmodel{USER} below.

\begin{table}[ht]
\centering 
  \caption{\texttt{OperatorIdClassType} Definition}
  \label{table:OperatorIdClassType}
\fontsize{9pt}{11pt}\selectfont
\tabulinesep=3pt
\begin{tabu} to 6in {|X[-1.35]|X[-0.7]|X[-1.75]|X[-1.5]|X[-1]|X[-0.7]|} \everyrow{\hline}
\hline
\rowfont\bfseries {Attribute} & \multicolumn{5}{|l|}{Value} \\
\tabucline[1.5pt]{}
BrowseName & \multicolumn{5}{|l|}{OperatorIdClassType} \\
IsAbstract & \multicolumn{5}{|l|}{False} \\
\tabucline[1.5pt]{}
\rowfont \bfseries References & NodeClass & BrowseName & DataType & Type\-Definition & {Modeling\-Rule} \\
\multicolumn{6}{|l|}{Subtype of MTStringEventClassType (See section \ref{type:MTStringEventClassType})} \\
\end{tabu}
\end{table} 


\FloatBarrier
\subsubsection{Defintion of \texttt{ PalletIdClassType}}
  \label{type:PalletIdClassType}

\FloatBarrier

The identifier for a pallet.

\begin{table}[ht]
\centering 
  \caption{\texttt{PalletIdClassType} Definition}
  \label{table:PalletIdClassType}
\fontsize{9pt}{11pt}\selectfont
\tabulinesep=3pt
\begin{tabu} to 6in {|X[-1.35]|X[-0.7]|X[-1.75]|X[-1.5]|X[-1]|X[-0.7]|} \everyrow{\hline}
\hline
\rowfont\bfseries {Attribute} & \multicolumn{5}{|l|}{Value} \\
\tabucline[1.5pt]{}
BrowseName & \multicolumn{5}{|l|}{PalletIdClassType} \\
IsAbstract & \multicolumn{5}{|l|}{False} \\
\tabucline[1.5pt]{}
\rowfont \bfseries References & NodeClass & BrowseName & DataType & Type\-Definition & {Modeling\-Rule} \\
\multicolumn{6}{|l|}{Subtype of MTStringEventClassType (See section \ref{type:MTStringEventClassType})} \\
\end{tabu}
\end{table} 


\FloatBarrier
\subsubsection{Defintion of \texttt{ PartIdClassType}}
  \label{type:PartIdClassType}

\FloatBarrier

An identifier of a part in a manufacturing operation.

\begin{table}[ht]
\centering 
  \caption{\texttt{PartIdClassType} Definition}
  \label{table:PartIdClassType}
\fontsize{9pt}{11pt}\selectfont
\tabulinesep=3pt
\begin{tabu} to 6in {|X[-1.35]|X[-0.7]|X[-1.75]|X[-1.5]|X[-1]|X[-0.7]|} \everyrow{\hline}
\hline
\rowfont\bfseries {Attribute} & \multicolumn{5}{|l|}{Value} \\
\tabucline[1.5pt]{}
BrowseName & \multicolumn{5}{|l|}{PartIdClassType} \\
IsAbstract & \multicolumn{5}{|l|}{False} \\
\tabucline[1.5pt]{}
\rowfont \bfseries References & NodeClass & BrowseName & DataType & Type\-Definition & {Modeling\-Rule} \\
\multicolumn{6}{|l|}{Subtype of MTStringEventClassType (See section \ref{type:MTStringEventClassType})} \\
\end{tabu}
\end{table} 


\FloatBarrier
\subsubsection{Defintion of \texttt{ PartNumberClassType}}
  \label{type:PartNumberClassType}

\FloatBarrier

An identifier of a part or product moving through the manufacturing process.

\begin{table}[ht]
\centering 
  \caption{\texttt{PartNumberClassType} Definition}
  \label{table:PartNumberClassType}
\fontsize{9pt}{11pt}\selectfont
\tabulinesep=3pt
\begin{tabu} to 6in {|X[-1.35]|X[-0.7]|X[-1.75]|X[-1.5]|X[-1]|X[-0.7]|} \everyrow{\hline}
\hline
\rowfont\bfseries {Attribute} & \multicolumn{5}{|l|}{Value} \\
\tabucline[1.5pt]{}
BrowseName & \multicolumn{5}{|l|}{PartNumberClassType} \\
IsAbstract & \multicolumn{5}{|l|}{False} \\
\tabucline[1.5pt]{}
\rowfont \bfseries References & NodeClass & BrowseName & DataType & Type\-Definition & {Modeling\-Rule} \\
\multicolumn{6}{|l|}{Subtype of MTStringEventClassType (See section \ref{type:MTStringEventClassType})} \\
\end{tabu}
\end{table} 


\FloatBarrier
\subsubsection{Defintion of \texttt{ ProgramClassType}}
  \label{type:ProgramClassType}

\FloatBarrier

The name of the logic or motion program being executed by the \mtmodel{Controller} or \mtmodel{Path} component.

\begin{table}[ht]
\centering 
  \caption{\texttt{ProgramClassType} Definition}
  \label{table:ProgramClassType}
\fontsize{9pt}{11pt}\selectfont
\tabulinesep=3pt
\begin{tabu} to 6in {|X[-1.35]|X[-0.7]|X[-1.75]|X[-1.5]|X[-1]|X[-0.7]|} \everyrow{\hline}
\hline
\rowfont\bfseries {Attribute} & \multicolumn{5}{|l|}{Value} \\
\tabucline[1.5pt]{}
BrowseName & \multicolumn{5}{|l|}{ProgramClassType} \\
IsAbstract & \multicolumn{5}{|l|}{False} \\
\tabucline[1.5pt]{}
\rowfont \bfseries References & NodeClass & BrowseName & DataType & Type\-Definition & {Modeling\-Rule} \\
\multicolumn{6}{|l|}{Subtype of MTStringEventClassType (See section \ref{type:MTStringEventClassType})} \\
\end{tabu}
\end{table} 


\FloatBarrier
\subsubsection{Defintion of \texttt{ ProgramCommentClassType}}
  \label{type:ProgramCommentClassType}

\FloatBarrier

A comment or non-executable statement in the control program.

\begin{table}[ht]
\centering 
  \caption{\texttt{ProgramCommentClassType} Definition}
  \label{table:ProgramCommentClassType}
\fontsize{9pt}{11pt}\selectfont
\tabulinesep=3pt
\begin{tabu} to 6in {|X[-1.35]|X[-0.7]|X[-1.75]|X[-1.5]|X[-1]|X[-0.7]|} \everyrow{\hline}
\hline
\rowfont\bfseries {Attribute} & \multicolumn{5}{|l|}{Value} \\
\tabucline[1.5pt]{}
BrowseName & \multicolumn{5}{|l|}{ProgramCommentClassType} \\
IsAbstract & \multicolumn{5}{|l|}{False} \\
\tabucline[1.5pt]{}
\rowfont \bfseries References & NodeClass & BrowseName & DataType & Type\-Definition & {Modeling\-Rule} \\
\multicolumn{6}{|l|}{Subtype of MTStringEventClassType (See section \ref{type:MTStringEventClassType})} \\
\end{tabu}
\end{table} 


\FloatBarrier
\subsubsection{Defintion of \texttt{ ProgramEditNameClassType}}
  \label{type:ProgramEditNameClassType}

\FloatBarrier

The name of the program being edited. This is used in conjunction with \mtmodel{PROGRAM_EDIT} when in \mtmodel{ACTIVE} state.

\begin{table}[ht]
\centering 
  \caption{\texttt{ProgramEditNameClassType} Definition}
  \label{table:ProgramEditNameClassType}
\fontsize{9pt}{11pt}\selectfont
\tabulinesep=3pt
\begin{tabu} to 6in {|X[-1.35]|X[-0.7]|X[-1.75]|X[-1.5]|X[-1]|X[-0.7]|} \everyrow{\hline}
\hline
\rowfont\bfseries {Attribute} & \multicolumn{5}{|l|}{Value} \\
\tabucline[1.5pt]{}
BrowseName & \multicolumn{5}{|l|}{ProgramEditNameClassType} \\
IsAbstract & \multicolumn{5}{|l|}{False} \\
\tabucline[1.5pt]{}
\rowfont \bfseries References & NodeClass & BrowseName & DataType & Type\-Definition & {Modeling\-Rule} \\
\multicolumn{6}{|l|}{Subtype of MTStringEventClassType (See section \ref{type:MTStringEventClassType})} \\
\end{tabu}
\end{table} 


\FloatBarrier
\subsubsection{Defintion of \texttt{ ProgramHeaderClassType}}
  \label{type:ProgramHeaderClassType}

\FloatBarrier

The non-executable header section of the control program.

\begin{table}[ht]
\centering 
  \caption{\texttt{ProgramHeaderClassType} Definition}
  \label{table:ProgramHeaderClassType}
\fontsize{9pt}{11pt}\selectfont
\tabulinesep=3pt
\begin{tabu} to 6in {|X[-1.35]|X[-0.7]|X[-1.75]|X[-1.5]|X[-1]|X[-0.7]|} \everyrow{\hline}
\hline
\rowfont\bfseries {Attribute} & \multicolumn{5}{|l|}{Value} \\
\tabucline[1.5pt]{}
BrowseName & \multicolumn{5}{|l|}{ProgramHeaderClassType} \\
IsAbstract & \multicolumn{5}{|l|}{False} \\
\tabucline[1.5pt]{}
\rowfont \bfseries References & NodeClass & BrowseName & DataType & Type\-Definition & {Modeling\-Rule} \\
\multicolumn{6}{|l|}{Subtype of MTStringEventClassType (See section \ref{type:MTStringEventClassType})} \\
\end{tabu}
\end{table} 


\FloatBarrier
\subsubsection{Defintion of \texttt{ SerialNumberClassType}}
  \label{type:SerialNumberClassType}

\FloatBarrier

The serial number associated with a \mtmodel{Component}, \mtmodel{Asset}, or \mtmodel{Device}.

\begin{table}[ht]
\centering 
  \caption{\texttt{SerialNumberClassType} Definition}
  \label{table:SerialNumberClassType}
\fontsize{9pt}{11pt}\selectfont
\tabulinesep=3pt
\begin{tabu} to 6in {|X[-1.35]|X[-0.7]|X[-1.75]|X[-1.5]|X[-1]|X[-0.7]|} \everyrow{\hline}
\hline
\rowfont\bfseries {Attribute} & \multicolumn{5}{|l|}{Value} \\
\tabucline[1.5pt]{}
BrowseName & \multicolumn{5}{|l|}{SerialNumberClassType} \\
IsAbstract & \multicolumn{5}{|l|}{False} \\
\tabucline[1.5pt]{}
\rowfont \bfseries References & NodeClass & BrowseName & DataType & Type\-Definition & {Modeling\-Rule} \\
\multicolumn{6}{|l|}{Subtype of MTStringEventClassType (See section \ref{type:MTStringEventClassType})} \\
\end{tabu}
\end{table} 


\FloatBarrier
\subsubsection{Defintion of \texttt{ ToolAssetIdClassType}}
  \label{type:ToolAssetIdClassType}

\FloatBarrier

The identifier of an individual tool asset

\begin{table}[ht]
\centering 
  \caption{\texttt{ToolAssetIdClassType} Definition}
  \label{table:ToolAssetIdClassType}
\fontsize{9pt}{11pt}\selectfont
\tabulinesep=3pt
\begin{tabu} to 6in {|X[-1.35]|X[-0.7]|X[-1.75]|X[-1.5]|X[-1]|X[-0.7]|} \everyrow{\hline}
\hline
\rowfont\bfseries {Attribute} & \multicolumn{5}{|l|}{Value} \\
\tabucline[1.5pt]{}
BrowseName & \multicolumn{5}{|l|}{ToolAssetIdClassType} \\
IsAbstract & \multicolumn{5}{|l|}{False} \\
\tabucline[1.5pt]{}
\rowfont \bfseries References & NodeClass & BrowseName & DataType & Type\-Definition & {Modeling\-Rule} \\
\multicolumn{6}{|l|}{Subtype of MTStringEventClassType (See section \ref{type:MTStringEventClassType})} \\
\end{tabu}
\end{table} 


\FloatBarrier
\subsubsection{Defintion of \texttt{ ToolNumberClassType}}
  \label{type:ToolNumberClassType}

\FloatBarrier

The identifier of a tool provided by the piece of equipment controller.

\begin{table}[ht]
\centering 
  \caption{\texttt{ToolNumberClassType} Definition}
  \label{table:ToolNumberClassType}
\fontsize{9pt}{11pt}\selectfont
\tabulinesep=3pt
\begin{tabu} to 6in {|X[-1.35]|X[-0.7]|X[-1.75]|X[-1.5]|X[-1]|X[-0.7]|} \everyrow{\hline}
\hline
\rowfont\bfseries {Attribute} & \multicolumn{5}{|l|}{Value} \\
\tabucline[1.5pt]{}
BrowseName & \multicolumn{5}{|l|}{ToolNumberClassType} \\
IsAbstract & \multicolumn{5}{|l|}{False} \\
\tabucline[1.5pt]{}
\rowfont \bfseries References & NodeClass & BrowseName & DataType & Type\-Definition & {Modeling\-Rule} \\
\multicolumn{6}{|l|}{Subtype of MTStringEventClassType (See section \ref{type:MTStringEventClassType})} \\
\end{tabu}
\end{table} 


\FloatBarrier
\subsubsection{Defintion of \texttt{ ToolOffsetClassType}}
  \label{type:ToolOffsetClassType}

\FloatBarrier

A reference to the tool offset variables applied to the active cutting tool associated with a Path in a Controller type component.

The valid data value MUST be a text string.

The reported value returned for \mtmodel{TOOL_OFFSET} identifies the location in
a table or list where the actual tool offset values are stored. 

A \gls{subType} MUST always be specified.

\begin{table}[ht]
\centering 
  \caption{\texttt{ToolOffsetClassType} Definition}
  \label{table:ToolOffsetClassType}
\fontsize{9pt}{11pt}\selectfont
\tabulinesep=3pt
\begin{tabu} to 6in {|X[-1.35]|X[-0.7]|X[-1.75]|X[-1.5]|X[-1]|X[-0.7]|} \everyrow{\hline}
\hline
\rowfont\bfseries {Attribute} & \multicolumn{5}{|l|}{Value} \\
\tabucline[1.5pt]{}
BrowseName & \multicolumn{5}{|l|}{ToolOffsetClassType} \\
IsAbstract & \multicolumn{5}{|l|}{False} \\
\tabucline[1.5pt]{}
\rowfont \bfseries References & NodeClass & BrowseName & DataType & Type\-Definition & {Modeling\-Rule} \\
\multicolumn{6}{|l|}{Subtype of MTStringEventClassType (See section \ref{type:MTStringEventClassType})} \\
\end{tabu}
\end{table} 


\FloatBarrier
\subsubsection{Defintion of \texttt{ UserClassType}}
  \label{type:UserClassType}

\FloatBarrier

The identifier of the person currently responsible for operating the piece of equipment.

A \gls{subType} MUST always be specified.

\begin{table}[ht]
\centering 
  \caption{\texttt{UserClassType} Definition}
  \label{table:UserClassType}
\fontsize{9pt}{11pt}\selectfont
\tabulinesep=3pt
\begin{tabu} to 6in {|X[-1.35]|X[-0.7]|X[-1.75]|X[-1.5]|X[-1]|X[-0.7]|} \everyrow{\hline}
\hline
\rowfont\bfseries {Attribute} & \multicolumn{5}{|l|}{Value} \\
\tabucline[1.5pt]{}
BrowseName & \multicolumn{5}{|l|}{UserClassType} \\
IsAbstract & \multicolumn{5}{|l|}{False} \\
\tabucline[1.5pt]{}
\rowfont \bfseries References & NodeClass & BrowseName & DataType & Type\-Definition & {Modeling\-Rule} \\
\multicolumn{6}{|l|}{Subtype of MTStringEventClassType (See section \ref{type:MTStringEventClassType})} \\
\end{tabu}
\end{table} 


\FloatBarrier
\subsubsection{Defintion of \texttt{ WireClassType}}
  \label{type:WireClassType}

\FloatBarrier

The identifier for the type of wire used as the cutting mechanism in Electrical Discharge Machining or similar processes.

\begin{table}[ht]
\centering 
  \caption{\texttt{WireClassType} Definition}
  \label{table:WireClassType}
\fontsize{9pt}{11pt}\selectfont
\tabulinesep=3pt
\begin{tabu} to 6in {|X[-1.35]|X[-0.7]|X[-1.75]|X[-1.5]|X[-1]|X[-0.7]|} \everyrow{\hline}
\hline
\rowfont\bfseries {Attribute} & \multicolumn{5}{|l|}{Value} \\
\tabucline[1.5pt]{}
BrowseName & \multicolumn{5}{|l|}{WireClassType} \\
IsAbstract & \multicolumn{5}{|l|}{False} \\
\tabucline[1.5pt]{}
\rowfont \bfseries References & NodeClass & BrowseName & DataType & Type\-Definition & {Modeling\-Rule} \\
\multicolumn{6}{|l|}{Subtype of MTStringEventClassType (See section \ref{type:MTStringEventClassType})} \\
\end{tabu}
\end{table} 


\FloatBarrier
\subsubsection{Defintion of \texttt{ WorkholdingClassType}}
  \label{type:WorkholdingClassType}

\FloatBarrier

The identifier for the workholding currently in use.

\begin{table}[ht]
\centering 
  \caption{\texttt{WorkholdingClassType} Definition}
  \label{table:WorkholdingClassType}
\fontsize{9pt}{11pt}\selectfont
\tabulinesep=3pt
\begin{tabu} to 6in {|X[-1.35]|X[-0.7]|X[-1.75]|X[-1.5]|X[-1]|X[-0.7]|} \everyrow{\hline}
\hline
\rowfont\bfseries {Attribute} & \multicolumn{5}{|l|}{Value} \\
\tabucline[1.5pt]{}
BrowseName & \multicolumn{5}{|l|}{WorkholdingClassType} \\
IsAbstract & \multicolumn{5}{|l|}{False} \\
\tabucline[1.5pt]{}
\rowfont \bfseries References & NodeClass & BrowseName & DataType & Type\-Definition & {Modeling\-Rule} \\
\multicolumn{6}{|l|}{Subtype of MTStringEventClassType (See section \ref{type:MTStringEventClassType})} \\
\end{tabu}
\end{table} 


\FloatBarrier
\subsubsection{Defintion of \texttt{ WorkOffsetClassType}}
  \label{type:WorkOffsetClassType}

\FloatBarrier

A reference to the offset variables for a work piece or part associated with a Path in a Controller type component.

The valid data value MUST be a text string.

The reported value returned for \mtmodel{WORK_OFFSET} identifies the location in a table or list 
where the actual tool offset values are stored.

\begin{table}[ht]
\centering 
  \caption{\texttt{WorkOffsetClassType} Definition}
  \label{table:WorkOffsetClassType}
\fontsize{9pt}{11pt}\selectfont
\tabulinesep=3pt
\begin{tabu} to 6in {|X[-1.35]|X[-0.7]|X[-1.75]|X[-1.5]|X[-1]|X[-0.7]|} \everyrow{\hline}
\hline
\rowfont\bfseries {Attribute} & \multicolumn{5}{|l|}{Value} \\
\tabucline[1.5pt]{}
BrowseName & \multicolumn{5}{|l|}{WorkOffsetClassType} \\
IsAbstract & \multicolumn{5}{|l|}{False} \\
\tabucline[1.5pt]{}
\rowfont \bfseries References & NodeClass & BrowseName & DataType & Type\-Definition & {Modeling\-Rule} \\
\multicolumn{6}{|l|}{Subtype of MTStringEventClassType (See section \ref{type:MTStringEventClassType})} \\
\end{tabu}
\end{table} 


\FloatBarrier
