% Generated 2019-05-02 22:39:15 -0700
\subsection{Components} \label{model:Components}

\begin{figure}[ht]
  \centering
    \includegraphics[width=1.0\textwidth]{./diagrams/types/Components.png}
  \caption{Components Diagram}
  \label{fig:Components}
\end{figure}

\FloatBarrier

\subsubsection{Defintion of \texttt{ MTChannelType}}
  \label{type:MTChannelType}

\FloatBarrier
\begin{table}[ht]
\centering 
  \caption{\texttt{MTChannelType} Definition}
  \label{table:MTChannelType}
\fontsize{9pt}{11pt}\selectfont
\tabulinesep=3pt
\begin{tabu} to 6in {|X[-1.35]|X[-0.7]|X[-1.75]|X[-1.5]|X[-1]|X[-0.7]|} \everyrow{\hline}
\hline
\rowfont\bfseries {Attribute} & \multicolumn{5}{|l|}{Value} \\
\tabucline[1.5pt]{}
BrowseName & \multicolumn{5}{|l|}{MTChannelType} \\
IsAbstract & \multicolumn{5}{|l|}{False} \\
\tabucline[1.5pt]{}
\rowfont \bfseries References & NodeClass & BrowseName & DataType & Type\-Definition & {Modeling\-Rule} \\
\multicolumn{6}{|l|}{Subtype of BaseObjectType (See \cite{UAPart05} Documentation)} \\
Has\-Property & Variable & Calibration\-Date & Utc\-Time & Property\-Type & Optional \\
Has\-Property & Variable & Calibration\-Initials & String & Property\-Type & Optional \\
Has\-Property & Variable & MT\-Description & String & Property\-Type & Optional \\
Has\-Property & Variable & Name & String & Property\-Type & Optional \\
Has\-Property & Variable & Next\-Calibration\-Date & Utc\-Time & Property\-Type & Optional \\
Has\-Property & Variable & Number & Int32 & Property\-Type & Mandatory \\
\end{tabu}
\end{table} 


\FloatBarrier
\subsubsection{Defintion of \texttt{ MTComponentType}}
  \label{type:MTComponentType}

\FloatBarrier

\begin{figure}[ht]
  \centering
    \includegraphics[width=1.0\textwidth]{./diagrams/types/MTComponentType.png}
  \caption{MTComponentType Diagram}
  \label{fig:MTComponentType}
\end{figure}

\FloatBarrier


All the sub types of components organized into top level organizational types.

\begin{table}[ht]
\centering 
  \caption{\texttt{MTComponentType} Definition}
  \label{table:MTComponentType}
\fontsize{9pt}{11pt}\selectfont
\tabulinesep=3pt
\begin{tabu} to 6in {|X[-1.35]|X[-0.7]|X[-1.75]|X[-1.5]|X[-1]|X[-0.7]|} \everyrow{\hline}
\hline
\rowfont\bfseries {Attribute} & \multicolumn{5}{|l|}{Value} \\
\tabucline[1.5pt]{}
BrowseName & \multicolumn{5}{|l|}{MTComponentType} \\
IsAbstract & \multicolumn{5}{|l|}{True} \\
\tabucline[1.5pt]{}
\rowfont \bfseries References & NodeClass & BrowseName & DataType & Type\-Definition & {Modeling\-Rule} \\
\multicolumn{6}{|l|}{Subtype of BaseObjectType (See \cite{UAPart05} Documentation)} \\
HasSubtype & ObjectType & \multicolumn{2}{l}{MTDeviceType} & \multicolumn{2}{|l|}{See section \ref{type:MTDeviceType}} \\
HasSubtype & ObjectType & \multicolumn{2}{l}{ActuatorType} & \multicolumn{2}{|l|}{See section \ref{type:ActuatorType}} \\
HasSubtype & ObjectType & \multicolumn{2}{l}{AuxiliariesType} & \multicolumn{2}{|l|}{See section \ref{type:AuxiliariesType}} \\
HasSubtype & ObjectType & \multicolumn{2}{l}{AxesType} & \multicolumn{2}{|l|}{See section \ref{type:AxesType}} \\
HasSubtype & ObjectType & \multicolumn{2}{l}{ControllerType} & \multicolumn{2}{|l|}{See section \ref{type:ControllerType}} \\
HasSubtype & ObjectType & \multicolumn{2}{l}{DoorType} & \multicolumn{2}{|l|}{See section \ref{type:DoorType}} \\
HasSubtype & ObjectType & \multicolumn{2}{l}{InterfacesType} & \multicolumn{2}{|l|}{See section \ref{type:InterfacesType}} \\
HasSubtype & ObjectType & \multicolumn{2}{l}{ResourcesType} & \multicolumn{2}{|l|}{See section \ref{type:ResourcesType}} \\
HasSubtype & ObjectType & \multicolumn{2}{l}{SystemsType} & \multicolumn{2}{|l|}{See section \ref{type:SystemsType}} \\
Has\-Property & Variable & Name & String & Property\-Type & Optional \\
Has\-Property & Variable & Native\-Name & String & Property\-Type & Optional \\
Has\-Property & Variable & Sample\-Interval & Float & Property\-Type & Optional \\
Has\-Property & Variable & Sample\-Rate & Float & Property\-Type & Optional \\
Has\-Property & Variable & Uuid & String & Property\-Type & Optional \\
Has\-Property & Variable & Xml\-Id & String & Property\-Type & Mandatory \\
Has\-Component & Object & Description & \multicolumn{2}{l|}{MTDescriptionType} & Optional \\
Has\-Component & Event & <MT\-Condition> & \multicolumn{2}{l|}{MTConditionType[]} & Optional \\
Has\-Component & Object & Configuration & \multicolumn{2}{l|}{MTConfigurationType} & Optional \\
Has\-Component & Variable & <MT\-Three\-Space\-Sample> & Three\-Space\-Sample\-Data\-Type & MT\-Three\-Space\-Sample\-Type & Mandatory \\
Has\-Component & Variable & <MT\-Controlled\-Vocab\-Event> & UInteger[] & MT\-Controlled\-Vocab\-Event\-Type & Optional \\
Has\-Condition & Event & <MT\-Condition> & \multicolumn{2}{l|}{MTConditionType[]} & Optional \\
Has\-Component & Variable & <MT\-Asset\-Event> & Asset\-Event\-Data\-Type[] & MT\-Asset\-Event\-Type & Optional \\
Has\-Component & Variable & <MT\-Sample> & Number[] & MT\-Sample\-Type & Optional \\
Organizes & Object & Compositions & MT\-Composition\-Type[] & Folder\-Type & Optional \\
Has\-Component & Variable & <MT\-String\-Event> & String[] & MT\-String\-Event\-Type & Optional \\
Has\-Component & Variable & <MT\-Message> & Message\-Data\-Type[] & MT\-Message\-Type & Optional \\
Has\-Component & Variable & <MT\-Numeric\-Event> & Number[] & MT\-Numeric\-Event\-Type & Optional \\
Organizes & Object & Components & MT\-Component\-Type[] & Folder\-Type & Optional \\
\end{tabu}
\end{table} 


\FloatBarrier
\subsubsection{Defintion of \texttt{ MTDeviceType}}
  \label{type:MTDeviceType}

\FloatBarrier

\input ./type-sections/MTDeviceType.tex

\begin{table}[ht]
\centering 
  \caption{\texttt{MTDeviceType} Definition}
  \label{table:MTDeviceType}
\fontsize{9pt}{11pt}\selectfont
\tabulinesep=3pt
\begin{tabu} to 6in {|X[-1.35]|X[-0.7]|X[-1.75]|X[-1.5]|X[-1]|X[-0.7]|} \everyrow{\hline}
\hline
\rowfont\bfseries {Attribute} & \multicolumn{5}{|l|}{Value} \\
\tabucline[1.5pt]{}
BrowseName & \multicolumn{5}{|l|}{MTDeviceType} \\
IsAbstract & \multicolumn{5}{|l|}{False} \\
\tabucline[1.5pt]{}
\rowfont \bfseries References & NodeClass & BrowseName & DataType & Type\-Definition & {Modeling\-Rule} \\
\multicolumn{6}{|l|}{Subtype of MTComponentType (See section \ref{type:MTComponentType})} \\
Has\-Property & Variable & Iso841Class & String & Property\-Type & Optional \\
Has\-Property & Variable & Version & String & Property\-Type & Optional \\
\end{tabu}
\end{table} 


\FloatBarrier
\subsubsection{Defintion of \texttt{ MTCompositionType}}
  \label{type:MTCompositionType}

\FloatBarrier
\begin{table}[ht]
\centering 
  \caption{\texttt{MTCompositionType} Definition}
  \label{table:MTCompositionType}
\fontsize{9pt}{11pt}\selectfont
\tabulinesep=3pt
\begin{tabu} to 6in {|X[-1.35]|X[-0.7]|X[-1.75]|X[-1.5]|X[-1]|X[-0.7]|} \everyrow{\hline}
\hline
\rowfont\bfseries {Attribute} & \multicolumn{5}{|l|}{Value} \\
\tabucline[1.5pt]{}
BrowseName & \multicolumn{5}{|l|}{MTCompositionType} \\
IsAbstract & \multicolumn{5}{|l|}{False} \\
\tabucline[1.5pt]{}
\rowfont \bfseries References & NodeClass & BrowseName & DataType & Type\-Definition & {Modeling\-Rule} \\
\multicolumn{6}{|l|}{Subtype of BaseObjectType (See \cite{UAPart05} Documentation)} \\
Has\-Property & Variable & MT\-Type\-Name & String & Property\-Type & Mandatory \\
Has\-Property & Variable & Name & String & Property\-Type & Optional \\
Has\-Property & Variable & Uuid & String & Property\-Type & Optional \\
Has\-Property & Variable & Xml\-Id & String & Property\-Type & Mandatory \\
\end{tabu}
\end{table} 


\FloatBarrier
\subsubsection{Defintion of \texttt{ MTConfigurationType}}
  \label{type:MTConfigurationType}

\FloatBarrier
\begin{table}[ht]
\centering 
  \caption{\texttt{MTConfigurationType} Definition}
  \label{table:MTConfigurationType}
\fontsize{9pt}{11pt}\selectfont
\tabulinesep=3pt
\begin{tabu} to 6in {|X[-1.35]|X[-0.7]|X[-1.75]|X[-1.5]|X[-1]|X[-0.7]|} \everyrow{\hline}
\hline
\rowfont\bfseries {Attribute} & \multicolumn{5}{|l|}{Value} \\
\tabucline[1.5pt]{}
BrowseName & \multicolumn{5}{|l|}{MTConfigurationType} \\
IsAbstract & \multicolumn{5}{|l|}{True} \\
\tabucline[1.5pt]{}
\rowfont \bfseries References & NodeClass & BrowseName & DataType & Type\-Definition & {Modeling\-Rule} \\
\multicolumn{6}{|l|}{Subtype of BaseObjectType (See \cite{UAPart05} Documentation)} \\
HasSubtype & ObjectType & \multicolumn{2}{l}{MTSensorConfigurationType} & \multicolumn{2}{|l|}{See section \ref{type:MTSensorConfigurationType}} \\
\end{tabu}
\end{table} 


\FloatBarrier
\subsubsection{Defintion of \texttt{ MTSensorConfigurationType}}
  \label{type:MTSensorConfigurationType}

\FloatBarrier
\begin{table}[ht]
\centering 
  \caption{\texttt{MTSensorConfigurationType} Definition}
  \label{table:MTSensorConfigurationType}
\fontsize{9pt}{11pt}\selectfont
\tabulinesep=3pt
\begin{tabu} to 6in {|X[-1.35]|X[-0.7]|X[-1.75]|X[-1.5]|X[-1]|X[-0.7]|} \everyrow{\hline}
\hline
\rowfont\bfseries {Attribute} & \multicolumn{5}{|l|}{Value} \\
\tabucline[1.5pt]{}
BrowseName & \multicolumn{5}{|l|}{MTSensorConfigurationType} \\
IsAbstract & \multicolumn{5}{|l|}{False} \\
\tabucline[1.5pt]{}
\rowfont \bfseries References & NodeClass & BrowseName & DataType & Type\-Definition & {Modeling\-Rule} \\
\multicolumn{6}{|l|}{Subtype of MTConfigurationType (See section \ref{type:MTConfigurationType})} \\
Has\-Property & Variable & Calibration\-Date & Utc\-Time & Property\-Type & Optional \\
Has\-Property & Variable & Calibration\-Initials & String & Property\-Type & Optional \\
Has\-Property & Variable & Firware\-Version & String & Property\-Type & Mandatory \\
Has\-Property & Variable & Next\-Calibration\-Date & Utc\-Time & Property\-Type & Optional \\
Organizes & Object & Channels & MT\-Channel\-Type[] & Folder\-Type & Optional \\
\end{tabu}
\end{table} 


\FloatBarrier
\subsubsection{Defintion of \texttt{ MTDescriptionType}}
  \label{type:MTDescriptionType}

\FloatBarrier
\begin{table}[ht]
\centering 
  \caption{\texttt{MTDescriptionType} Definition}
  \label{table:MTDescriptionType}
\fontsize{9pt}{11pt}\selectfont
\tabulinesep=3pt
\begin{tabu} to 6in {|X[-1.35]|X[-0.7]|X[-1.75]|X[-1.5]|X[-1]|X[-0.7]|} \everyrow{\hline}
\hline
\rowfont\bfseries {Attribute} & \multicolumn{5}{|l|}{Value} \\
\tabucline[1.5pt]{}
BrowseName & \multicolumn{5}{|l|}{MTDescriptionType} \\
IsAbstract & \multicolumn{5}{|l|}{False} \\
\tabucline[1.5pt]{}
\rowfont \bfseries References & NodeClass & BrowseName & DataType & Type\-Definition & {Modeling\-Rule} \\
\multicolumn{6}{|l|}{Subtype of BaseObjectType (See \cite{UAPart05} Documentation)} \\
Has\-Property & Variable & Data & String & Property\-Type & Optional \\
Has\-Property & Variable & Manufacturer & String & Property\-Type & Optional \\
Has\-Property & Variable & Serial\-Number & String & Property\-Type & Optional \\
Has\-Property & Variable & Station & String & Property\-Type & Optional \\
\end{tabu}
\end{table} 


\FloatBarrier
\subsection{Component Types} \label{model:ComponentTypes}

\begin{figure}[ht]
  \centering
    \includegraphics[width=1.0\textwidth]{./diagrams/types/ComponentTypes.png}
  \caption{Component Types Diagram}
  \label{fig:ComponentTypes}
\end{figure}

\FloatBarrier

\subsubsection{Defintion of \texttt{ ActuatorType}}
  \label{type:ActuatorType}

\FloatBarrier
\begin{table}[ht]
\centering 
  \caption{\texttt{ActuatorType} Definition}
  \label{table:ActuatorType}
\fontsize{9pt}{11pt}\selectfont
\tabulinesep=3pt
\begin{tabu} to 6in {|X[-1.35]|X[-0.7]|X[-1.75]|X[-1.5]|X[-1]|X[-0.7]|} \everyrow{\hline}
\hline
\rowfont\bfseries {Attribute} & \multicolumn{5}{|l|}{Value} \\
\tabucline[1.5pt]{}
BrowseName & \multicolumn{5}{|l|}{ActuatorType} \\
IsAbstract & \multicolumn{5}{|l|}{False} \\
\tabucline[1.5pt]{}
\rowfont \bfseries References & NodeClass & BrowseName & DataType & Type\-Definition & {Modeling\-Rule} \\
\multicolumn{6}{|l|}{Subtype of MTComponentType (See Components Documentation)} \\
\end{tabu}
\end{table} 


\FloatBarrier
\subsubsection{Defintion of \texttt{ AuxiliariesType}}
  \label{type:AuxiliariesType}

\FloatBarrier
\begin{table}[ht]
\centering 
  \caption{\texttt{AuxiliariesType} Definition}
  \label{table:AuxiliariesType}
\fontsize{9pt}{11pt}\selectfont
\tabulinesep=3pt
\begin{tabu} to 6in {|X[-1.35]|X[-0.7]|X[-1.75]|X[-1.5]|X[-1]|X[-0.7]|} \everyrow{\hline}
\hline
\rowfont\bfseries {Attribute} & \multicolumn{5}{|l|}{Value} \\
\tabucline[1.5pt]{}
BrowseName & \multicolumn{5}{|l|}{AuxiliariesType} \\
IsAbstract & \multicolumn{5}{|l|}{False} \\
\tabucline[1.5pt]{}
\rowfont \bfseries References & NodeClass & BrowseName & DataType & Type\-Definition & {Modeling\-Rule} \\
\multicolumn{6}{|l|}{Subtype of MTComponentType (See Components Documentation)} \\
HasSubtype & ObjectType & \multicolumn{2}{l}{BarFeederType} & \multicolumn{2}{|l|}{See section \ref{type:BarFeederType}} \\
HasSubtype & ObjectType & \multicolumn{2}{l}{EnvironmentalType} & \multicolumn{2}{|l|}{See section \ref{type:EnvironmentalType}} \\
HasSubtype & ObjectType & \multicolumn{2}{l}{LoaderType} & \multicolumn{2}{|l|}{See section \ref{type:LoaderType}} \\
HasSubtype & ObjectType & \multicolumn{2}{l}{SensorType} & \multicolumn{2}{|l|}{See section \ref{type:SensorType}} \\
HasSubtype & ObjectType & \multicolumn{2}{l}{ToolingDeliveryType} & \multicolumn{2}{|l|}{See section \ref{type:ToolingDeliveryType}} \\
HasSubtype & ObjectType & \multicolumn{2}{l}{WasteDisposalType} & \multicolumn{2}{|l|}{See section \ref{type:WasteDisposalType}} \\
\end{tabu}
\end{table} 


\FloatBarrier
\subsubsection{Defintion of \texttt{ BarFeederType}}
  \label{type:BarFeederType}

\FloatBarrier
\begin{table}[ht]
\centering 
  \caption{\texttt{BarFeederType} Definition}
  \label{table:BarFeederType}
\fontsize{9pt}{11pt}\selectfont
\tabulinesep=3pt
\begin{tabu} to 6in {|X[-1.35]|X[-0.7]|X[-1.75]|X[-1.5]|X[-1]|X[-0.7]|} \everyrow{\hline}
\hline
\rowfont\bfseries {Attribute} & \multicolumn{5}{|l|}{Value} \\
\tabucline[1.5pt]{}
BrowseName & \multicolumn{5}{|l|}{BarFeederType} \\
IsAbstract & \multicolumn{5}{|l|}{False} \\
\tabucline[1.5pt]{}
\rowfont \bfseries References & NodeClass & BrowseName & DataType & Type\-Definition & {Modeling\-Rule} \\
\multicolumn{6}{|l|}{Subtype of AuxiliariesType (See section \ref{type:AuxiliariesType})} \\
\end{tabu}
\end{table} 


\FloatBarrier
\subsubsection{Defintion of \texttt{ EnvironmentalType}}
  \label{type:EnvironmentalType}

\FloatBarrier
\begin{table}[ht]
\centering 
  \caption{\texttt{EnvironmentalType} Definition}
  \label{table:EnvironmentalType}
\fontsize{9pt}{11pt}\selectfont
\tabulinesep=3pt
\begin{tabu} to 6in {|X[-1.35]|X[-0.7]|X[-1.75]|X[-1.5]|X[-1]|X[-0.7]|} \everyrow{\hline}
\hline
\rowfont\bfseries {Attribute} & \multicolumn{5}{|l|}{Value} \\
\tabucline[1.5pt]{}
BrowseName & \multicolumn{5}{|l|}{EnvironmentalType} \\
IsAbstract & \multicolumn{5}{|l|}{False} \\
\tabucline[1.5pt]{}
\rowfont \bfseries References & NodeClass & BrowseName & DataType & Type\-Definition & {Modeling\-Rule} \\
\multicolumn{6}{|l|}{Subtype of AuxiliariesType (See section \ref{type:AuxiliariesType})} \\
\end{tabu}
\end{table} 


\FloatBarrier
\subsubsection{Defintion of \texttt{ LoaderType}}
  \label{type:LoaderType}

\FloatBarrier
\begin{table}[ht]
\centering 
  \caption{\texttt{LoaderType} Definition}
  \label{table:LoaderType}
\fontsize{9pt}{11pt}\selectfont
\tabulinesep=3pt
\begin{tabu} to 6in {|X[-1.35]|X[-0.7]|X[-1.75]|X[-1.5]|X[-1]|X[-0.7]|} \everyrow{\hline}
\hline
\rowfont\bfseries {Attribute} & \multicolumn{5}{|l|}{Value} \\
\tabucline[1.5pt]{}
BrowseName & \multicolumn{5}{|l|}{LoaderType} \\
IsAbstract & \multicolumn{5}{|l|}{False} \\
\tabucline[1.5pt]{}
\rowfont \bfseries References & NodeClass & BrowseName & DataType & Type\-Definition & {Modeling\-Rule} \\
\multicolumn{6}{|l|}{Subtype of AuxiliariesType (See section \ref{type:AuxiliariesType})} \\
\end{tabu}
\end{table} 


\FloatBarrier
\subsubsection{Defintion of \texttt{ SensorType}}
  \label{type:SensorType}

\FloatBarrier
\begin{table}[ht]
\centering 
  \caption{\texttt{SensorType} Definition}
  \label{table:SensorType}
\fontsize{9pt}{11pt}\selectfont
\tabulinesep=3pt
\begin{tabu} to 6in {|X[-1.35]|X[-0.7]|X[-1.75]|X[-1.5]|X[-1]|X[-0.7]|} \everyrow{\hline}
\hline
\rowfont\bfseries {Attribute} & \multicolumn{5}{|l|}{Value} \\
\tabucline[1.5pt]{}
BrowseName & \multicolumn{5}{|l|}{SensorType} \\
IsAbstract & \multicolumn{5}{|l|}{False} \\
\tabucline[1.5pt]{}
\rowfont \bfseries References & NodeClass & BrowseName & DataType & Type\-Definition & {Modeling\-Rule} \\
\multicolumn{6}{|l|}{Subtype of AuxiliariesType (See section \ref{type:AuxiliariesType})} \\
\end{tabu}
\end{table} 


\FloatBarrier
\subsubsection{Defintion of \texttt{ ToolingDeliveryType}}
  \label{type:ToolingDeliveryType}

\FloatBarrier
\begin{table}[ht]
\centering 
  \caption{\texttt{ToolingDeliveryType} Definition}
  \label{table:ToolingDeliveryType}
\fontsize{9pt}{11pt}\selectfont
\tabulinesep=3pt
\begin{tabu} to 6in {|X[-1.35]|X[-0.7]|X[-1.75]|X[-1.5]|X[-1]|X[-0.7]|} \everyrow{\hline}
\hline
\rowfont\bfseries {Attribute} & \multicolumn{5}{|l|}{Value} \\
\tabucline[1.5pt]{}
BrowseName & \multicolumn{5}{|l|}{ToolingDeliveryType} \\
IsAbstract & \multicolumn{5}{|l|}{False} \\
\tabucline[1.5pt]{}
\rowfont \bfseries References & NodeClass & BrowseName & DataType & Type\-Definition & {Modeling\-Rule} \\
\multicolumn{6}{|l|}{Subtype of AuxiliariesType (See section \ref{type:AuxiliariesType})} \\
\end{tabu}
\end{table} 


\FloatBarrier
\subsubsection{Defintion of \texttt{ WasteDisposalType}}
  \label{type:WasteDisposalType}

\FloatBarrier
\begin{table}[ht]
\centering 
  \caption{\texttt{WasteDisposalType} Definition}
  \label{table:WasteDisposalType}
\fontsize{9pt}{11pt}\selectfont
\tabulinesep=3pt
\begin{tabu} to 6in {|X[-1.35]|X[-0.7]|X[-1.75]|X[-1.5]|X[-1]|X[-0.7]|} \everyrow{\hline}
\hline
\rowfont\bfseries {Attribute} & \multicolumn{5}{|l|}{Value} \\
\tabucline[1.5pt]{}
BrowseName & \multicolumn{5}{|l|}{WasteDisposalType} \\
IsAbstract & \multicolumn{5}{|l|}{False} \\
\tabucline[1.5pt]{}
\rowfont \bfseries References & NodeClass & BrowseName & DataType & Type\-Definition & {Modeling\-Rule} \\
\multicolumn{6}{|l|}{Subtype of AuxiliariesType (See section \ref{type:AuxiliariesType})} \\
\end{tabu}
\end{table} 


\FloatBarrier
\subsubsection{Defintion of \texttt{ AxesType}}
  \label{type:AxesType}

\FloatBarrier
\begin{table}[ht]
\centering 
  \caption{\texttt{AxesType} Definition}
  \label{table:AxesType}
\fontsize{9pt}{11pt}\selectfont
\tabulinesep=3pt
\begin{tabu} to 6in {|X[-1.35]|X[-0.7]|X[-1.75]|X[-1.5]|X[-1]|X[-0.7]|} \everyrow{\hline}
\hline
\rowfont\bfseries {Attribute} & \multicolumn{5}{|l|}{Value} \\
\tabucline[1.5pt]{}
BrowseName & \multicolumn{5}{|l|}{AxesType} \\
IsAbstract & \multicolumn{5}{|l|}{False} \\
\tabucline[1.5pt]{}
\rowfont \bfseries References & NodeClass & BrowseName & DataType & Type\-Definition & {Modeling\-Rule} \\
\multicolumn{6}{|l|}{Subtype of MTComponentType (See Components Documentation)} \\
HasSubtype & ObjectType & \multicolumn{2}{l}{LinearType} & \multicolumn{2}{|l|}{See section \ref{type:LinearType}} \\
HasSubtype & ObjectType & \multicolumn{2}{l}{RotaryType} & \multicolumn{2}{|l|}{See section \ref{type:RotaryType}} \\
\end{tabu}
\end{table} 


\FloatBarrier
\subsubsection{Defintion of \texttt{ LinearType}}
  \label{type:LinearType}

\FloatBarrier
\begin{table}[ht]
\centering 
  \caption{\texttt{LinearType} Definition}
  \label{table:LinearType}
\fontsize{9pt}{11pt}\selectfont
\tabulinesep=3pt
\begin{tabu} to 6in {|X[-1.35]|X[-0.7]|X[-1.75]|X[-1.5]|X[-1]|X[-0.7]|} \everyrow{\hline}
\hline
\rowfont\bfseries {Attribute} & \multicolumn{5}{|l|}{Value} \\
\tabucline[1.5pt]{}
BrowseName & \multicolumn{5}{|l|}{LinearType} \\
IsAbstract & \multicolumn{5}{|l|}{False} \\
\tabucline[1.5pt]{}
\rowfont \bfseries References & NodeClass & BrowseName & DataType & Type\-Definition & {Modeling\-Rule} \\
\multicolumn{6}{|l|}{Subtype of AxesType (See section \ref{type:AxesType})} \\
\end{tabu}
\end{table} 


\FloatBarrier
\subsubsection{Defintion of \texttt{ RotaryType}}
  \label{type:RotaryType}

\FloatBarrier
\begin{table}[ht]
\centering 
  \caption{\texttt{RotaryType} Definition}
  \label{table:RotaryType}
\fontsize{9pt}{11pt}\selectfont
\tabulinesep=3pt
\begin{tabu} to 6in {|X[-1.35]|X[-0.7]|X[-1.75]|X[-1.5]|X[-1]|X[-0.7]|} \everyrow{\hline}
\hline
\rowfont\bfseries {Attribute} & \multicolumn{5}{|l|}{Value} \\
\tabucline[1.5pt]{}
BrowseName & \multicolumn{5}{|l|}{RotaryType} \\
IsAbstract & \multicolumn{5}{|l|}{False} \\
\tabucline[1.5pt]{}
\rowfont \bfseries References & NodeClass & BrowseName & DataType & Type\-Definition & {Modeling\-Rule} \\
\multicolumn{6}{|l|}{Subtype of AxesType (See section \ref{type:AxesType})} \\
HasSubtype & ObjectType & \multicolumn{2}{l}{ChuckType} & \multicolumn{2}{|l|}{See section \ref{type:ChuckType}} \\
\end{tabu}
\end{table} 


\FloatBarrier
\subsubsection{Defintion of \texttt{ ChuckType}}
  \label{type:ChuckType}

\FloatBarrier
\begin{table}[ht]
\centering 
  \caption{\texttt{ChuckType} Definition}
  \label{table:ChuckType}
\fontsize{9pt}{11pt}\selectfont
\tabulinesep=3pt
\begin{tabu} to 6in {|X[-1.35]|X[-0.7]|X[-1.75]|X[-1.5]|X[-1]|X[-0.7]|} \everyrow{\hline}
\hline
\rowfont\bfseries {Attribute} & \multicolumn{5}{|l|}{Value} \\
\tabucline[1.5pt]{}
BrowseName & \multicolumn{5}{|l|}{ChuckType} \\
IsAbstract & \multicolumn{5}{|l|}{False} \\
\tabucline[1.5pt]{}
\rowfont \bfseries References & NodeClass & BrowseName & DataType & Type\-Definition & {Modeling\-Rule} \\
\multicolumn{6}{|l|}{Subtype of RotaryType (See section \ref{type:RotaryType})} \\
\end{tabu}
\end{table} 


\FloatBarrier
\subsubsection{Defintion of \texttt{ ControllerType}}
  \label{type:ControllerType}

\FloatBarrier
\begin{table}[ht]
\centering 
  \caption{\texttt{ControllerType} Definition}
  \label{table:ControllerType}
\fontsize{9pt}{11pt}\selectfont
\tabulinesep=3pt
\begin{tabu} to 6in {|X[-1.35]|X[-0.7]|X[-1.75]|X[-1.5]|X[-1]|X[-0.7]|} \everyrow{\hline}
\hline
\rowfont\bfseries {Attribute} & \multicolumn{5}{|l|}{Value} \\
\tabucline[1.5pt]{}
BrowseName & \multicolumn{5}{|l|}{ControllerType} \\
IsAbstract & \multicolumn{5}{|l|}{False} \\
\tabucline[1.5pt]{}
\rowfont \bfseries References & NodeClass & BrowseName & DataType & Type\-Definition & {Modeling\-Rule} \\
\multicolumn{6}{|l|}{Subtype of MTComponentType (See Components Documentation)} \\
HasSubtype & ObjectType & \multicolumn{2}{l}{PathType} & \multicolumn{2}{|l|}{See section \ref{type:PathType}} \\
\end{tabu}
\end{table} 


\FloatBarrier
\subsubsection{Defintion of \texttt{ PathType}}
  \label{type:PathType}

\FloatBarrier
\begin{table}[ht]
\centering 
  \caption{\texttt{PathType} Definition}
  \label{table:PathType}
\fontsize{9pt}{11pt}\selectfont
\tabulinesep=3pt
\begin{tabu} to 6in {|X[-1.35]|X[-0.7]|X[-1.75]|X[-1.5]|X[-1]|X[-0.7]|} \everyrow{\hline}
\hline
\rowfont\bfseries {Attribute} & \multicolumn{5}{|l|}{Value} \\
\tabucline[1.5pt]{}
BrowseName & \multicolumn{5}{|l|}{PathType} \\
IsAbstract & \multicolumn{5}{|l|}{False} \\
\tabucline[1.5pt]{}
\rowfont \bfseries References & NodeClass & BrowseName & DataType & Type\-Definition & {Modeling\-Rule} \\
\multicolumn{6}{|l|}{Subtype of ControllerType (See section \ref{type:ControllerType})} \\
\end{tabu}
\end{table} 


\FloatBarrier
\subsubsection{Defintion of \texttt{ DoorType}}
  \label{type:DoorType}

\FloatBarrier
\begin{table}[ht]
\centering 
  \caption{\texttt{DoorType} Definition}
  \label{table:DoorType}
\fontsize{9pt}{11pt}\selectfont
\tabulinesep=3pt
\begin{tabu} to 6in {|X[-1.35]|X[-0.7]|X[-1.75]|X[-1.5]|X[-1]|X[-0.7]|} \everyrow{\hline}
\hline
\rowfont\bfseries {Attribute} & \multicolumn{5}{|l|}{Value} \\
\tabucline[1.5pt]{}
BrowseName & \multicolumn{5}{|l|}{DoorType} \\
IsAbstract & \multicolumn{5}{|l|}{False} \\
\tabucline[1.5pt]{}
\rowfont \bfseries References & NodeClass & BrowseName & DataType & Type\-Definition & {Modeling\-Rule} \\
\multicolumn{6}{|l|}{Subtype of MTComponentType (See Components Documentation)} \\
\end{tabu}
\end{table} 


\FloatBarrier
\subsubsection{Defintion of \texttt{ InterfacesType}}
  \label{type:InterfacesType}

\FloatBarrier
\begin{table}[ht]
\centering 
  \caption{\texttt{InterfacesType} Definition}
  \label{table:InterfacesType}
\fontsize{9pt}{11pt}\selectfont
\tabulinesep=3pt
\begin{tabu} to 6in {|X[-1.35]|X[-0.7]|X[-1.75]|X[-1.5]|X[-1]|X[-0.7]|} \everyrow{\hline}
\hline
\rowfont\bfseries {Attribute} & \multicolumn{5}{|l|}{Value} \\
\tabucline[1.5pt]{}
BrowseName & \multicolumn{5}{|l|}{InterfacesType} \\
IsAbstract & \multicolumn{5}{|l|}{False} \\
\tabucline[1.5pt]{}
\rowfont \bfseries References & NodeClass & BrowseName & DataType & Type\-Definition & {Modeling\-Rule} \\
\multicolumn{6}{|l|}{Subtype of MTComponentType (See Components Documentation)} \\
HasSubtype & ObjectType & \multicolumn{2}{l}{BarFeederInterfaceType} & \multicolumn{2}{|l|}{See section \ref{type:BarFeederInterfaceType}} \\
HasSubtype & ObjectType & \multicolumn{2}{l}{ChuckInterfaceType} & \multicolumn{2}{|l|}{See section \ref{type:ChuckInterfaceType}} \\
HasSubtype & ObjectType & \multicolumn{2}{l}{DoorInterfaceType} & \multicolumn{2}{|l|}{See section \ref{type:DoorInterfaceType}} \\
HasSubtype & ObjectType & \multicolumn{2}{l}{MaterialHandlerInterfaceType} & \multicolumn{2}{|l|}{See section \ref{type:MaterialHandlerInterfaceType}} \\
\end{tabu}
\end{table} 


\FloatBarrier
\subsubsection{Defintion of \texttt{ BarFeederInterfaceType}}
  \label{type:BarFeederInterfaceType}

\FloatBarrier
\begin{table}[ht]
\centering 
  \caption{\texttt{BarFeederInterfaceType} Definition}
  \label{table:BarFeederInterfaceType}
\fontsize{9pt}{11pt}\selectfont
\tabulinesep=3pt
\begin{tabu} to 6in {|X[-1.35]|X[-0.7]|X[-1.75]|X[-1.5]|X[-1]|X[-0.7]|} \everyrow{\hline}
\hline
\rowfont\bfseries {Attribute} & \multicolumn{5}{|l|}{Value} \\
\tabucline[1.5pt]{}
BrowseName & \multicolumn{5}{|l|}{BarFeederInterfaceType} \\
IsAbstract & \multicolumn{5}{|l|}{False} \\
\tabucline[1.5pt]{}
\rowfont \bfseries References & NodeClass & BrowseName & DataType & Type\-Definition & {Modeling\-Rule} \\
\multicolumn{6}{|l|}{Subtype of InterfacesType (See section \ref{type:InterfacesType})} \\
\end{tabu}
\end{table} 


\FloatBarrier
\subsubsection{Defintion of \texttt{ ChuckInterfaceType}}
  \label{type:ChuckInterfaceType}

\FloatBarrier
\begin{table}[ht]
\centering 
  \caption{\texttt{ChuckInterfaceType} Definition}
  \label{table:ChuckInterfaceType}
\fontsize{9pt}{11pt}\selectfont
\tabulinesep=3pt
\begin{tabu} to 6in {|X[-1.35]|X[-0.7]|X[-1.75]|X[-1.5]|X[-1]|X[-0.7]|} \everyrow{\hline}
\hline
\rowfont\bfseries {Attribute} & \multicolumn{5}{|l|}{Value} \\
\tabucline[1.5pt]{}
BrowseName & \multicolumn{5}{|l|}{ChuckInterfaceType} \\
IsAbstract & \multicolumn{5}{|l|}{False} \\
\tabucline[1.5pt]{}
\rowfont \bfseries References & NodeClass & BrowseName & DataType & Type\-Definition & {Modeling\-Rule} \\
\multicolumn{6}{|l|}{Subtype of InterfacesType (See section \ref{type:InterfacesType})} \\
\end{tabu}
\end{table} 


\FloatBarrier
\subsubsection{Defintion of \texttt{ DoorInterfaceType}}
  \label{type:DoorInterfaceType}

\FloatBarrier
\begin{table}[ht]
\centering 
  \caption{\texttt{DoorInterfaceType} Definition}
  \label{table:DoorInterfaceType}
\fontsize{9pt}{11pt}\selectfont
\tabulinesep=3pt
\begin{tabu} to 6in {|X[-1.35]|X[-0.7]|X[-1.75]|X[-1.5]|X[-1]|X[-0.7]|} \everyrow{\hline}
\hline
\rowfont\bfseries {Attribute} & \multicolumn{5}{|l|}{Value} \\
\tabucline[1.5pt]{}
BrowseName & \multicolumn{5}{|l|}{DoorInterfaceType} \\
IsAbstract & \multicolumn{5}{|l|}{False} \\
\tabucline[1.5pt]{}
\rowfont \bfseries References & NodeClass & BrowseName & DataType & Type\-Definition & {Modeling\-Rule} \\
\multicolumn{6}{|l|}{Subtype of InterfacesType (See section \ref{type:InterfacesType})} \\
\end{tabu}
\end{table} 


\FloatBarrier
\subsubsection{Defintion of \texttt{ MaterialHandlerInterfaceType}}
  \label{type:MaterialHandlerInterfaceType}

\FloatBarrier
\begin{table}[ht]
\centering 
  \caption{\texttt{MaterialHandlerInterfaceType} Definition}
  \label{table:MaterialHandlerInterfaceType}
\fontsize{9pt}{11pt}\selectfont
\tabulinesep=3pt
\begin{tabu} to 6in {|X[-1.35]|X[-0.7]|X[-1.75]|X[-1.5]|X[-1]|X[-0.7]|} \everyrow{\hline}
\hline
\rowfont\bfseries {Attribute} & \multicolumn{5}{|l|}{Value} \\
\tabucline[1.5pt]{}
BrowseName & \multicolumn{5}{|l|}{MaterialHandlerInterfaceType} \\
IsAbstract & \multicolumn{5}{|l|}{False} \\
\tabucline[1.5pt]{}
\rowfont \bfseries References & NodeClass & BrowseName & DataType & Type\-Definition & {Modeling\-Rule} \\
\multicolumn{6}{|l|}{Subtype of InterfacesType (See section \ref{type:InterfacesType})} \\
\end{tabu}
\end{table} 


\FloatBarrier
\subsubsection{Defintion of \texttt{ ResourcesType}}
  \label{type:ResourcesType}

\FloatBarrier
\begin{table}[ht]
\centering 
  \caption{\texttt{ResourcesType} Definition}
  \label{table:ResourcesType}
\fontsize{9pt}{11pt}\selectfont
\tabulinesep=3pt
\begin{tabu} to 6in {|X[-1.35]|X[-0.7]|X[-1.75]|X[-1.5]|X[-1]|X[-0.7]|} \everyrow{\hline}
\hline
\rowfont\bfseries {Attribute} & \multicolumn{5}{|l|}{Value} \\
\tabucline[1.5pt]{}
BrowseName & \multicolumn{5}{|l|}{ResourcesType} \\
IsAbstract & \multicolumn{5}{|l|}{False} \\
\tabucline[1.5pt]{}
\rowfont \bfseries References & NodeClass & BrowseName & DataType & Type\-Definition & {Modeling\-Rule} \\
\multicolumn{6}{|l|}{Subtype of MTComponentType (See Components Documentation)} \\
HasSubtype & ObjectType & \multicolumn{2}{l}{MaterialsType} & \multicolumn{2}{|l|}{See section \ref{type:MaterialsType}} \\
HasSubtype & ObjectType & \multicolumn{2}{l}{PersonnelType} & \multicolumn{2}{|l|}{See section \ref{type:PersonnelType}} \\
\end{tabu}
\end{table} 


\FloatBarrier
\subsubsection{Defintion of \texttt{ MaterialsType}}
  \label{type:MaterialsType}

\FloatBarrier
\begin{table}[ht]
\centering 
  \caption{\texttt{MaterialsType} Definition}
  \label{table:MaterialsType}
\fontsize{9pt}{11pt}\selectfont
\tabulinesep=3pt
\begin{tabu} to 6in {|X[-1.35]|X[-0.7]|X[-1.75]|X[-1.5]|X[-1]|X[-0.7]|} \everyrow{\hline}
\hline
\rowfont\bfseries {Attribute} & \multicolumn{5}{|l|}{Value} \\
\tabucline[1.5pt]{}
BrowseName & \multicolumn{5}{|l|}{MaterialsType} \\
IsAbstract & \multicolumn{5}{|l|}{False} \\
\tabucline[1.5pt]{}
\rowfont \bfseries References & NodeClass & BrowseName & DataType & Type\-Definition & {Modeling\-Rule} \\
\multicolumn{6}{|l|}{Subtype of ResourcesType (See section \ref{type:ResourcesType})} \\
HasSubtype & ObjectType & \multicolumn{2}{l}{StockType} & \multicolumn{2}{|l|}{See section \ref{type:StockType}} \\
\end{tabu}
\end{table} 


\FloatBarrier
\subsubsection{Defintion of \texttt{ StockType}}
  \label{type:StockType}

\FloatBarrier
\begin{table}[ht]
\centering 
  \caption{\texttt{StockType} Definition}
  \label{table:StockType}
\fontsize{9pt}{11pt}\selectfont
\tabulinesep=3pt
\begin{tabu} to 6in {|X[-1.35]|X[-0.7]|X[-1.75]|X[-1.5]|X[-1]|X[-0.7]|} \everyrow{\hline}
\hline
\rowfont\bfseries {Attribute} & \multicolumn{5}{|l|}{Value} \\
\tabucline[1.5pt]{}
BrowseName & \multicolumn{5}{|l|}{StockType} \\
IsAbstract & \multicolumn{5}{|l|}{False} \\
\tabucline[1.5pt]{}
\rowfont \bfseries References & NodeClass & BrowseName & DataType & Type\-Definition & {Modeling\-Rule} \\
\multicolumn{6}{|l|}{Subtype of MaterialsType (See section \ref{type:MaterialsType})} \\
\end{tabu}
\end{table} 


\FloatBarrier
\subsubsection{Defintion of \texttt{ PersonnelType}}
  \label{type:PersonnelType}

\FloatBarrier
\begin{table}[ht]
\centering 
  \caption{\texttt{PersonnelType} Definition}
  \label{table:PersonnelType}
\fontsize{9pt}{11pt}\selectfont
\tabulinesep=3pt
\begin{tabu} to 6in {|X[-1.35]|X[-0.7]|X[-1.75]|X[-1.5]|X[-1]|X[-0.7]|} \everyrow{\hline}
\hline
\rowfont\bfseries {Attribute} & \multicolumn{5}{|l|}{Value} \\
\tabucline[1.5pt]{}
BrowseName & \multicolumn{5}{|l|}{PersonnelType} \\
IsAbstract & \multicolumn{5}{|l|}{False} \\
\tabucline[1.5pt]{}
\rowfont \bfseries References & NodeClass & BrowseName & DataType & Type\-Definition & {Modeling\-Rule} \\
\multicolumn{6}{|l|}{Subtype of ResourcesType (See section \ref{type:ResourcesType})} \\
\end{tabu}
\end{table} 


\FloatBarrier
\subsubsection{Defintion of \texttt{ SystemsType}}
  \label{type:SystemsType}

\FloatBarrier
\begin{table}[ht]
\centering 
  \caption{\texttt{SystemsType} Definition}
  \label{table:SystemsType}
\fontsize{9pt}{11pt}\selectfont
\tabulinesep=3pt
\begin{tabu} to 6in {|X[-1.35]|X[-0.7]|X[-1.75]|X[-1.5]|X[-1]|X[-0.7]|} \everyrow{\hline}
\hline
\rowfont\bfseries {Attribute} & \multicolumn{5}{|l|}{Value} \\
\tabucline[1.5pt]{}
BrowseName & \multicolumn{5}{|l|}{SystemsType} \\
IsAbstract & \multicolumn{5}{|l|}{False} \\
\tabucline[1.5pt]{}
\rowfont \bfseries References & NodeClass & BrowseName & DataType & Type\-Definition & {Modeling\-Rule} \\
\multicolumn{6}{|l|}{Subtype of MTComponentType (See Components Documentation)} \\
HasSubtype & ObjectType & \multicolumn{2}{l}{CoolantType} & \multicolumn{2}{|l|}{See section \ref{type:CoolantType}} \\
HasSubtype & ObjectType & \multicolumn{2}{l}{DielectricType} & \multicolumn{2}{|l|}{See section \ref{type:DielectricType}} \\
HasSubtype & ObjectType & \multicolumn{2}{l}{ElectricType} & \multicolumn{2}{|l|}{See section \ref{type:ElectricType}} \\
HasSubtype & ObjectType & \multicolumn{2}{l}{EnclosureType} & \multicolumn{2}{|l|}{See section \ref{type:EnclosureType}} \\
HasSubtype & ObjectType & \multicolumn{2}{l}{FeederType} & \multicolumn{2}{|l|}{See section \ref{type:FeederType}} \\
HasSubtype & ObjectType & \multicolumn{2}{l}{HydraulicType} & \multicolumn{2}{|l|}{See section \ref{type:HydraulicType}} \\
HasSubtype & ObjectType & \multicolumn{2}{l}{LubricationType} & \multicolumn{2}{|l|}{See section \ref{type:LubricationType}} \\
HasSubtype & ObjectType & \multicolumn{2}{l}{PneumaticType} & \multicolumn{2}{|l|}{See section \ref{type:PneumaticType}} \\
HasSubtype & ObjectType & \multicolumn{2}{l}{ProcessPowerType} & \multicolumn{2}{|l|}{See section \ref{type:ProcessPowerType}} \\
HasSubtype & ObjectType & \multicolumn{2}{l}{ProtectiveType} & \multicolumn{2}{|l|}{See section \ref{type:ProtectiveType}} \\
\end{tabu}
\end{table} 


\FloatBarrier
\subsubsection{Defintion of \texttt{ CoolantType}}
  \label{type:CoolantType}

\FloatBarrier
\begin{table}[ht]
\centering 
  \caption{\texttt{CoolantType} Definition}
  \label{table:CoolantType}
\fontsize{9pt}{11pt}\selectfont
\tabulinesep=3pt
\begin{tabu} to 6in {|X[-1.35]|X[-0.7]|X[-1.75]|X[-1.5]|X[-1]|X[-0.7]|} \everyrow{\hline}
\hline
\rowfont\bfseries {Attribute} & \multicolumn{5}{|l|}{Value} \\
\tabucline[1.5pt]{}
BrowseName & \multicolumn{5}{|l|}{CoolantType} \\
IsAbstract & \multicolumn{5}{|l|}{False} \\
\tabucline[1.5pt]{}
\rowfont \bfseries References & NodeClass & BrowseName & DataType & Type\-Definition & {Modeling\-Rule} \\
\multicolumn{6}{|l|}{Subtype of SystemsType (See section \ref{type:SystemsType})} \\
\end{tabu}
\end{table} 


\FloatBarrier
\subsubsection{Defintion of \texttt{ DielectricType}}
  \label{type:DielectricType}

\FloatBarrier
\begin{table}[ht]
\centering 
  \caption{\texttt{DielectricType} Definition}
  \label{table:DielectricType}
\fontsize{9pt}{11pt}\selectfont
\tabulinesep=3pt
\begin{tabu} to 6in {|X[-1.35]|X[-0.7]|X[-1.75]|X[-1.5]|X[-1]|X[-0.7]|} \everyrow{\hline}
\hline
\rowfont\bfseries {Attribute} & \multicolumn{5}{|l|}{Value} \\
\tabucline[1.5pt]{}
BrowseName & \multicolumn{5}{|l|}{DielectricType} \\
IsAbstract & \multicolumn{5}{|l|}{False} \\
\tabucline[1.5pt]{}
\rowfont \bfseries References & NodeClass & BrowseName & DataType & Type\-Definition & {Modeling\-Rule} \\
\multicolumn{6}{|l|}{Subtype of SystemsType (See section \ref{type:SystemsType})} \\
\end{tabu}
\end{table} 


\FloatBarrier
\subsubsection{Defintion of \texttt{ ElectricType}}
  \label{type:ElectricType}

\FloatBarrier
\begin{table}[ht]
\centering 
  \caption{\texttt{ElectricType} Definition}
  \label{table:ElectricType}
\fontsize{9pt}{11pt}\selectfont
\tabulinesep=3pt
\begin{tabu} to 6in {|X[-1.35]|X[-0.7]|X[-1.75]|X[-1.5]|X[-1]|X[-0.7]|} \everyrow{\hline}
\hline
\rowfont\bfseries {Attribute} & \multicolumn{5}{|l|}{Value} \\
\tabucline[1.5pt]{}
BrowseName & \multicolumn{5}{|l|}{ElectricType} \\
IsAbstract & \multicolumn{5}{|l|}{False} \\
\tabucline[1.5pt]{}
\rowfont \bfseries References & NodeClass & BrowseName & DataType & Type\-Definition & {Modeling\-Rule} \\
\multicolumn{6}{|l|}{Subtype of SystemsType (See section \ref{type:SystemsType})} \\
\end{tabu}
\end{table} 


\FloatBarrier
\subsubsection{Defintion of \texttt{ EnclosureType}}
  \label{type:EnclosureType}

\FloatBarrier
\begin{table}[ht]
\centering 
  \caption{\texttt{EnclosureType} Definition}
  \label{table:EnclosureType}
\fontsize{9pt}{11pt}\selectfont
\tabulinesep=3pt
\begin{tabu} to 6in {|X[-1.35]|X[-0.7]|X[-1.75]|X[-1.5]|X[-1]|X[-0.7]|} \everyrow{\hline}
\hline
\rowfont\bfseries {Attribute} & \multicolumn{5}{|l|}{Value} \\
\tabucline[1.5pt]{}
BrowseName & \multicolumn{5}{|l|}{EnclosureType} \\
IsAbstract & \multicolumn{5}{|l|}{False} \\
\tabucline[1.5pt]{}
\rowfont \bfseries References & NodeClass & BrowseName & DataType & Type\-Definition & {Modeling\-Rule} \\
\multicolumn{6}{|l|}{Subtype of SystemsType (See section \ref{type:SystemsType})} \\
\end{tabu}
\end{table} 


\FloatBarrier
\subsubsection{Defintion of \texttt{ FeederType}}
  \label{type:FeederType}

\FloatBarrier
\begin{table}[ht]
\centering 
  \caption{\texttt{FeederType} Definition}
  \label{table:FeederType}
\fontsize{9pt}{11pt}\selectfont
\tabulinesep=3pt
\begin{tabu} to 6in {|X[-1.35]|X[-0.7]|X[-1.75]|X[-1.5]|X[-1]|X[-0.7]|} \everyrow{\hline}
\hline
\rowfont\bfseries {Attribute} & \multicolumn{5}{|l|}{Value} \\
\tabucline[1.5pt]{}
BrowseName & \multicolumn{5}{|l|}{FeederType} \\
IsAbstract & \multicolumn{5}{|l|}{False} \\
\tabucline[1.5pt]{}
\rowfont \bfseries References & NodeClass & BrowseName & DataType & Type\-Definition & {Modeling\-Rule} \\
\multicolumn{6}{|l|}{Subtype of SystemsType (See section \ref{type:SystemsType})} \\
\end{tabu}
\end{table} 


\FloatBarrier
\subsubsection{Defintion of \texttt{ HydraulicType}}
  \label{type:HydraulicType}

\FloatBarrier
\begin{table}[ht]
\centering 
  \caption{\texttt{HydraulicType} Definition}
  \label{table:HydraulicType}
\fontsize{9pt}{11pt}\selectfont
\tabulinesep=3pt
\begin{tabu} to 6in {|X[-1.35]|X[-0.7]|X[-1.75]|X[-1.5]|X[-1]|X[-0.7]|} \everyrow{\hline}
\hline
\rowfont\bfseries {Attribute} & \multicolumn{5}{|l|}{Value} \\
\tabucline[1.5pt]{}
BrowseName & \multicolumn{5}{|l|}{HydraulicType} \\
IsAbstract & \multicolumn{5}{|l|}{False} \\
\tabucline[1.5pt]{}
\rowfont \bfseries References & NodeClass & BrowseName & DataType & Type\-Definition & {Modeling\-Rule} \\
\multicolumn{6}{|l|}{Subtype of SystemsType (See section \ref{type:SystemsType})} \\
\end{tabu}
\end{table} 


\FloatBarrier
\subsubsection{Defintion of \texttt{ LubricationType}}
  \label{type:LubricationType}

\FloatBarrier
\begin{table}[ht]
\centering 
  \caption{\texttt{LubricationType} Definition}
  \label{table:LubricationType}
\fontsize{9pt}{11pt}\selectfont
\tabulinesep=3pt
\begin{tabu} to 6in {|X[-1.35]|X[-0.7]|X[-1.75]|X[-1.5]|X[-1]|X[-0.7]|} \everyrow{\hline}
\hline
\rowfont\bfseries {Attribute} & \multicolumn{5}{|l|}{Value} \\
\tabucline[1.5pt]{}
BrowseName & \multicolumn{5}{|l|}{LubricationType} \\
IsAbstract & \multicolumn{5}{|l|}{False} \\
\tabucline[1.5pt]{}
\rowfont \bfseries References & NodeClass & BrowseName & DataType & Type\-Definition & {Modeling\-Rule} \\
\multicolumn{6}{|l|}{Subtype of SystemsType (See section \ref{type:SystemsType})} \\
\end{tabu}
\end{table} 


\FloatBarrier
\subsubsection{Defintion of \texttt{ PneumaticType}}
  \label{type:PneumaticType}

\FloatBarrier
\begin{table}[ht]
\centering 
  \caption{\texttt{PneumaticType} Definition}
  \label{table:PneumaticType}
\fontsize{9pt}{11pt}\selectfont
\tabulinesep=3pt
\begin{tabu} to 6in {|X[-1.35]|X[-0.7]|X[-1.75]|X[-1.5]|X[-1]|X[-0.7]|} \everyrow{\hline}
\hline
\rowfont\bfseries {Attribute} & \multicolumn{5}{|l|}{Value} \\
\tabucline[1.5pt]{}
BrowseName & \multicolumn{5}{|l|}{PneumaticType} \\
IsAbstract & \multicolumn{5}{|l|}{False} \\
\tabucline[1.5pt]{}
\rowfont \bfseries References & NodeClass & BrowseName & DataType & Type\-Definition & {Modeling\-Rule} \\
\multicolumn{6}{|l|}{Subtype of SystemsType (See section \ref{type:SystemsType})} \\
\end{tabu}
\end{table} 


\FloatBarrier
\subsubsection{Defintion of \texttt{ ProcessPowerType}}
  \label{type:ProcessPowerType}

\FloatBarrier
\begin{table}[ht]
\centering 
  \caption{\texttt{ProcessPowerType} Definition}
  \label{table:ProcessPowerType}
\fontsize{9pt}{11pt}\selectfont
\tabulinesep=3pt
\begin{tabu} to 6in {|X[-1.35]|X[-0.7]|X[-1.75]|X[-1.5]|X[-1]|X[-0.7]|} \everyrow{\hline}
\hline
\rowfont\bfseries {Attribute} & \multicolumn{5}{|l|}{Value} \\
\tabucline[1.5pt]{}
BrowseName & \multicolumn{5}{|l|}{ProcessPowerType} \\
IsAbstract & \multicolumn{5}{|l|}{False} \\
\tabucline[1.5pt]{}
\rowfont \bfseries References & NodeClass & BrowseName & DataType & Type\-Definition & {Modeling\-Rule} \\
\multicolumn{6}{|l|}{Subtype of SystemsType (See section \ref{type:SystemsType})} \\
\end{tabu}
\end{table} 


\FloatBarrier
\subsubsection{Defintion of \texttt{ ProtectiveType}}
  \label{type:ProtectiveType}

\FloatBarrier
\begin{table}[ht]
\centering 
  \caption{\texttt{ProtectiveType} Definition}
  \label{table:ProtectiveType}
\fontsize{9pt}{11pt}\selectfont
\tabulinesep=3pt
\begin{tabu} to 6in {|X[-1.35]|X[-0.7]|X[-1.75]|X[-1.5]|X[-1]|X[-0.7]|} \everyrow{\hline}
\hline
\rowfont\bfseries {Attribute} & \multicolumn{5}{|l|}{Value} \\
\tabucline[1.5pt]{}
BrowseName & \multicolumn{5}{|l|}{ProtectiveType} \\
IsAbstract & \multicolumn{5}{|l|}{False} \\
\tabucline[1.5pt]{}
\rowfont \bfseries References & NodeClass & BrowseName & DataType & Type\-Definition & {Modeling\-Rule} \\
\multicolumn{6}{|l|}{Subtype of SystemsType (See section \ref{type:SystemsType})} \\
\end{tabu}
\end{table} 


\FloatBarrier
\subsection{Data Items} \label{model:DataItems}

\begin{figure}[ht]
  \centering
    \includegraphics[width=1.0\textwidth]{./diagrams/types/DataItems.png}
  \caption{Data Items Diagram}
  \label{fig:DataItems}
\end{figure}

\FloatBarrier


\input ./type-sections/DataItems.tex

\subsubsection{Defintion of \texttt{ AssetEventDataType}}
  \label{type:AssetEventDataType}

\FloatBarrier
\begin{table}[ht]
\centering 
  \caption{\texttt{AssetEventDataType} DataType}
  \label{data-type:AssetEventDataType}
\tabulinesep=3pt
\begin{tabu} to 6in {|l|l|l|} \everyrow{\hline}
\hline
\rowfont\bfseries {Field} & {Type} & {Optional} \\
\tabucline[1.5pt]{}
\texttt{AssetId} & \texttt{String} & \texttt{Mandatory} \\
\texttt{AssetType} & \texttt{String} & \texttt{Mandatory} \\
\end{tabu}
\end{table} 

\FloatBarrier
\subsubsection{Defintion of \texttt{ MTAssetEventType}}
  \label{type:MTAssetEventType}

\FloatBarrier
\begin{table}[ht]
\centering 
  \caption{\texttt{MTAssetEventType} Definition}
  \label{table:MTAssetEventType}
\fontsize{9pt}{11pt}\selectfont
\tabulinesep=3pt
\begin{tabu} to 6in {|X[-1.35]|X[-0.7]|X[-1.75]|X[-1.5]|X[-1]|X[-0.7]|} \everyrow{\hline}
\hline
\rowfont\bfseries {Attribute} & \multicolumn{5}{|l|}{Value} \\
\tabucline[1.5pt]{}
BrowseName & \multicolumn{5}{|l|}{MTAssetEventType} \\
IsAbstract & \multicolumn{5}{|l|}{False} \\
ValueRank & \multicolumn{5}{|l|}{} \\
DataType & \multicolumn{5}{|l|}{AssetEventDataType} \\
\tabucline[1.5pt]{}
\rowfont \bfseries References & NodeClass & BrowseName & DataType & Type\-Definition & {Modeling\-Rule} \\
\multicolumn{6}{|l|}{Subtype of BaseDataVariableType (See \cite{UAPart08} Documentation)} \\
Has\-Property & Variable & Category & MT\-Category\-Type & Property\-Type & Mandatory \\
Has\-Property & Variable & MT\-Sub\-Type\-Name & String & Property\-Type & Optional \\
Has\-Property & Variable & MT\-Type\-Name & String & Property\-Type & Mandatory \\
Has\-Property & Variable & Name & String & Property\-Type & Optional \\
Has\-Property & Variable & Period\-Filter & Float & Property\-Type & Optional \\
Has\-Property & Variable & Representation & MT\-Representation\-Type & Property\-Type & Optional \\
Has\-Property & Variable & Sample\-Rate & Double & Property\-Type & Optional \\
Has\-Property & Variable & Source\-Data & String & Property\-Type & Optional \\
Has\-Property & Variable & Xml\-Id & String & Property\-Type & Mandatory \\
Has\-MT\-Source & Object & <Base\-Object> & \multicolumn{2}{l|}{BaseObjectType} & Optional \\
Has\-MT\-Composition & Object & <MT\-Composition> & \multicolumn{2}{l|}{MTCompositionType} & Optional \\
Has\-MT\-Sub\-Class\-Type & Object & <MT\-Data\-Item\-Sub\-Class> & \multicolumn{2}{l|}{MTDataItemSubClassType} & Optional \\
Has\-Condition & Event & <MT\-Condition> & \multicolumn{2}{l|}{MTConditionType} & Optional \\
Has\-Component & Object & Constraints & \multicolumn{2}{l|}{MTConstraintType} & Optional \\
Has\-MT\-Class\-Type & Object & <MT\-Data\-Item\-Class> & \multicolumn{2}{l|}{MTDataItemClassType} & Mandatory \\
\end{tabu}
\end{table} 


\paragraph{Dependencies and Relationships}

\begin{itemize}
\item Mixes in \texttt{MTDataItemType}, see See section \ref{type:MTDataItemType}
\end{itemize}
\FloatBarrier
\subsubsection{Defintion of \texttt{ MTConditionClassType}}
  \label{type:MTConditionClassType}

\FloatBarrier
\begin{table}[ht]
\centering 
  \caption{\texttt{MTConditionClassType} Definition}
  \label{table:MTConditionClassType}
\fontsize{9pt}{11pt}\selectfont
\tabulinesep=3pt
\begin{tabu} to 6in {|X[-1.35]|X[-0.7]|X[-1.75]|X[-1.5]|X[-1]|X[-0.7]|} \everyrow{\hline}
\hline
\rowfont\bfseries {Attribute} & \multicolumn{5}{|l|}{Value} \\
\tabucline[1.5pt]{}
BrowseName & \multicolumn{5}{|l|}{MTConditionClassType} \\
IsAbstract & \multicolumn{5}{|l|}{True} \\
\tabucline[1.5pt]{}
\rowfont \bfseries References & NodeClass & BrowseName & DataType & Type\-Definition & {Modeling\-Rule} \\
\multicolumn{6}{|l|}{Subtype of MTDataItemClassType (See Data Item Types Documentation)} \\
HasSubtype & ObjectType & \multicolumn{2}{l}{ActuatorClassType} & \multicolumn{2}{|l|}{See section \ref{type:ActuatorClassType}} \\
HasSubtype & ObjectType & \multicolumn{2}{l}{CommunicationsClassType} & \multicolumn{2}{|l|}{See section \ref{type:CommunicationsClassType}} \\
HasSubtype & ObjectType & \multicolumn{2}{l}{DataRangeClassType} & \multicolumn{2}{|l|}{See section \ref{type:DataRangeClassType}} \\
HasSubtype & ObjectType & \multicolumn{2}{l}{HardwareClassType} & \multicolumn{2}{|l|}{See section \ref{type:HardwareClassType}} \\
HasSubtype & ObjectType & \multicolumn{2}{l}{LogicProgramClassType} & \multicolumn{2}{|l|}{See section \ref{type:LogicProgramClassType}} \\
HasSubtype & ObjectType & \multicolumn{2}{l}{MotionProgramClassType} & \multicolumn{2}{|l|}{See section \ref{type:MotionProgramClassType}} \\
HasSubtype & ObjectType & \multicolumn{2}{l}{SystemClassType} & \multicolumn{2}{|l|}{See section \ref{type:SystemClassType}} \\
\end{tabu}
\end{table} 


\FloatBarrier
\subsubsection{Defintion of \texttt{ MTConstraintType}}
  \label{type:MTConstraintType}

\FloatBarrier
\begin{table}[ht]
\centering 
  \caption{\texttt{MTConstraintType} Definition}
  \label{table:MTConstraintType}
\fontsize{9pt}{11pt}\selectfont
\tabulinesep=3pt
\begin{tabu} to 6in {|X[-1.35]|X[-0.7]|X[-1.75]|X[-1.5]|X[-1]|X[-0.7]|} \everyrow{\hline}
\hline
\rowfont\bfseries {Attribute} & \multicolumn{5}{|l|}{Value} \\
\tabucline[1.5pt]{}
BrowseName & \multicolumn{5}{|l|}{MTConstraintType} \\
IsAbstract & \multicolumn{5}{|l|}{False} \\
\tabucline[1.5pt]{}
\rowfont \bfseries References & NodeClass & BrowseName & DataType & Type\-Definition & {Modeling\-Rule} \\
Has\-Property & Variable & Maximum & Float & Property\-Type & Optional \\
Has\-Property & Variable & Minimum & Float & Property\-Type & Optional \\
Has\-Property & Variable & Nominal & Float & Property\-Type & Optional \\
Has\-Property & Variable & Values & String[] & Property\-Type & Optional \\
\end{tabu}
\end{table} 


\FloatBarrier
\subsubsection{Defintion of \texttt{ MTControlledVocabEventType}}
  \label{type:MTControlledVocabEventType}

\FloatBarrier
\begin{table}[ht]
\centering 
  \caption{\texttt{MTControlledVocabEventType} Definition}
  \label{table:MTControlledVocabEventType}
\fontsize{9pt}{11pt}\selectfont
\tabulinesep=3pt
\begin{tabu} to 6in {|X[-1.35]|X[-0.7]|X[-1.75]|X[-1.5]|X[-1]|X[-0.7]|} \everyrow{\hline}
\hline
\rowfont\bfseries {Attribute} & \multicolumn{5}{|l|}{Value} \\
\tabucline[1.5pt]{}
BrowseName & \multicolumn{5}{|l|}{MTControlledVocabEventType} \\
IsAbstract & \multicolumn{5}{|l|}{False} \\
ValueRank & \multicolumn{5}{|l|}{} \\
DataType & \multicolumn{5}{|l|}{UInteger} \\
\tabucline[1.5pt]{}
\rowfont \bfseries References & NodeClass & BrowseName & DataType & Type\-Definition & {Modeling\-Rule} \\
\multicolumn{6}{|l|}{Subtype of MultiStateDiscreteType (See \cite{UAPart08} Documentation)} \\
Has\-Property & Variable & Category & MT\-Category\-Type & Property\-Type & Mandatory \\
Has\-Property & Variable & MT\-Sub\-Type\-Name & String & Property\-Type & Optional \\
Has\-Property & Variable & MT\-Type\-Name & String & Property\-Type & Mandatory \\
Has\-Property & Variable & Name & String & Property\-Type & Optional \\
Has\-Property & Variable & Period\-Filter & Float & Property\-Type & Optional \\
Has\-Property & Variable & Representation & MT\-Representation\-Type & Property\-Type & Optional \\
Has\-Property & Variable & Sample\-Rate & Double & Property\-Type & Optional \\
Has\-Property & Variable & Source\-Data & String & Property\-Type & Optional \\
Has\-Property & Variable & Xml\-Id & String & Property\-Type & Mandatory \\
Has\-MT\-Source & Object & <Base\-Object> & \multicolumn{2}{l|}{BaseObjectType} & Optional \\
Has\-MT\-Composition & Object & <MT\-Composition> & \multicolumn{2}{l|}{MTCompositionType} & Optional \\
Has\-MT\-Sub\-Class\-Type & Object & <MT\-Data\-Item\-Sub\-Class> & \multicolumn{2}{l|}{MTDataItemSubClassType} & Optional \\
Has\-Condition & Event & <MT\-Condition> & \multicolumn{2}{l|}{MTConditionType} & Optional \\
Has\-Component & Object & Constraints & \multicolumn{2}{l|}{MTConstraintType} & Optional \\
Has\-MT\-Class\-Type & Object & <MT\-Data\-Item\-Class> & \multicolumn{2}{l|}{MTDataItemClassType} & Mandatory \\
Has\-Property & Variable & Value\-As\-Text & String & Property\-Type & Mandatory \\
\end{tabu}
\end{table} 


\paragraph{Dependencies and Relationships}

\begin{itemize}
\item Mixes in \texttt{MTDataItemType}, see See section \ref{type:MTDataItemType}
\end{itemize}
\FloatBarrier
\subsubsection{Defintion of \texttt{<<mixin>> MTDataItemType}}
  \label{type:MTDataItemType}

\FloatBarrier
\begin{table}[ht]
\centering 
  \caption{\texttt{MTDataItemType} Definition}
  \label{table:MTDataItemType}
\fontsize{9pt}{11pt}\selectfont
\tabulinesep=3pt
\begin{tabu} to 6in {|X[-1.35]|X[-0.7]|X[-1.75]|X[-1.5]|X[-1]|X[-0.7]|} \everyrow{\hline}
\hline
\rowfont\bfseries {Attribute} & \multicolumn{5}{|l|}{Value} \\
\tabucline[1.5pt]{}
BrowseName & \multicolumn{5}{|l|}{MTDataItemType} \\
IsAbstract & \multicolumn{5}{|l|}{False} \\
\tabucline[1.5pt]{}
\rowfont \bfseries References & NodeClass & BrowseName & DataType & Type\-Definition & {Modeling\-Rule} \\
HasSubtype & ObjectType & \multicolumn{2}{l}{MTNumericDataItemType} & \multicolumn{2}{|l|}{See section \ref{type:MTNumericDataItemType}} \\
Has\-Property & Variable & Category & MT\-Category\-Type & Property\-Type & Mandatory \\
Has\-Property & Variable & MT\-Sub\-Type\-Name & String & Property\-Type & Optional \\
Has\-Property & Variable & MT\-Type\-Name & String & Property\-Type & Mandatory \\
Has\-Property & Variable & Name & String & Property\-Type & Optional \\
Has\-Property & Variable & Period\-Filter & Float & Property\-Type & Optional \\
Has\-Property & Variable & Representation & MT\-Representation\-Type & Property\-Type & Optional \\
Has\-Property & Variable & Sample\-Rate & Double & Property\-Type & Optional \\
Has\-Property & Variable & Source\-Data & String & Property\-Type & Optional \\
Has\-Property & Variable & Xml\-Id & String & Property\-Type & Mandatory \\
Has\-MT\-Source & Object & <Base\-Object> & \multicolumn{2}{l|}{BaseObjectType} & Optional \\
Has\-MT\-Composition & Object & <MT\-Composition> & \multicolumn{2}{l|}{MTCompositionType} & Optional \\
Has\-MT\-Sub\-Class\-Type & Object & <MT\-Data\-Item\-Sub\-Class> & \multicolumn{2}{l|}{MTDataItemSubClassType} & Optional \\
Has\-Condition & Event & <MT\-Condition> & \multicolumn{2}{l|}{MTConditionType} & Optional \\
Has\-Component & Object & Constraints & \multicolumn{2}{l|}{MTConstraintType} & Optional \\
Has\-MT\-Class\-Type & Object & <MT\-Data\-Item\-Class> & \multicolumn{2}{l|}{MTDataItemClassType} & Mandatory \\
\end{tabu}
\end{table} 


\FloatBarrier
\paragraph{Referenced Properties and Objects}

\begin{itemize}
\item \textbf{Allowable Values} for \texttt{MTCategoryType}
\FloatBarrier
\begin{table}[ht]
\centering 
  \caption{\texttt{MTCategoryType} Enumeration}
  \label{enum:MTCategoryType}
\tabulinesep=3pt
\begin{tabu} to 6in {|l|r|} \everyrow{\hline}
\hline
\rowfont\bfseries {Name} & {Index} \\
\tabucline[1.5pt]{}
\texttt{EVENT} & \texttt{0} \\
\texttt{CONDITION} & \texttt{1} \\
\texttt{SAMPLE} & \texttt{2} \\
\end{tabu}
\end{table} 
\FloatBarrier
\item \textbf{Allowable Values} for \texttt{MTRepresentationType}
\FloatBarrier
\begin{table}[ht]
\centering 
  \caption{\texttt{MTRepresentationType} Enumeration}
  \label{enum:MTRepresentationType}
\tabulinesep=3pt
\begin{tabu} to 6in {|l|r|} \everyrow{\hline}
\hline
\rowfont\bfseries {Name} & {Index} \\
\tabucline[1.5pt]{}
\texttt{DISCRETE} & \texttt{0} \\
\texttt{TIME_SERIES} & \texttt{1} \\
\texttt{VALUE} & \texttt{2} \\
\end{tabu}
\end{table} 
\FloatBarrier
\end{itemize}
\FloatBarrier
\subsubsection{Defintion of \texttt{<<mixin>> MTNumericDataItemType}}
  \label{type:MTNumericDataItemType}

\FloatBarrier
\begin{table}[ht]
\centering 
  \caption{\texttt{MTNumericDataItemType} Definition}
  \label{table:MTNumericDataItemType}
\fontsize{9pt}{11pt}\selectfont
\tabulinesep=3pt
\begin{tabu} to 6in {|X[-1.35]|X[-0.7]|X[-1.75]|X[-1.5]|X[-1]|X[-0.7]|} \everyrow{\hline}
\hline
\rowfont\bfseries {Attribute} & \multicolumn{5}{|l|}{Value} \\
\tabucline[1.5pt]{}
BrowseName & \multicolumn{5}{|l|}{MTNumericDataItemType} \\
IsAbstract & \multicolumn{5}{|l|}{False} \\
\tabucline[1.5pt]{}
\rowfont \bfseries References & NodeClass & BrowseName & DataType & Type\-Definition & {Modeling\-Rule} \\
\multicolumn{6}{|l|}{Subtype of MTDataItemType (See section \ref{type:MTDataItemType})} \\
Has\-Property & Variable & Coordinate\-System & MT\-Coordinate\-System\-Type & Property\-Type & Optional \\
Has\-Property & Variable & Initial\-Value & Double & Property\-Type & Optional \\
Has\-Property & Variable & Minimum\-Delta\-Filter & Float & Property\-Type & Optional \\
Has\-Property & Variable & Native\-Units & String & Property\-Type & Optional \\
Has\-Property & Variable & Reset\-Trigger & MT\-Reset\-Trigger\-Type & Property\-Type & Optional \\
Has\-Property & Variable & Significant\-Digits & Int16 & Property\-Type & Optional \\
Has\-Property & Variable & Statistic & MT\-Statistic\-Type & Property\-Type & Optional \\
Has\-Property & Variable & Units & String & Property\-Type & Optional \\
\end{tabu}
\end{table} 


\FloatBarrier
\paragraph{Referenced Properties and Objects}

\begin{itemize}
\item \textbf{Allowable Values} for \texttt{MTCoordinateSystemType}
\FloatBarrier
\begin{table}[ht]
\centering 
  \caption{\texttt{MTCoordinateSystemType} Enumeration}
  \label{enum:MTCoordinateSystemType}
\tabulinesep=3pt
\begin{tabu} to 6in {|l|r|} \everyrow{\hline}
\hline
\rowfont\bfseries {Name} & {Index} \\
\tabucline[1.5pt]{}
\texttt{MACHINE} & \texttt{0} \\
\texttt{WORK} & \texttt{1} \\
\end{tabu}
\end{table} 
\FloatBarrier
\item \textbf{Allowable Values} for \texttt{MTResetTriggerType}
\FloatBarrier
\begin{table}[ht]
\centering 
  \caption{\texttt{MTResetTriggerType} Enumeration}
  \label{enum:MTResetTriggerType}
\tabulinesep=3pt
\begin{tabu} to 6in {|l|r|} \everyrow{\hline}
\hline
\rowfont\bfseries {Name} & {Index} \\
\tabucline[1.5pt]{}
\texttt{ACTION_COMPLETE} & \texttt{0} \\
\texttt{ANNUAL} & \texttt{1} \\
\texttt{DAY} & \texttt{2} \\
\texttt{MAINTENANCE} & \texttt{3} \\
\texttt{MANUAL} & \texttt{4} \\
\texttt{MONTH} & \texttt{5} \\
\texttt{POWER_ON} & \texttt{6} \\
\texttt{SHIFT} & \texttt{7} \\
\texttt{WEEK} & \texttt{8} \\
\end{tabu}
\end{table} 
\FloatBarrier
\item \textbf{Allowable Values} for \texttt{MTStatisticType}
\FloatBarrier
\begin{table}[ht]
\centering 
  \caption{\texttt{MTStatisticType} Enumeration}
  \label{enum:MTStatisticType}
\tabulinesep=3pt
\begin{tabu} to 6in {|l|r|} \everyrow{\hline}
\hline
\rowfont\bfseries {Name} & {Index} \\
\tabucline[1.5pt]{}
\texttt{AVERAGE} & \texttt{0} \\
\texttt{MAXIMUM} & \texttt{1} \\
\texttt{MEDIAN} & \texttt{2} \\
\texttt{MINIMUM} & \texttt{3} \\
\texttt{MODE} & \texttt{4} \\
\texttt{RANGE} & \texttt{5} \\
\texttt{ROOT_MEAN_SQUARE} & \texttt{6} \\
\texttt{STANDARD_DEVIATION} & \texttt{7} \\
\end{tabu}
\end{table} 
\FloatBarrier
\end{itemize}
\FloatBarrier
\subsubsection{Defintion of \texttt{ MTEventClassType}}
  \label{type:MTEventClassType}

\FloatBarrier
\begin{table}[ht]
\centering 
  \caption{\texttt{MTEventClassType} Definition}
  \label{table:MTEventClassType}
\fontsize{9pt}{11pt}\selectfont
\tabulinesep=3pt
\begin{tabu} to 6in {|X[-1.35]|X[-0.7]|X[-1.75]|X[-1.5]|X[-1]|X[-0.7]|} \everyrow{\hline}
\hline
\rowfont\bfseries {Attribute} & \multicolumn{5}{|l|}{Value} \\
\tabucline[1.5pt]{}
BrowseName & \multicolumn{5}{|l|}{MTEventClassType} \\
IsAbstract & \multicolumn{5}{|l|}{True} \\
\tabucline[1.5pt]{}
\rowfont \bfseries References & NodeClass & BrowseName & DataType & Type\-Definition & {Modeling\-Rule} \\
\multicolumn{6}{|l|}{Subtype of MTDataItemClassType (See Data Item Types Documentation)} \\
HasSubtype & ObjectType & \multicolumn{2}{l}{MTMessageClassType} & \multicolumn{2}{|l|}{See section \ref{type:MTMessageClassType}} \\
HasSubtype & ObjectType & \multicolumn{2}{l}{MTControlledVocabEventClassType} & \multicolumn{2}{|l|}{See section \ref{type:MTControlledVocabEventClassType}} \\
HasSubtype & ObjectType & \multicolumn{2}{l}{MTNumericEventClassType} & \multicolumn{2}{|l|}{See section \ref{type:MTNumericEventClassType}} \\
HasSubtype & ObjectType & \multicolumn{2}{l}{MTStringEventClassType} & \multicolumn{2}{|l|}{See section \ref{type:MTStringEventClassType}} \\
\end{tabu}
\end{table} 


\FloatBarrier
\subsubsection{Defintion of \texttt{ MTMessageEventType}}
  \label{type:MTMessageEventType}

\FloatBarrier
\begin{table}[ht]
\centering 
  \caption{\texttt{MTMessageEventType} Definition}
  \label{table:MTMessageEventType}
\fontsize{9pt}{11pt}\selectfont
\tabulinesep=3pt
\begin{tabu} to 6in {|X[-1.35]|X[-0.7]|X[-1.75]|X[-1.5]|X[-1]|X[-0.7]|} \everyrow{\hline}
\hline
\rowfont\bfseries {Attribute} & \multicolumn{5}{|l|}{Value} \\
\tabucline[1.5pt]{}
BrowseName & \multicolumn{5}{|l|}{MTMessageEventType} \\
IsAbstract & \multicolumn{5}{|l|}{False} \\
\tabucline[1.5pt]{}
\rowfont \bfseries References & NodeClass & BrowseName & DataType & Type\-Definition & {Modeling\-Rule} \\
\multicolumn{6}{|l|}{Subtype of BaseEventType (See \cite{UAPart05} Documentation)} \\
Has\-Property & Variable & Native\-Code & String & Property\-Type & Optional \\
\end{tabu}
\end{table} 


\FloatBarrier
\subsubsection{Defintion of \texttt{ MTMessageType}}
  \label{type:MTMessageType}

\FloatBarrier
\begin{table}[ht]
\centering 
  \caption{\texttt{MTMessageType} Definition}
  \label{table:MTMessageType}
\fontsize{9pt}{11pt}\selectfont
\tabulinesep=3pt
\begin{tabu} to 6in {|X[-1.35]|X[-0.7]|X[-1.75]|X[-1.5]|X[-1]|X[-0.7]|} \everyrow{\hline}
\hline
\rowfont\bfseries {Attribute} & \multicolumn{5}{|l|}{Value} \\
\tabucline[1.5pt]{}
BrowseName & \multicolumn{5}{|l|}{MTMessageType} \\
IsAbstract & \multicolumn{5}{|l|}{False} \\
ValueRank & \multicolumn{5}{|l|}{} \\
DataType & \multicolumn{5}{|l|}{MessageDataType} \\
\tabucline[1.5pt]{}
\rowfont \bfseries References & NodeClass & BrowseName & DataType & Type\-Definition & {Modeling\-Rule} \\
\multicolumn{6}{|l|}{Subtype of BaseDataVariableType (See \cite{UAPart08} Documentation)} \\
Has\-Property & Variable & Category & MT\-Category\-Type & Property\-Type & Mandatory \\
Has\-Property & Variable & MT\-Sub\-Type\-Name & String & Property\-Type & Optional \\
Has\-Property & Variable & MT\-Type\-Name & String & Property\-Type & Mandatory \\
Has\-Property & Variable & Name & String & Property\-Type & Optional \\
Has\-Property & Variable & Period\-Filter & Float & Property\-Type & Optional \\
Has\-Property & Variable & Representation & MT\-Representation\-Type & Property\-Type & Optional \\
Has\-Property & Variable & Sample\-Rate & Double & Property\-Type & Optional \\
Has\-Property & Variable & Source\-Data & String & Property\-Type & Optional \\
Has\-Property & Variable & Xml\-Id & String & Property\-Type & Mandatory \\
Has\-MT\-Source & Object & <Base\-Object> & \multicolumn{2}{l|}{BaseObjectType} & Optional \\
Has\-MT\-Composition & Object & <MT\-Composition> & \multicolumn{2}{l|}{MTCompositionType} & Optional \\
Has\-MT\-Sub\-Class\-Type & Object & <MT\-Data\-Item\-Sub\-Class> & \multicolumn{2}{l|}{MTDataItemSubClassType} & Optional \\
Has\-Condition & Event & <MT\-Condition> & \multicolumn{2}{l|}{MTConditionType} & Optional \\
Has\-Component & Object & Constraints & \multicolumn{2}{l|}{MTConstraintType} & Optional \\
Has\-MT\-Class\-Type & Object & <MT\-Data\-Item\-Class> & \multicolumn{2}{l|}{MTDataItemClassType} & Mandatory \\
\end{tabu}
\end{table} 


\paragraph{Dependencies and Relationships}

\begin{itemize}
\item Mixes in \texttt{MTDataItemType}, see See section \ref{type:MTDataItemType}
\end{itemize}
\FloatBarrier
\subsubsection{Defintion of \texttt{ MTNumericEventType}}
  \label{type:MTNumericEventType}

\FloatBarrier
\begin{table}[ht]
\centering 
  \caption{\texttt{MTNumericEventType} Definition}
  \label{table:MTNumericEventType}
\fontsize{9pt}{11pt}\selectfont
\tabulinesep=3pt
\begin{tabu} to 6in {|X[-1.35]|X[-0.7]|X[-1.75]|X[-1.5]|X[-1]|X[-0.7]|} \everyrow{\hline}
\hline
\rowfont\bfseries {Attribute} & \multicolumn{5}{|l|}{Value} \\
\tabucline[1.5pt]{}
BrowseName & \multicolumn{5}{|l|}{MTNumericEventType} \\
IsAbstract & \multicolumn{5}{|l|}{False} \\
ValueRank & \multicolumn{5}{|l|}{} \\
DataType & \multicolumn{5}{|l|}{Number} \\
\tabucline[1.5pt]{}
\rowfont \bfseries References & NodeClass & BrowseName & DataType & Type\-Definition & {Modeling\-Rule} \\
\multicolumn{6}{|l|}{Subtype of DataItemType (See \cite{UAPart08} Documentation)} \\
Has\-Property & Variable & Category & MT\-Category\-Type & Property\-Type & Mandatory \\
Has\-Property & Variable & MT\-Sub\-Type\-Name & String & Property\-Type & Optional \\
Has\-Property & Variable & MT\-Type\-Name & String & Property\-Type & Mandatory \\
Has\-Property & Variable & Name & String & Property\-Type & Optional \\
Has\-Property & Variable & Period\-Filter & Float & Property\-Type & Optional \\
Has\-Property & Variable & Representation & MT\-Representation\-Type & Property\-Type & Optional \\
Has\-Property & Variable & Sample\-Rate & Double & Property\-Type & Optional \\
Has\-Property & Variable & Source\-Data & String & Property\-Type & Optional \\
Has\-Property & Variable & Xml\-Id & String & Property\-Type & Mandatory \\
Has\-MT\-Source & Object & <Base\-Object> & \multicolumn{2}{l|}{BaseObjectType} & Optional \\
Has\-MT\-Composition & Object & <MT\-Composition> & \multicolumn{2}{l|}{MTCompositionType} & Optional \\
Has\-MT\-Sub\-Class\-Type & Object & <MT\-Data\-Item\-Sub\-Class> & \multicolumn{2}{l|}{MTDataItemSubClassType} & Optional \\
Has\-Condition & Event & <MT\-Condition> & \multicolumn{2}{l|}{MTConditionType} & Optional \\
Has\-Component & Object & Constraints & \multicolumn{2}{l|}{MTConstraintType} & Optional \\
Has\-MT\-Class\-Type & Object & <MT\-Data\-Item\-Class> & \multicolumn{2}{l|}{MTDataItemClassType} & Mandatory \\
Has\-Property & Variable & Coordinate\-System & MT\-Coordinate\-System\-Type & Property\-Type & Optional \\
Has\-Property & Variable & Initial\-Value & Double & Property\-Type & Optional \\
Has\-Property & Variable & Minimum\-Delta\-Filter & Float & Property\-Type & Optional \\
Has\-Property & Variable & Native\-Units & String & Property\-Type & Optional \\
Has\-Property & Variable & Reset\-Trigger & MT\-Reset\-Trigger\-Type & Property\-Type & Optional \\
Has\-Property & Variable & Significant\-Digits & Int16 & Property\-Type & Optional \\
Has\-Property & Variable & Statistic & MT\-Statistic\-Type & Property\-Type & Optional \\
Has\-Property & Variable & Units & String & Property\-Type & Optional \\
\end{tabu}
\end{table} 


\paragraph{Dependencies and Relationships}

\begin{itemize}
\item Mixes in \texttt{MTNumericDataItemType}, see See section \ref{type:MTNumericDataItemType}
\end{itemize}
\FloatBarrier
\subsubsection{Defintion of \texttt{ MTSampleType}}
  \label{type:MTSampleType}

\FloatBarrier
\begin{table}[ht]
\centering 
  \caption{\texttt{MTSampleType} Definition}
  \label{table:MTSampleType}
\fontsize{9pt}{11pt}\selectfont
\tabulinesep=3pt
\begin{tabu} to 6in {|X[-1.35]|X[-0.7]|X[-1.75]|X[-1.5]|X[-1]|X[-0.7]|} \everyrow{\hline}
\hline
\rowfont\bfseries {Attribute} & \multicolumn{5}{|l|}{Value} \\
\tabucline[1.5pt]{}
BrowseName & \multicolumn{5}{|l|}{MTSampleType} \\
IsAbstract & \multicolumn{5}{|l|}{False} \\
ValueRank & \multicolumn{5}{|l|}{} \\
DataType & \multicolumn{5}{|l|}{Number} \\
\tabucline[1.5pt]{}
\rowfont \bfseries References & NodeClass & BrowseName & DataType & Type\-Definition & {Modeling\-Rule} \\
\multicolumn{6}{|l|}{Subtype of AnalogUnitType (See \cite{UAAmend01} Documentation)} \\
Has\-Property & Variable & Category & MT\-Category\-Type & Property\-Type & Mandatory \\
Has\-Property & Variable & MT\-Sub\-Type\-Name & String & Property\-Type & Optional \\
Has\-Property & Variable & MT\-Type\-Name & String & Property\-Type & Mandatory \\
Has\-Property & Variable & Name & String & Property\-Type & Optional \\
Has\-Property & Variable & Period\-Filter & Float & Property\-Type & Optional \\
Has\-Property & Variable & Representation & MT\-Representation\-Type & Property\-Type & Optional \\
Has\-Property & Variable & Sample\-Rate & Double & Property\-Type & Optional \\
Has\-Property & Variable & Source\-Data & String & Property\-Type & Optional \\
Has\-Property & Variable & Xml\-Id & String & Property\-Type & Mandatory \\
Has\-MT\-Source & Object & <Base\-Object> & \multicolumn{2}{l|}{BaseObjectType} & Optional \\
Has\-MT\-Composition & Object & <MT\-Composition> & \multicolumn{2}{l|}{MTCompositionType} & Optional \\
Has\-MT\-Sub\-Class\-Type & Object & <MT\-Data\-Item\-Sub\-Class> & \multicolumn{2}{l|}{MTDataItemSubClassType} & Optional \\
Has\-Condition & Event & <MT\-Condition> & \multicolumn{2}{l|}{MTConditionType} & Optional \\
Has\-Component & Object & Constraints & \multicolumn{2}{l|}{MTConstraintType} & Optional \\
Has\-MT\-Class\-Type & Object & <MT\-Data\-Item\-Class> & \multicolumn{2}{l|}{MTDataItemClassType} & Mandatory \\
Has\-Property & Variable & Coordinate\-System & MT\-Coordinate\-System\-Type & Property\-Type & Optional \\
Has\-Property & Variable & Initial\-Value & Double & Property\-Type & Optional \\
Has\-Property & Variable & Minimum\-Delta\-Filter & Float & Property\-Type & Optional \\
Has\-Property & Variable & Native\-Units & String & Property\-Type & Optional \\
Has\-Property & Variable & Reset\-Trigger & MT\-Reset\-Trigger\-Type & Property\-Type & Optional \\
Has\-Property & Variable & Significant\-Digits & Int16 & Property\-Type & Optional \\
Has\-Property & Variable & Statistic & MT\-Statistic\-Type & Property\-Type & Optional \\
Has\-Property & Variable & Units & String & Property\-Type & Optional \\
\end{tabu}
\end{table} 


\paragraph{Dependencies and Relationships}

\begin{itemize}
\item Mixes in \texttt{MTNumericDataItemType}, see See section \ref{type:MTNumericDataItemType}
\end{itemize}
\FloatBarrier
\subsubsection{Defintion of \texttt{ MTStringEventType}}
  \label{type:MTStringEventType}

\FloatBarrier
\begin{table}[ht]
\centering 
  \caption{\texttt{MTStringEventType} Definition}
  \label{table:MTStringEventType}
\fontsize{9pt}{11pt}\selectfont
\tabulinesep=3pt
\begin{tabu} to 6in {|X[-1.35]|X[-0.7]|X[-1.75]|X[-1.5]|X[-1]|X[-0.7]|} \everyrow{\hline}
\hline
\rowfont\bfseries {Attribute} & \multicolumn{5}{|l|}{Value} \\
\tabucline[1.5pt]{}
BrowseName & \multicolumn{5}{|l|}{MTStringEventType} \\
IsAbstract & \multicolumn{5}{|l|}{False} \\
ValueRank & \multicolumn{5}{|l|}{} \\
DataType & \multicolumn{5}{|l|}{String} \\
\tabucline[1.5pt]{}
\rowfont \bfseries References & NodeClass & BrowseName & DataType & Type\-Definition & {Modeling\-Rule} \\
\multicolumn{6}{|l|}{Subtype of BaseDataVariableType (See \cite{UAPart08} Documentation)} \\
Has\-Property & Variable & Category & MT\-Category\-Type & Property\-Type & Mandatory \\
Has\-Property & Variable & MT\-Sub\-Type\-Name & String & Property\-Type & Optional \\
Has\-Property & Variable & MT\-Type\-Name & String & Property\-Type & Mandatory \\
Has\-Property & Variable & Name & String & Property\-Type & Optional \\
Has\-Property & Variable & Period\-Filter & Float & Property\-Type & Optional \\
Has\-Property & Variable & Representation & MT\-Representation\-Type & Property\-Type & Optional \\
Has\-Property & Variable & Sample\-Rate & Double & Property\-Type & Optional \\
Has\-Property & Variable & Source\-Data & String & Property\-Type & Optional \\
Has\-Property & Variable & Xml\-Id & String & Property\-Type & Mandatory \\
Has\-MT\-Source & Object & <Base\-Object> & \multicolumn{2}{l|}{BaseObjectType} & Optional \\
Has\-MT\-Composition & Object & <MT\-Composition> & \multicolumn{2}{l|}{MTCompositionType} & Optional \\
Has\-MT\-Sub\-Class\-Type & Object & <MT\-Data\-Item\-Sub\-Class> & \multicolumn{2}{l|}{MTDataItemSubClassType} & Optional \\
Has\-Condition & Event & <MT\-Condition> & \multicolumn{2}{l|}{MTConditionType} & Optional \\
Has\-Component & Object & Constraints & \multicolumn{2}{l|}{MTConstraintType} & Optional \\
Has\-MT\-Class\-Type & Object & <MT\-Data\-Item\-Class> & \multicolumn{2}{l|}{MTDataItemClassType} & Mandatory \\
\end{tabu}
\end{table} 


\paragraph{Dependencies and Relationships}

\begin{itemize}
\item Mixes in \texttt{MTDataItemType}, see See section \ref{type:MTDataItemType}
\end{itemize}
\FloatBarrier
\subsubsection{Defintion of \texttt{ MTThreeSpaceSampleType}}
  \label{type:MTThreeSpaceSampleType}

\FloatBarrier
\begin{table}[ht]
\centering 
  \caption{\texttt{MTThreeSpaceSampleType} Definition}
  \label{table:MTThreeSpaceSampleType}
\fontsize{9pt}{11pt}\selectfont
\tabulinesep=3pt
\begin{tabu} to 6in {|X[-1.35]|X[-0.7]|X[-1.75]|X[-1.5]|X[-1]|X[-0.7]|} \everyrow{\hline}
\hline
\rowfont\bfseries {Attribute} & \multicolumn{5}{|l|}{Value} \\
\tabucline[1.5pt]{}
BrowseName & \multicolumn{5}{|l|}{MTThreeSpaceSampleType} \\
IsAbstract & \multicolumn{5}{|l|}{False} \\
ValueRank & \multicolumn{5}{|l|}{} \\
DataType & \multicolumn{5}{|l|}{ThreeSpaceSampleDataType} \\
\tabucline[1.5pt]{}
\rowfont \bfseries References & NodeClass & BrowseName & DataType & Type\-Definition & {Modeling\-Rule} \\
\multicolumn{6}{|l|}{Subtype of DataItemType (See \cite{UAPart08} Documentation)} \\
Has\-Property & Variable & Category & MT\-Category\-Type & Property\-Type & Mandatory \\
Has\-Property & Variable & MT\-Sub\-Type\-Name & String & Property\-Type & Optional \\
Has\-Property & Variable & MT\-Type\-Name & String & Property\-Type & Mandatory \\
Has\-Property & Variable & Name & String & Property\-Type & Optional \\
Has\-Property & Variable & Period\-Filter & Float & Property\-Type & Optional \\
Has\-Property & Variable & Representation & MT\-Representation\-Type & Property\-Type & Optional \\
Has\-Property & Variable & Sample\-Rate & Double & Property\-Type & Optional \\
Has\-Property & Variable & Source\-Data & String & Property\-Type & Optional \\
Has\-Property & Variable & Xml\-Id & String & Property\-Type & Mandatory \\
Has\-MT\-Source & Object & <Base\-Object> & \multicolumn{2}{l|}{BaseObjectType} & Optional \\
Has\-MT\-Composition & Object & <MT\-Composition> & \multicolumn{2}{l|}{MTCompositionType} & Optional \\
Has\-MT\-Sub\-Class\-Type & Object & <MT\-Data\-Item\-Sub\-Class> & \multicolumn{2}{l|}{MTDataItemSubClassType} & Optional \\
Has\-Condition & Event & <MT\-Condition> & \multicolumn{2}{l|}{MTConditionType} & Optional \\
Has\-Component & Object & Constraints & \multicolumn{2}{l|}{MTConstraintType} & Optional \\
Has\-MT\-Class\-Type & Object & <MT\-Data\-Item\-Class> & \multicolumn{2}{l|}{MTDataItemClassType} & Mandatory \\
Has\-Property & Variable & Coordinate\-System & MT\-Coordinate\-System\-Type & Property\-Type & Optional \\
Has\-Property & Variable & Initial\-Value & Double & Property\-Type & Optional \\
Has\-Property & Variable & Minimum\-Delta\-Filter & Float & Property\-Type & Optional \\
Has\-Property & Variable & Native\-Units & String & Property\-Type & Optional \\
Has\-Property & Variable & Reset\-Trigger & MT\-Reset\-Trigger\-Type & Property\-Type & Optional \\
Has\-Property & Variable & Significant\-Digits & Int16 & Property\-Type & Optional \\
Has\-Property & Variable & Statistic & MT\-Statistic\-Type & Property\-Type & Optional \\
Has\-Property & Variable & Units & String & Property\-Type & Optional \\
Has\-Property & Variable & Engineering\-Units & EUInformation & Property\-Type & Mandatory \\
\end{tabu}
\end{table} 


\paragraph{Dependencies and Relationships}

\begin{itemize}
\item Mixes in \texttt{MTNumericDataItemType}, see See section \ref{type:MTNumericDataItemType}
\end{itemize}
\FloatBarrier
\subsubsection{Defintion of \texttt{ MessageDataType}}
  \label{type:MessageDataType}

\FloatBarrier
\begin{table}[ht]
\centering 
  \caption{\texttt{MessageDataType} DataType}
  \label{data-type:MessageDataType}
\tabulinesep=3pt
\begin{tabu} to 6in {|l|l|l|} \everyrow{\hline}
\hline
\rowfont\bfseries {Field} & {Type} & {Optional} \\
\tabucline[1.5pt]{}
\texttt{NativeCode} & \texttt{String} & \texttt{Optional} \\
\texttt{Text} & \texttt{String} & \texttt{Mandatory} \\
\end{tabu}
\end{table} 

\FloatBarrier
\subsubsection{Defintion of \texttt{ ThreeSpaceSampleDataType}}
  \label{type:ThreeSpaceSampleDataType}

\FloatBarrier
\begin{table}[ht]
\centering 
  \caption{\texttt{ThreeSpaceSampleDataType} DataType}
  \label{data-type:ThreeSpaceSampleDataType}
\tabulinesep=3pt
\begin{tabu} to 6in {|l|l|l|} \everyrow{\hline}
\hline
\rowfont\bfseries {Field} & {Type} & {Optional} \\
\tabucline[1.5pt]{}
\texttt{X} & \texttt{Double} & \texttt{Mandatory} \\
\texttt{Y} & \texttt{Double} & \texttt{Mandatory} \\
\texttt{Z} & \texttt{Double} & \texttt{Mandatory} \\
\end{tabu}
\end{table} 

\FloatBarrier
\subsection{Conditions} \label{model:Conditions}

\begin{figure}[ht]
  \centering
    \includegraphics[width=1.0\textwidth]{./diagrams/types/Conditions.png}
  \caption{Conditions Diagram}
  \label{fig:Conditions}
\end{figure}

\FloatBarrier


\input ./type-sections/Conditions.tex

\subsubsection{Defintion of \texttt{ MTConditionType}}
  \label{type:MTConditionType}

\FloatBarrier
\begin{table}[ht]
\centering 
  \caption{\texttt{MTConditionType} Definition}
  \label{table:MTConditionType}
\fontsize{9pt}{11pt}\selectfont
\tabulinesep=3pt
\begin{tabu} to 6in {|X[-1.35]|X[-0.7]|X[-1.75]|X[-1.5]|X[-1]|X[-0.7]|} \everyrow{\hline}
\hline
\rowfont\bfseries {Attribute} & \multicolumn{5}{|l|}{Value} \\
\tabucline[1.5pt]{}
BrowseName & \multicolumn{5}{|l|}{MTConditionType} \\
IsAbstract & \multicolumn{5}{|l|}{False} \\
\tabucline[1.5pt]{}
\rowfont \bfseries References & NodeClass & BrowseName & DataType & Type\-Definition & {Modeling\-Rule} \\
\multicolumn{6}{|l|}{Subtype of ConditionType (See \cite{UAPart09} Documentation)} \\
Has\-Property & Variable & Category & MT\-Category\-Type & Property\-Type & Mandatory \\
Has\-Property & Variable & MT\-Sub\-Type\-Name & String & Property\-Type & Optional \\
Has\-Property & Variable & MT\-Type\-Name & String & Property\-Type & Mandatory \\
Has\-Property & Variable & Name & String & Property\-Type & Optional \\
Has\-Property & Variable & Period\-Filter & Float & Property\-Type & Optional \\
Has\-Property & Variable & Representation & MT\-Representation\-Type & Property\-Type & Optional \\
Has\-Property & Variable & Sample\-Rate & Double & Property\-Type & Optional \\
Has\-Property & Variable & Source\-Data & String & Property\-Type & Optional \\
Has\-Property & Variable & Xml\-Id & String & Property\-Type & Mandatory \\
Has\-MT\-Source & Object & <Base\-Object> & \multicolumn{2}{l|}{BaseObjectType} & Optional \\
Has\-MT\-Composition & Object & <MT\-Composition> & \multicolumn{2}{l|}{MTCompositionType} & Optional \\
Has\-MT\-Sub\-Class\-Type & Object & <MT\-Data\-Item\-Sub\-Class> & \multicolumn{2}{l|}{MTDataItemSubClassType} & Optional \\
Has\-Condition & Event & <MT\-Condition> & \multicolumn{2}{l|}{MTConditionType} & Optional \\
Has\-Component & Object & Constraints & \multicolumn{2}{l|}{MTConstraintType} & Optional \\
Has\-MT\-Class\-Type & Object & <MT\-Data\-Item\-Class> & \multicolumn{2}{l|}{MTDataItemClassType} & Mandatory \\
Has\-Property & Variable & MT\-Severity & MT\-Severity\-Data\-Type & Property\-Type & Mandatory \\
Has\-Property & Variable & Native\-Code & String & Property\-Type & Optional \\
Has\-Property & Variable & Native\-Severity & String & Property\-Type & Optional \\
Has\-Property & Variable & Qualifier & Qualifier\-Data\-Type & Property\-Type & Optional \\
Has\-Component & Variable & Active\-State & Localized\-Text & Two\-State\-Variable\-Type & Mandatory \\
\end{tabu}
\end{table} 


\FloatBarrier
\paragraph{Referenced Properties and Objects}

\begin{itemize}
\item \textbf{Allowable Values} for \texttt{MTSeverityDataType}
\FloatBarrier
\begin{table}[ht]
\centering 
  \caption{\texttt{MTSeverityDataType} Enumeration}
  \label{enum:MTSeverityDataType}
\tabulinesep=3pt
\begin{tabu} to 6in {|l|r|} \everyrow{\hline}
\hline
\rowfont\bfseries {Name} & {Index} \\
\tabucline[1.5pt]{}
\texttt{FAULT} & \texttt{0} \\
\texttt{NORMAL} & \texttt{1} \\
\texttt{WARNING} & \texttt{2} \\
\end{tabu}
\end{table} 
\FloatBarrier
\item \textbf{Allowable Values} for \texttt{QualifierDataType}
\FloatBarrier
\begin{table}[ht]
\centering 
  \caption{\texttt{QualifierDataType} Enumeration}
  \label{enum:QualifierDataType}
\tabulinesep=3pt
\begin{tabu} to 6in {|l|r|} \everyrow{\hline}
\hline
\rowfont\bfseries {Name} & {Index} \\
\tabucline[1.5pt]{}
\texttt{HIGH} & \texttt{0} \\
\texttt{LOW} & \texttt{1} \\
\end{tabu}
\end{table} 
\FloatBarrier
\end{itemize}
\paragraph{Dependencies and Relationships}

\begin{itemize}
\item Mixes in \texttt{MTDataItemType}, see See section \ref{type:MTDataItemType}
\end{itemize}
\FloatBarrier
\subsection{Data Item Types} \label{model:DataItemTypes}
\subsubsection{Defintion of \texttt{ MTDataItemClassType}}
  \label{type:MTDataItemClassType}

\FloatBarrier
\begin{table}[ht]
\centering 
  \caption{\texttt{MTDataItemClassType} Definition}
  \label{table:MTDataItemClassType}
\fontsize{9pt}{11pt}\selectfont
\tabulinesep=3pt
\begin{tabu} to 6in {|X[-1.35]|X[-0.7]|X[-1.75]|X[-1.5]|X[-1]|X[-0.7]|} \everyrow{\hline}
\hline
\rowfont\bfseries {Attribute} & \multicolumn{5}{|l|}{Value} \\
\tabucline[1.5pt]{}
BrowseName & \multicolumn{5}{|l|}{MTDataItemClassType} \\
IsAbstract & \multicolumn{5}{|l|}{True} \\
\tabucline[1.5pt]{}
\rowfont \bfseries References & NodeClass & BrowseName & DataType & Type\-Definition & {Modeling\-Rule} \\
\multicolumn{6}{|l|}{Subtype of BaseConditionClassType (See \cite{UAPart09} Documentation)} \\
HasSubtype & ObjectType & \multicolumn{2}{l}{MTConditionClassType} & \multicolumn{2}{|l|}{See section \ref{type:MTConditionClassType}} \\
HasSubtype & ObjectType & \multicolumn{2}{l}{MTEventClassType} & \multicolumn{2}{|l|}{See section \ref{type:MTEventClassType}} \\
HasSubtype & ObjectType & \multicolumn{2}{l}{MTSampleClassType} & \multicolumn{2}{|l|}{See section \ref{type:MTSampleClassType}} \\
\end{tabu}
\end{table} 


\FloatBarrier
\subsubsection{Defintion of \texttt{ MTMessageClassType}}
  \label{type:MTMessageClassType}

\FloatBarrier
\begin{table}[ht]
\centering 
  \caption{\texttt{MTMessageClassType} Definition}
  \label{table:MTMessageClassType}
\fontsize{9pt}{11pt}\selectfont
\tabulinesep=3pt
\begin{tabu} to 6in {|X[-1.35]|X[-0.7]|X[-1.75]|X[-1.5]|X[-1]|X[-0.7]|} \everyrow{\hline}
\hline
\rowfont\bfseries {Attribute} & \multicolumn{5}{|l|}{Value} \\
\tabucline[1.5pt]{}
BrowseName & \multicolumn{5}{|l|}{MTMessageClassType} \\
IsAbstract & \multicolumn{5}{|l|}{False} \\
\tabucline[1.5pt]{}
\rowfont \bfseries References & NodeClass & BrowseName & DataType & Type\-Definition & {Modeling\-Rule} \\
\multicolumn{6}{|l|}{Subtype of MTEventClassType (See Data Items Documentation)} \\
\end{tabu}
\end{table} 


\FloatBarrier
\subsection{Sample Data Item Types} \label{model:SampleDataItemTypes}
\subsubsection{Defintion of \texttt{ MTSampleClassType}}
  \label{type:MTSampleClassType}

\FloatBarrier
\begin{table}[ht]
\centering 
  \caption{\texttt{MTSampleClassType} Definition}
  \label{table:MTSampleClassType}
\fontsize{9pt}{11pt}\selectfont
\tabulinesep=3pt
\begin{tabu} to 6in {|X[-1.35]|X[-0.7]|X[-1.75]|X[-1.5]|X[-1]|X[-0.7]|} \everyrow{\hline}
\hline
\rowfont\bfseries {Attribute} & \multicolumn{5}{|l|}{Value} \\
\tabucline[1.5pt]{}
BrowseName & \multicolumn{5}{|l|}{MTSampleClassType} \\
IsAbstract & \multicolumn{5}{|l|}{True} \\
\tabucline[1.5pt]{}
\rowfont \bfseries References & NodeClass & BrowseName & DataType & Type\-Definition & {Modeling\-Rule} \\
\multicolumn{6}{|l|}{Subtype of MTDataItemClassType (See Data Item Types Documentation)} \\
HasSubtype & ObjectType & \multicolumn{2}{l}{MassClassType} & \multicolumn{2}{|l|}{See section \ref{type:MassClassType}} \\
HasSubtype & ObjectType & \multicolumn{2}{l}{PathFeedrateClassType} & \multicolumn{2}{|l|}{See section \ref{type:PathFeedrateClassType}} \\
HasSubtype & ObjectType & \multicolumn{2}{l}{PathPositionClassType} & \multicolumn{2}{|l|}{See section \ref{type:PathPositionClassType}} \\
HasSubtype & ObjectType & \multicolumn{2}{l}{PHClassType} & \multicolumn{2}{|l|}{See section \ref{type:PHClassType}} \\
HasSubtype & ObjectType & \multicolumn{2}{l}{PositionClassType} & \multicolumn{2}{|l|}{See section \ref{type:PositionClassType}} \\
HasSubtype & ObjectType & \multicolumn{2}{l}{PowerFactorClassType} & \multicolumn{2}{|l|}{See section \ref{type:PowerFactorClassType}} \\
HasSubtype & ObjectType & \multicolumn{2}{l}{PressureClassType} & \multicolumn{2}{|l|}{See section \ref{type:PressureClassType}} \\
HasSubtype & ObjectType & \multicolumn{2}{l}{ProcessTimerClassType} & \multicolumn{2}{|l|}{See section \ref{type:ProcessTimerClassType}} \\
HasSubtype & ObjectType & \multicolumn{2}{l}{ResistenceClassType} & \multicolumn{2}{|l|}{See section \ref{type:ResistenceClassType}} \\
HasSubtype & ObjectType & \multicolumn{2}{l}{RotaryVelocityClassType} & \multicolumn{2}{|l|}{See section \ref{type:RotaryVelocityClassType}} \\
HasSubtype & ObjectType & \multicolumn{2}{l}{SoundLevelClassType} & \multicolumn{2}{|l|}{See section \ref{type:SoundLevelClassType}} \\
HasSubtype & ObjectType & \multicolumn{2}{l}{StrainClassType} & \multicolumn{2}{|l|}{See section \ref{type:StrainClassType}} \\
HasSubtype & ObjectType & \multicolumn{2}{l}{TemperatureClassType} & \multicolumn{2}{|l|}{See section \ref{type:TemperatureClassType}} \\
HasSubtype & ObjectType & \multicolumn{2}{l}{TensionClassType} & \multicolumn{2}{|l|}{See section \ref{type:TensionClassType}} \\
HasSubtype & ObjectType & \multicolumn{2}{l}{TiltClassType} & \multicolumn{2}{|l|}{See section \ref{type:TiltClassType}} \\
HasSubtype & ObjectType & \multicolumn{2}{l}{TorqueClassType} & \multicolumn{2}{|l|}{See section \ref{type:TorqueClassType}} \\
HasSubtype & ObjectType & \multicolumn{2}{l}{VelocityClassType} & \multicolumn{2}{|l|}{See section \ref{type:VelocityClassType}} \\
HasSubtype & ObjectType & \multicolumn{2}{l}{ViscosityClassType} & \multicolumn{2}{|l|}{See section \ref{type:ViscosityClassType}} \\
HasSubtype & ObjectType & \multicolumn{2}{l}{VoltageClassType} & \multicolumn{2}{|l|}{See section \ref{type:VoltageClassType}} \\
HasSubtype & ObjectType & \multicolumn{2}{l}{VoltAmpereClassType} & \multicolumn{2}{|l|}{See section \ref{type:VoltAmpereClassType}} \\
HasSubtype & ObjectType & \multicolumn{2}{l}{VoltAmpereReactiveClassType} & \multicolumn{2}{|l|}{See section \ref{type:VoltAmpereReactiveClassType}} \\
HasSubtype & ObjectType & \multicolumn{2}{l}{WattageClassType} & \multicolumn{2}{|l|}{See section \ref{type:WattageClassType}} \\
\multicolumn{6}{|l|}{Continued...} \\
\end{tabu}
\end{table}
\begin{table}[ht]
\fontsize{9pt}{11pt}\selectfont
\tabulinesep=3pt
\begin{tabu} to 6in {|X[-1.35]|X[-0.7]|X[-1.75]|X[-1.5]|X[-1]|X[-0.7]|} \everyrow{\hline}
\hline
\rowfont \bfseries References & NodeClass & BrowseName & DataType & Type\-Definition & {Modeling\-Rule} \\
HasSubtype & ObjectType & \multicolumn{2}{l}{AccelerationClassType} & \multicolumn{2}{|l|}{See section \ref{type:AccelerationClassType}} \\
HasSubtype & ObjectType & \multicolumn{2}{l}{AccumulatedTimeClassType} & \multicolumn{2}{|l|}{See section \ref{type:AccumulatedTimeClassType}} \\
HasSubtype & ObjectType & \multicolumn{2}{l}{AmperageClassType} & \multicolumn{2}{|l|}{See section \ref{type:AmperageClassType}} \\
HasSubtype & ObjectType & \multicolumn{2}{l}{AngleClassType} & \multicolumn{2}{|l|}{See section \ref{type:AngleClassType}} \\
HasSubtype & ObjectType & \multicolumn{2}{l}{AngularAccelerationClassType} & \multicolumn{2}{|l|}{See section \ref{type:AngularAccelerationClassType}} \\
HasSubtype & ObjectType & \multicolumn{2}{l}{AngularVelocityClassType} & \multicolumn{2}{|l|}{See section \ref{type:AngularVelocityClassType}} \\
HasSubtype & ObjectType & \multicolumn{2}{l}{AxisFeedrateClassType} & \multicolumn{2}{|l|}{See section \ref{type:AxisFeedrateClassType}} \\
HasSubtype & ObjectType & \multicolumn{2}{l}{ClockTimeClassType} & \multicolumn{2}{|l|}{See section \ref{type:ClockTimeClassType}} \\
HasSubtype & ObjectType & \multicolumn{2}{l}{ConcentrationClassType} & \multicolumn{2}{|l|}{See section \ref{type:ConcentrationClassType}} \\
HasSubtype & ObjectType & \multicolumn{2}{l}{ConductivityClassType} & \multicolumn{2}{|l|}{See section \ref{type:ConductivityClassType}} \\
HasSubtype & ObjectType & \multicolumn{2}{l}{DisplacementClassType} & \multicolumn{2}{|l|}{See section \ref{type:DisplacementClassType}} \\
HasSubtype & ObjectType & \multicolumn{2}{l}{ElectricalEnergyClassType} & \multicolumn{2}{|l|}{See section \ref{type:ElectricalEnergyClassType}} \\
HasSubtype & ObjectType & \multicolumn{2}{l}{EquipmentTimerClassType} & \multicolumn{2}{|l|}{See section \ref{type:EquipmentTimerClassType}} \\
HasSubtype & ObjectType & \multicolumn{2}{l}{FillLevelClassType} & \multicolumn{2}{|l|}{See section \ref{type:FillLevelClassType}} \\
HasSubtype & ObjectType & \multicolumn{2}{l}{FlowClassType} & \multicolumn{2}{|l|}{See section \ref{type:FlowClassType}} \\
HasSubtype & ObjectType & \multicolumn{2}{l}{FrequencyClassType} & \multicolumn{2}{|l|}{See section \ref{type:FrequencyClassType}} \\
HasSubtype & ObjectType & \multicolumn{2}{l}{LengthClassType} & \multicolumn{2}{|l|}{See section \ref{type:LengthClassType}} \\
HasSubtype & ObjectType & \multicolumn{2}{l}{LinearForceClassType} & \multicolumn{2}{|l|}{See section \ref{type:LinearForceClassType}} \\
HasSubtype & ObjectType & \multicolumn{2}{l}{LoadClassType} & \multicolumn{2}{|l|}{See section \ref{type:LoadClassType}} \\
\end{tabu}
\end{table} 


\FloatBarrier
\subsubsection{Defintion of \texttt{ AccelerationClassType}}
  \label{type:AccelerationClassType}

\FloatBarrier
\begin{table}[ht]
\centering 
  \caption{\texttt{AccelerationClassType} Definition}
  \label{table:AccelerationClassType}
\fontsize{9pt}{11pt}\selectfont
\tabulinesep=3pt
\begin{tabu} to 6in {|X[-1.35]|X[-0.7]|X[-1.75]|X[-1.5]|X[-1]|X[-0.7]|} \everyrow{\hline}
\hline
\rowfont\bfseries {Attribute} & \multicolumn{5}{|l|}{Value} \\
\tabucline[1.5pt]{}
BrowseName & \multicolumn{5}{|l|}{AccelerationClassType} \\
IsAbstract & \multicolumn{5}{|l|}{False} \\
\tabucline[1.5pt]{}
\rowfont \bfseries References & NodeClass & BrowseName & DataType & Type\-Definition & {Modeling\-Rule} \\
\multicolumn{6}{|l|}{Subtype of MTSampleClassType (See section \ref{type:MTSampleClassType})} \\
\end{tabu}
\end{table} 


\FloatBarrier
\subsubsection{Defintion of \texttt{ AccumulatedTimeClassType}}
  \label{type:AccumulatedTimeClassType}

\FloatBarrier
\begin{table}[ht]
\centering 
  \caption{\texttt{AccumulatedTimeClassType} Definition}
  \label{table:AccumulatedTimeClassType}
\fontsize{9pt}{11pt}\selectfont
\tabulinesep=3pt
\begin{tabu} to 6in {|X[-1.35]|X[-0.7]|X[-1.75]|X[-1.5]|X[-1]|X[-0.7]|} \everyrow{\hline}
\hline
\rowfont\bfseries {Attribute} & \multicolumn{5}{|l|}{Value} \\
\tabucline[1.5pt]{}
BrowseName & \multicolumn{5}{|l|}{AccumulatedTimeClassType} \\
IsAbstract & \multicolumn{5}{|l|}{False} \\
\tabucline[1.5pt]{}
\rowfont \bfseries References & NodeClass & BrowseName & DataType & Type\-Definition & {Modeling\-Rule} \\
\multicolumn{6}{|l|}{Subtype of MTSampleClassType (See section \ref{type:MTSampleClassType})} \\
\end{tabu}
\end{table} 


\FloatBarrier
\subsubsection{Defintion of \texttt{ AmperageClassType}}
  \label{type:AmperageClassType}

\FloatBarrier
\begin{table}[ht]
\centering 
  \caption{\texttt{AmperageClassType} Definition}
  \label{table:AmperageClassType}
\fontsize{9pt}{11pt}\selectfont
\tabulinesep=3pt
\begin{tabu} to 6in {|X[-1.35]|X[-0.7]|X[-1.75]|X[-1.5]|X[-1]|X[-0.7]|} \everyrow{\hline}
\hline
\rowfont\bfseries {Attribute} & \multicolumn{5}{|l|}{Value} \\
\tabucline[1.5pt]{}
BrowseName & \multicolumn{5}{|l|}{AmperageClassType} \\
IsAbstract & \multicolumn{5}{|l|}{False} \\
\tabucline[1.5pt]{}
\rowfont \bfseries References & NodeClass & BrowseName & DataType & Type\-Definition & {Modeling\-Rule} \\
\multicolumn{6}{|l|}{Subtype of MTSampleClassType (See section \ref{type:MTSampleClassType})} \\
\end{tabu}
\end{table} 


\FloatBarrier
\subsubsection{Defintion of \texttt{ AngleClassType}}
  \label{type:AngleClassType}

\FloatBarrier
\begin{table}[ht]
\centering 
  \caption{\texttt{AngleClassType} Definition}
  \label{table:AngleClassType}
\fontsize{9pt}{11pt}\selectfont
\tabulinesep=3pt
\begin{tabu} to 6in {|X[-1.35]|X[-0.7]|X[-1.75]|X[-1.5]|X[-1]|X[-0.7]|} \everyrow{\hline}
\hline
\rowfont\bfseries {Attribute} & \multicolumn{5}{|l|}{Value} \\
\tabucline[1.5pt]{}
BrowseName & \multicolumn{5}{|l|}{AngleClassType} \\
IsAbstract & \multicolumn{5}{|l|}{False} \\
\tabucline[1.5pt]{}
\rowfont \bfseries References & NodeClass & BrowseName & DataType & Type\-Definition & {Modeling\-Rule} \\
\multicolumn{6}{|l|}{Subtype of MTSampleClassType (See section \ref{type:MTSampleClassType})} \\
\end{tabu}
\end{table} 


\FloatBarrier
\subsubsection{Defintion of \texttt{ AngularAccelerationClassType}}
  \label{type:AngularAccelerationClassType}

\FloatBarrier
\begin{table}[ht]
\centering 
  \caption{\texttt{AngularAccelerationClassType} Definition}
  \label{table:AngularAccelerationClassType}
\fontsize{9pt}{11pt}\selectfont
\tabulinesep=3pt
\begin{tabu} to 6in {|X[-1.35]|X[-0.7]|X[-1.75]|X[-1.5]|X[-1]|X[-0.7]|} \everyrow{\hline}
\hline
\rowfont\bfseries {Attribute} & \multicolumn{5}{|l|}{Value} \\
\tabucline[1.5pt]{}
BrowseName & \multicolumn{5}{|l|}{AngularAccelerationClassType} \\
IsAbstract & \multicolumn{5}{|l|}{False} \\
\tabucline[1.5pt]{}
\rowfont \bfseries References & NodeClass & BrowseName & DataType & Type\-Definition & {Modeling\-Rule} \\
\multicolumn{6}{|l|}{Subtype of MTSampleClassType (See section \ref{type:MTSampleClassType})} \\
\end{tabu}
\end{table} 


\FloatBarrier
\subsubsection{Defintion of \texttt{ AngularVelocityClassType}}
  \label{type:AngularVelocityClassType}

\FloatBarrier
\begin{table}[ht]
\centering 
  \caption{\texttt{AngularVelocityClassType} Definition}
  \label{table:AngularVelocityClassType}
\fontsize{9pt}{11pt}\selectfont
\tabulinesep=3pt
\begin{tabu} to 6in {|X[-1.35]|X[-0.7]|X[-1.75]|X[-1.5]|X[-1]|X[-0.7]|} \everyrow{\hline}
\hline
\rowfont\bfseries {Attribute} & \multicolumn{5}{|l|}{Value} \\
\tabucline[1.5pt]{}
BrowseName & \multicolumn{5}{|l|}{AngularVelocityClassType} \\
IsAbstract & \multicolumn{5}{|l|}{False} \\
\tabucline[1.5pt]{}
\rowfont \bfseries References & NodeClass & BrowseName & DataType & Type\-Definition & {Modeling\-Rule} \\
\multicolumn{6}{|l|}{Subtype of MTSampleClassType (See section \ref{type:MTSampleClassType})} \\
\end{tabu}
\end{table} 


\FloatBarrier
\subsubsection{Defintion of \texttt{ AxisFeedrateClassType}}
  \label{type:AxisFeedrateClassType}

\FloatBarrier
\begin{table}[ht]
\centering 
  \caption{\texttt{AxisFeedrateClassType} Definition}
  \label{table:AxisFeedrateClassType}
\fontsize{9pt}{11pt}\selectfont
\tabulinesep=3pt
\begin{tabu} to 6in {|X[-1.35]|X[-0.7]|X[-1.75]|X[-1.5]|X[-1]|X[-0.7]|} \everyrow{\hline}
\hline
\rowfont\bfseries {Attribute} & \multicolumn{5}{|l|}{Value} \\
\tabucline[1.5pt]{}
BrowseName & \multicolumn{5}{|l|}{AxisFeedrateClassType} \\
IsAbstract & \multicolumn{5}{|l|}{False} \\
\tabucline[1.5pt]{}
\rowfont \bfseries References & NodeClass & BrowseName & DataType & Type\-Definition & {Modeling\-Rule} \\
\multicolumn{6}{|l|}{Subtype of MTSampleClassType (See section \ref{type:MTSampleClassType})} \\
\end{tabu}
\end{table} 


\FloatBarrier
\subsubsection{Defintion of \texttt{ ClockTimeClassType}}
  \label{type:ClockTimeClassType}

\FloatBarrier
\begin{table}[ht]
\centering 
  \caption{\texttt{ClockTimeClassType} Definition}
  \label{table:ClockTimeClassType}
\fontsize{9pt}{11pt}\selectfont
\tabulinesep=3pt
\begin{tabu} to 6in {|X[-1.35]|X[-0.7]|X[-1.75]|X[-1.5]|X[-1]|X[-0.7]|} \everyrow{\hline}
\hline
\rowfont\bfseries {Attribute} & \multicolumn{5}{|l|}{Value} \\
\tabucline[1.5pt]{}
BrowseName & \multicolumn{5}{|l|}{ClockTimeClassType} \\
IsAbstract & \multicolumn{5}{|l|}{False} \\
\tabucline[1.5pt]{}
\rowfont \bfseries References & NodeClass & BrowseName & DataType & Type\-Definition & {Modeling\-Rule} \\
\multicolumn{6}{|l|}{Subtype of MTSampleClassType (See section \ref{type:MTSampleClassType})} \\
\end{tabu}
\end{table} 


\FloatBarrier
\subsubsection{Defintion of \texttt{ ConcentrationClassType}}
  \label{type:ConcentrationClassType}

\FloatBarrier
\begin{table}[ht]
\centering 
  \caption{\texttt{ConcentrationClassType} Definition}
  \label{table:ConcentrationClassType}
\fontsize{9pt}{11pt}\selectfont
\tabulinesep=3pt
\begin{tabu} to 6in {|X[-1.35]|X[-0.7]|X[-1.75]|X[-1.5]|X[-1]|X[-0.7]|} \everyrow{\hline}
\hline
\rowfont\bfseries {Attribute} & \multicolumn{5}{|l|}{Value} \\
\tabucline[1.5pt]{}
BrowseName & \multicolumn{5}{|l|}{ConcentrationClassType} \\
IsAbstract & \multicolumn{5}{|l|}{False} \\
\tabucline[1.5pt]{}
\rowfont \bfseries References & NodeClass & BrowseName & DataType & Type\-Definition & {Modeling\-Rule} \\
\multicolumn{6}{|l|}{Subtype of MTSampleClassType (See section \ref{type:MTSampleClassType})} \\
\end{tabu}
\end{table} 


\FloatBarrier
\subsubsection{Defintion of \texttt{ ConductivityClassType}}
  \label{type:ConductivityClassType}

\FloatBarrier
\begin{table}[ht]
\centering 
  \caption{\texttt{ConductivityClassType} Definition}
  \label{table:ConductivityClassType}
\fontsize{9pt}{11pt}\selectfont
\tabulinesep=3pt
\begin{tabu} to 6in {|X[-1.35]|X[-0.7]|X[-1.75]|X[-1.5]|X[-1]|X[-0.7]|} \everyrow{\hline}
\hline
\rowfont\bfseries {Attribute} & \multicolumn{5}{|l|}{Value} \\
\tabucline[1.5pt]{}
BrowseName & \multicolumn{5}{|l|}{ConductivityClassType} \\
IsAbstract & \multicolumn{5}{|l|}{False} \\
\tabucline[1.5pt]{}
\rowfont \bfseries References & NodeClass & BrowseName & DataType & Type\-Definition & {Modeling\-Rule} \\
\multicolumn{6}{|l|}{Subtype of MTSampleClassType (See section \ref{type:MTSampleClassType})} \\
\end{tabu}
\end{table} 


\FloatBarrier
\subsubsection{Defintion of \texttt{ DisplacementClassType}}
  \label{type:DisplacementClassType}

\FloatBarrier
\begin{table}[ht]
\centering 
  \caption{\texttt{DisplacementClassType} Definition}
  \label{table:DisplacementClassType}
\fontsize{9pt}{11pt}\selectfont
\tabulinesep=3pt
\begin{tabu} to 6in {|X[-1.35]|X[-0.7]|X[-1.75]|X[-1.5]|X[-1]|X[-0.7]|} \everyrow{\hline}
\hline
\rowfont\bfseries {Attribute} & \multicolumn{5}{|l|}{Value} \\
\tabucline[1.5pt]{}
BrowseName & \multicolumn{5}{|l|}{DisplacementClassType} \\
IsAbstract & \multicolumn{5}{|l|}{False} \\
\tabucline[1.5pt]{}
\rowfont \bfseries References & NodeClass & BrowseName & DataType & Type\-Definition & {Modeling\-Rule} \\
\multicolumn{6}{|l|}{Subtype of MTSampleClassType (See section \ref{type:MTSampleClassType})} \\
\end{tabu}
\end{table} 


\FloatBarrier
\subsubsection{Defintion of \texttt{ ElectricalEnergyClassType}}
  \label{type:ElectricalEnergyClassType}

\FloatBarrier
\begin{table}[ht]
\centering 
  \caption{\texttt{ElectricalEnergyClassType} Definition}
  \label{table:ElectricalEnergyClassType}
\fontsize{9pt}{11pt}\selectfont
\tabulinesep=3pt
\begin{tabu} to 6in {|X[-1.35]|X[-0.7]|X[-1.75]|X[-1.5]|X[-1]|X[-0.7]|} \everyrow{\hline}
\hline
\rowfont\bfseries {Attribute} & \multicolumn{5}{|l|}{Value} \\
\tabucline[1.5pt]{}
BrowseName & \multicolumn{5}{|l|}{ElectricalEnergyClassType} \\
IsAbstract & \multicolumn{5}{|l|}{False} \\
\tabucline[1.5pt]{}
\rowfont \bfseries References & NodeClass & BrowseName & DataType & Type\-Definition & {Modeling\-Rule} \\
\multicolumn{6}{|l|}{Subtype of MTSampleClassType (See section \ref{type:MTSampleClassType})} \\
\end{tabu}
\end{table} 


\FloatBarrier
\subsubsection{Defintion of \texttt{ EquipmentTimerClassType}}
  \label{type:EquipmentTimerClassType}

\FloatBarrier
\begin{table}[ht]
\centering 
  \caption{\texttt{EquipmentTimerClassType} Definition}
  \label{table:EquipmentTimerClassType}
\fontsize{9pt}{11pt}\selectfont
\tabulinesep=3pt
\begin{tabu} to 6in {|X[-1.35]|X[-0.7]|X[-1.75]|X[-1.5]|X[-1]|X[-0.7]|} \everyrow{\hline}
\hline
\rowfont\bfseries {Attribute} & \multicolumn{5}{|l|}{Value} \\
\tabucline[1.5pt]{}
BrowseName & \multicolumn{5}{|l|}{EquipmentTimerClassType} \\
IsAbstract & \multicolumn{5}{|l|}{False} \\
\tabucline[1.5pt]{}
\rowfont \bfseries References & NodeClass & BrowseName & DataType & Type\-Definition & {Modeling\-Rule} \\
\multicolumn{6}{|l|}{Subtype of MTSampleClassType (See section \ref{type:MTSampleClassType})} \\
\end{tabu}
\end{table} 


\FloatBarrier
\subsubsection{Defintion of \texttt{ FillLevelClassType}}
  \label{type:FillLevelClassType}

\FloatBarrier
\begin{table}[ht]
\centering 
  \caption{\texttt{FillLevelClassType} Definition}
  \label{table:FillLevelClassType}
\fontsize{9pt}{11pt}\selectfont
\tabulinesep=3pt
\begin{tabu} to 6in {|X[-1.35]|X[-0.7]|X[-1.75]|X[-1.5]|X[-1]|X[-0.7]|} \everyrow{\hline}
\hline
\rowfont\bfseries {Attribute} & \multicolumn{5}{|l|}{Value} \\
\tabucline[1.5pt]{}
BrowseName & \multicolumn{5}{|l|}{FillLevelClassType} \\
IsAbstract & \multicolumn{5}{|l|}{False} \\
\tabucline[1.5pt]{}
\rowfont \bfseries References & NodeClass & BrowseName & DataType & Type\-Definition & {Modeling\-Rule} \\
\multicolumn{6}{|l|}{Subtype of MTSampleClassType (See section \ref{type:MTSampleClassType})} \\
\end{tabu}
\end{table} 


\FloatBarrier
\subsubsection{Defintion of \texttt{ FlowClassType}}
  \label{type:FlowClassType}

\FloatBarrier
\begin{table}[ht]
\centering 
  \caption{\texttt{FlowClassType} Definition}
  \label{table:FlowClassType}
\fontsize{9pt}{11pt}\selectfont
\tabulinesep=3pt
\begin{tabu} to 6in {|X[-1.35]|X[-0.7]|X[-1.75]|X[-1.5]|X[-1]|X[-0.7]|} \everyrow{\hline}
\hline
\rowfont\bfseries {Attribute} & \multicolumn{5}{|l|}{Value} \\
\tabucline[1.5pt]{}
BrowseName & \multicolumn{5}{|l|}{FlowClassType} \\
IsAbstract & \multicolumn{5}{|l|}{False} \\
\tabucline[1.5pt]{}
\rowfont \bfseries References & NodeClass & BrowseName & DataType & Type\-Definition & {Modeling\-Rule} \\
\multicolumn{6}{|l|}{Subtype of MTSampleClassType (See section \ref{type:MTSampleClassType})} \\
\end{tabu}
\end{table} 


\FloatBarrier
\subsubsection{Defintion of \texttt{ FrequencyClassType}}
  \label{type:FrequencyClassType}

\FloatBarrier
\begin{table}[ht]
\centering 
  \caption{\texttt{FrequencyClassType} Definition}
  \label{table:FrequencyClassType}
\fontsize{9pt}{11pt}\selectfont
\tabulinesep=3pt
\begin{tabu} to 6in {|X[-1.35]|X[-0.7]|X[-1.75]|X[-1.5]|X[-1]|X[-0.7]|} \everyrow{\hline}
\hline
\rowfont\bfseries {Attribute} & \multicolumn{5}{|l|}{Value} \\
\tabucline[1.5pt]{}
BrowseName & \multicolumn{5}{|l|}{FrequencyClassType} \\
IsAbstract & \multicolumn{5}{|l|}{False} \\
\tabucline[1.5pt]{}
\rowfont \bfseries References & NodeClass & BrowseName & DataType & Type\-Definition & {Modeling\-Rule} \\
\multicolumn{6}{|l|}{Subtype of MTSampleClassType (See section \ref{type:MTSampleClassType})} \\
\end{tabu}
\end{table} 


\FloatBarrier
\subsubsection{Defintion of \texttt{ LengthClassType}}
  \label{type:LengthClassType}

\FloatBarrier
\begin{table}[ht]
\centering 
  \caption{\texttt{LengthClassType} Definition}
  \label{table:LengthClassType}
\fontsize{9pt}{11pt}\selectfont
\tabulinesep=3pt
\begin{tabu} to 6in {|X[-1.35]|X[-0.7]|X[-1.75]|X[-1.5]|X[-1]|X[-0.7]|} \everyrow{\hline}
\hline
\rowfont\bfseries {Attribute} & \multicolumn{5}{|l|}{Value} \\
\tabucline[1.5pt]{}
BrowseName & \multicolumn{5}{|l|}{LengthClassType} \\
IsAbstract & \multicolumn{5}{|l|}{False} \\
\tabucline[1.5pt]{}
\rowfont \bfseries References & NodeClass & BrowseName & DataType & Type\-Definition & {Modeling\-Rule} \\
\multicolumn{6}{|l|}{Subtype of MTSampleClassType (See section \ref{type:MTSampleClassType})} \\
\end{tabu}
\end{table} 


\FloatBarrier
\subsubsection{Defintion of \texttt{ LinearForceClassType}}
  \label{type:LinearForceClassType}

\FloatBarrier
\begin{table}[ht]
\centering 
  \caption{\texttt{LinearForceClassType} Definition}
  \label{table:LinearForceClassType}
\fontsize{9pt}{11pt}\selectfont
\tabulinesep=3pt
\begin{tabu} to 6in {|X[-1.35]|X[-0.7]|X[-1.75]|X[-1.5]|X[-1]|X[-0.7]|} \everyrow{\hline}
\hline
\rowfont\bfseries {Attribute} & \multicolumn{5}{|l|}{Value} \\
\tabucline[1.5pt]{}
BrowseName & \multicolumn{5}{|l|}{LinearForceClassType} \\
IsAbstract & \multicolumn{5}{|l|}{False} \\
\tabucline[1.5pt]{}
\rowfont \bfseries References & NodeClass & BrowseName & DataType & Type\-Definition & {Modeling\-Rule} \\
\multicolumn{6}{|l|}{Subtype of MTSampleClassType (See section \ref{type:MTSampleClassType})} \\
\end{tabu}
\end{table} 


\FloatBarrier
\subsubsection{Defintion of \texttt{ LoadClassType}}
  \label{type:LoadClassType}

\FloatBarrier
\begin{table}[ht]
\centering 
  \caption{\texttt{LoadClassType} Definition}
  \label{table:LoadClassType}
\fontsize{9pt}{11pt}\selectfont
\tabulinesep=3pt
\begin{tabu} to 6in {|X[-1.35]|X[-0.7]|X[-1.75]|X[-1.5]|X[-1]|X[-0.7]|} \everyrow{\hline}
\hline
\rowfont\bfseries {Attribute} & \multicolumn{5}{|l|}{Value} \\
\tabucline[1.5pt]{}
BrowseName & \multicolumn{5}{|l|}{LoadClassType} \\
IsAbstract & \multicolumn{5}{|l|}{False} \\
\tabucline[1.5pt]{}
\rowfont \bfseries References & NodeClass & BrowseName & DataType & Type\-Definition & {Modeling\-Rule} \\
\multicolumn{6}{|l|}{Subtype of MTSampleClassType (See section \ref{type:MTSampleClassType})} \\
\end{tabu}
\end{table} 


\FloatBarrier
\subsubsection{Defintion of \texttt{ MassClassType}}
  \label{type:MassClassType}

\FloatBarrier
\begin{table}[ht]
\centering 
  \caption{\texttt{MassClassType} Definition}
  \label{table:MassClassType}
\fontsize{9pt}{11pt}\selectfont
\tabulinesep=3pt
\begin{tabu} to 6in {|X[-1.35]|X[-0.7]|X[-1.75]|X[-1.5]|X[-1]|X[-0.7]|} \everyrow{\hline}
\hline
\rowfont\bfseries {Attribute} & \multicolumn{5}{|l|}{Value} \\
\tabucline[1.5pt]{}
BrowseName & \multicolumn{5}{|l|}{MassClassType} \\
IsAbstract & \multicolumn{5}{|l|}{False} \\
\tabucline[1.5pt]{}
\rowfont \bfseries References & NodeClass & BrowseName & DataType & Type\-Definition & {Modeling\-Rule} \\
\multicolumn{6}{|l|}{Subtype of MTSampleClassType (See section \ref{type:MTSampleClassType})} \\
\end{tabu}
\end{table} 


\FloatBarrier
\subsubsection{Defintion of \texttt{ PathFeedrateClassType}}
  \label{type:PathFeedrateClassType}

\FloatBarrier
\begin{table}[ht]
\centering 
  \caption{\texttt{PathFeedrateClassType} Definition}
  \label{table:PathFeedrateClassType}
\fontsize{9pt}{11pt}\selectfont
\tabulinesep=3pt
\begin{tabu} to 6in {|X[-1.35]|X[-0.7]|X[-1.75]|X[-1.5]|X[-1]|X[-0.7]|} \everyrow{\hline}
\hline
\rowfont\bfseries {Attribute} & \multicolumn{5}{|l|}{Value} \\
\tabucline[1.5pt]{}
BrowseName & \multicolumn{5}{|l|}{PathFeedrateClassType} \\
IsAbstract & \multicolumn{5}{|l|}{False} \\
\tabucline[1.5pt]{}
\rowfont \bfseries References & NodeClass & BrowseName & DataType & Type\-Definition & {Modeling\-Rule} \\
\multicolumn{6}{|l|}{Subtype of MTSampleClassType (See section \ref{type:MTSampleClassType})} \\
\end{tabu}
\end{table} 


\FloatBarrier
\subsubsection{Defintion of \texttt{ PathPositionClassType}}
  \label{type:PathPositionClassType}

\FloatBarrier
\begin{table}[ht]
\centering 
  \caption{\texttt{PathPositionClassType} Definition}
  \label{table:PathPositionClassType}
\fontsize{9pt}{11pt}\selectfont
\tabulinesep=3pt
\begin{tabu} to 6in {|X[-1.35]|X[-0.7]|X[-1.75]|X[-1.5]|X[-1]|X[-0.7]|} \everyrow{\hline}
\hline
\rowfont\bfseries {Attribute} & \multicolumn{5}{|l|}{Value} \\
\tabucline[1.5pt]{}
BrowseName & \multicolumn{5}{|l|}{PathPositionClassType} \\
IsAbstract & \multicolumn{5}{|l|}{False} \\
\tabucline[1.5pt]{}
\rowfont \bfseries References & NodeClass & BrowseName & DataType & Type\-Definition & {Modeling\-Rule} \\
\multicolumn{6}{|l|}{Subtype of MTSampleClassType (See section \ref{type:MTSampleClassType})} \\
\end{tabu}
\end{table} 


\FloatBarrier
\subsubsection{Defintion of \texttt{ PHClassType}}
  \label{type:PHClassType}

\FloatBarrier
\begin{table}[ht]
\centering 
  \caption{\texttt{PHClassType} Definition}
  \label{table:PHClassType}
\fontsize{9pt}{11pt}\selectfont
\tabulinesep=3pt
\begin{tabu} to 6in {|X[-1.35]|X[-0.7]|X[-1.75]|X[-1.5]|X[-1]|X[-0.7]|} \everyrow{\hline}
\hline
\rowfont\bfseries {Attribute} & \multicolumn{5}{|l|}{Value} \\
\tabucline[1.5pt]{}
BrowseName & \multicolumn{5}{|l|}{PHClassType} \\
IsAbstract & \multicolumn{5}{|l|}{False} \\
\tabucline[1.5pt]{}
\rowfont \bfseries References & NodeClass & BrowseName & DataType & Type\-Definition & {Modeling\-Rule} \\
\multicolumn{6}{|l|}{Subtype of MTSampleClassType (See section \ref{type:MTSampleClassType})} \\
\end{tabu}
\end{table} 


\FloatBarrier
\subsubsection{Defintion of \texttt{ PositionClassType}}
  \label{type:PositionClassType}

\FloatBarrier
\begin{table}[ht]
\centering 
  \caption{\texttt{PositionClassType} Definition}
  \label{table:PositionClassType}
\fontsize{9pt}{11pt}\selectfont
\tabulinesep=3pt
\begin{tabu} to 6in {|X[-1.35]|X[-0.7]|X[-1.75]|X[-1.5]|X[-1]|X[-0.7]|} \everyrow{\hline}
\hline
\rowfont\bfseries {Attribute} & \multicolumn{5}{|l|}{Value} \\
\tabucline[1.5pt]{}
BrowseName & \multicolumn{5}{|l|}{PositionClassType} \\
IsAbstract & \multicolumn{5}{|l|}{False} \\
\tabucline[1.5pt]{}
\rowfont \bfseries References & NodeClass & BrowseName & DataType & Type\-Definition & {Modeling\-Rule} \\
\multicolumn{6}{|l|}{Subtype of MTSampleClassType (See section \ref{type:MTSampleClassType})} \\
\end{tabu}
\end{table} 


\FloatBarrier
\subsubsection{Defintion of \texttt{ PowerFactorClassType}}
  \label{type:PowerFactorClassType}

\FloatBarrier
\begin{table}[ht]
\centering 
  \caption{\texttt{PowerFactorClassType} Definition}
  \label{table:PowerFactorClassType}
\fontsize{9pt}{11pt}\selectfont
\tabulinesep=3pt
\begin{tabu} to 6in {|X[-1.35]|X[-0.7]|X[-1.75]|X[-1.5]|X[-1]|X[-0.7]|} \everyrow{\hline}
\hline
\rowfont\bfseries {Attribute} & \multicolumn{5}{|l|}{Value} \\
\tabucline[1.5pt]{}
BrowseName & \multicolumn{5}{|l|}{PowerFactorClassType} \\
IsAbstract & \multicolumn{5}{|l|}{False} \\
\tabucline[1.5pt]{}
\rowfont \bfseries References & NodeClass & BrowseName & DataType & Type\-Definition & {Modeling\-Rule} \\
\multicolumn{6}{|l|}{Subtype of MTSampleClassType (See section \ref{type:MTSampleClassType})} \\
\end{tabu}
\end{table} 


\FloatBarrier
\subsubsection{Defintion of \texttt{ PressureClassType}}
  \label{type:PressureClassType}

\FloatBarrier
\begin{table}[ht]
\centering 
  \caption{\texttt{PressureClassType} Definition}
  \label{table:PressureClassType}
\fontsize{9pt}{11pt}\selectfont
\tabulinesep=3pt
\begin{tabu} to 6in {|X[-1.35]|X[-0.7]|X[-1.75]|X[-1.5]|X[-1]|X[-0.7]|} \everyrow{\hline}
\hline
\rowfont\bfseries {Attribute} & \multicolumn{5}{|l|}{Value} \\
\tabucline[1.5pt]{}
BrowseName & \multicolumn{5}{|l|}{PressureClassType} \\
IsAbstract & \multicolumn{5}{|l|}{False} \\
\tabucline[1.5pt]{}
\rowfont \bfseries References & NodeClass & BrowseName & DataType & Type\-Definition & {Modeling\-Rule} \\
\multicolumn{6}{|l|}{Subtype of MTSampleClassType (See section \ref{type:MTSampleClassType})} \\
\end{tabu}
\end{table} 


\FloatBarrier
\subsubsection{Defintion of \texttt{ ProcessTimerClassType}}
  \label{type:ProcessTimerClassType}

\FloatBarrier
\begin{table}[ht]
\centering 
  \caption{\texttt{ProcessTimerClassType} Definition}
  \label{table:ProcessTimerClassType}
\fontsize{9pt}{11pt}\selectfont
\tabulinesep=3pt
\begin{tabu} to 6in {|X[-1.35]|X[-0.7]|X[-1.75]|X[-1.5]|X[-1]|X[-0.7]|} \everyrow{\hline}
\hline
\rowfont\bfseries {Attribute} & \multicolumn{5}{|l|}{Value} \\
\tabucline[1.5pt]{}
BrowseName & \multicolumn{5}{|l|}{ProcessTimerClassType} \\
IsAbstract & \multicolumn{5}{|l|}{False} \\
\tabucline[1.5pt]{}
\rowfont \bfseries References & NodeClass & BrowseName & DataType & Type\-Definition & {Modeling\-Rule} \\
\multicolumn{6}{|l|}{Subtype of MTSampleClassType (See section \ref{type:MTSampleClassType})} \\
\end{tabu}
\end{table} 


\FloatBarrier
\subsubsection{Defintion of \texttt{ ResistenceClassType}}
  \label{type:ResistenceClassType}

\FloatBarrier
\begin{table}[ht]
\centering 
  \caption{\texttt{ResistenceClassType} Definition}
  \label{table:ResistenceClassType}
\fontsize{9pt}{11pt}\selectfont
\tabulinesep=3pt
\begin{tabu} to 6in {|X[-1.35]|X[-0.7]|X[-1.75]|X[-1.5]|X[-1]|X[-0.7]|} \everyrow{\hline}
\hline
\rowfont\bfseries {Attribute} & \multicolumn{5}{|l|}{Value} \\
\tabucline[1.5pt]{}
BrowseName & \multicolumn{5}{|l|}{ResistenceClassType} \\
IsAbstract & \multicolumn{5}{|l|}{False} \\
\tabucline[1.5pt]{}
\rowfont \bfseries References & NodeClass & BrowseName & DataType & Type\-Definition & {Modeling\-Rule} \\
\multicolumn{6}{|l|}{Subtype of MTSampleClassType (See section \ref{type:MTSampleClassType})} \\
\end{tabu}
\end{table} 


\FloatBarrier
\subsubsection{Defintion of \texttt{ RotaryVelocityClassType}}
  \label{type:RotaryVelocityClassType}

\FloatBarrier
\begin{table}[ht]
\centering 
  \caption{\texttt{RotaryVelocityClassType} Definition}
  \label{table:RotaryVelocityClassType}
\fontsize{9pt}{11pt}\selectfont
\tabulinesep=3pt
\begin{tabu} to 6in {|X[-1.35]|X[-0.7]|X[-1.75]|X[-1.5]|X[-1]|X[-0.7]|} \everyrow{\hline}
\hline
\rowfont\bfseries {Attribute} & \multicolumn{5}{|l|}{Value} \\
\tabucline[1.5pt]{}
BrowseName & \multicolumn{5}{|l|}{RotaryVelocityClassType} \\
IsAbstract & \multicolumn{5}{|l|}{False} \\
\tabucline[1.5pt]{}
\rowfont \bfseries References & NodeClass & BrowseName & DataType & Type\-Definition & {Modeling\-Rule} \\
\multicolumn{6}{|l|}{Subtype of MTSampleClassType (See section \ref{type:MTSampleClassType})} \\
\end{tabu}
\end{table} 


\FloatBarrier
\subsubsection{Defintion of \texttt{ SoundLevelClassType}}
  \label{type:SoundLevelClassType}

\FloatBarrier
\begin{table}[ht]
\centering 
  \caption{\texttt{SoundLevelClassType} Definition}
  \label{table:SoundLevelClassType}
\fontsize{9pt}{11pt}\selectfont
\tabulinesep=3pt
\begin{tabu} to 6in {|X[-1.35]|X[-0.7]|X[-1.75]|X[-1.5]|X[-1]|X[-0.7]|} \everyrow{\hline}
\hline
\rowfont\bfseries {Attribute} & \multicolumn{5}{|l|}{Value} \\
\tabucline[1.5pt]{}
BrowseName & \multicolumn{5}{|l|}{SoundLevelClassType} \\
IsAbstract & \multicolumn{5}{|l|}{False} \\
\tabucline[1.5pt]{}
\rowfont \bfseries References & NodeClass & BrowseName & DataType & Type\-Definition & {Modeling\-Rule} \\
\multicolumn{6}{|l|}{Subtype of MTSampleClassType (See section \ref{type:MTSampleClassType})} \\
\end{tabu}
\end{table} 


\FloatBarrier
\subsubsection{Defintion of \texttt{ StrainClassType}}
  \label{type:StrainClassType}

\FloatBarrier
\begin{table}[ht]
\centering 
  \caption{\texttt{StrainClassType} Definition}
  \label{table:StrainClassType}
\fontsize{9pt}{11pt}\selectfont
\tabulinesep=3pt
\begin{tabu} to 6in {|X[-1.35]|X[-0.7]|X[-1.75]|X[-1.5]|X[-1]|X[-0.7]|} \everyrow{\hline}
\hline
\rowfont\bfseries {Attribute} & \multicolumn{5}{|l|}{Value} \\
\tabucline[1.5pt]{}
BrowseName & \multicolumn{5}{|l|}{StrainClassType} \\
IsAbstract & \multicolumn{5}{|l|}{False} \\
\tabucline[1.5pt]{}
\rowfont \bfseries References & NodeClass & BrowseName & DataType & Type\-Definition & {Modeling\-Rule} \\
\multicolumn{6}{|l|}{Subtype of MTSampleClassType (See section \ref{type:MTSampleClassType})} \\
\end{tabu}
\end{table} 


\FloatBarrier
\subsubsection{Defintion of \texttt{ TemperatureClassType}}
  \label{type:TemperatureClassType}

\FloatBarrier
\begin{table}[ht]
\centering 
  \caption{\texttt{TemperatureClassType} Definition}
  \label{table:TemperatureClassType}
\fontsize{9pt}{11pt}\selectfont
\tabulinesep=3pt
\begin{tabu} to 6in {|X[-1.35]|X[-0.7]|X[-1.75]|X[-1.5]|X[-1]|X[-0.7]|} \everyrow{\hline}
\hline
\rowfont\bfseries {Attribute} & \multicolumn{5}{|l|}{Value} \\
\tabucline[1.5pt]{}
BrowseName & \multicolumn{5}{|l|}{TemperatureClassType} \\
IsAbstract & \multicolumn{5}{|l|}{False} \\
\tabucline[1.5pt]{}
\rowfont \bfseries References & NodeClass & BrowseName & DataType & Type\-Definition & {Modeling\-Rule} \\
\multicolumn{6}{|l|}{Subtype of MTSampleClassType (See section \ref{type:MTSampleClassType})} \\
\end{tabu}
\end{table} 


\FloatBarrier
\subsubsection{Defintion of \texttt{ TensionClassType}}
  \label{type:TensionClassType}

\FloatBarrier
\begin{table}[ht]
\centering 
  \caption{\texttt{TensionClassType} Definition}
  \label{table:TensionClassType}
\fontsize{9pt}{11pt}\selectfont
\tabulinesep=3pt
\begin{tabu} to 6in {|X[-1.35]|X[-0.7]|X[-1.75]|X[-1.5]|X[-1]|X[-0.7]|} \everyrow{\hline}
\hline
\rowfont\bfseries {Attribute} & \multicolumn{5}{|l|}{Value} \\
\tabucline[1.5pt]{}
BrowseName & \multicolumn{5}{|l|}{TensionClassType} \\
IsAbstract & \multicolumn{5}{|l|}{False} \\
\tabucline[1.5pt]{}
\rowfont \bfseries References & NodeClass & BrowseName & DataType & Type\-Definition & {Modeling\-Rule} \\
\multicolumn{6}{|l|}{Subtype of MTSampleClassType (See section \ref{type:MTSampleClassType})} \\
\end{tabu}
\end{table} 


\FloatBarrier
\subsubsection{Defintion of \texttt{ TiltClassType}}
  \label{type:TiltClassType}

\FloatBarrier
\begin{table}[ht]
\centering 
  \caption{\texttt{TiltClassType} Definition}
  \label{table:TiltClassType}
\fontsize{9pt}{11pt}\selectfont
\tabulinesep=3pt
\begin{tabu} to 6in {|X[-1.35]|X[-0.7]|X[-1.75]|X[-1.5]|X[-1]|X[-0.7]|} \everyrow{\hline}
\hline
\rowfont\bfseries {Attribute} & \multicolumn{5}{|l|}{Value} \\
\tabucline[1.5pt]{}
BrowseName & \multicolumn{5}{|l|}{TiltClassType} \\
IsAbstract & \multicolumn{5}{|l|}{False} \\
\tabucline[1.5pt]{}
\rowfont \bfseries References & NodeClass & BrowseName & DataType & Type\-Definition & {Modeling\-Rule} \\
\multicolumn{6}{|l|}{Subtype of MTSampleClassType (See section \ref{type:MTSampleClassType})} \\
\end{tabu}
\end{table} 


\FloatBarrier
\subsubsection{Defintion of \texttt{ TorqueClassType}}
  \label{type:TorqueClassType}

\FloatBarrier
\begin{table}[ht]
\centering 
  \caption{\texttt{TorqueClassType} Definition}
  \label{table:TorqueClassType}
\fontsize{9pt}{11pt}\selectfont
\tabulinesep=3pt
\begin{tabu} to 6in {|X[-1.35]|X[-0.7]|X[-1.75]|X[-1.5]|X[-1]|X[-0.7]|} \everyrow{\hline}
\hline
\rowfont\bfseries {Attribute} & \multicolumn{5}{|l|}{Value} \\
\tabucline[1.5pt]{}
BrowseName & \multicolumn{5}{|l|}{TorqueClassType} \\
IsAbstract & \multicolumn{5}{|l|}{False} \\
\tabucline[1.5pt]{}
\rowfont \bfseries References & NodeClass & BrowseName & DataType & Type\-Definition & {Modeling\-Rule} \\
\multicolumn{6}{|l|}{Subtype of MTSampleClassType (See section \ref{type:MTSampleClassType})} \\
\end{tabu}
\end{table} 


\FloatBarrier
\subsubsection{Defintion of \texttt{ VelocityClassType}}
  \label{type:VelocityClassType}

\FloatBarrier
\begin{table}[ht]
\centering 
  \caption{\texttt{VelocityClassType} Definition}
  \label{table:VelocityClassType}
\fontsize{9pt}{11pt}\selectfont
\tabulinesep=3pt
\begin{tabu} to 6in {|X[-1.35]|X[-0.7]|X[-1.75]|X[-1.5]|X[-1]|X[-0.7]|} \everyrow{\hline}
\hline
\rowfont\bfseries {Attribute} & \multicolumn{5}{|l|}{Value} \\
\tabucline[1.5pt]{}
BrowseName & \multicolumn{5}{|l|}{VelocityClassType} \\
IsAbstract & \multicolumn{5}{|l|}{False} \\
\tabucline[1.5pt]{}
\rowfont \bfseries References & NodeClass & BrowseName & DataType & Type\-Definition & {Modeling\-Rule} \\
\multicolumn{6}{|l|}{Subtype of MTSampleClassType (See section \ref{type:MTSampleClassType})} \\
\end{tabu}
\end{table} 


\FloatBarrier
\subsubsection{Defintion of \texttt{ ViscosityClassType}}
  \label{type:ViscosityClassType}

\FloatBarrier
\begin{table}[ht]
\centering 
  \caption{\texttt{ViscosityClassType} Definition}
  \label{table:ViscosityClassType}
\fontsize{9pt}{11pt}\selectfont
\tabulinesep=3pt
\begin{tabu} to 6in {|X[-1.35]|X[-0.7]|X[-1.75]|X[-1.5]|X[-1]|X[-0.7]|} \everyrow{\hline}
\hline
\rowfont\bfseries {Attribute} & \multicolumn{5}{|l|}{Value} \\
\tabucline[1.5pt]{}
BrowseName & \multicolumn{5}{|l|}{ViscosityClassType} \\
IsAbstract & \multicolumn{5}{|l|}{False} \\
\tabucline[1.5pt]{}
\rowfont \bfseries References & NodeClass & BrowseName & DataType & Type\-Definition & {Modeling\-Rule} \\
\multicolumn{6}{|l|}{Subtype of MTSampleClassType (See section \ref{type:MTSampleClassType})} \\
\end{tabu}
\end{table} 


\FloatBarrier
\subsubsection{Defintion of \texttt{ VoltageClassType}}
  \label{type:VoltageClassType}

\FloatBarrier
\begin{table}[ht]
\centering 
  \caption{\texttt{VoltageClassType} Definition}
  \label{table:VoltageClassType}
\fontsize{9pt}{11pt}\selectfont
\tabulinesep=3pt
\begin{tabu} to 6in {|X[-1.35]|X[-0.7]|X[-1.75]|X[-1.5]|X[-1]|X[-0.7]|} \everyrow{\hline}
\hline
\rowfont\bfseries {Attribute} & \multicolumn{5}{|l|}{Value} \\
\tabucline[1.5pt]{}
BrowseName & \multicolumn{5}{|l|}{VoltageClassType} \\
IsAbstract & \multicolumn{5}{|l|}{False} \\
\tabucline[1.5pt]{}
\rowfont \bfseries References & NodeClass & BrowseName & DataType & Type\-Definition & {Modeling\-Rule} \\
\multicolumn{6}{|l|}{Subtype of MTSampleClassType (See section \ref{type:MTSampleClassType})} \\
\end{tabu}
\end{table} 


\FloatBarrier
\subsubsection{Defintion of \texttt{ VoltAmpereClassType}}
  \label{type:VoltAmpereClassType}

\FloatBarrier
\begin{table}[ht]
\centering 
  \caption{\texttt{VoltAmpereClassType} Definition}
  \label{table:VoltAmpereClassType}
\fontsize{9pt}{11pt}\selectfont
\tabulinesep=3pt
\begin{tabu} to 6in {|X[-1.35]|X[-0.7]|X[-1.75]|X[-1.5]|X[-1]|X[-0.7]|} \everyrow{\hline}
\hline
\rowfont\bfseries {Attribute} & \multicolumn{5}{|l|}{Value} \\
\tabucline[1.5pt]{}
BrowseName & \multicolumn{5}{|l|}{VoltAmpereClassType} \\
IsAbstract & \multicolumn{5}{|l|}{False} \\
\tabucline[1.5pt]{}
\rowfont \bfseries References & NodeClass & BrowseName & DataType & Type\-Definition & {Modeling\-Rule} \\
\multicolumn{6}{|l|}{Subtype of MTSampleClassType (See section \ref{type:MTSampleClassType})} \\
\end{tabu}
\end{table} 


\FloatBarrier
\subsubsection{Defintion of \texttt{ VoltAmpereReactiveClassType}}
  \label{type:VoltAmpereReactiveClassType}

\FloatBarrier
\begin{table}[ht]
\centering 
  \caption{\texttt{VoltAmpereReactiveClassType} Definition}
  \label{table:VoltAmpereReactiveClassType}
\fontsize{9pt}{11pt}\selectfont
\tabulinesep=3pt
\begin{tabu} to 6in {|X[-1.35]|X[-0.7]|X[-1.75]|X[-1.5]|X[-1]|X[-0.7]|} \everyrow{\hline}
\hline
\rowfont\bfseries {Attribute} & \multicolumn{5}{|l|}{Value} \\
\tabucline[1.5pt]{}
BrowseName & \multicolumn{5}{|l|}{VoltAmpereReactiveClassType} \\
IsAbstract & \multicolumn{5}{|l|}{False} \\
\tabucline[1.5pt]{}
\rowfont \bfseries References & NodeClass & BrowseName & DataType & Type\-Definition & {Modeling\-Rule} \\
\multicolumn{6}{|l|}{Subtype of MTSampleClassType (See section \ref{type:MTSampleClassType})} \\
\end{tabu}
\end{table} 


\FloatBarrier
\subsubsection{Defintion of \texttt{ WattageClassType}}
  \label{type:WattageClassType}

\FloatBarrier
\begin{table}[ht]
\centering 
  \caption{\texttt{WattageClassType} Definition}
  \label{table:WattageClassType}
\fontsize{9pt}{11pt}\selectfont
\tabulinesep=3pt
\begin{tabu} to 6in {|X[-1.35]|X[-0.7]|X[-1.75]|X[-1.5]|X[-1]|X[-0.7]|} \everyrow{\hline}
\hline
\rowfont\bfseries {Attribute} & \multicolumn{5}{|l|}{Value} \\
\tabucline[1.5pt]{}
BrowseName & \multicolumn{5}{|l|}{WattageClassType} \\
IsAbstract & \multicolumn{5}{|l|}{False} \\
\tabucline[1.5pt]{}
\rowfont \bfseries References & NodeClass & BrowseName & DataType & Type\-Definition & {Modeling\-Rule} \\
\multicolumn{6}{|l|}{Subtype of MTSampleClassType (See section \ref{type:MTSampleClassType})} \\
\end{tabu}
\end{table} 


\FloatBarrier
\subsection{Controlled Vocab Data Item Types} \label{model:ControlledVocabDataItemTypes}
\subsubsection{Defintion of \texttt{ MTControlledVocabEventClassType}}
  \label{type:MTControlledVocabEventClassType}

\FloatBarrier
\begin{table}[ht]
\centering 
  \caption{\texttt{MTControlledVocabEventClassType} Definition}
  \label{table:MTControlledVocabEventClassType}
\fontsize{9pt}{11pt}\selectfont
\tabulinesep=3pt
\begin{tabu} to 6in {|X[-1.35]|X[-0.7]|X[-1.75]|X[-1.5]|X[-1]|X[-0.7]|} \everyrow{\hline}
\hline
\rowfont\bfseries {Attribute} & \multicolumn{5}{|l|}{Value} \\
\tabucline[1.5pt]{}
BrowseName & \multicolumn{5}{|l|}{MTControlledVocabEventClassType} \\
IsAbstract & \multicolumn{5}{|l|}{True} \\
\tabucline[1.5pt]{}
\rowfont \bfseries References & NodeClass & BrowseName & DataType & Type\-Definition & {Modeling\-Rule} \\
\multicolumn{6}{|l|}{Subtype of MTEventClassType (See Data Items Documentation)} \\
HasSubtype & ObjectType & \multicolumn{2}{l}{ControllerModeOverrideClassType} & \multicolumn{2}{|l|}{See section \ref{type:ControllerModeOverrideClassType}} \\
HasSubtype & ObjectType & \multicolumn{2}{l}{DirectionClassType} & \multicolumn{2}{|l|}{See section \ref{type:DirectionClassType}} \\
HasSubtype & ObjectType & \multicolumn{2}{l}{DoorStateClassType} & \multicolumn{2}{|l|}{See section \ref{type:DoorStateClassType}} \\
HasSubtype & ObjectType & \multicolumn{2}{l}{EmergencyStopClassType} & \multicolumn{2}{|l|}{See section \ref{type:EmergencyStopClassType}} \\
HasSubtype & ObjectType & \multicolumn{2}{l}{EndOfBarClassType} & \multicolumn{2}{|l|}{See section \ref{type:EndOfBarClassType}} \\
HasSubtype & ObjectType & \multicolumn{2}{l}{EquipmentModeClassType} & \multicolumn{2}{|l|}{See section \ref{type:EquipmentModeClassType}} \\
HasSubtype & ObjectType & \multicolumn{2}{l}{ExecutionClassType} & \multicolumn{2}{|l|}{See section \ref{type:ExecutionClassType}} \\
HasSubtype & ObjectType & \multicolumn{2}{l}{FunctionalModeClassType} & \multicolumn{2}{|l|}{See section \ref{type:FunctionalModeClassType}} \\
HasSubtype & ObjectType & \multicolumn{2}{l}{InterfaceStateClassType} & \multicolumn{2}{|l|}{See section \ref{type:InterfaceStateClassType}} \\
HasSubtype & ObjectType & \multicolumn{2}{l}{MaterialChangeClassType} & \multicolumn{2}{|l|}{See section \ref{type:MaterialChangeClassType}} \\
HasSubtype & ObjectType & \multicolumn{2}{l}{MaterialFeedClassType} & \multicolumn{2}{|l|}{See section \ref{type:MaterialFeedClassType}} \\
HasSubtype & ObjectType & \multicolumn{2}{l}{MaterialLoadClassType} & \multicolumn{2}{|l|}{See section \ref{type:MaterialLoadClassType}} \\
HasSubtype & ObjectType & \multicolumn{2}{l}{MaterialRetractClassType} & \multicolumn{2}{|l|}{See section \ref{type:MaterialRetractClassType}} \\
HasSubtype & ObjectType & \multicolumn{2}{l}{MaterialUnloadClassType} & \multicolumn{2}{|l|}{See section \ref{type:MaterialUnloadClassType}} \\
HasSubtype & ObjectType & \multicolumn{2}{l}{OpenChuckClassType} & \multicolumn{2}{|l|}{See section \ref{type:OpenChuckClassType}} \\
HasSubtype & ObjectType & \multicolumn{2}{l}{OpenDoorClassType} & \multicolumn{2}{|l|}{See section \ref{type:OpenDoorClassType}} \\
HasSubtype & ObjectType & \multicolumn{2}{l}{PartChangeClassType} & \multicolumn{2}{|l|}{See section \ref{type:PartChangeClassType}} \\
HasSubtype & ObjectType & \multicolumn{2}{l}{PathModeClassType} & \multicolumn{2}{|l|}{See section \ref{type:PathModeClassType}} \\
HasSubtype & ObjectType & \multicolumn{2}{l}{PowerStateClassType} & \multicolumn{2}{|l|}{See section \ref{type:PowerStateClassType}} \\
HasSubtype & ObjectType & \multicolumn{2}{l}{ProgramEditClassType} & \multicolumn{2}{|l|}{See section \ref{type:ProgramEditClassType}} \\
HasSubtype & ObjectType & \multicolumn{2}{l}{RotaryModeClassType} & \multicolumn{2}{|l|}{See section \ref{type:RotaryModeClassType}} \\
HasSubtype & ObjectType & \multicolumn{2}{l}{SpindleInterlockClassType} & \multicolumn{2}{|l|}{See section \ref{type:SpindleInterlockClassType}} \\
\multicolumn{6}{|l|}{Continued...} \\
\end{tabu}
\end{table}
\begin{table}[ht]
\fontsize{9pt}{11pt}\selectfont
\tabulinesep=3pt
\begin{tabu} to 6in {|X[-1.35]|X[-0.7]|X[-1.75]|X[-1.5]|X[-1]|X[-0.7]|} \everyrow{\hline}
\hline
\rowfont \bfseries References & NodeClass & BrowseName & DataType & Type\-Definition & {Modeling\-Rule} \\
HasSubtype & ObjectType & \multicolumn{2}{l}{ActuatorStateClassType} & \multicolumn{2}{|l|}{See section \ref{type:ActuatorStateClassType}} \\
HasSubtype & ObjectType & \multicolumn{2}{l}{AvailabilityClassType} & \multicolumn{2}{|l|}{See section \ref{type:AvailabilityClassType}} \\
HasSubtype & ObjectType & \multicolumn{2}{l}{AxisCouplingClassType} & \multicolumn{2}{|l|}{See section \ref{type:AxisCouplingClassType}} \\
HasSubtype & ObjectType & \multicolumn{2}{l}{AxisInterlockClassType} & \multicolumn{2}{|l|}{See section \ref{type:AxisInterlockClassType}} \\
HasSubtype & ObjectType & \multicolumn{2}{l}{AxisStateClassType} & \multicolumn{2}{|l|}{See section \ref{type:AxisStateClassType}} \\
HasSubtype & ObjectType & \multicolumn{2}{l}{ChuckInterlockClassType} & \multicolumn{2}{|l|}{See section \ref{type:ChuckInterlockClassType}} \\
HasSubtype & ObjectType & \multicolumn{2}{l}{ChuckStateClassType} & \multicolumn{2}{|l|}{See section \ref{type:ChuckStateClassType}} \\
HasSubtype & ObjectType & \multicolumn{2}{l}{CloseChuckClassType} & \multicolumn{2}{|l|}{See section \ref{type:CloseChuckClassType}} \\
HasSubtype & ObjectType & \multicolumn{2}{l}{CloseDoorClassType} & \multicolumn{2}{|l|}{See section \ref{type:CloseDoorClassType}} \\
HasSubtype & ObjectType & \multicolumn{2}{l}{CompositionStateClassType} & \multicolumn{2}{|l|}{See section \ref{type:CompositionStateClassType}} \\
HasSubtype & ObjectType & \multicolumn{2}{l}{ControllerModeClassType} & \multicolumn{2}{|l|}{See section \ref{type:ControllerModeClassType}} \\
\end{tabu}
\end{table} 


\FloatBarrier
\subsubsection{Defintion of \texttt{ ActuatorStateClassType}}
  \label{type:ActuatorStateClassType}

\FloatBarrier
\begin{table}[ht]
\centering 
  \caption{\texttt{ActuatorStateClassType} Definition}
  \label{table:ActuatorStateClassType}
\fontsize{9pt}{11pt}\selectfont
\tabulinesep=3pt
\begin{tabu} to 6in {|X[-1.35]|X[-0.7]|X[-1.75]|X[-1.5]|X[-1]|X[-0.7]|} \everyrow{\hline}
\hline
\rowfont\bfseries {Attribute} & \multicolumn{5}{|l|}{Value} \\
\tabucline[1.5pt]{}
BrowseName & \multicolumn{5}{|l|}{ActuatorStateClassType} \\
IsAbstract & \multicolumn{5}{|l|}{False} \\
\tabucline[1.5pt]{}
\rowfont \bfseries References & NodeClass & BrowseName & DataType & Type\-Definition & {Modeling\-Rule} \\
\multicolumn{6}{|l|}{Subtype of MTControlledVocabEventClassType (See section \ref{type:MTControlledVocabEventClassType})} \\
values & Variable & Enum\-Strings & Active\-State\-Data\-Type & Active\-State\-Data\-Type & Mandatory \\
\end{tabu}
\end{table} 


\FloatBarrier
\paragraph{Referenced Properties and Objects}

\begin{itemize}
\item \textbf{Allowable Values} for \texttt{ActiveStateDataType}
\FloatBarrier
\begin{table}[ht]
\centering 
  \caption{\texttt{ActiveStateDataType} Enumeration}
  \label{enum:ActiveStateDataType}
\tabulinesep=3pt
\begin{tabu} to 6in {|l|r|} \everyrow{\hline}
\hline
\rowfont\bfseries {Name} & {Index} \\
\tabucline[1.5pt]{}
\texttt{ACTIVE} & \texttt{0} \\
\texttt{INACTIVE} & \texttt{1} \\
\end{tabu}
\end{table} 
\FloatBarrier
\end{itemize}
\FloatBarrier
\subsubsection{Defintion of \texttt{ AvailabilityClassType}}
  \label{type:AvailabilityClassType}

\FloatBarrier
\begin{table}[ht]
\centering 
  \caption{\texttt{AvailabilityClassType} Definition}
  \label{table:AvailabilityClassType}
\fontsize{9pt}{11pt}\selectfont
\tabulinesep=3pt
\begin{tabu} to 6in {|X[-1.35]|X[-0.7]|X[-1.75]|X[-1.5]|X[-1]|X[-0.7]|} \everyrow{\hline}
\hline
\rowfont\bfseries {Attribute} & \multicolumn{5}{|l|}{Value} \\
\tabucline[1.5pt]{}
BrowseName & \multicolumn{5}{|l|}{AvailabilityClassType} \\
IsAbstract & \multicolumn{5}{|l|}{False} \\
\tabucline[1.5pt]{}
\rowfont \bfseries References & NodeClass & BrowseName & DataType & Type\-Definition & {Modeling\-Rule} \\
\multicolumn{6}{|l|}{Subtype of MTControlledVocabEventClassType (See section \ref{type:MTControlledVocabEventClassType})} \\
values & Variable & Enum\-Strings & Availability\-Data\-Type & Availability\-Data\-Type & Mandatory \\
\end{tabu}
\end{table} 


\FloatBarrier
\paragraph{Referenced Properties and Objects}

\begin{itemize}
\item \textbf{Allowable Values} for \texttt{AvailabilityDataType}
\FloatBarrier
\begin{table}[ht]
\centering 
  \caption{\texttt{AvailabilityDataType} Enumeration}
  \label{enum:AvailabilityDataType}
\tabulinesep=3pt
\begin{tabu} to 6in {|l|r|} \everyrow{\hline}
\hline
\rowfont\bfseries {Name} & {Index} \\
\tabucline[1.5pt]{}
\texttt{AVAILABLE} & \texttt{0} \\
\texttt{UNAVAILABLE} & \texttt{1} \\
\end{tabu}
\end{table} 
\FloatBarrier
\end{itemize}
\FloatBarrier
\subsubsection{Defintion of \texttt{ AxisCouplingClassType}}
  \label{type:AxisCouplingClassType}

\FloatBarrier
\begin{table}[ht]
\centering 
  \caption{\texttt{AxisCouplingClassType} Definition}
  \label{table:AxisCouplingClassType}
\fontsize{9pt}{11pt}\selectfont
\tabulinesep=3pt
\begin{tabu} to 6in {|X[-1.35]|X[-0.7]|X[-1.75]|X[-1.5]|X[-1]|X[-0.7]|} \everyrow{\hline}
\hline
\rowfont\bfseries {Attribute} & \multicolumn{5}{|l|}{Value} \\
\tabucline[1.5pt]{}
BrowseName & \multicolumn{5}{|l|}{AxisCouplingClassType} \\
IsAbstract & \multicolumn{5}{|l|}{False} \\
\tabucline[1.5pt]{}
\rowfont \bfseries References & NodeClass & BrowseName & DataType & Type\-Definition & {Modeling\-Rule} \\
\multicolumn{6}{|l|}{Subtype of MTControlledVocabEventClassType (See section \ref{type:MTControlledVocabEventClassType})} \\
values & Variable & Enum\-Strings & Axis\-Coupling\-Data\-Type & Axis\-Coupling\-Data\-Type & Mandatory \\
\end{tabu}
\end{table} 


\FloatBarrier
\paragraph{Referenced Properties and Objects}

\begin{itemize}
\item \textbf{Allowable Values} for \texttt{AxisCouplingDataType}
\FloatBarrier
\begin{table}[ht]
\centering 
  \caption{\texttt{AxisCouplingDataType} Enumeration}
  \label{enum:AxisCouplingDataType}
\tabulinesep=3pt
\begin{tabu} to 6in {|l|r|} \everyrow{\hline}
\hline
\rowfont\bfseries {Name} & {Index} \\
\tabucline[1.5pt]{}
\texttt{MASTER} & \texttt{0} \\
\texttt{SLAVE} & \texttt{1} \\
\texttt{SYNCHRONOUS} & \texttt{2} \\
\texttt{TANDEM} & \texttt{3} \\
\end{tabu}
\end{table} 
\FloatBarrier
\end{itemize}
\FloatBarrier
\subsubsection{Defintion of \texttt{ AxisInterlockClassType}}
  \label{type:AxisInterlockClassType}

\FloatBarrier
\begin{table}[ht]
\centering 
  \caption{\texttt{AxisInterlockClassType} Definition}
  \label{table:AxisInterlockClassType}
\fontsize{9pt}{11pt}\selectfont
\tabulinesep=3pt
\begin{tabu} to 6in {|X[-1.35]|X[-0.7]|X[-1.75]|X[-1.5]|X[-1]|X[-0.7]|} \everyrow{\hline}
\hline
\rowfont\bfseries {Attribute} & \multicolumn{5}{|l|}{Value} \\
\tabucline[1.5pt]{}
BrowseName & \multicolumn{5}{|l|}{AxisInterlockClassType} \\
IsAbstract & \multicolumn{5}{|l|}{False} \\
\tabucline[1.5pt]{}
\rowfont \bfseries References & NodeClass & BrowseName & DataType & Type\-Definition & {Modeling\-Rule} \\
\multicolumn{6}{|l|}{Subtype of MTControlledVocabEventClassType (See section \ref{type:MTControlledVocabEventClassType})} \\
values & Variable & Enum\-Strings & Active\-State\-Data\-Type & Active\-State\-Data\-Type & Mandatory \\
\end{tabu}
\end{table} 


\FloatBarrier
\paragraph{Referenced Properties and Objects}

\begin{itemize}
\item \textbf{Allowable Values} for \texttt{ActiveStateDataType}
\FloatBarrier
\begin{table}[ht]
\centering 
  \caption{\texttt{ActiveStateDataType} Enumeration}
\tabulinesep=3pt
\begin{tabu} to 6in {|l|r|} \everyrow{\hline}
\hline
\rowfont\bfseries {Name} & {Index} \\
\tabucline[1.5pt]{}
\texttt{ACTIVE} & \texttt{0} \\
\texttt{INACTIVE} & \texttt{1} \\
\end{tabu}
\end{table} 
\FloatBarrier
\end{itemize}
\FloatBarrier
\subsubsection{Defintion of \texttt{ AxisStateClassType}}
  \label{type:AxisStateClassType}

\FloatBarrier
\begin{table}[ht]
\centering 
  \caption{\texttt{AxisStateClassType} Definition}
  \label{table:AxisStateClassType}
\fontsize{9pt}{11pt}\selectfont
\tabulinesep=3pt
\begin{tabu} to 6in {|X[-1.35]|X[-0.7]|X[-1.75]|X[-1.5]|X[-1]|X[-0.7]|} \everyrow{\hline}
\hline
\rowfont\bfseries {Attribute} & \multicolumn{5}{|l|}{Value} \\
\tabucline[1.5pt]{}
BrowseName & \multicolumn{5}{|l|}{AxisStateClassType} \\
IsAbstract & \multicolumn{5}{|l|}{False} \\
\tabucline[1.5pt]{}
\rowfont \bfseries References & NodeClass & BrowseName & DataType & Type\-Definition & {Modeling\-Rule} \\
\multicolumn{6}{|l|}{Subtype of MTControlledVocabEventClassType (See section \ref{type:MTControlledVocabEventClassType})} \\
values & Variable & Enum\-Strings & Axis\-State\-Data\-Type & Axis\-State\-Data\-Type & Mandatory \\
\end{tabu}
\end{table} 


\FloatBarrier
\paragraph{Referenced Properties and Objects}

\begin{itemize}
\item \textbf{Allowable Values} for \texttt{AxisStateDataType}
\FloatBarrier
\begin{table}[ht]
\centering 
  \caption{\texttt{AxisStateDataType} Enumeration}
  \label{enum:AxisStateDataType}
\tabulinesep=3pt
\begin{tabu} to 6in {|l|r|} \everyrow{\hline}
\hline
\rowfont\bfseries {Name} & {Index} \\
\tabucline[1.5pt]{}
\texttt{HOME} & \texttt{0} \\
\texttt{PARKED} & \texttt{1} \\
\texttt{STOPPED} & \texttt{2} \\
\texttt{TRAVEL} & \texttt{3} \\
\end{tabu}
\end{table} 
\FloatBarrier
\end{itemize}
\FloatBarrier
\subsubsection{Defintion of \texttt{ ChuckInterlockClassType}}
  \label{type:ChuckInterlockClassType}

\FloatBarrier
\begin{table}[ht]
\centering 
  \caption{\texttt{ChuckInterlockClassType} Definition}
  \label{table:ChuckInterlockClassType}
\fontsize{9pt}{11pt}\selectfont
\tabulinesep=3pt
\begin{tabu} to 6in {|X[-1.35]|X[-0.7]|X[-1.75]|X[-1.5]|X[-1]|X[-0.7]|} \everyrow{\hline}
\hline
\rowfont\bfseries {Attribute} & \multicolumn{5}{|l|}{Value} \\
\tabucline[1.5pt]{}
BrowseName & \multicolumn{5}{|l|}{ChuckInterlockClassType} \\
IsAbstract & \multicolumn{5}{|l|}{False} \\
\tabucline[1.5pt]{}
\rowfont \bfseries References & NodeClass & BrowseName & DataType & Type\-Definition & {Modeling\-Rule} \\
\multicolumn{6}{|l|}{Subtype of MTControlledVocabEventClassType (See section \ref{type:MTControlledVocabEventClassType})} \\
values & Variable & Enum\-Strings & Active\-State\-Data\-Type & Active\-State\-Data\-Type & Mandatory \\
\end{tabu}
\end{table} 


\FloatBarrier
\paragraph{Referenced Properties and Objects}

\begin{itemize}
\item \textbf{Allowable Values} for \texttt{ActiveStateDataType}
\FloatBarrier
\begin{table}[ht]
\centering 
  \caption{\texttt{ActiveStateDataType} Enumeration}
\tabulinesep=3pt
\begin{tabu} to 6in {|l|r|} \everyrow{\hline}
\hline
\rowfont\bfseries {Name} & {Index} \\
\tabucline[1.5pt]{}
\texttt{ACTIVE} & \texttt{0} \\
\texttt{INACTIVE} & \texttt{1} \\
\end{tabu}
\end{table} 
\FloatBarrier
\end{itemize}
\FloatBarrier
\subsubsection{Defintion of \texttt{ ChuckStateClassType}}
  \label{type:ChuckStateClassType}

\FloatBarrier
\begin{table}[ht]
\centering 
  \caption{\texttt{ChuckStateClassType} Definition}
  \label{table:ChuckStateClassType}
\fontsize{9pt}{11pt}\selectfont
\tabulinesep=3pt
\begin{tabu} to 6in {|X[-1.35]|X[-0.7]|X[-1.75]|X[-1.5]|X[-1]|X[-0.7]|} \everyrow{\hline}
\hline
\rowfont\bfseries {Attribute} & \multicolumn{5}{|l|}{Value} \\
\tabucline[1.5pt]{}
BrowseName & \multicolumn{5}{|l|}{ChuckStateClassType} \\
IsAbstract & \multicolumn{5}{|l|}{False} \\
\tabucline[1.5pt]{}
\rowfont \bfseries References & NodeClass & BrowseName & DataType & Type\-Definition & {Modeling\-Rule} \\
\multicolumn{6}{|l|}{Subtype of MTControlledVocabEventClassType (See section \ref{type:MTControlledVocabEventClassType})} \\
values & Variable & Enum\-Strings & Open\-State\-Data\-Type & Open\-State\-Data\-Type & Mandatory \\
\end{tabu}
\end{table} 


\FloatBarrier
\paragraph{Referenced Properties and Objects}

\begin{itemize}
\item \textbf{Allowable Values} for \texttt{OpenStateDataType}
\FloatBarrier
\begin{table}[ht]
\centering 
  \caption{\texttt{OpenStateDataType} Enumeration}
  \label{enum:OpenStateDataType}
\tabulinesep=3pt
\begin{tabu} to 6in {|l|r|} \everyrow{\hline}
\hline
\rowfont\bfseries {Name} & {Index} \\
\tabucline[1.5pt]{}
\texttt{CLOSED} & \texttt{0} \\
\texttt{OPEN} & \texttt{1} \\
\texttt{UNLATCHED} & \texttt{2} \\
\end{tabu}
\end{table} 
\FloatBarrier
\end{itemize}
\FloatBarrier
\subsubsection{Defintion of \texttt{ CloseChuckClassType}}
  \label{type:CloseChuckClassType}

\FloatBarrier
\begin{table}[ht]
\centering 
  \caption{\texttt{CloseChuckClassType} Definition}
  \label{table:CloseChuckClassType}
\fontsize{9pt}{11pt}\selectfont
\tabulinesep=3pt
\begin{tabu} to 6in {|X[-1.35]|X[-0.7]|X[-1.75]|X[-1.5]|X[-1]|X[-0.7]|} \everyrow{\hline}
\hline
\rowfont\bfseries {Attribute} & \multicolumn{5}{|l|}{Value} \\
\tabucline[1.5pt]{}
BrowseName & \multicolumn{5}{|l|}{CloseChuckClassType} \\
IsAbstract & \multicolumn{5}{|l|}{False} \\
\tabucline[1.5pt]{}
\rowfont \bfseries References & NodeClass & BrowseName & DataType & Type\-Definition & {Modeling\-Rule} \\
\multicolumn{6}{|l|}{Subtype of MTControlledVocabEventClassType (See section \ref{type:MTControlledVocabEventClassType})} \\
values & Variable & Enum\-Strings & Interface\-State\-Data\-Type & Interface\-State\-Data\-Type & Mandatory \\
\end{tabu}
\end{table} 


\FloatBarrier
\paragraph{Referenced Properties and Objects}

\begin{itemize}
\item \textbf{Allowable Values} for \texttt{InterfaceStateDataType}
\FloatBarrier
\begin{table}[ht]
\centering 
  \caption{\texttt{InterfaceStateDataType} Enumeration}
  \label{enum:InterfaceStateDataType}
\tabulinesep=3pt
\begin{tabu} to 6in {|l|r|} \everyrow{\hline}
\hline
\rowfont\bfseries {Name} & {Index} \\
\tabucline[1.5pt]{}
\texttt{ACTIVE} & \texttt{0} \\
\texttt{COMPLETE} & \texttt{1} \\
\texttt{FAIL} & \texttt{2} \\
\texttt{NOT_READY} & \texttt{4} \\
\texttt{READY} & \texttt{5} \\
\end{tabu}
\end{table} 
\FloatBarrier
\end{itemize}
\FloatBarrier
\subsubsection{Defintion of \texttt{ CloseDoorClassType}}
  \label{type:CloseDoorClassType}

\FloatBarrier
\begin{table}[ht]
\centering 
  \caption{\texttt{CloseDoorClassType} Definition}
  \label{table:CloseDoorClassType}
\fontsize{9pt}{11pt}\selectfont
\tabulinesep=3pt
\begin{tabu} to 6in {|X[-1.35]|X[-0.7]|X[-1.75]|X[-1.5]|X[-1]|X[-0.7]|} \everyrow{\hline}
\hline
\rowfont\bfseries {Attribute} & \multicolumn{5}{|l|}{Value} \\
\tabucline[1.5pt]{}
BrowseName & \multicolumn{5}{|l|}{CloseDoorClassType} \\
IsAbstract & \multicolumn{5}{|l|}{False} \\
\tabucline[1.5pt]{}
\rowfont \bfseries References & NodeClass & BrowseName & DataType & Type\-Definition & {Modeling\-Rule} \\
\multicolumn{6}{|l|}{Subtype of MTControlledVocabEventClassType (See section \ref{type:MTControlledVocabEventClassType})} \\
values & Variable & Enum\-Strings & Interface\-State\-Data\-Type & Interface\-State\-Data\-Type & Mandatory \\
\end{tabu}
\end{table} 


\FloatBarrier
\paragraph{Referenced Properties and Objects}

\begin{itemize}
\item \textbf{Allowable Values} for \texttt{InterfaceStateDataType}
\FloatBarrier
\begin{table}[ht]
\centering 
  \caption{\texttt{InterfaceStateDataType} Enumeration}
\tabulinesep=3pt
\begin{tabu} to 6in {|l|r|} \everyrow{\hline}
\hline
\rowfont\bfseries {Name} & {Index} \\
\tabucline[1.5pt]{}
\texttt{ACTIVE} & \texttt{0} \\
\texttt{COMPLETE} & \texttt{1} \\
\texttt{FAIL} & \texttt{2} \\
\texttt{NOT_READY} & \texttt{4} \\
\texttt{READY} & \texttt{5} \\
\end{tabu}
\end{table} 
\FloatBarrier
\end{itemize}
\FloatBarrier
\subsubsection{Defintion of \texttt{ CompositionStateClassType}}
  \label{type:CompositionStateClassType}

\FloatBarrier
\begin{table}[ht]
\centering 
  \caption{\texttt{CompositionStateClassType} Definition}
  \label{table:CompositionStateClassType}
\fontsize{9pt}{11pt}\selectfont
\tabulinesep=3pt
\begin{tabu} to 6in {|X[-1.35]|X[-0.7]|X[-1.75]|X[-1.5]|X[-1]|X[-0.7]|} \everyrow{\hline}
\hline
\rowfont\bfseries {Attribute} & \multicolumn{5}{|l|}{Value} \\
\tabucline[1.5pt]{}
BrowseName & \multicolumn{5}{|l|}{CompositionStateClassType} \\
IsAbstract & \multicolumn{5}{|l|}{False} \\
\tabucline[1.5pt]{}
\rowfont \bfseries References & NodeClass & BrowseName & DataType & Type\-Definition & {Modeling\-Rule} \\
\multicolumn{6}{|l|}{Subtype of MTControlledVocabEventClassType (See section \ref{type:MTControlledVocabEventClassType})} \\
values & Variable & Enum\-Strings & Composition\-State\-Data\-Type & Composition\-State\-Data\-Type & Mandatory \\
\end{tabu}
\end{table} 


\FloatBarrier
\paragraph{Referenced Properties and Objects}

\begin{itemize}
\item \textbf{Allowable Values} for \texttt{CompositionStateDataType}
\FloatBarrier
\begin{table}[ht]
\centering 
  \caption{\texttt{CompositionStateDataType} Enumeration}
  \label{enum:CompositionStateDataType}
\tabulinesep=3pt
\begin{tabu} to 6in {|l|r|} \everyrow{\hline}
\hline
\rowfont\bfseries {Name} & {Index} \\
\tabucline[1.5pt]{}
\texttt{ACTIVE} & \texttt{0} \\
\texttt{CLOSED} & \texttt{1} \\
\texttt{DOWN} & \texttt{2} \\
\texttt{INACTIVE} & \texttt{3} \\
\texttt{LEFT} & \texttt{4} \\
\texttt{OFF} & \texttt{5} \\
\texttt{ON} & \texttt{6} \\
\texttt{OPEN} & \texttt{7} \\
\texttt{RIGHT} & \texttt{8} \\
\texttt{TRANSITIONING} & \texttt{9} \\
\texttt{UNLATCHED} & \texttt{10} \\
\texttt{UP} & \texttt{11} \\
\end{tabu}
\end{table} 
\FloatBarrier
\end{itemize}
\FloatBarrier
\subsubsection{Defintion of \texttt{ ControllerModeClassType}}
  \label{type:ControllerModeClassType}

\FloatBarrier
\begin{table}[ht]
\centering 
  \caption{\texttt{ControllerModeClassType} Definition}
  \label{table:ControllerModeClassType}
\fontsize{9pt}{11pt}\selectfont
\tabulinesep=3pt
\begin{tabu} to 6in {|X[-1.35]|X[-0.7]|X[-1.75]|X[-1.5]|X[-1]|X[-0.7]|} \everyrow{\hline}
\hline
\rowfont\bfseries {Attribute} & \multicolumn{5}{|l|}{Value} \\
\tabucline[1.5pt]{}
BrowseName & \multicolumn{5}{|l|}{ControllerModeClassType} \\
IsAbstract & \multicolumn{5}{|l|}{False} \\
\tabucline[1.5pt]{}
\rowfont \bfseries References & NodeClass & BrowseName & DataType & Type\-Definition & {Modeling\-Rule} \\
\multicolumn{6}{|l|}{Subtype of MTControlledVocabEventClassType (See section \ref{type:MTControlledVocabEventClassType})} \\
values & Variable & Enum\-Strings & Controller\-Mode\-Data\-Type & Controller\-Mode\-Data\-Type & Mandatory \\
\end{tabu}
\end{table} 


\FloatBarrier
\paragraph{Referenced Properties and Objects}

\begin{itemize}
\item \textbf{Allowable Values} for \texttt{ControllerModeDataType}
\FloatBarrier
\begin{table}[ht]
\centering 
  \caption{\texttt{ControllerModeDataType} Enumeration}
  \label{enum:ControllerModeDataType}
\tabulinesep=3pt
\begin{tabu} to 6in {|l|r|} \everyrow{\hline}
\hline
\rowfont\bfseries {Name} & {Index} \\
\tabucline[1.5pt]{}
\texttt{AUTOMATIC} & \texttt{0} \\
\texttt{EDIT} & \texttt{1} \\
\texttt{MANUAL} & \texttt{2} \\
\texttt{MANUAL_DATA_INPUT} & \texttt{3} \\
\texttt{SEMI_AUTOMATIC} & \texttt{4} \\
\end{tabu}
\end{table} 
\FloatBarrier
\end{itemize}
\FloatBarrier
\subsubsection{Defintion of \texttt{ ControllerModeOverrideClassType}}
  \label{type:ControllerModeOverrideClassType}

\FloatBarrier
\begin{table}[ht]
\centering 
  \caption{\texttt{ControllerModeOverrideClassType} Definition}
  \label{table:ControllerModeOverrideClassType}
\fontsize{9pt}{11pt}\selectfont
\tabulinesep=3pt
\begin{tabu} to 6in {|X[-1.35]|X[-0.7]|X[-1.75]|X[-1.5]|X[-1]|X[-0.7]|} \everyrow{\hline}
\hline
\rowfont\bfseries {Attribute} & \multicolumn{5}{|l|}{Value} \\
\tabucline[1.5pt]{}
BrowseName & \multicolumn{5}{|l|}{ControllerModeOverrideClassType} \\
IsAbstract & \multicolumn{5}{|l|}{False} \\
\tabucline[1.5pt]{}
\rowfont \bfseries References & NodeClass & BrowseName & DataType & Type\-Definition & {Modeling\-Rule} \\
\multicolumn{6}{|l|}{Subtype of MTControlledVocabEventClassType (See section \ref{type:MTControlledVocabEventClassType})} \\
values & Variable & Enum\-Strings & On\-Off\-Data\-Type & On\-Off\-Data\-Type & Mandatory \\
\end{tabu}
\end{table} 


\FloatBarrier
\paragraph{Referenced Properties and Objects}

\begin{itemize}
\item \textbf{Allowable Values} for \texttt{OnOffDataType}
\FloatBarrier
\begin{table}[ht]
\centering 
  \caption{\texttt{OnOffDataType} Enumeration}
  \label{enum:OnOffDataType}
\tabulinesep=3pt
\begin{tabu} to 6in {|l|r|} \everyrow{\hline}
\hline
\rowfont\bfseries {Name} & {Index} \\
\tabucline[1.5pt]{}
\texttt{OFF} & \texttt{0} \\
\texttt{ON} & \texttt{1} \\
\end{tabu}
\end{table} 
\FloatBarrier
\end{itemize}
\FloatBarrier
\subsubsection{Defintion of \texttt{ DirectionClassType}}
  \label{type:DirectionClassType}

\FloatBarrier
\begin{table}[ht]
\centering 
  \caption{\texttt{DirectionClassType} Definition}
  \label{table:DirectionClassType}
\fontsize{9pt}{11pt}\selectfont
\tabulinesep=3pt
\begin{tabu} to 6in {|X[-1.35]|X[-0.7]|X[-1.75]|X[-1.5]|X[-1]|X[-0.7]|} \everyrow{\hline}
\hline
\rowfont\bfseries {Attribute} & \multicolumn{5}{|l|}{Value} \\
\tabucline[1.5pt]{}
BrowseName & \multicolumn{5}{|l|}{DirectionClassType} \\
IsAbstract & \multicolumn{5}{|l|}{False} \\
\tabucline[1.5pt]{}
\rowfont \bfseries References & NodeClass & BrowseName & DataType & Type\-Definition & {Modeling\-Rule} \\
\multicolumn{6}{|l|}{Subtype of MTControlledVocabEventClassType (See section \ref{type:MTControlledVocabEventClassType})} \\
values & Variable & Enum\-Strings & Direction\-Data\-Type & Direction\-Data\-Type & Mandatory \\
\end{tabu}
\end{table} 


\FloatBarrier
\paragraph{Referenced Properties and Objects}

\begin{itemize}
\item \textbf{Allowable Values} for \texttt{DirectionDataType}
\FloatBarrier
\begin{table}[ht]
\centering 
  \caption{\texttt{DirectionDataType} Enumeration}
  \label{enum:DirectionDataType}
\tabulinesep=3pt
\begin{tabu} to 6in {|l|r|} \everyrow{\hline}
\hline
\rowfont\bfseries {Name} & {Index} \\
\tabucline[1.5pt]{}
\texttt{CLOCKWISE} & \texttt{0} \\
\texttt{COUNTER_CLOCKWISE} & \texttt{1} \\
\texttt{NEGATIVE} & \texttt{2} \\
\texttt{POSITIVE} & \texttt{3} \\
\end{tabu}
\end{table} 
\FloatBarrier
\end{itemize}
\FloatBarrier
\subsubsection{Defintion of \texttt{ DoorStateClassType}}
  \label{type:DoorStateClassType}

\FloatBarrier
\begin{table}[ht]
\centering 
  \caption{\texttt{DoorStateClassType} Definition}
  \label{table:DoorStateClassType}
\fontsize{9pt}{11pt}\selectfont
\tabulinesep=3pt
\begin{tabu} to 6in {|X[-1.35]|X[-0.7]|X[-1.75]|X[-1.5]|X[-1]|X[-0.7]|} \everyrow{\hline}
\hline
\rowfont\bfseries {Attribute} & \multicolumn{5}{|l|}{Value} \\
\tabucline[1.5pt]{}
BrowseName & \multicolumn{5}{|l|}{DoorStateClassType} \\
IsAbstract & \multicolumn{5}{|l|}{False} \\
\tabucline[1.5pt]{}
\rowfont \bfseries References & NodeClass & BrowseName & DataType & Type\-Definition & {Modeling\-Rule} \\
\multicolumn{6}{|l|}{Subtype of MTControlledVocabEventClassType (See section \ref{type:MTControlledVocabEventClassType})} \\
values & Variable & Enum\-Strings & Open\-State\-Data\-Type & Open\-State\-Data\-Type & Mandatory \\
\end{tabu}
\end{table} 


\FloatBarrier
\paragraph{Referenced Properties and Objects}

\begin{itemize}
\item \textbf{Allowable Values} for \texttt{OpenStateDataType}
\FloatBarrier
\begin{table}[ht]
\centering 
  \caption{\texttt{OpenStateDataType} Enumeration}
\tabulinesep=3pt
\begin{tabu} to 6in {|l|r|} \everyrow{\hline}
\hline
\rowfont\bfseries {Name} & {Index} \\
\tabucline[1.5pt]{}
\texttt{CLOSED} & \texttt{0} \\
\texttt{OPEN} & \texttt{1} \\
\texttt{UNLATCHED} & \texttt{2} \\
\end{tabu}
\end{table} 
\FloatBarrier
\end{itemize}
\FloatBarrier
\subsubsection{Defintion of \texttt{ EmergencyStopClassType}}
  \label{type:EmergencyStopClassType}

\FloatBarrier
\begin{table}[ht]
\centering 
  \caption{\texttt{EmergencyStopClassType} Definition}
  \label{table:EmergencyStopClassType}
\fontsize{9pt}{11pt}\selectfont
\tabulinesep=3pt
\begin{tabu} to 6in {|X[-1.35]|X[-0.7]|X[-1.75]|X[-1.5]|X[-1]|X[-0.7]|} \everyrow{\hline}
\hline
\rowfont\bfseries {Attribute} & \multicolumn{5}{|l|}{Value} \\
\tabucline[1.5pt]{}
BrowseName & \multicolumn{5}{|l|}{EmergencyStopClassType} \\
IsAbstract & \multicolumn{5}{|l|}{False} \\
\tabucline[1.5pt]{}
\rowfont \bfseries References & NodeClass & BrowseName & DataType & Type\-Definition & {Modeling\-Rule} \\
\multicolumn{6}{|l|}{Subtype of MTControlledVocabEventClassType (See section \ref{type:MTControlledVocabEventClassType})} \\
values & Variable & Enum\-Strings & Emergency\-Stop\-Data\-Type & Emergency\-Stop\-Data\-Type & Mandatory \\
\end{tabu}
\end{table} 


\FloatBarrier
\paragraph{Referenced Properties and Objects}

\begin{itemize}
\item \textbf{Allowable Values} for \texttt{EmergencyStopDataType}
\FloatBarrier
\begin{table}[ht]
\centering 
  \caption{\texttt{EmergencyStopDataType} Enumeration}
  \label{enum:EmergencyStopDataType}
\tabulinesep=3pt
\begin{tabu} to 6in {|l|r|} \everyrow{\hline}
\hline
\rowfont\bfseries {Name} & {Index} \\
\tabucline[1.5pt]{}
\texttt{ARMED} & \texttt{0} \\
\texttt{TRIGGERED} & \texttt{1} \\
\end{tabu}
\end{table} 
\FloatBarrier
\end{itemize}
\FloatBarrier
\subsubsection{Defintion of \texttt{ EndOfBarClassType}}
  \label{type:EndOfBarClassType}

\FloatBarrier
\begin{table}[ht]
\centering 
  \caption{\texttt{EndOfBarClassType} Definition}
  \label{table:EndOfBarClassType}
\fontsize{9pt}{11pt}\selectfont
\tabulinesep=3pt
\begin{tabu} to 6in {|X[-1.35]|X[-0.7]|X[-1.75]|X[-1.5]|X[-1]|X[-0.7]|} \everyrow{\hline}
\hline
\rowfont\bfseries {Attribute} & \multicolumn{5}{|l|}{Value} \\
\tabucline[1.5pt]{}
BrowseName & \multicolumn{5}{|l|}{EndOfBarClassType} \\
IsAbstract & \multicolumn{5}{|l|}{False} \\
\tabucline[1.5pt]{}
\rowfont \bfseries References & NodeClass & BrowseName & DataType & Type\-Definition & {Modeling\-Rule} \\
\multicolumn{6}{|l|}{Subtype of MTControlledVocabEventClassType (See section \ref{type:MTControlledVocabEventClassType})} \\
values & Variable & Enum\-Strings & Yes\-No\-Data\-Type & Yes\-No\-Data\-Type & Mandatory \\
\end{tabu}
\end{table} 


\FloatBarrier
\paragraph{Referenced Properties and Objects}

\begin{itemize}
\item \textbf{Allowable Values} for \texttt{YesNoDataType}
\FloatBarrier
\begin{table}[ht]
\centering 
  \caption{\texttt{YesNoDataType} Enumeration}
  \label{enum:YesNoDataType}
\tabulinesep=3pt
\begin{tabu} to 6in {|l|r|} \everyrow{\hline}
\hline
\rowfont\bfseries {Name} & {Index} \\
\tabucline[1.5pt]{}
\texttt{NO} & \texttt{0} \\
\texttt{YES} & \texttt{1} \\
\end{tabu}
\end{table} 
\FloatBarrier
\end{itemize}
\FloatBarrier
\subsubsection{Defintion of \texttt{ EquipmentModeClassType}}
  \label{type:EquipmentModeClassType}

\FloatBarrier
\begin{table}[ht]
\centering 
  \caption{\texttt{EquipmentModeClassType} Definition}
  \label{table:EquipmentModeClassType}
\fontsize{9pt}{11pt}\selectfont
\tabulinesep=3pt
\begin{tabu} to 6in {|X[-1.35]|X[-0.7]|X[-1.75]|X[-1.5]|X[-1]|X[-0.7]|} \everyrow{\hline}
\hline
\rowfont\bfseries {Attribute} & \multicolumn{5}{|l|}{Value} \\
\tabucline[1.5pt]{}
BrowseName & \multicolumn{5}{|l|}{EquipmentModeClassType} \\
IsAbstract & \multicolumn{5}{|l|}{False} \\
\tabucline[1.5pt]{}
\rowfont \bfseries References & NodeClass & BrowseName & DataType & Type\-Definition & {Modeling\-Rule} \\
\multicolumn{6}{|l|}{Subtype of MTControlledVocabEventClassType (See section \ref{type:MTControlledVocabEventClassType})} \\
values & Variable & Enum\-Strings & On\-Off\-Data\-Type & On\-Off\-Data\-Type & Mandatory \\
\end{tabu}
\end{table} 


\FloatBarrier
\paragraph{Referenced Properties and Objects}

\begin{itemize}
\item \textbf{Allowable Values} for \texttt{OnOffDataType}
\FloatBarrier
\begin{table}[ht]
\centering 
  \caption{\texttt{OnOffDataType} Enumeration}
\tabulinesep=3pt
\begin{tabu} to 6in {|l|r|} \everyrow{\hline}
\hline
\rowfont\bfseries {Name} & {Index} \\
\tabucline[1.5pt]{}
\texttt{OFF} & \texttt{0} \\
\texttt{ON} & \texttt{1} \\
\end{tabu}
\end{table} 
\FloatBarrier
\end{itemize}
\FloatBarrier
\subsubsection{Defintion of \texttt{ ExecutionClassType}}
  \label{type:ExecutionClassType}

\FloatBarrier
\begin{table}[ht]
\centering 
  \caption{\texttt{ExecutionClassType} Definition}
  \label{table:ExecutionClassType}
\fontsize{9pt}{11pt}\selectfont
\tabulinesep=3pt
\begin{tabu} to 6in {|X[-1.35]|X[-0.7]|X[-1.75]|X[-1.5]|X[-1]|X[-0.7]|} \everyrow{\hline}
\hline
\rowfont\bfseries {Attribute} & \multicolumn{5}{|l|}{Value} \\
\tabucline[1.5pt]{}
BrowseName & \multicolumn{5}{|l|}{ExecutionClassType} \\
IsAbstract & \multicolumn{5}{|l|}{False} \\
\tabucline[1.5pt]{}
\rowfont \bfseries References & NodeClass & BrowseName & DataType & Type\-Definition & {Modeling\-Rule} \\
\multicolumn{6}{|l|}{Subtype of MTControlledVocabEventClassType (See section \ref{type:MTControlledVocabEventClassType})} \\
values & Variable & Enum\-Strings & Execution\-Data\-Type & Execution\-Data\-Type & Mandatory \\
\end{tabu}
\end{table} 


\FloatBarrier
\paragraph{Referenced Properties and Objects}

\begin{itemize}
\item \textbf{Allowable Values} for \texttt{ExecutionDataType}
\FloatBarrier
\begin{table}[ht]
\centering 
  \caption{\texttt{ExecutionDataType} Enumeration}
  \label{enum:ExecutionDataType}
\tabulinesep=3pt
\begin{tabu} to 6in {|l|r|} \everyrow{\hline}
\hline
\rowfont\bfseries {Name} & {Index} \\
\tabucline[1.5pt]{}
\texttt{ACTIVE} & \texttt{0} \\
\texttt{FEED_HOLD} & \texttt{1} \\
\texttt{INTERRUPTED} & \texttt{2} \\
\texttt{OPTIONAL_STOP} & \texttt{3} \\
\texttt{READY} & \texttt{4} \\
\texttt{PROGRAM_COMPLETED} & \texttt{5} \\
\texttt{PROGRAM_STOPPED} & \texttt{6} \\
\texttt{STOPPED} & \texttt{7} \\
\end{tabu}
\end{table} 
\FloatBarrier
\end{itemize}
\FloatBarrier
\subsubsection{Defintion of \texttt{ FunctionalModeClassType}}
  \label{type:FunctionalModeClassType}

\FloatBarrier
\begin{table}[ht]
\centering 
  \caption{\texttt{FunctionalModeClassType} Definition}
  \label{table:FunctionalModeClassType}
\fontsize{9pt}{11pt}\selectfont
\tabulinesep=3pt
\begin{tabu} to 6in {|X[-1.35]|X[-0.7]|X[-1.75]|X[-1.5]|X[-1]|X[-0.7]|} \everyrow{\hline}
\hline
\rowfont\bfseries {Attribute} & \multicolumn{5}{|l|}{Value} \\
\tabucline[1.5pt]{}
BrowseName & \multicolumn{5}{|l|}{FunctionalModeClassType} \\
IsAbstract & \multicolumn{5}{|l|}{False} \\
\tabucline[1.5pt]{}
\rowfont \bfseries References & NodeClass & BrowseName & DataType & Type\-Definition & {Modeling\-Rule} \\
\multicolumn{6}{|l|}{Subtype of MTControlledVocabEventClassType (See section \ref{type:MTControlledVocabEventClassType})} \\
values & Variable & Enum\-Strings & Functional\-Mode\-Data\-Type & Functional\-Mode\-Data\-Type & Mandatory \\
\end{tabu}
\end{table} 


\FloatBarrier
\paragraph{Referenced Properties and Objects}

\begin{itemize}
\item \textbf{Allowable Values} for \texttt{FunctionalModeDataType}
\FloatBarrier
\begin{table}[ht]
\centering 
  \caption{\texttt{FunctionalModeDataType} Enumeration}
  \label{enum:FunctionalModeDataType}
\tabulinesep=3pt
\begin{tabu} to 6in {|l|r|} \everyrow{\hline}
\hline
\rowfont\bfseries {Name} & {Index} \\
\tabucline[1.5pt]{}
\texttt{MAINTENANCE} & \texttt{0} \\
\texttt{PRODUCTION} & \texttt{1} \\
\texttt{PROCESS_DEVELOPMENT} & \texttt{2} \\
\texttt{SETUP} & \texttt{3} \\
\texttt{TEARDOWN} & \texttt{4} \\
\end{tabu}
\end{table} 
\FloatBarrier
\end{itemize}
\FloatBarrier
\subsubsection{Defintion of \texttt{ InterfaceStateClassType}}
  \label{type:InterfaceStateClassType}

\FloatBarrier
\begin{table}[ht]
\centering 
  \caption{\texttt{InterfaceStateClassType} Definition}
  \label{table:InterfaceStateClassType}
\fontsize{9pt}{11pt}\selectfont
\tabulinesep=3pt
\begin{tabu} to 6in {|X[-1.35]|X[-0.7]|X[-1.75]|X[-1.5]|X[-1]|X[-0.7]|} \everyrow{\hline}
\hline
\rowfont\bfseries {Attribute} & \multicolumn{5}{|l|}{Value} \\
\tabucline[1.5pt]{}
BrowseName & \multicolumn{5}{|l|}{InterfaceStateClassType} \\
IsAbstract & \multicolumn{5}{|l|}{False} \\
\tabucline[1.5pt]{}
\rowfont \bfseries References & NodeClass & BrowseName & DataType & Type\-Definition & {Modeling\-Rule} \\
\multicolumn{6}{|l|}{Subtype of MTControlledVocabEventClassType (See section \ref{type:MTControlledVocabEventClassType})} \\
Has\-Component & Variable & Enum\-Strings & Interface\-Status\-Data\-Type & Interface\-Status\-Data\-Type & Mandatory \\
\end{tabu}
\end{table} 


\FloatBarrier
\paragraph{Referenced Properties and Objects}

\begin{itemize}
\item \textbf{Allowable Values} for \texttt{InterfaceStatusDataType}
\FloatBarrier
\begin{table}[ht]
\centering 
  \caption{\texttt{InterfaceStatusDataType} Enumeration}
  \label{enum:InterfaceStatusDataType}
\tabulinesep=3pt
\begin{tabu} to 6in {|l|r|} \everyrow{\hline}
\hline
\rowfont\bfseries {Name} & {Index} \\
\tabucline[1.5pt]{}
\texttt{DISABLED} & \texttt{0} \\
\texttt{ENABLED} & \texttt{1} \\
\end{tabu}
\end{table} 
\FloatBarrier
\end{itemize}
\FloatBarrier
\subsubsection{Defintion of \texttt{ MaterialChangeClassType}}
  \label{type:MaterialChangeClassType}

\FloatBarrier
\begin{table}[ht]
\centering 
  \caption{\texttt{MaterialChangeClassType} Definition}
  \label{table:MaterialChangeClassType}
\fontsize{9pt}{11pt}\selectfont
\tabulinesep=3pt
\begin{tabu} to 6in {|X[-1.35]|X[-0.7]|X[-1.75]|X[-1.5]|X[-1]|X[-0.7]|} \everyrow{\hline}
\hline
\rowfont\bfseries {Attribute} & \multicolumn{5}{|l|}{Value} \\
\tabucline[1.5pt]{}
BrowseName & \multicolumn{5}{|l|}{MaterialChangeClassType} \\
IsAbstract & \multicolumn{5}{|l|}{False} \\
\tabucline[1.5pt]{}
\rowfont \bfseries References & NodeClass & BrowseName & DataType & Type\-Definition & {Modeling\-Rule} \\
\multicolumn{6}{|l|}{Subtype of MTControlledVocabEventClassType (See section \ref{type:MTControlledVocabEventClassType})} \\
values & Variable & Enum\-Strings & Interface\-State\-Data\-Type & Interface\-State\-Data\-Type & Mandatory \\
\end{tabu}
\end{table} 


\FloatBarrier
\paragraph{Referenced Properties and Objects}

\begin{itemize}
\item \textbf{Allowable Values} for \texttt{InterfaceStateDataType}
\FloatBarrier
\begin{table}[ht]
\centering 
  \caption{\texttt{InterfaceStateDataType} Enumeration}
\tabulinesep=3pt
\begin{tabu} to 6in {|l|r|} \everyrow{\hline}
\hline
\rowfont\bfseries {Name} & {Index} \\
\tabucline[1.5pt]{}
\texttt{ACTIVE} & \texttt{0} \\
\texttt{COMPLETE} & \texttt{1} \\
\texttt{FAIL} & \texttt{2} \\
\texttt{NOT_READY} & \texttt{4} \\
\texttt{READY} & \texttt{5} \\
\end{tabu}
\end{table} 
\FloatBarrier
\end{itemize}
\FloatBarrier
\subsubsection{Defintion of \texttt{ MaterialFeedClassType}}
  \label{type:MaterialFeedClassType}

\FloatBarrier
\begin{table}[ht]
\centering 
  \caption{\texttt{MaterialFeedClassType} Definition}
  \label{table:MaterialFeedClassType}
\fontsize{9pt}{11pt}\selectfont
\tabulinesep=3pt
\begin{tabu} to 6in {|X[-1.35]|X[-0.7]|X[-1.75]|X[-1.5]|X[-1]|X[-0.7]|} \everyrow{\hline}
\hline
\rowfont\bfseries {Attribute} & \multicolumn{5}{|l|}{Value} \\
\tabucline[1.5pt]{}
BrowseName & \multicolumn{5}{|l|}{MaterialFeedClassType} \\
IsAbstract & \multicolumn{5}{|l|}{False} \\
\tabucline[1.5pt]{}
\rowfont \bfseries References & NodeClass & BrowseName & DataType & Type\-Definition & {Modeling\-Rule} \\
\multicolumn{6}{|l|}{Subtype of MTControlledVocabEventClassType (See section \ref{type:MTControlledVocabEventClassType})} \\
values & Variable & Enum\-Strings & Interface\-State\-Data\-Type & Interface\-State\-Data\-Type & Mandatory \\
\end{tabu}
\end{table} 


\FloatBarrier
\paragraph{Referenced Properties and Objects}

\begin{itemize}
\item \textbf{Allowable Values} for \texttt{InterfaceStateDataType}
\FloatBarrier
\begin{table}[ht]
\centering 
  \caption{\texttt{InterfaceStateDataType} Enumeration}
\tabulinesep=3pt
\begin{tabu} to 6in {|l|r|} \everyrow{\hline}
\hline
\rowfont\bfseries {Name} & {Index} \\
\tabucline[1.5pt]{}
\texttt{ACTIVE} & \texttt{0} \\
\texttt{COMPLETE} & \texttt{1} \\
\texttt{FAIL} & \texttt{2} \\
\texttt{NOT_READY} & \texttt{4} \\
\texttt{READY} & \texttt{5} \\
\end{tabu}
\end{table} 
\FloatBarrier
\end{itemize}
\FloatBarrier
\subsubsection{Defintion of \texttt{ MaterialLoadClassType}}
  \label{type:MaterialLoadClassType}

\FloatBarrier
\begin{table}[ht]
\centering 
  \caption{\texttt{MaterialLoadClassType} Definition}
  \label{table:MaterialLoadClassType}
\fontsize{9pt}{11pt}\selectfont
\tabulinesep=3pt
\begin{tabu} to 6in {|X[-1.35]|X[-0.7]|X[-1.75]|X[-1.5]|X[-1]|X[-0.7]|} \everyrow{\hline}
\hline
\rowfont\bfseries {Attribute} & \multicolumn{5}{|l|}{Value} \\
\tabucline[1.5pt]{}
BrowseName & \multicolumn{5}{|l|}{MaterialLoadClassType} \\
IsAbstract & \multicolumn{5}{|l|}{False} \\
\tabucline[1.5pt]{}
\rowfont \bfseries References & NodeClass & BrowseName & DataType & Type\-Definition & {Modeling\-Rule} \\
\multicolumn{6}{|l|}{Subtype of MTControlledVocabEventClassType (See section \ref{type:MTControlledVocabEventClassType})} \\
values & Variable & Enum\-Strings & Interface\-State\-Data\-Type & Interface\-State\-Data\-Type & Mandatory \\
\end{tabu}
\end{table} 


\FloatBarrier
\paragraph{Referenced Properties and Objects}

\begin{itemize}
\item \textbf{Allowable Values} for \texttt{InterfaceStateDataType}
\FloatBarrier
\begin{table}[ht]
\centering 
  \caption{\texttt{InterfaceStateDataType} Enumeration}
\tabulinesep=3pt
\begin{tabu} to 6in {|l|r|} \everyrow{\hline}
\hline
\rowfont\bfseries {Name} & {Index} \\
\tabucline[1.5pt]{}
\texttt{ACTIVE} & \texttt{0} \\
\texttt{COMPLETE} & \texttt{1} \\
\texttt{FAIL} & \texttt{2} \\
\texttt{NOT_READY} & \texttt{4} \\
\texttt{READY} & \texttt{5} \\
\end{tabu}
\end{table} 
\FloatBarrier
\end{itemize}
\FloatBarrier
\subsubsection{Defintion of \texttt{ MaterialRetractClassType}}
  \label{type:MaterialRetractClassType}

\FloatBarrier
\begin{table}[ht]
\centering 
  \caption{\texttt{MaterialRetractClassType} Definition}
  \label{table:MaterialRetractClassType}
\fontsize{9pt}{11pt}\selectfont
\tabulinesep=3pt
\begin{tabu} to 6in {|X[-1.35]|X[-0.7]|X[-1.75]|X[-1.5]|X[-1]|X[-0.7]|} \everyrow{\hline}
\hline
\rowfont\bfseries {Attribute} & \multicolumn{5}{|l|}{Value} \\
\tabucline[1.5pt]{}
BrowseName & \multicolumn{5}{|l|}{MaterialRetractClassType} \\
IsAbstract & \multicolumn{5}{|l|}{False} \\
\tabucline[1.5pt]{}
\rowfont \bfseries References & NodeClass & BrowseName & DataType & Type\-Definition & {Modeling\-Rule} \\
\multicolumn{6}{|l|}{Subtype of MTControlledVocabEventClassType (See section \ref{type:MTControlledVocabEventClassType})} \\
values & Variable & Enum\-Strings & Interface\-State\-Data\-Type & Interface\-State\-Data\-Type & Mandatory \\
\end{tabu}
\end{table} 


\FloatBarrier
\paragraph{Referenced Properties and Objects}

\begin{itemize}
\item \textbf{Allowable Values} for \texttt{InterfaceStateDataType}
\FloatBarrier
\begin{table}[ht]
\centering 
  \caption{\texttt{InterfaceStateDataType} Enumeration}
\tabulinesep=3pt
\begin{tabu} to 6in {|l|r|} \everyrow{\hline}
\hline
\rowfont\bfseries {Name} & {Index} \\
\tabucline[1.5pt]{}
\texttt{ACTIVE} & \texttt{0} \\
\texttt{COMPLETE} & \texttt{1} \\
\texttt{FAIL} & \texttt{2} \\
\texttt{NOT_READY} & \texttt{4} \\
\texttt{READY} & \texttt{5} \\
\end{tabu}
\end{table} 
\FloatBarrier
\end{itemize}
\FloatBarrier
\subsubsection{Defintion of \texttt{ MaterialUnloadClassType}}
  \label{type:MaterialUnloadClassType}

\FloatBarrier
\begin{table}[ht]
\centering 
  \caption{\texttt{MaterialUnloadClassType} Definition}
  \label{table:MaterialUnloadClassType}
\fontsize{9pt}{11pt}\selectfont
\tabulinesep=3pt
\begin{tabu} to 6in {|X[-1.35]|X[-0.7]|X[-1.75]|X[-1.5]|X[-1]|X[-0.7]|} \everyrow{\hline}
\hline
\rowfont\bfseries {Attribute} & \multicolumn{5}{|l|}{Value} \\
\tabucline[1.5pt]{}
BrowseName & \multicolumn{5}{|l|}{MaterialUnloadClassType} \\
IsAbstract & \multicolumn{5}{|l|}{False} \\
\tabucline[1.5pt]{}
\rowfont \bfseries References & NodeClass & BrowseName & DataType & Type\-Definition & {Modeling\-Rule} \\
\multicolumn{6}{|l|}{Subtype of MTControlledVocabEventClassType (See section \ref{type:MTControlledVocabEventClassType})} \\
values & Variable & Enum\-Strings & Interface\-State\-Data\-Type & Interface\-State\-Data\-Type & Mandatory \\
\end{tabu}
\end{table} 


\FloatBarrier
\paragraph{Referenced Properties and Objects}

\begin{itemize}
\item \textbf{Allowable Values} for \texttt{InterfaceStateDataType}
\FloatBarrier
\begin{table}[ht]
\centering 
  \caption{\texttt{InterfaceStateDataType} Enumeration}
\tabulinesep=3pt
\begin{tabu} to 6in {|l|r|} \everyrow{\hline}
\hline
\rowfont\bfseries {Name} & {Index} \\
\tabucline[1.5pt]{}
\texttt{ACTIVE} & \texttt{0} \\
\texttt{COMPLETE} & \texttt{1} \\
\texttt{FAIL} & \texttt{2} \\
\texttt{NOT_READY} & \texttt{4} \\
\texttt{READY} & \texttt{5} \\
\end{tabu}
\end{table} 
\FloatBarrier
\end{itemize}
\FloatBarrier
\subsubsection{Defintion of \texttt{ OpenChuckClassType}}
  \label{type:OpenChuckClassType}

\FloatBarrier
\begin{table}[ht]
\centering 
  \caption{\texttt{OpenChuckClassType} Definition}
  \label{table:OpenChuckClassType}
\fontsize{9pt}{11pt}\selectfont
\tabulinesep=3pt
\begin{tabu} to 6in {|X[-1.35]|X[-0.7]|X[-1.75]|X[-1.5]|X[-1]|X[-0.7]|} \everyrow{\hline}
\hline
\rowfont\bfseries {Attribute} & \multicolumn{5}{|l|}{Value} \\
\tabucline[1.5pt]{}
BrowseName & \multicolumn{5}{|l|}{OpenChuckClassType} \\
IsAbstract & \multicolumn{5}{|l|}{False} \\
\tabucline[1.5pt]{}
\rowfont \bfseries References & NodeClass & BrowseName & DataType & Type\-Definition & {Modeling\-Rule} \\
\multicolumn{6}{|l|}{Subtype of MTControlledVocabEventClassType (See section \ref{type:MTControlledVocabEventClassType})} \\
values & Variable & Enum\-Strings & Interface\-State\-Data\-Type & Interface\-State\-Data\-Type & Mandatory \\
\end{tabu}
\end{table} 


\FloatBarrier
\paragraph{Referenced Properties and Objects}

\begin{itemize}
\item \textbf{Allowable Values} for \texttt{InterfaceStateDataType}
\FloatBarrier
\begin{table}[ht]
\centering 
  \caption{\texttt{InterfaceStateDataType} Enumeration}
\tabulinesep=3pt
\begin{tabu} to 6in {|l|r|} \everyrow{\hline}
\hline
\rowfont\bfseries {Name} & {Index} \\
\tabucline[1.5pt]{}
\texttt{ACTIVE} & \texttt{0} \\
\texttt{COMPLETE} & \texttt{1} \\
\texttt{FAIL} & \texttt{2} \\
\texttt{NOT_READY} & \texttt{4} \\
\texttt{READY} & \texttt{5} \\
\end{tabu}
\end{table} 
\FloatBarrier
\end{itemize}
\FloatBarrier
\subsubsection{Defintion of \texttt{ OpenDoorClassType}}
  \label{type:OpenDoorClassType}

\FloatBarrier
\begin{table}[ht]
\centering 
  \caption{\texttt{OpenDoorClassType} Definition}
  \label{table:OpenDoorClassType}
\fontsize{9pt}{11pt}\selectfont
\tabulinesep=3pt
\begin{tabu} to 6in {|X[-1.35]|X[-0.7]|X[-1.75]|X[-1.5]|X[-1]|X[-0.7]|} \everyrow{\hline}
\hline
\rowfont\bfseries {Attribute} & \multicolumn{5}{|l|}{Value} \\
\tabucline[1.5pt]{}
BrowseName & \multicolumn{5}{|l|}{OpenDoorClassType} \\
IsAbstract & \multicolumn{5}{|l|}{False} \\
\tabucline[1.5pt]{}
\rowfont \bfseries References & NodeClass & BrowseName & DataType & Type\-Definition & {Modeling\-Rule} \\
\multicolumn{6}{|l|}{Subtype of MTControlledVocabEventClassType (See section \ref{type:MTControlledVocabEventClassType})} \\
values & Variable & Enum\-Strings & Interface\-State\-Data\-Type & Interface\-State\-Data\-Type & Mandatory \\
\end{tabu}
\end{table} 


\FloatBarrier
\paragraph{Referenced Properties and Objects}

\begin{itemize}
\item \textbf{Allowable Values} for \texttt{InterfaceStateDataType}
\FloatBarrier
\begin{table}[ht]
\centering 
  \caption{\texttt{InterfaceStateDataType} Enumeration}
\tabulinesep=3pt
\begin{tabu} to 6in {|l|r|} \everyrow{\hline}
\hline
\rowfont\bfseries {Name} & {Index} \\
\tabucline[1.5pt]{}
\texttt{ACTIVE} & \texttt{0} \\
\texttt{COMPLETE} & \texttt{1} \\
\texttt{FAIL} & \texttt{2} \\
\texttt{NOT_READY} & \texttt{4} \\
\texttt{READY} & \texttt{5} \\
\end{tabu}
\end{table} 
\FloatBarrier
\end{itemize}
\FloatBarrier
\subsubsection{Defintion of \texttt{ PartChangeClassType}}
  \label{type:PartChangeClassType}

\FloatBarrier
\begin{table}[ht]
\centering 
  \caption{\texttt{PartChangeClassType} Definition}
  \label{table:PartChangeClassType}
\fontsize{9pt}{11pt}\selectfont
\tabulinesep=3pt
\begin{tabu} to 6in {|X[-1.35]|X[-0.7]|X[-1.75]|X[-1.5]|X[-1]|X[-0.7]|} \everyrow{\hline}
\hline
\rowfont\bfseries {Attribute} & \multicolumn{5}{|l|}{Value} \\
\tabucline[1.5pt]{}
BrowseName & \multicolumn{5}{|l|}{PartChangeClassType} \\
IsAbstract & \multicolumn{5}{|l|}{False} \\
\tabucline[1.5pt]{}
\rowfont \bfseries References & NodeClass & BrowseName & DataType & Type\-Definition & {Modeling\-Rule} \\
\multicolumn{6}{|l|}{Subtype of MTControlledVocabEventClassType (See section \ref{type:MTControlledVocabEventClassType})} \\
values & Variable & Enum\-Strings & Interface\-State\-Data\-Type & Interface\-State\-Data\-Type & Mandatory \\
\end{tabu}
\end{table} 


\FloatBarrier
\paragraph{Referenced Properties and Objects}

\begin{itemize}
\item \textbf{Allowable Values} for \texttt{InterfaceStateDataType}
\FloatBarrier
\begin{table}[ht]
\centering 
  \caption{\texttt{InterfaceStateDataType} Enumeration}
\tabulinesep=3pt
\begin{tabu} to 6in {|l|r|} \everyrow{\hline}
\hline
\rowfont\bfseries {Name} & {Index} \\
\tabucline[1.5pt]{}
\texttt{ACTIVE} & \texttt{0} \\
\texttt{COMPLETE} & \texttt{1} \\
\texttt{FAIL} & \texttt{2} \\
\texttt{NOT_READY} & \texttt{4} \\
\texttt{READY} & \texttt{5} \\
\end{tabu}
\end{table} 
\FloatBarrier
\end{itemize}
\FloatBarrier
\subsubsection{Defintion of \texttt{ PathModeClassType}}
  \label{type:PathModeClassType}

\FloatBarrier
\begin{table}[ht]
\centering 
  \caption{\texttt{PathModeClassType} Definition}
  \label{table:PathModeClassType}
\fontsize{9pt}{11pt}\selectfont
\tabulinesep=3pt
\begin{tabu} to 6in {|X[-1.35]|X[-0.7]|X[-1.75]|X[-1.5]|X[-1]|X[-0.7]|} \everyrow{\hline}
\hline
\rowfont\bfseries {Attribute} & \multicolumn{5}{|l|}{Value} \\
\tabucline[1.5pt]{}
BrowseName & \multicolumn{5}{|l|}{PathModeClassType} \\
IsAbstract & \multicolumn{5}{|l|}{False} \\
\tabucline[1.5pt]{}
\rowfont \bfseries References & NodeClass & BrowseName & DataType & Type\-Definition & {Modeling\-Rule} \\
\multicolumn{6}{|l|}{Subtype of MTControlledVocabEventClassType (See section \ref{type:MTControlledVocabEventClassType})} \\
values & Variable & Enum\-Strings & Path\-Mode\-Data\-Type & Path\-Mode\-Data\-Type & Mandatory \\
\end{tabu}
\end{table} 


\FloatBarrier
\paragraph{Referenced Properties and Objects}

\begin{itemize}
\item \textbf{Allowable Values} for \texttt{PathModeDataType}
\FloatBarrier
\begin{table}[ht]
\centering 
  \caption{\texttt{PathModeDataType} Enumeration}
  \label{enum:PathModeDataType}
\tabulinesep=3pt
\begin{tabu} to 6in {|l|r|} \everyrow{\hline}
\hline
\rowfont\bfseries {Name} & {Index} \\
\tabucline[1.5pt]{}
\texttt{INDEPENDENT} & \texttt{0} \\
\texttt{MASTER} & \texttt{1} \\
\texttt{MIRROR} & \texttt{2} \\
\texttt{SYNCHRONOUS} & \texttt{3} \\
\end{tabu}
\end{table} 
\FloatBarrier
\end{itemize}
\FloatBarrier
\subsubsection{Defintion of \texttt{ PowerStateClassType}}
  \label{type:PowerStateClassType}

\FloatBarrier
\begin{table}[ht]
\centering 
  \caption{\texttt{PowerStateClassType} Definition}
  \label{table:PowerStateClassType}
\fontsize{9pt}{11pt}\selectfont
\tabulinesep=3pt
\begin{tabu} to 6in {|X[-1.35]|X[-0.7]|X[-1.75]|X[-1.5]|X[-1]|X[-0.7]|} \everyrow{\hline}
\hline
\rowfont\bfseries {Attribute} & \multicolumn{5}{|l|}{Value} \\
\tabucline[1.5pt]{}
BrowseName & \multicolumn{5}{|l|}{PowerStateClassType} \\
IsAbstract & \multicolumn{5}{|l|}{False} \\
\tabucline[1.5pt]{}
\rowfont \bfseries References & NodeClass & BrowseName & DataType & Type\-Definition & {Modeling\-Rule} \\
\multicolumn{6}{|l|}{Subtype of MTControlledVocabEventClassType (See section \ref{type:MTControlledVocabEventClassType})} \\
values & Variable & Enum\-Strings & On\-Off\-Data\-Type & On\-Off\-Data\-Type & Mandatory \\
\end{tabu}
\end{table} 


\FloatBarrier
\paragraph{Referenced Properties and Objects}

\begin{itemize}
\item \textbf{Allowable Values} for \texttt{OnOffDataType}
\FloatBarrier
\begin{table}[ht]
\centering 
  \caption{\texttt{OnOffDataType} Enumeration}
\tabulinesep=3pt
\begin{tabu} to 6in {|l|r|} \everyrow{\hline}
\hline
\rowfont\bfseries {Name} & {Index} \\
\tabucline[1.5pt]{}
\texttt{OFF} & \texttt{0} \\
\texttt{ON} & \texttt{1} \\
\end{tabu}
\end{table} 
\FloatBarrier
\end{itemize}
\FloatBarrier
\subsubsection{Defintion of \texttt{ ProgramEditClassType}}
  \label{type:ProgramEditClassType}

\FloatBarrier
\begin{table}[ht]
\centering 
  \caption{\texttt{ProgramEditClassType} Definition}
  \label{table:ProgramEditClassType}
\fontsize{9pt}{11pt}\selectfont
\tabulinesep=3pt
\begin{tabu} to 6in {|X[-1.35]|X[-0.7]|X[-1.75]|X[-1.5]|X[-1]|X[-0.7]|} \everyrow{\hline}
\hline
\rowfont\bfseries {Attribute} & \multicolumn{5}{|l|}{Value} \\
\tabucline[1.5pt]{}
BrowseName & \multicolumn{5}{|l|}{ProgramEditClassType} \\
IsAbstract & \multicolumn{5}{|l|}{False} \\
\tabucline[1.5pt]{}
\rowfont \bfseries References & NodeClass & BrowseName & DataType & Type\-Definition & {Modeling\-Rule} \\
\multicolumn{6}{|l|}{Subtype of MTControlledVocabEventClassType (See section \ref{type:MTControlledVocabEventClassType})} \\
values & Variable & Enum\-Strings & Program\-Edit\-Data\-Type & Program\-Edit\-Data\-Type & Mandatory \\
\end{tabu}
\end{table} 


\FloatBarrier
\paragraph{Referenced Properties and Objects}

\begin{itemize}
\item \textbf{Allowable Values} for \texttt{ProgramEditDataType}
\FloatBarrier
\begin{table}[ht]
\centering 
  \caption{\texttt{ProgramEditDataType} Enumeration}
  \label{enum:ProgramEditDataType}
\tabulinesep=3pt
\begin{tabu} to 6in {|l|r|} \everyrow{\hline}
\hline
\rowfont\bfseries {Name} & {Index} \\
\tabucline[1.5pt]{}
\texttt{ACTIVE} & \texttt{0} \\
\texttt{NOT_READY} & \texttt{1} \\
\texttt{READY} & \texttt{2} \\
\end{tabu}
\end{table} 
\FloatBarrier
\end{itemize}
\FloatBarrier
\subsubsection{Defintion of \texttt{ RotaryModeClassType}}
  \label{type:RotaryModeClassType}

\FloatBarrier
\begin{table}[ht]
\centering 
  \caption{\texttt{RotaryModeClassType} Definition}
  \label{table:RotaryModeClassType}
\fontsize{9pt}{11pt}\selectfont
\tabulinesep=3pt
\begin{tabu} to 6in {|X[-1.35]|X[-0.7]|X[-1.75]|X[-1.5]|X[-1]|X[-0.7]|} \everyrow{\hline}
\hline
\rowfont\bfseries {Attribute} & \multicolumn{5}{|l|}{Value} \\
\tabucline[1.5pt]{}
BrowseName & \multicolumn{5}{|l|}{RotaryModeClassType} \\
IsAbstract & \multicolumn{5}{|l|}{False} \\
\tabucline[1.5pt]{}
\rowfont \bfseries References & NodeClass & BrowseName & DataType & Type\-Definition & {Modeling\-Rule} \\
\multicolumn{6}{|l|}{Subtype of MTControlledVocabEventClassType (See section \ref{type:MTControlledVocabEventClassType})} \\
values & Variable & Enum\-Strings & Rotary\-Mode\-Data\-Type & Rotary\-Mode\-Data\-Type & Mandatory \\
\end{tabu}
\end{table} 


\FloatBarrier
\paragraph{Referenced Properties and Objects}

\begin{itemize}
\item \textbf{Allowable Values} for \texttt{RotaryModeDataType}
\FloatBarrier
\begin{table}[ht]
\centering 
  \caption{\texttt{RotaryModeDataType} Enumeration}
  \label{enum:RotaryModeDataType}
\tabulinesep=3pt
\begin{tabu} to 6in {|l|r|} \everyrow{\hline}
\hline
\rowfont\bfseries {Name} & {Index} \\
\tabucline[1.5pt]{}
\texttt{CONTOUR} & \texttt{0} \\
\texttt{INDEX} & \texttt{1} \\
\texttt{SPINDLE} & \texttt{2} \\
\end{tabu}
\end{table} 
\FloatBarrier
\end{itemize}
\FloatBarrier
\subsubsection{Defintion of \texttt{ SpindleInterlockClassType}}
  \label{type:SpindleInterlockClassType}

\FloatBarrier
\begin{table}[ht]
\centering 
  \caption{\texttt{SpindleInterlockClassType} Definition}
  \label{table:SpindleInterlockClassType}
\fontsize{9pt}{11pt}\selectfont
\tabulinesep=3pt
\begin{tabu} to 6in {|X[-1.35]|X[-0.7]|X[-1.75]|X[-1.5]|X[-1]|X[-0.7]|} \everyrow{\hline}
\hline
\rowfont\bfseries {Attribute} & \multicolumn{5}{|l|}{Value} \\
\tabucline[1.5pt]{}
BrowseName & \multicolumn{5}{|l|}{SpindleInterlockClassType} \\
IsAbstract & \multicolumn{5}{|l|}{False} \\
\tabucline[1.5pt]{}
\rowfont \bfseries References & NodeClass & BrowseName & DataType & Type\-Definition & {Modeling\-Rule} \\
\multicolumn{6}{|l|}{Subtype of MTControlledVocabEventClassType (See section \ref{type:MTControlledVocabEventClassType})} \\
values & Variable & Enum\-Strings & Active\-State\-Data\-Type & Active\-State\-Data\-Type & Mandatory \\
\end{tabu}
\end{table} 


\FloatBarrier
\paragraph{Referenced Properties and Objects}

\begin{itemize}
\item \textbf{Allowable Values} for \texttt{ActiveStateDataType}
\FloatBarrier
\begin{table}[ht]
\centering 
  \caption{\texttt{ActiveStateDataType} Enumeration}
\tabulinesep=3pt
\begin{tabu} to 6in {|l|r|} \everyrow{\hline}
\hline
\rowfont\bfseries {Name} & {Index} \\
\tabucline[1.5pt]{}
\texttt{ACTIVE} & \texttt{0} \\
\texttt{INACTIVE} & \texttt{1} \\
\end{tabu}
\end{table} 
\FloatBarrier
\end{itemize}
\FloatBarrier
\subsection{Numeric Event Data Item Types} \label{model:NumericEventDataItemTypes}
\subsubsection{Defintion of \texttt{ MTNumericEventClassType}}
  \label{type:MTNumericEventClassType}

\FloatBarrier
\begin{table}[ht]
\centering 
  \caption{\texttt{MTNumericEventClassType} Definition}
  \label{table:MTNumericEventClassType}
\fontsize{9pt}{11pt}\selectfont
\tabulinesep=3pt
\begin{tabu} to 6in {|X[-1.35]|X[-0.7]|X[-1.75]|X[-1.5]|X[-1]|X[-0.7]|} \everyrow{\hline}
\hline
\rowfont\bfseries {Attribute} & \multicolumn{5}{|l|}{Value} \\
\tabucline[1.5pt]{}
BrowseName & \multicolumn{5}{|l|}{MTNumericEventClassType} \\
IsAbstract & \multicolumn{5}{|l|}{True} \\
\tabucline[1.5pt]{}
\rowfont \bfseries References & NodeClass & BrowseName & DataType & Type\-Definition & {Modeling\-Rule} \\
\multicolumn{6}{|l|}{Subtype of MTEventClassType (See Data Items Documentation)} \\
HasSubtype & ObjectType & \multicolumn{2}{l}{AxisFeedrateOverrideClassType} & \multicolumn{2}{|l|}{See section \ref{type:AxisFeedrateOverrideClassType}} \\
HasSubtype & ObjectType & \multicolumn{2}{l}{BlockCountClassType} & \multicolumn{2}{|l|}{See section \ref{type:BlockCountClassType}} \\
HasSubtype & ObjectType & \multicolumn{2}{l}{HardnessClassType} & \multicolumn{2}{|l|}{See section \ref{type:HardnessClassType}} \\
HasSubtype & ObjectType & \multicolumn{2}{l}{LineNumberClassType} & \multicolumn{2}{|l|}{See section \ref{type:LineNumberClassType}} \\
HasSubtype & ObjectType & \multicolumn{2}{l}{PartCountClassType} & \multicolumn{2}{|l|}{See section \ref{type:PartCountClassType}} \\
HasSubtype & ObjectType & \multicolumn{2}{l}{PathFeedrateOverrideClassType} & \multicolumn{2}{|l|}{See section \ref{type:PathFeedrateOverrideClassType}} \\
HasSubtype & ObjectType & \multicolumn{2}{l}{RotaryVelocityOverrideClassType} & \multicolumn{2}{|l|}{See section \ref{type:RotaryVelocityOverrideClassType}} \\
\end{tabu}
\end{table} 


\FloatBarrier
\subsubsection{Defintion of \texttt{ AxisFeedrateOverrideClassType}}
  \label{type:AxisFeedrateOverrideClassType}

\FloatBarrier
\begin{table}[ht]
\centering 
  \caption{\texttt{AxisFeedrateOverrideClassType} Definition}
  \label{table:AxisFeedrateOverrideClassType}
\fontsize{9pt}{11pt}\selectfont
\tabulinesep=3pt
\begin{tabu} to 6in {|X[-1.35]|X[-0.7]|X[-1.75]|X[-1.5]|X[-1]|X[-0.7]|} \everyrow{\hline}
\hline
\rowfont\bfseries {Attribute} & \multicolumn{5}{|l|}{Value} \\
\tabucline[1.5pt]{}
BrowseName & \multicolumn{5}{|l|}{AxisFeedrateOverrideClassType} \\
IsAbstract & \multicolumn{5}{|l|}{False} \\
\tabucline[1.5pt]{}
\rowfont \bfseries References & NodeClass & BrowseName & DataType & Type\-Definition & {Modeling\-Rule} \\
\multicolumn{6}{|l|}{Subtype of MTNumericEventClassType (See section \ref{type:MTNumericEventClassType})} \\
\end{tabu}
\end{table} 


\FloatBarrier
\subsubsection{Defintion of \texttt{ BlockCountClassType}}
  \label{type:BlockCountClassType}

\FloatBarrier
\begin{table}[ht]
\centering 
  \caption{\texttt{BlockCountClassType} Definition}
  \label{table:BlockCountClassType}
\fontsize{9pt}{11pt}\selectfont
\tabulinesep=3pt
\begin{tabu} to 6in {|X[-1.35]|X[-0.7]|X[-1.75]|X[-1.5]|X[-1]|X[-0.7]|} \everyrow{\hline}
\hline
\rowfont\bfseries {Attribute} & \multicolumn{5}{|l|}{Value} \\
\tabucline[1.5pt]{}
BrowseName & \multicolumn{5}{|l|}{BlockCountClassType} \\
IsAbstract & \multicolumn{5}{|l|}{False} \\
\tabucline[1.5pt]{}
\rowfont \bfseries References & NodeClass & BrowseName & DataType & Type\-Definition & {Modeling\-Rule} \\
\multicolumn{6}{|l|}{Subtype of MTNumericEventClassType (See section \ref{type:MTNumericEventClassType})} \\
\end{tabu}
\end{table} 


\FloatBarrier
\subsubsection{Defintion of \texttt{ HardnessClassType}}
  \label{type:HardnessClassType}

\FloatBarrier
\begin{table}[ht]
\centering 
  \caption{\texttt{HardnessClassType} Definition}
  \label{table:HardnessClassType}
\fontsize{9pt}{11pt}\selectfont
\tabulinesep=3pt
\begin{tabu} to 6in {|X[-1.35]|X[-0.7]|X[-1.75]|X[-1.5]|X[-1]|X[-0.7]|} \everyrow{\hline}
\hline
\rowfont\bfseries {Attribute} & \multicolumn{5}{|l|}{Value} \\
\tabucline[1.5pt]{}
BrowseName & \multicolumn{5}{|l|}{HardnessClassType} \\
IsAbstract & \multicolumn{5}{|l|}{False} \\
\tabucline[1.5pt]{}
\rowfont \bfseries References & NodeClass & BrowseName & DataType & Type\-Definition & {Modeling\-Rule} \\
\multicolumn{6}{|l|}{Subtype of MTNumericEventClassType (See section \ref{type:MTNumericEventClassType})} \\
\end{tabu}
\end{table} 


\FloatBarrier
\subsubsection{Defintion of \texttt{ LineNumberClassType}}
  \label{type:LineNumberClassType}

\FloatBarrier
\begin{table}[ht]
\centering 
  \caption{\texttt{LineNumberClassType} Definition}
  \label{table:LineNumberClassType}
\fontsize{9pt}{11pt}\selectfont
\tabulinesep=3pt
\begin{tabu} to 6in {|X[-1.35]|X[-0.7]|X[-1.75]|X[-1.5]|X[-1]|X[-0.7]|} \everyrow{\hline}
\hline
\rowfont\bfseries {Attribute} & \multicolumn{5}{|l|}{Value} \\
\tabucline[1.5pt]{}
BrowseName & \multicolumn{5}{|l|}{LineNumberClassType} \\
IsAbstract & \multicolumn{5}{|l|}{False} \\
\tabucline[1.5pt]{}
\rowfont \bfseries References & NodeClass & BrowseName & DataType & Type\-Definition & {Modeling\-Rule} \\
\multicolumn{6}{|l|}{Subtype of MTNumericEventClassType (See section \ref{type:MTNumericEventClassType})} \\
\end{tabu}
\end{table} 


\FloatBarrier
\subsubsection{Defintion of \texttt{ PartCountClassType}}
  \label{type:PartCountClassType}

\FloatBarrier
\begin{table}[ht]
\centering 
  \caption{\texttt{PartCountClassType} Definition}
  \label{table:PartCountClassType}
\fontsize{9pt}{11pt}\selectfont
\tabulinesep=3pt
\begin{tabu} to 6in {|X[-1.35]|X[-0.7]|X[-1.75]|X[-1.5]|X[-1]|X[-0.7]|} \everyrow{\hline}
\hline
\rowfont\bfseries {Attribute} & \multicolumn{5}{|l|}{Value} \\
\tabucline[1.5pt]{}
BrowseName & \multicolumn{5}{|l|}{PartCountClassType} \\
IsAbstract & \multicolumn{5}{|l|}{False} \\
\tabucline[1.5pt]{}
\rowfont \bfseries References & NodeClass & BrowseName & DataType & Type\-Definition & {Modeling\-Rule} \\
\multicolumn{6}{|l|}{Subtype of MTNumericEventClassType (See section \ref{type:MTNumericEventClassType})} \\
\end{tabu}
\end{table} 


\FloatBarrier
\subsubsection{Defintion of \texttt{ PathFeedrateOverrideClassType}}
  \label{type:PathFeedrateOverrideClassType}

\FloatBarrier
\begin{table}[ht]
\centering 
  \caption{\texttt{PathFeedrateOverrideClassType} Definition}
  \label{table:PathFeedrateOverrideClassType}
\fontsize{9pt}{11pt}\selectfont
\tabulinesep=3pt
\begin{tabu} to 6in {|X[-1.35]|X[-0.7]|X[-1.75]|X[-1.5]|X[-1]|X[-0.7]|} \everyrow{\hline}
\hline
\rowfont\bfseries {Attribute} & \multicolumn{5}{|l|}{Value} \\
\tabucline[1.5pt]{}
BrowseName & \multicolumn{5}{|l|}{PathFeedrateOverrideClassType} \\
IsAbstract & \multicolumn{5}{|l|}{False} \\
\tabucline[1.5pt]{}
\rowfont \bfseries References & NodeClass & BrowseName & DataType & Type\-Definition & {Modeling\-Rule} \\
\multicolumn{6}{|l|}{Subtype of MTNumericEventClassType (See section \ref{type:MTNumericEventClassType})} \\
\end{tabu}
\end{table} 


\FloatBarrier
\subsubsection{Defintion of \texttt{ RotaryVelocityOverrideClassType}}
  \label{type:RotaryVelocityOverrideClassType}

\FloatBarrier
\begin{table}[ht]
\centering 
  \caption{\texttt{RotaryVelocityOverrideClassType} Definition}
  \label{table:RotaryVelocityOverrideClassType}
\fontsize{9pt}{11pt}\selectfont
\tabulinesep=3pt
\begin{tabu} to 6in {|X[-1.35]|X[-0.7]|X[-1.75]|X[-1.5]|X[-1]|X[-0.7]|} \everyrow{\hline}
\hline
\rowfont\bfseries {Attribute} & \multicolumn{5}{|l|}{Value} \\
\tabucline[1.5pt]{}
BrowseName & \multicolumn{5}{|l|}{RotaryVelocityOverrideClassType} \\
IsAbstract & \multicolumn{5}{|l|}{False} \\
\tabucline[1.5pt]{}
\rowfont \bfseries References & NodeClass & BrowseName & DataType & Type\-Definition & {Modeling\-Rule} \\
\multicolumn{6}{|l|}{Subtype of MTNumericEventClassType (See section \ref{type:MTNumericEventClassType})} \\
\end{tabu}
\end{table} 


\FloatBarrier
\subsection{String Event Data Item Types} \label{model:StringEventDataItemTypes}
\subsubsection{Defintion of \texttt{ MTStringEventClassType}}
  \label{type:MTStringEventClassType}

\FloatBarrier
\begin{table}[ht]
\centering 
  \caption{\texttt{MTStringEventClassType} Definition}
  \label{table:MTStringEventClassType}
\fontsize{9pt}{11pt}\selectfont
\tabulinesep=3pt
\begin{tabu} to 6in {|X[-1.35]|X[-0.7]|X[-1.75]|X[-1.5]|X[-1]|X[-0.7]|} \everyrow{\hline}
\hline
\rowfont\bfseries {Attribute} & \multicolumn{5}{|l|}{Value} \\
\tabucline[1.5pt]{}
BrowseName & \multicolumn{5}{|l|}{MTStringEventClassType} \\
IsAbstract & \multicolumn{5}{|l|}{True} \\
\tabucline[1.5pt]{}
\rowfont \bfseries References & NodeClass & BrowseName & DataType & Type\-Definition & {Modeling\-Rule} \\
\multicolumn{6}{|l|}{Subtype of MTEventClassType (See Data Items Documentation)} \\
HasSubtype & ObjectType & \multicolumn{2}{l}{BlockClassType} & \multicolumn{2}{|l|}{See section \ref{type:BlockClassType}} \\
HasSubtype & ObjectType & \multicolumn{2}{l}{CoupledAxesClassType} & \multicolumn{2}{|l|}{See section \ref{type:CoupledAxesClassType}} \\
HasSubtype & ObjectType & \multicolumn{2}{l}{LineClassType} & \multicolumn{2}{|l|}{See section \ref{type:LineClassType}} \\
HasSubtype & ObjectType & \multicolumn{2}{l}{LineLabelClassType} & \multicolumn{2}{|l|}{See section \ref{type:LineLabelClassType}} \\
HasSubtype & ObjectType & \multicolumn{2}{l}{MaterialClassType} & \multicolumn{2}{|l|}{See section \ref{type:MaterialClassType}} \\
HasSubtype & ObjectType & \multicolumn{2}{l}{MessageClassType} & \multicolumn{2}{|l|}{See section \ref{type:MessageClassType}} \\
HasSubtype & ObjectType & \multicolumn{2}{l}{OperatorIdClassType} & \multicolumn{2}{|l|}{See section \ref{type:OperatorIdClassType}} \\
HasSubtype & ObjectType & \multicolumn{2}{l}{PalletIdClassType} & \multicolumn{2}{|l|}{See section \ref{type:PalletIdClassType}} \\
HasSubtype & ObjectType & \multicolumn{2}{l}{PartIdClassType} & \multicolumn{2}{|l|}{See section \ref{type:PartIdClassType}} \\
HasSubtype & ObjectType & \multicolumn{2}{l}{PartNumberClassType} & \multicolumn{2}{|l|}{See section \ref{type:PartNumberClassType}} \\
HasSubtype & ObjectType & \multicolumn{2}{l}{ProgramClassType} & \multicolumn{2}{|l|}{See section \ref{type:ProgramClassType}} \\
HasSubtype & ObjectType & \multicolumn{2}{l}{ProgramCommentClassType} & \multicolumn{2}{|l|}{See section \ref{type:ProgramCommentClassType}} \\
HasSubtype & ObjectType & \multicolumn{2}{l}{ProgramEditNameClassType} & \multicolumn{2}{|l|}{See section \ref{type:ProgramEditNameClassType}} \\
HasSubtype & ObjectType & \multicolumn{2}{l}{ProgramHeaderClassType} & \multicolumn{2}{|l|}{See section \ref{type:ProgramHeaderClassType}} \\
HasSubtype & ObjectType & \multicolumn{2}{l}{SerialNumberClassType} & \multicolumn{2}{|l|}{See section \ref{type:SerialNumberClassType}} \\
HasSubtype & ObjectType & \multicolumn{2}{l}{ToolAssetIdClassType} & \multicolumn{2}{|l|}{See section \ref{type:ToolAssetIdClassType}} \\
HasSubtype & ObjectType & \multicolumn{2}{l}{ToolNumberClassType} & \multicolumn{2}{|l|}{See section \ref{type:ToolNumberClassType}} \\
HasSubtype & ObjectType & \multicolumn{2}{l}{ToolOffsetClassType} & \multicolumn{2}{|l|}{See section \ref{type:ToolOffsetClassType}} \\
HasSubtype & ObjectType & \multicolumn{2}{l}{UserClassType} & \multicolumn{2}{|l|}{See section \ref{type:UserClassType}} \\
HasSubtype & ObjectType & \multicolumn{2}{l}{WireClassType} & \multicolumn{2}{|l|}{See section \ref{type:WireClassType}} \\
HasSubtype & ObjectType & \multicolumn{2}{l}{WorkholdingClassType} & \multicolumn{2}{|l|}{See section \ref{type:WorkholdingClassType}} \\
HasSubtype & ObjectType & \multicolumn{2}{l}{WorkOffsetClassType} & \multicolumn{2}{|l|}{See section \ref{type:WorkOffsetClassType}} \\
\multicolumn{6}{|l|}{Continued...} \\
\end{tabu}
\end{table}
\begin{table}[ht]
\fontsize{9pt}{11pt}\selectfont
\tabulinesep=3pt
\begin{tabu} to 6in {|X[-1.35]|X[-0.7]|X[-1.75]|X[-1.5]|X[-1]|X[-0.7]|} \everyrow{\hline}
\hline
\rowfont \bfseries References & NodeClass & BrowseName & DataType & Type\-Definition & {Modeling\-Rule} \\
HasSubtype & ObjectType & \multicolumn{2}{l}{AssetChangedClassType} & \multicolumn{2}{|l|}{See section \ref{type:AssetChangedClassType}} \\
HasSubtype & ObjectType & \multicolumn{2}{l}{AssetRemovedClassType} & \multicolumn{2}{|l|}{See section \ref{type:AssetRemovedClassType}} \\
\end{tabu}
\end{table} 


\FloatBarrier
\subsubsection{Defintion of \texttt{ AssetChangedClassType}}
  \label{type:AssetChangedClassType}

\FloatBarrier
\begin{table}[ht]
\centering 
  \caption{\texttt{AssetChangedClassType} Definition}
  \label{table:AssetChangedClassType}
\fontsize{9pt}{11pt}\selectfont
\tabulinesep=3pt
\begin{tabu} to 6in {|X[-1.35]|X[-0.7]|X[-1.75]|X[-1.5]|X[-1]|X[-0.7]|} \everyrow{\hline}
\hline
\rowfont\bfseries {Attribute} & \multicolumn{5}{|l|}{Value} \\
\tabucline[1.5pt]{}
BrowseName & \multicolumn{5}{|l|}{AssetChangedClassType} \\
IsAbstract & \multicolumn{5}{|l|}{False} \\
\tabucline[1.5pt]{}
\rowfont \bfseries References & NodeClass & BrowseName & DataType & Type\-Definition & {Modeling\-Rule} \\
\multicolumn{6}{|l|}{Subtype of MTStringEventClassType (See section \ref{type:MTStringEventClassType})} \\
\end{tabu}
\end{table} 


\FloatBarrier
\subsubsection{Defintion of \texttt{ AssetRemovedClassType}}
  \label{type:AssetRemovedClassType}

\FloatBarrier
\begin{table}[ht]
\centering 
  \caption{\texttt{AssetRemovedClassType} Definition}
  \label{table:AssetRemovedClassType}
\fontsize{9pt}{11pt}\selectfont
\tabulinesep=3pt
\begin{tabu} to 6in {|X[-1.35]|X[-0.7]|X[-1.75]|X[-1.5]|X[-1]|X[-0.7]|} \everyrow{\hline}
\hline
\rowfont\bfseries {Attribute} & \multicolumn{5}{|l|}{Value} \\
\tabucline[1.5pt]{}
BrowseName & \multicolumn{5}{|l|}{AssetRemovedClassType} \\
IsAbstract & \multicolumn{5}{|l|}{False} \\
\tabucline[1.5pt]{}
\rowfont \bfseries References & NodeClass & BrowseName & DataType & Type\-Definition & {Modeling\-Rule} \\
\multicolumn{6}{|l|}{Subtype of MTStringEventClassType (See section \ref{type:MTStringEventClassType})} \\
\end{tabu}
\end{table} 


\FloatBarrier
\subsubsection{Defintion of \texttt{ BlockClassType}}
  \label{type:BlockClassType}

\FloatBarrier
\begin{table}[ht]
\centering 
  \caption{\texttt{BlockClassType} Definition}
  \label{table:BlockClassType}
\fontsize{9pt}{11pt}\selectfont
\tabulinesep=3pt
\begin{tabu} to 6in {|X[-1.35]|X[-0.7]|X[-1.75]|X[-1.5]|X[-1]|X[-0.7]|} \everyrow{\hline}
\hline
\rowfont\bfseries {Attribute} & \multicolumn{5}{|l|}{Value} \\
\tabucline[1.5pt]{}
BrowseName & \multicolumn{5}{|l|}{BlockClassType} \\
IsAbstract & \multicolumn{5}{|l|}{False} \\
\tabucline[1.5pt]{}
\rowfont \bfseries References & NodeClass & BrowseName & DataType & Type\-Definition & {Modeling\-Rule} \\
\multicolumn{6}{|l|}{Subtype of MTStringEventClassType (See section \ref{type:MTStringEventClassType})} \\
\end{tabu}
\end{table} 


\FloatBarrier
\subsubsection{Defintion of \texttt{ CoupledAxesClassType}}
  \label{type:CoupledAxesClassType}

\FloatBarrier
\begin{table}[ht]
\centering 
  \caption{\texttt{CoupledAxesClassType} Definition}
  \label{table:CoupledAxesClassType}
\fontsize{9pt}{11pt}\selectfont
\tabulinesep=3pt
\begin{tabu} to 6in {|X[-1.35]|X[-0.7]|X[-1.75]|X[-1.5]|X[-1]|X[-0.7]|} \everyrow{\hline}
\hline
\rowfont\bfseries {Attribute} & \multicolumn{5}{|l|}{Value} \\
\tabucline[1.5pt]{}
BrowseName & \multicolumn{5}{|l|}{CoupledAxesClassType} \\
IsAbstract & \multicolumn{5}{|l|}{False} \\
\tabucline[1.5pt]{}
\rowfont \bfseries References & NodeClass & BrowseName & DataType & Type\-Definition & {Modeling\-Rule} \\
\multicolumn{6}{|l|}{Subtype of MTStringEventClassType (See section \ref{type:MTStringEventClassType})} \\
\end{tabu}
\end{table} 


\FloatBarrier
\subsubsection{Defintion of \texttt{ LineClassType}}
  \label{type:LineClassType}

\FloatBarrier
\begin{table}[ht]
\centering 
  \caption{\texttt{LineClassType} Definition}
  \label{table:LineClassType}
\fontsize{9pt}{11pt}\selectfont
\tabulinesep=3pt
\begin{tabu} to 6in {|X[-1.35]|X[-0.7]|X[-1.75]|X[-1.5]|X[-1]|X[-0.7]|} \everyrow{\hline}
\hline
\rowfont\bfseries {Attribute} & \multicolumn{5}{|l|}{Value} \\
\tabucline[1.5pt]{}
BrowseName & \multicolumn{5}{|l|}{LineClassType} \\
IsAbstract & \multicolumn{5}{|l|}{False} \\
\tabucline[1.5pt]{}
\rowfont \bfseries References & NodeClass & BrowseName & DataType & Type\-Definition & {Modeling\-Rule} \\
\multicolumn{6}{|l|}{Subtype of MTStringEventClassType (See section \ref{type:MTStringEventClassType})} \\
\end{tabu}
\end{table} 


\FloatBarrier
\subsubsection{Defintion of \texttt{ LineLabelClassType}}
  \label{type:LineLabelClassType}

\FloatBarrier
\begin{table}[ht]
\centering 
  \caption{\texttt{LineLabelClassType} Definition}
  \label{table:LineLabelClassType}
\fontsize{9pt}{11pt}\selectfont
\tabulinesep=3pt
\begin{tabu} to 6in {|X[-1.35]|X[-0.7]|X[-1.75]|X[-1.5]|X[-1]|X[-0.7]|} \everyrow{\hline}
\hline
\rowfont\bfseries {Attribute} & \multicolumn{5}{|l|}{Value} \\
\tabucline[1.5pt]{}
BrowseName & \multicolumn{5}{|l|}{LineLabelClassType} \\
IsAbstract & \multicolumn{5}{|l|}{False} \\
\tabucline[1.5pt]{}
\rowfont \bfseries References & NodeClass & BrowseName & DataType & Type\-Definition & {Modeling\-Rule} \\
\multicolumn{6}{|l|}{Subtype of MTStringEventClassType (See section \ref{type:MTStringEventClassType})} \\
\end{tabu}
\end{table} 


\FloatBarrier
\subsubsection{Defintion of \texttt{ MaterialClassType}}
  \label{type:MaterialClassType}

\FloatBarrier
\begin{table}[ht]
\centering 
  \caption{\texttt{MaterialClassType} Definition}
  \label{table:MaterialClassType}
\fontsize{9pt}{11pt}\selectfont
\tabulinesep=3pt
\begin{tabu} to 6in {|X[-1.35]|X[-0.7]|X[-1.75]|X[-1.5]|X[-1]|X[-0.7]|} \everyrow{\hline}
\hline
\rowfont\bfseries {Attribute} & \multicolumn{5}{|l|}{Value} \\
\tabucline[1.5pt]{}
BrowseName & \multicolumn{5}{|l|}{MaterialClassType} \\
IsAbstract & \multicolumn{5}{|l|}{False} \\
\tabucline[1.5pt]{}
\rowfont \bfseries References & NodeClass & BrowseName & DataType & Type\-Definition & {Modeling\-Rule} \\
\multicolumn{6}{|l|}{Subtype of MTStringEventClassType (See section \ref{type:MTStringEventClassType})} \\
\end{tabu}
\end{table} 


\FloatBarrier
\subsubsection{Defintion of \texttt{ MessageClassType}}
  \label{type:MessageClassType}

\FloatBarrier
\begin{table}[ht]
\centering 
  \caption{\texttt{MessageClassType} Definition}
  \label{table:MessageClassType}
\fontsize{9pt}{11pt}\selectfont
\tabulinesep=3pt
\begin{tabu} to 6in {|X[-1.35]|X[-0.7]|X[-1.75]|X[-1.5]|X[-1]|X[-0.7]|} \everyrow{\hline}
\hline
\rowfont\bfseries {Attribute} & \multicolumn{5}{|l|}{Value} \\
\tabucline[1.5pt]{}
BrowseName & \multicolumn{5}{|l|}{MessageClassType} \\
IsAbstract & \multicolumn{5}{|l|}{False} \\
\tabucline[1.5pt]{}
\rowfont \bfseries References & NodeClass & BrowseName & DataType & Type\-Definition & {Modeling\-Rule} \\
\multicolumn{6}{|l|}{Subtype of MTStringEventClassType (See section \ref{type:MTStringEventClassType})} \\
\end{tabu}
\end{table} 


\FloatBarrier
\subsubsection{Defintion of \texttt{ OperatorIdClassType}}
  \label{type:OperatorIdClassType}

\FloatBarrier
\begin{table}[ht]
\centering 
  \caption{\texttt{OperatorIdClassType} Definition}
  \label{table:OperatorIdClassType}
\fontsize{9pt}{11pt}\selectfont
\tabulinesep=3pt
\begin{tabu} to 6in {|X[-1.35]|X[-0.7]|X[-1.75]|X[-1.5]|X[-1]|X[-0.7]|} \everyrow{\hline}
\hline
\rowfont\bfseries {Attribute} & \multicolumn{5}{|l|}{Value} \\
\tabucline[1.5pt]{}
BrowseName & \multicolumn{5}{|l|}{OperatorIdClassType} \\
IsAbstract & \multicolumn{5}{|l|}{False} \\
\tabucline[1.5pt]{}
\rowfont \bfseries References & NodeClass & BrowseName & DataType & Type\-Definition & {Modeling\-Rule} \\
\multicolumn{6}{|l|}{Subtype of MTStringEventClassType (See section \ref{type:MTStringEventClassType})} \\
\end{tabu}
\end{table} 


\FloatBarrier
\subsubsection{Defintion of \texttt{ PalletIdClassType}}
  \label{type:PalletIdClassType}

\FloatBarrier
\begin{table}[ht]
\centering 
  \caption{\texttt{PalletIdClassType} Definition}
  \label{table:PalletIdClassType}
\fontsize{9pt}{11pt}\selectfont
\tabulinesep=3pt
\begin{tabu} to 6in {|X[-1.35]|X[-0.7]|X[-1.75]|X[-1.5]|X[-1]|X[-0.7]|} \everyrow{\hline}
\hline
\rowfont\bfseries {Attribute} & \multicolumn{5}{|l|}{Value} \\
\tabucline[1.5pt]{}
BrowseName & \multicolumn{5}{|l|}{PalletIdClassType} \\
IsAbstract & \multicolumn{5}{|l|}{False} \\
\tabucline[1.5pt]{}
\rowfont \bfseries References & NodeClass & BrowseName & DataType & Type\-Definition & {Modeling\-Rule} \\
\multicolumn{6}{|l|}{Subtype of MTStringEventClassType (See section \ref{type:MTStringEventClassType})} \\
\end{tabu}
\end{table} 


\FloatBarrier
\subsubsection{Defintion of \texttt{ PartIdClassType}}
  \label{type:PartIdClassType}

\FloatBarrier
\begin{table}[ht]
\centering 
  \caption{\texttt{PartIdClassType} Definition}
  \label{table:PartIdClassType}
\fontsize{9pt}{11pt}\selectfont
\tabulinesep=3pt
\begin{tabu} to 6in {|X[-1.35]|X[-0.7]|X[-1.75]|X[-1.5]|X[-1]|X[-0.7]|} \everyrow{\hline}
\hline
\rowfont\bfseries {Attribute} & \multicolumn{5}{|l|}{Value} \\
\tabucline[1.5pt]{}
BrowseName & \multicolumn{5}{|l|}{PartIdClassType} \\
IsAbstract & \multicolumn{5}{|l|}{False} \\
\tabucline[1.5pt]{}
\rowfont \bfseries References & NodeClass & BrowseName & DataType & Type\-Definition & {Modeling\-Rule} \\
\multicolumn{6}{|l|}{Subtype of MTStringEventClassType (See section \ref{type:MTStringEventClassType})} \\
\end{tabu}
\end{table} 


\FloatBarrier
\subsubsection{Defintion of \texttt{ PartNumberClassType}}
  \label{type:PartNumberClassType}

\FloatBarrier
\begin{table}[ht]
\centering 
  \caption{\texttt{PartNumberClassType} Definition}
  \label{table:PartNumberClassType}
\fontsize{9pt}{11pt}\selectfont
\tabulinesep=3pt
\begin{tabu} to 6in {|X[-1.35]|X[-0.7]|X[-1.75]|X[-1.5]|X[-1]|X[-0.7]|} \everyrow{\hline}
\hline
\rowfont\bfseries {Attribute} & \multicolumn{5}{|l|}{Value} \\
\tabucline[1.5pt]{}
BrowseName & \multicolumn{5}{|l|}{PartNumberClassType} \\
IsAbstract & \multicolumn{5}{|l|}{False} \\
\tabucline[1.5pt]{}
\rowfont \bfseries References & NodeClass & BrowseName & DataType & Type\-Definition & {Modeling\-Rule} \\
\multicolumn{6}{|l|}{Subtype of MTStringEventClassType (See section \ref{type:MTStringEventClassType})} \\
\end{tabu}
\end{table} 


\FloatBarrier
\subsubsection{Defintion of \texttt{ ProgramClassType}}
  \label{type:ProgramClassType}

\FloatBarrier
\begin{table}[ht]
\centering 
  \caption{\texttt{ProgramClassType} Definition}
  \label{table:ProgramClassType}
\fontsize{9pt}{11pt}\selectfont
\tabulinesep=3pt
\begin{tabu} to 6in {|X[-1.35]|X[-0.7]|X[-1.75]|X[-1.5]|X[-1]|X[-0.7]|} \everyrow{\hline}
\hline
\rowfont\bfseries {Attribute} & \multicolumn{5}{|l|}{Value} \\
\tabucline[1.5pt]{}
BrowseName & \multicolumn{5}{|l|}{ProgramClassType} \\
IsAbstract & \multicolumn{5}{|l|}{False} \\
\tabucline[1.5pt]{}
\rowfont \bfseries References & NodeClass & BrowseName & DataType & Type\-Definition & {Modeling\-Rule} \\
\multicolumn{6}{|l|}{Subtype of MTStringEventClassType (See section \ref{type:MTStringEventClassType})} \\
\end{tabu}
\end{table} 


\FloatBarrier
\subsubsection{Defintion of \texttt{ ProgramCommentClassType}}
  \label{type:ProgramCommentClassType}

\FloatBarrier
\begin{table}[ht]
\centering 
  \caption{\texttt{ProgramCommentClassType} Definition}
  \label{table:ProgramCommentClassType}
\fontsize{9pt}{11pt}\selectfont
\tabulinesep=3pt
\begin{tabu} to 6in {|X[-1.35]|X[-0.7]|X[-1.75]|X[-1.5]|X[-1]|X[-0.7]|} \everyrow{\hline}
\hline
\rowfont\bfseries {Attribute} & \multicolumn{5}{|l|}{Value} \\
\tabucline[1.5pt]{}
BrowseName & \multicolumn{5}{|l|}{ProgramCommentClassType} \\
IsAbstract & \multicolumn{5}{|l|}{False} \\
\tabucline[1.5pt]{}
\rowfont \bfseries References & NodeClass & BrowseName & DataType & Type\-Definition & {Modeling\-Rule} \\
\multicolumn{6}{|l|}{Subtype of MTStringEventClassType (See section \ref{type:MTStringEventClassType})} \\
\end{tabu}
\end{table} 


\FloatBarrier
\subsubsection{Defintion of \texttt{ ProgramEditNameClassType}}
  \label{type:ProgramEditNameClassType}

\FloatBarrier
\begin{table}[ht]
\centering 
  \caption{\texttt{ProgramEditNameClassType} Definition}
  \label{table:ProgramEditNameClassType}
\fontsize{9pt}{11pt}\selectfont
\tabulinesep=3pt
\begin{tabu} to 6in {|X[-1.35]|X[-0.7]|X[-1.75]|X[-1.5]|X[-1]|X[-0.7]|} \everyrow{\hline}
\hline
\rowfont\bfseries {Attribute} & \multicolumn{5}{|l|}{Value} \\
\tabucline[1.5pt]{}
BrowseName & \multicolumn{5}{|l|}{ProgramEditNameClassType} \\
IsAbstract & \multicolumn{5}{|l|}{False} \\
\tabucline[1.5pt]{}
\rowfont \bfseries References & NodeClass & BrowseName & DataType & Type\-Definition & {Modeling\-Rule} \\
\multicolumn{6}{|l|}{Subtype of MTStringEventClassType (See section \ref{type:MTStringEventClassType})} \\
\end{tabu}
\end{table} 


\FloatBarrier
\subsubsection{Defintion of \texttt{ ProgramHeaderClassType}}
  \label{type:ProgramHeaderClassType}

\FloatBarrier
\begin{table}[ht]
\centering 
  \caption{\texttt{ProgramHeaderClassType} Definition}
  \label{table:ProgramHeaderClassType}
\fontsize{9pt}{11pt}\selectfont
\tabulinesep=3pt
\begin{tabu} to 6in {|X[-1.35]|X[-0.7]|X[-1.75]|X[-1.5]|X[-1]|X[-0.7]|} \everyrow{\hline}
\hline
\rowfont\bfseries {Attribute} & \multicolumn{5}{|l|}{Value} \\
\tabucline[1.5pt]{}
BrowseName & \multicolumn{5}{|l|}{ProgramHeaderClassType} \\
IsAbstract & \multicolumn{5}{|l|}{False} \\
\tabucline[1.5pt]{}
\rowfont \bfseries References & NodeClass & BrowseName & DataType & Type\-Definition & {Modeling\-Rule} \\
\multicolumn{6}{|l|}{Subtype of MTStringEventClassType (See section \ref{type:MTStringEventClassType})} \\
\end{tabu}
\end{table} 


\FloatBarrier
\subsubsection{Defintion of \texttt{ SerialNumberClassType}}
  \label{type:SerialNumberClassType}

\FloatBarrier
\begin{table}[ht]
\centering 
  \caption{\texttt{SerialNumberClassType} Definition}
  \label{table:SerialNumberClassType}
\fontsize{9pt}{11pt}\selectfont
\tabulinesep=3pt
\begin{tabu} to 6in {|X[-1.35]|X[-0.7]|X[-1.75]|X[-1.5]|X[-1]|X[-0.7]|} \everyrow{\hline}
\hline
\rowfont\bfseries {Attribute} & \multicolumn{5}{|l|}{Value} \\
\tabucline[1.5pt]{}
BrowseName & \multicolumn{5}{|l|}{SerialNumberClassType} \\
IsAbstract & \multicolumn{5}{|l|}{False} \\
\tabucline[1.5pt]{}
\rowfont \bfseries References & NodeClass & BrowseName & DataType & Type\-Definition & {Modeling\-Rule} \\
\multicolumn{6}{|l|}{Subtype of MTStringEventClassType (See section \ref{type:MTStringEventClassType})} \\
\end{tabu}
\end{table} 


\FloatBarrier
\subsubsection{Defintion of \texttt{ ToolAssetIdClassType}}
  \label{type:ToolAssetIdClassType}

\FloatBarrier
\begin{table}[ht]
\centering 
  \caption{\texttt{ToolAssetIdClassType} Definition}
  \label{table:ToolAssetIdClassType}
\fontsize{9pt}{11pt}\selectfont
\tabulinesep=3pt
\begin{tabu} to 6in {|X[-1.35]|X[-0.7]|X[-1.75]|X[-1.5]|X[-1]|X[-0.7]|} \everyrow{\hline}
\hline
\rowfont\bfseries {Attribute} & \multicolumn{5}{|l|}{Value} \\
\tabucline[1.5pt]{}
BrowseName & \multicolumn{5}{|l|}{ToolAssetIdClassType} \\
IsAbstract & \multicolumn{5}{|l|}{False} \\
\tabucline[1.5pt]{}
\rowfont \bfseries References & NodeClass & BrowseName & DataType & Type\-Definition & {Modeling\-Rule} \\
\multicolumn{6}{|l|}{Subtype of MTStringEventClassType (See section \ref{type:MTStringEventClassType})} \\
\end{tabu}
\end{table} 


\FloatBarrier
\subsubsection{Defintion of \texttt{ ToolNumberClassType}}
  \label{type:ToolNumberClassType}

\FloatBarrier
\begin{table}[ht]
\centering 
  \caption{\texttt{ToolNumberClassType} Definition}
  \label{table:ToolNumberClassType}
\fontsize{9pt}{11pt}\selectfont
\tabulinesep=3pt
\begin{tabu} to 6in {|X[-1.35]|X[-0.7]|X[-1.75]|X[-1.5]|X[-1]|X[-0.7]|} \everyrow{\hline}
\hline
\rowfont\bfseries {Attribute} & \multicolumn{5}{|l|}{Value} \\
\tabucline[1.5pt]{}
BrowseName & \multicolumn{5}{|l|}{ToolNumberClassType} \\
IsAbstract & \multicolumn{5}{|l|}{False} \\
\tabucline[1.5pt]{}
\rowfont \bfseries References & NodeClass & BrowseName & DataType & Type\-Definition & {Modeling\-Rule} \\
\multicolumn{6}{|l|}{Subtype of MTStringEventClassType (See section \ref{type:MTStringEventClassType})} \\
\end{tabu}
\end{table} 


\FloatBarrier
\subsubsection{Defintion of \texttt{ ToolOffsetClassType}}
  \label{type:ToolOffsetClassType}

\FloatBarrier
\begin{table}[ht]
\centering 
  \caption{\texttt{ToolOffsetClassType} Definition}
  \label{table:ToolOffsetClassType}
\fontsize{9pt}{11pt}\selectfont
\tabulinesep=3pt
\begin{tabu} to 6in {|X[-1.35]|X[-0.7]|X[-1.75]|X[-1.5]|X[-1]|X[-0.7]|} \everyrow{\hline}
\hline
\rowfont\bfseries {Attribute} & \multicolumn{5}{|l|}{Value} \\
\tabucline[1.5pt]{}
BrowseName & \multicolumn{5}{|l|}{ToolOffsetClassType} \\
IsAbstract & \multicolumn{5}{|l|}{False} \\
\tabucline[1.5pt]{}
\rowfont \bfseries References & NodeClass & BrowseName & DataType & Type\-Definition & {Modeling\-Rule} \\
\multicolumn{6}{|l|}{Subtype of MTStringEventClassType (See section \ref{type:MTStringEventClassType})} \\
\end{tabu}
\end{table} 


\FloatBarrier
\subsubsection{Defintion of \texttt{ UserClassType}}
  \label{type:UserClassType}

\FloatBarrier
\begin{table}[ht]
\centering 
  \caption{\texttt{UserClassType} Definition}
  \label{table:UserClassType}
\fontsize{9pt}{11pt}\selectfont
\tabulinesep=3pt
\begin{tabu} to 6in {|X[-1.35]|X[-0.7]|X[-1.75]|X[-1.5]|X[-1]|X[-0.7]|} \everyrow{\hline}
\hline
\rowfont\bfseries {Attribute} & \multicolumn{5}{|l|}{Value} \\
\tabucline[1.5pt]{}
BrowseName & \multicolumn{5}{|l|}{UserClassType} \\
IsAbstract & \multicolumn{5}{|l|}{False} \\
\tabucline[1.5pt]{}
\rowfont \bfseries References & NodeClass & BrowseName & DataType & Type\-Definition & {Modeling\-Rule} \\
\multicolumn{6}{|l|}{Subtype of MTStringEventClassType (See section \ref{type:MTStringEventClassType})} \\
\end{tabu}
\end{table} 


\FloatBarrier
\subsubsection{Defintion of \texttt{ WireClassType}}
  \label{type:WireClassType}

\FloatBarrier
\begin{table}[ht]
\centering 
  \caption{\texttt{WireClassType} Definition}
  \label{table:WireClassType}
\fontsize{9pt}{11pt}\selectfont
\tabulinesep=3pt
\begin{tabu} to 6in {|X[-1.35]|X[-0.7]|X[-1.75]|X[-1.5]|X[-1]|X[-0.7]|} \everyrow{\hline}
\hline
\rowfont\bfseries {Attribute} & \multicolumn{5}{|l|}{Value} \\
\tabucline[1.5pt]{}
BrowseName & \multicolumn{5}{|l|}{WireClassType} \\
IsAbstract & \multicolumn{5}{|l|}{False} \\
\tabucline[1.5pt]{}
\rowfont \bfseries References & NodeClass & BrowseName & DataType & Type\-Definition & {Modeling\-Rule} \\
\multicolumn{6}{|l|}{Subtype of MTStringEventClassType (See section \ref{type:MTStringEventClassType})} \\
\end{tabu}
\end{table} 


\FloatBarrier
\subsubsection{Defintion of \texttt{ WorkholdingClassType}}
  \label{type:WorkholdingClassType}

\FloatBarrier
\begin{table}[ht]
\centering 
  \caption{\texttt{WorkholdingClassType} Definition}
  \label{table:WorkholdingClassType}
\fontsize{9pt}{11pt}\selectfont
\tabulinesep=3pt
\begin{tabu} to 6in {|X[-1.35]|X[-0.7]|X[-1.75]|X[-1.5]|X[-1]|X[-0.7]|} \everyrow{\hline}
\hline
\rowfont\bfseries {Attribute} & \multicolumn{5}{|l|}{Value} \\
\tabucline[1.5pt]{}
BrowseName & \multicolumn{5}{|l|}{WorkholdingClassType} \\
IsAbstract & \multicolumn{5}{|l|}{False} \\
\tabucline[1.5pt]{}
\rowfont \bfseries References & NodeClass & BrowseName & DataType & Type\-Definition & {Modeling\-Rule} \\
\multicolumn{6}{|l|}{Subtype of MTStringEventClassType (See section \ref{type:MTStringEventClassType})} \\
\end{tabu}
\end{table} 


\FloatBarrier
\subsubsection{Defintion of \texttt{ WorkOffsetClassType}}
  \label{type:WorkOffsetClassType}

\FloatBarrier
\begin{table}[ht]
\centering 
  \caption{\texttt{WorkOffsetClassType} Definition}
  \label{table:WorkOffsetClassType}
\fontsize{9pt}{11pt}\selectfont
\tabulinesep=3pt
\begin{tabu} to 6in {|X[-1.35]|X[-0.7]|X[-1.75]|X[-1.5]|X[-1]|X[-0.7]|} \everyrow{\hline}
\hline
\rowfont\bfseries {Attribute} & \multicolumn{5}{|l|}{Value} \\
\tabucline[1.5pt]{}
BrowseName & \multicolumn{5}{|l|}{WorkOffsetClassType} \\
IsAbstract & \multicolumn{5}{|l|}{False} \\
\tabucline[1.5pt]{}
\rowfont \bfseries References & NodeClass & BrowseName & DataType & Type\-Definition & {Modeling\-Rule} \\
\multicolumn{6}{|l|}{Subtype of MTStringEventClassType (See section \ref{type:MTStringEventClassType})} \\
\end{tabu}
\end{table} 


\FloatBarrier
\subsection{Condition Data Item Types} \label{model:ConditionDataItemTypes}
\subsubsection{Defintion of \texttt{ ActuatorClassType}}
  \label{type:ActuatorClassType}

\FloatBarrier
\begin{table}[ht]
\centering 
  \caption{\texttt{ActuatorClassType} Definition}
  \label{table:ActuatorClassType}
\fontsize{9pt}{11pt}\selectfont
\tabulinesep=3pt
\begin{tabu} to 6in {|X[-1.35]|X[-0.7]|X[-1.75]|X[-1.5]|X[-1]|X[-0.7]|} \everyrow{\hline}
\hline
\rowfont\bfseries {Attribute} & \multicolumn{5}{|l|}{Value} \\
\tabucline[1.5pt]{}
BrowseName & \multicolumn{5}{|l|}{ActuatorClassType} \\
IsAbstract & \multicolumn{5}{|l|}{False} \\
\tabucline[1.5pt]{}
\rowfont \bfseries References & NodeClass & BrowseName & DataType & Type\-Definition & {Modeling\-Rule} \\
\multicolumn{6}{|l|}{Subtype of MTConditionClassType (See Data Items Documentation)} \\
\end{tabu}
\end{table} 


\FloatBarrier
\subsubsection{Defintion of \texttt{ CommunicationsClassType}}
  \label{type:CommunicationsClassType}

\FloatBarrier
\begin{table}[ht]
\centering 
  \caption{\texttt{CommunicationsClassType} Definition}
  \label{table:CommunicationsClassType}
\fontsize{9pt}{11pt}\selectfont
\tabulinesep=3pt
\begin{tabu} to 6in {|X[-1.35]|X[-0.7]|X[-1.75]|X[-1.5]|X[-1]|X[-0.7]|} \everyrow{\hline}
\hline
\rowfont\bfseries {Attribute} & \multicolumn{5}{|l|}{Value} \\
\tabucline[1.5pt]{}
BrowseName & \multicolumn{5}{|l|}{CommunicationsClassType} \\
IsAbstract & \multicolumn{5}{|l|}{False} \\
\tabucline[1.5pt]{}
\rowfont \bfseries References & NodeClass & BrowseName & DataType & Type\-Definition & {Modeling\-Rule} \\
\multicolumn{6}{|l|}{Subtype of MTConditionClassType (See Data Items Documentation)} \\
\end{tabu}
\end{table} 


\FloatBarrier
\subsubsection{Defintion of \texttt{ DataRangeClassType}}
  \label{type:DataRangeClassType}

\FloatBarrier
\begin{table}[ht]
\centering 
  \caption{\texttt{DataRangeClassType} Definition}
  \label{table:DataRangeClassType}
\fontsize{9pt}{11pt}\selectfont
\tabulinesep=3pt
\begin{tabu} to 6in {|X[-1.35]|X[-0.7]|X[-1.75]|X[-1.5]|X[-1]|X[-0.7]|} \everyrow{\hline}
\hline
\rowfont\bfseries {Attribute} & \multicolumn{5}{|l|}{Value} \\
\tabucline[1.5pt]{}
BrowseName & \multicolumn{5}{|l|}{DataRangeClassType} \\
IsAbstract & \multicolumn{5}{|l|}{False} \\
\tabucline[1.5pt]{}
\rowfont \bfseries References & NodeClass & BrowseName & DataType & Type\-Definition & {Modeling\-Rule} \\
\multicolumn{6}{|l|}{Subtype of MTConditionClassType (See Data Items Documentation)} \\
\end{tabu}
\end{table} 


\FloatBarrier
\subsubsection{Defintion of \texttt{ HardwareClassType}}
  \label{type:HardwareClassType}

\FloatBarrier
\begin{table}[ht]
\centering 
  \caption{\texttt{HardwareClassType} Definition}
  \label{table:HardwareClassType}
\fontsize{9pt}{11pt}\selectfont
\tabulinesep=3pt
\begin{tabu} to 6in {|X[-1.35]|X[-0.7]|X[-1.75]|X[-1.5]|X[-1]|X[-0.7]|} \everyrow{\hline}
\hline
\rowfont\bfseries {Attribute} & \multicolumn{5}{|l|}{Value} \\
\tabucline[1.5pt]{}
BrowseName & \multicolumn{5}{|l|}{HardwareClassType} \\
IsAbstract & \multicolumn{5}{|l|}{False} \\
\tabucline[1.5pt]{}
\rowfont \bfseries References & NodeClass & BrowseName & DataType & Type\-Definition & {Modeling\-Rule} \\
\multicolumn{6}{|l|}{Subtype of MTConditionClassType (See Data Items Documentation)} \\
\end{tabu}
\end{table} 


\FloatBarrier
\subsubsection{Defintion of \texttt{ LogicProgramClassType}}
  \label{type:LogicProgramClassType}

\FloatBarrier
\begin{table}[ht]
\centering 
  \caption{\texttt{LogicProgramClassType} Definition}
  \label{table:LogicProgramClassType}
\fontsize{9pt}{11pt}\selectfont
\tabulinesep=3pt
\begin{tabu} to 6in {|X[-1.35]|X[-0.7]|X[-1.75]|X[-1.5]|X[-1]|X[-0.7]|} \everyrow{\hline}
\hline
\rowfont\bfseries {Attribute} & \multicolumn{5}{|l|}{Value} \\
\tabucline[1.5pt]{}
BrowseName & \multicolumn{5}{|l|}{LogicProgramClassType} \\
IsAbstract & \multicolumn{5}{|l|}{False} \\
\tabucline[1.5pt]{}
\rowfont \bfseries References & NodeClass & BrowseName & DataType & Type\-Definition & {Modeling\-Rule} \\
\multicolumn{6}{|l|}{Subtype of MTConditionClassType (See Data Items Documentation)} \\
\end{tabu}
\end{table} 


\FloatBarrier
\subsubsection{Defintion of \texttt{ MotionProgramClassType}}
  \label{type:MotionProgramClassType}

\FloatBarrier
\begin{table}[ht]
\centering 
  \caption{\texttt{MotionProgramClassType} Definition}
  \label{table:MotionProgramClassType}
\fontsize{9pt}{11pt}\selectfont
\tabulinesep=3pt
\begin{tabu} to 6in {|X[-1.35]|X[-0.7]|X[-1.75]|X[-1.5]|X[-1]|X[-0.7]|} \everyrow{\hline}
\hline
\rowfont\bfseries {Attribute} & \multicolumn{5}{|l|}{Value} \\
\tabucline[1.5pt]{}
BrowseName & \multicolumn{5}{|l|}{MotionProgramClassType} \\
IsAbstract & \multicolumn{5}{|l|}{False} \\
\tabucline[1.5pt]{}
\rowfont \bfseries References & NodeClass & BrowseName & DataType & Type\-Definition & {Modeling\-Rule} \\
\multicolumn{6}{|l|}{Subtype of MTConditionClassType (See Data Items Documentation)} \\
\end{tabu}
\end{table} 


\FloatBarrier
\subsubsection{Defintion of \texttt{ SystemClassType}}
  \label{type:SystemClassType}

\FloatBarrier
\begin{table}[ht]
\centering 
  \caption{\texttt{SystemClassType} Definition}
  \label{table:SystemClassType}
\fontsize{9pt}{11pt}\selectfont
\tabulinesep=3pt
\begin{tabu} to 6in {|X[-1.35]|X[-0.7]|X[-1.75]|X[-1.5]|X[-1]|X[-0.7]|} \everyrow{\hline}
\hline
\rowfont\bfseries {Attribute} & \multicolumn{5}{|l|}{Value} \\
\tabucline[1.5pt]{}
BrowseName & \multicolumn{5}{|l|}{SystemClassType} \\
IsAbstract & \multicolumn{5}{|l|}{False} \\
\tabucline[1.5pt]{}
\rowfont \bfseries References & NodeClass & BrowseName & DataType & Type\-Definition & {Modeling\-Rule} \\
\multicolumn{6}{|l|}{Subtype of MTConditionClassType (See Data Items Documentation)} \\
\end{tabu}
\end{table} 


\FloatBarrier
\subsection{Data Item Sub Types} \label{model:DataItemSubTypes}
\subsubsection{Defintion of \texttt{ MTDataItemSubClassType}}
  \label{type:MTDataItemSubClassType}

\FloatBarrier
\begin{table}[ht]
\centering 
  \caption{\texttt{MTDataItemSubClassType} Definition}
  \label{table:MTDataItemSubClassType}
\fontsize{9pt}{11pt}\selectfont
\tabulinesep=3pt
\begin{tabu} to 6in {|X[-1.35]|X[-0.7]|X[-1.75]|X[-1.5]|X[-1]|X[-0.7]|} \everyrow{\hline}
\hline
\rowfont\bfseries {Attribute} & \multicolumn{5}{|l|}{Value} \\
\tabucline[1.5pt]{}
BrowseName & \multicolumn{5}{|l|}{MTDataItemSubClassType} \\
IsAbstract & \multicolumn{5}{|l|}{True} \\
\tabucline[1.5pt]{}
\rowfont \bfseries References & NodeClass & BrowseName & DataType & Type\-Definition & {Modeling\-Rule} \\
\multicolumn{6}{|l|}{Subtype of BaseConditionClassType (See \cite{UAPart09} Documentation)} \\
HasSubtype & ObjectType & \multicolumn{2}{l}{RelativeSubClassType} & \multicolumn{2}{|l|}{See section \ref{type:RelativeSubClassType}} \\
HasSubtype & ObjectType & \multicolumn{2}{l}{RemainingSubClassType} & \multicolumn{2}{|l|}{See section \ref{type:RemainingSubClassType}} \\
HasSubtype & ObjectType & \multicolumn{2}{l}{RequestSubClassType} & \multicolumn{2}{|l|}{See section \ref{type:RequestSubClassType}} \\
HasSubtype & ObjectType & \multicolumn{2}{l}{ResponseSubClassType} & \multicolumn{2}{|l|}{See section \ref{type:ResponseSubClassType}} \\
HasSubtype & ObjectType & \multicolumn{2}{l}{RockwellSubClassType} & \multicolumn{2}{|l|}{See section \ref{type:RockwellSubClassType}} \\
HasSubtype & ObjectType & \multicolumn{2}{l}{RotarySubClassType} & \multicolumn{2}{|l|}{See section \ref{type:RotarySubClassType}} \\
HasSubtype & ObjectType & \multicolumn{2}{l}{SetUpSubClassType} & \multicolumn{2}{|l|}{See section \ref{type:SetUpSubClassType}} \\
HasSubtype & ObjectType & \multicolumn{2}{l}{ShoreSubClassType} & \multicolumn{2}{|l|}{See section \ref{type:ShoreSubClassType}} \\
HasSubtype & ObjectType & \multicolumn{2}{l}{StandardSubClassType} & \multicolumn{2}{|l|}{See section \ref{type:StandardSubClassType}} \\
HasSubtype & ObjectType & \multicolumn{2}{l}{SwitchedSubClassType} & \multicolumn{2}{|l|}{See section \ref{type:SwitchedSubClassType}} \\
HasSubtype & ObjectType & \multicolumn{2}{l}{TargetSubClassType} & \multicolumn{2}{|l|}{See section \ref{type:TargetSubClassType}} \\
HasSubtype & ObjectType & \multicolumn{2}{l}{ToolChangeStopSubClassType} & \multicolumn{2}{|l|}{See section \ref{type:ToolChangeStopSubClassType}} \\
HasSubtype & ObjectType & \multicolumn{2}{l}{ToolEdgeSubClassType} & \multicolumn{2}{|l|}{See section \ref{type:ToolEdgeSubClassType}} \\
HasSubtype & ObjectType & \multicolumn{2}{l}{ToolGroupSubClassType} & \multicolumn{2}{|l|}{See section \ref{type:ToolGroupSubClassType}} \\
HasSubtype & ObjectType & \multicolumn{2}{l}{ToolSubClassType} & \multicolumn{2}{|l|}{See section \ref{type:ToolSubClassType}} \\
HasSubtype & ObjectType & \multicolumn{2}{l}{UasbleSubClassType} & \multicolumn{2}{|l|}{See section \ref{type:UasbleSubClassType}} \\
HasSubtype & ObjectType & \multicolumn{2}{l}{VerticalSubClassType} & \multicolumn{2}{|l|}{See section \ref{type:VerticalSubClassType}} \\
HasSubtype & ObjectType & \multicolumn{2}{l}{VickersSubClassType} & \multicolumn{2}{|l|}{See section \ref{type:VickersSubClassType}} \\
HasSubtype & ObjectType & \multicolumn{2}{l}{VolumeSubClassType} & \multicolumn{2}{|l|}{See section \ref{type:VolumeSubClassType}} \\
HasSubtype & ObjectType & \multicolumn{2}{l}{WeightSubClassType} & \multicolumn{2}{|l|}{See section \ref{type:WeightSubClassType}} \\
HasSubtype & ObjectType & \multicolumn{2}{l}{WorkingSubClassType} & \multicolumn{2}{|l|}{See section \ref{type:WorkingSubClassType}} \\
HasSubtype & ObjectType & \multicolumn{2}{l}{WorkpieceSubClassType} & \multicolumn{2}{|l|}{See section \ref{type:WorkpieceSubClassType}} \\
\multicolumn{6}{|l|}{Continued...} \\
\end{tabu}
\end{table}
\begin{table}[ht]
\fontsize{9pt}{11pt}\selectfont
\tabulinesep=3pt
\begin{tabu} to 6in {|X[-1.35]|X[-0.7]|X[-1.75]|X[-1.5]|X[-1]|X[-0.7]|} \everyrow{\hline}
\hline
\rowfont \bfseries References & NodeClass & BrowseName & DataType & Type\-Definition & {Modeling\-Rule} \\
HasSubtype & ObjectType & \multicolumn{2}{l}{LineSubClassType} & \multicolumn{2}{|l|}{See section \ref{type:LineSubClassType}} \\
HasSubtype & ObjectType & \multicolumn{2}{l}{LoadedSubClassType} & \multicolumn{2}{|l|}{See section \ref{type:LoadedSubClassType}} \\
HasSubtype & ObjectType & \multicolumn{2}{l}{MachineAxisLockSubClassType} & \multicolumn{2}{|l|}{See section \ref{type:MachineAxisLockSubClassType}} \\
HasSubtype & ObjectType & \multicolumn{2}{l}{MaintenanceSubClassType} & \multicolumn{2}{|l|}{See section \ref{type:MaintenanceSubClassType}} \\
HasSubtype & ObjectType & \multicolumn{2}{l}{ManualUnclampSubClassType} & \multicolumn{2}{|l|}{See section \ref{type:ManualUnclampSubClassType}} \\
HasSubtype & ObjectType & \multicolumn{2}{l}{MaximumSubClassType} & \multicolumn{2}{|l|}{See section \ref{type:MaximumSubClassType}} \\
HasSubtype & ObjectType & \multicolumn{2}{l}{MinimumSubClassType} & \multicolumn{2}{|l|}{See section \ref{type:MinimumSubClassType}} \\
HasSubtype & ObjectType & \multicolumn{2}{l}{MohsSubClassType} & \multicolumn{2}{|l|}{See section \ref{type:MohsSubClassType}} \\
HasSubtype & ObjectType & \multicolumn{2}{l}{MoleSubClassType} & \multicolumn{2}{|l|}{See section \ref{type:MoleSubClassType}} \\
HasSubtype & ObjectType & \multicolumn{2}{l}{MotionSubClassType} & \multicolumn{2}{|l|}{See section \ref{type:MotionSubClassType}} \\
HasSubtype & ObjectType & \multicolumn{2}{l}{NoScaleSubClassType} & \multicolumn{2}{|l|}{See section \ref{type:NoScaleSubClassType}} \\
HasSubtype & ObjectType & \multicolumn{2}{l}{OperatingSubClassType} & \multicolumn{2}{|l|}{See section \ref{type:OperatingSubClassType}} \\
HasSubtype & ObjectType & \multicolumn{2}{l}{OperatorSubClassType} & \multicolumn{2}{|l|}{See section \ref{type:OperatorSubClassType}} \\
HasSubtype & ObjectType & \multicolumn{2}{l}{OptionalStopSubClassType} & \multicolumn{2}{|l|}{See section \ref{type:OptionalStopSubClassType}} \\
HasSubtype & ObjectType & \multicolumn{2}{l}{OverrideSubClassType} & \multicolumn{2}{|l|}{See section \ref{type:OverrideSubClassType}} \\
HasSubtype & ObjectType & \multicolumn{2}{l}{PoweredSubClassType} & \multicolumn{2}{|l|}{See section \ref{type:PoweredSubClassType}} \\
HasSubtype & ObjectType & \multicolumn{2}{l}{PrimarySubClassType} & \multicolumn{2}{|l|}{See section \ref{type:PrimarySubClassType}} \\
HasSubtype & ObjectType & \multicolumn{2}{l}{ProbeSubClassType} & \multicolumn{2}{|l|}{See section \ref{type:ProbeSubClassType}} \\
HasSubtype & ObjectType & \multicolumn{2}{l}{ProcessSubClassType} & \multicolumn{2}{|l|}{See section \ref{type:ProcessSubClassType}} \\
HasSubtype & ObjectType & \multicolumn{2}{l}{ProgrammedSubClassType} & \multicolumn{2}{|l|}{See section \ref{type:ProgrammedSubClassType}} \\
HasSubtype & ObjectType & \multicolumn{2}{l}{RadialSubClassType} & \multicolumn{2}{|l|}{See section \ref{type:RadialSubClassType}} \\
HasSubtype & ObjectType & \multicolumn{2}{l}{RapidSubClassType} & \multicolumn{2}{|l|}{See section \ref{type:RapidSubClassType}} \\
\multicolumn{6}{|l|}{Continued...} \\
\end{tabu}
\end{table}
\begin{table}[ht]
\fontsize{9pt}{11pt}\selectfont
\tabulinesep=3pt
\begin{tabu} to 6in {|X[-1.35]|X[-0.7]|X[-1.75]|X[-1.5]|X[-1]|X[-0.7]|} \everyrow{\hline}
\hline
\rowfont \bfseries References & NodeClass & BrowseName & DataType & Type\-Definition & {Modeling\-Rule} \\
HasSubtype & ObjectType & \multicolumn{2}{l}{AlternatingSubClassType} & \multicolumn{2}{|l|}{See section \ref{type:AlternatingSubClassType}} \\
HasSubtype & ObjectType & \multicolumn{2}{l}{AScaleSubClassType} & \multicolumn{2}{|l|}{See section \ref{type:AScaleSubClassType}} \\
HasSubtype & ObjectType & \multicolumn{2}{l}{AuxiliarySubClassType} & \multicolumn{2}{|l|}{See section \ref{type:AuxiliarySubClassType}} \\
HasSubtype & ObjectType & \multicolumn{2}{l}{BadSubClassType} & \multicolumn{2}{|l|}{See section \ref{type:BadSubClassType}} \\
HasSubtype & ObjectType & \multicolumn{2}{l}{BrinellSubClassType} & \multicolumn{2}{|l|}{See section \ref{type:BrinellSubClassType}} \\
HasSubtype & ObjectType & \multicolumn{2}{l}{BScaleSubClassType} & \multicolumn{2}{|l|}{See section \ref{type:BScaleSubClassType}} \\
HasSubtype & ObjectType & \multicolumn{2}{l}{CommandedSubClassType} & \multicolumn{2}{|l|}{See section \ref{type:CommandedSubClassType}} \\
HasSubtype & ObjectType & \multicolumn{2}{l}{ControlSubClassType} & \multicolumn{2}{|l|}{See section \ref{type:ControlSubClassType}} \\
HasSubtype & ObjectType & \multicolumn{2}{l}{CScaleSubClassType} & \multicolumn{2}{|l|}{See section \ref{type:CScaleSubClassType}} \\
HasSubtype & ObjectType & \multicolumn{2}{l}{DelaySubClassType} & \multicolumn{2}{|l|}{See section \ref{type:DelaySubClassType}} \\
HasSubtype & ObjectType & \multicolumn{2}{l}{DirectSubClassType} & \multicolumn{2}{|l|}{See section \ref{type:DirectSubClassType}} \\
HasSubtype & ObjectType & \multicolumn{2}{l}{DryRunSubClassType} & \multicolumn{2}{|l|}{See section \ref{type:DryRunSubClassType}} \\
HasSubtype & ObjectType & \multicolumn{2}{l}{DScaleSubClassType} & \multicolumn{2}{|l|}{See section \ref{type:DScaleSubClassType}} \\
HasSubtype & ObjectType & \multicolumn{2}{l}{FixtureSubClassType} & \multicolumn{2}{|l|}{See section \ref{type:FixtureSubClassType}} \\
HasSubtype & ObjectType & \multicolumn{2}{l}{GoodSubClassType} & \multicolumn{2}{|l|}{See section \ref{type:GoodSubClassType}} \\
HasSubtype & ObjectType & \multicolumn{2}{l}{IncrementalSubClassType} & \multicolumn{2}{|l|}{See section \ref{type:IncrementalSubClassType}} \\
HasSubtype & ObjectType & \multicolumn{2}{l}{JobSubClassType} & \multicolumn{2}{|l|}{See section \ref{type:JobSubClassType}} \\
HasSubtype & ObjectType & \multicolumn{2}{l}{KineticSubClassType} & \multicolumn{2}{|l|}{See section \ref{type:KineticSubClassType}} \\
HasSubtype & ObjectType & \multicolumn{2}{l}{LateralSubClassType} & \multicolumn{2}{|l|}{See section \ref{type:LateralSubClassType}} \\
HasSubtype & ObjectType & \multicolumn{2}{l}{LeebSubClassType} & \multicolumn{2}{|l|}{See section \ref{type:LeebSubClassType}} \\
HasSubtype & ObjectType & \multicolumn{2}{l}{LengthSubClassType} & \multicolumn{2}{|l|}{See section \ref{type:LengthSubClassType}} \\
HasSubtype & ObjectType & \multicolumn{2}{l}{LinearSubClassType} & \multicolumn{2}{|l|}{See section \ref{type:LinearSubClassType}} \\
\multicolumn{6}{|l|}{Continued...} \\
\end{tabu}
\end{table}
\begin{table}[ht]
\fontsize{9pt}{11pt}\selectfont
\tabulinesep=3pt
\begin{tabu} to 6in {|X[-1.35]|X[-0.7]|X[-1.75]|X[-1.5]|X[-1]|X[-0.7]|} \everyrow{\hline}
\hline
\rowfont \bfseries References & NodeClass & BrowseName & DataType & Type\-Definition & {Modeling\-Rule} \\
HasSubtype & ObjectType & \multicolumn{2}{l}{AbsoluteSubClassType} & \multicolumn{2}{|l|}{See section \ref{type:AbsoluteSubClassType}} \\
HasSubtype & ObjectType & \multicolumn{2}{l}{ActionSubClassType} & \multicolumn{2}{|l|}{See section \ref{type:ActionSubClassType}} \\
HasSubtype & ObjectType & \multicolumn{2}{l}{ActualSubClassType} & \multicolumn{2}{|l|}{See section \ref{type:ActualSubClassType}} \\
HasSubtype & ObjectType & \multicolumn{2}{l}{AllSubClassType} & \multicolumn{2}{|l|}{See section \ref{type:AllSubClassType}} \\
\end{tabu}
\end{table} 


\FloatBarrier
\subsubsection{Defintion of \texttt{ AbsoluteSubClassType}}
  \label{type:AbsoluteSubClassType}

\FloatBarrier
\begin{table}[ht]
\centering 
  \caption{\texttt{AbsoluteSubClassType} Definition}
  \label{table:AbsoluteSubClassType}
\fontsize{9pt}{11pt}\selectfont
\tabulinesep=3pt
\begin{tabu} to 6in {|X[-1.35]|X[-0.7]|X[-1.75]|X[-1.5]|X[-1]|X[-0.7]|} \everyrow{\hline}
\hline
\rowfont\bfseries {Attribute} & \multicolumn{5}{|l|}{Value} \\
\tabucline[1.5pt]{}
BrowseName & \multicolumn{5}{|l|}{AbsoluteSubClassType} \\
IsAbstract & \multicolumn{5}{|l|}{False} \\
\tabucline[1.5pt]{}
\rowfont \bfseries References & NodeClass & BrowseName & DataType & Type\-Definition & {Modeling\-Rule} \\
\multicolumn{6}{|l|}{Subtype of MTDataItemSubClassType (See section \ref{type:MTDataItemSubClassType})} \\
\end{tabu}
\end{table} 


\FloatBarrier
\subsubsection{Defintion of \texttt{ ActionSubClassType}}
  \label{type:ActionSubClassType}

\FloatBarrier
\begin{table}[ht]
\centering 
  \caption{\texttt{ActionSubClassType} Definition}
  \label{table:ActionSubClassType}
\fontsize{9pt}{11pt}\selectfont
\tabulinesep=3pt
\begin{tabu} to 6in {|X[-1.35]|X[-0.7]|X[-1.75]|X[-1.5]|X[-1]|X[-0.7]|} \everyrow{\hline}
\hline
\rowfont\bfseries {Attribute} & \multicolumn{5}{|l|}{Value} \\
\tabucline[1.5pt]{}
BrowseName & \multicolumn{5}{|l|}{ActionSubClassType} \\
IsAbstract & \multicolumn{5}{|l|}{False} \\
\tabucline[1.5pt]{}
\rowfont \bfseries References & NodeClass & BrowseName & DataType & Type\-Definition & {Modeling\-Rule} \\
\multicolumn{6}{|l|}{Subtype of MTDataItemSubClassType (See section \ref{type:MTDataItemSubClassType})} \\
\end{tabu}
\end{table} 


\FloatBarrier
\subsubsection{Defintion of \texttt{ ActualSubClassType}}
  \label{type:ActualSubClassType}

\FloatBarrier
\begin{table}[ht]
\centering 
  \caption{\texttt{ActualSubClassType} Definition}
  \label{table:ActualSubClassType}
\fontsize{9pt}{11pt}\selectfont
\tabulinesep=3pt
\begin{tabu} to 6in {|X[-1.35]|X[-0.7]|X[-1.75]|X[-1.5]|X[-1]|X[-0.7]|} \everyrow{\hline}
\hline
\rowfont\bfseries {Attribute} & \multicolumn{5}{|l|}{Value} \\
\tabucline[1.5pt]{}
BrowseName & \multicolumn{5}{|l|}{ActualSubClassType} \\
IsAbstract & \multicolumn{5}{|l|}{False} \\
\tabucline[1.5pt]{}
\rowfont \bfseries References & NodeClass & BrowseName & DataType & Type\-Definition & {Modeling\-Rule} \\
\multicolumn{6}{|l|}{Subtype of MTDataItemSubClassType (See section \ref{type:MTDataItemSubClassType})} \\
\end{tabu}
\end{table} 


\FloatBarrier
\subsubsection{Defintion of \texttt{ AllSubClassType}}
  \label{type:AllSubClassType}

\FloatBarrier
\begin{table}[ht]
\centering 
  \caption{\texttt{AllSubClassType} Definition}
  \label{table:AllSubClassType}
\fontsize{9pt}{11pt}\selectfont
\tabulinesep=3pt
\begin{tabu} to 6in {|X[-1.35]|X[-0.7]|X[-1.75]|X[-1.5]|X[-1]|X[-0.7]|} \everyrow{\hline}
\hline
\rowfont\bfseries {Attribute} & \multicolumn{5}{|l|}{Value} \\
\tabucline[1.5pt]{}
BrowseName & \multicolumn{5}{|l|}{AllSubClassType} \\
IsAbstract & \multicolumn{5}{|l|}{False} \\
\tabucline[1.5pt]{}
\rowfont \bfseries References & NodeClass & BrowseName & DataType & Type\-Definition & {Modeling\-Rule} \\
\multicolumn{6}{|l|}{Subtype of MTDataItemSubClassType (See section \ref{type:MTDataItemSubClassType})} \\
\end{tabu}
\end{table} 


\FloatBarrier
\subsubsection{Defintion of \texttt{ AlternatingSubClassType}}
  \label{type:AlternatingSubClassType}

\FloatBarrier
\begin{table}[ht]
\centering 
  \caption{\texttt{AlternatingSubClassType} Definition}
  \label{table:AlternatingSubClassType}
\fontsize{9pt}{11pt}\selectfont
\tabulinesep=3pt
\begin{tabu} to 6in {|X[-1.35]|X[-0.7]|X[-1.75]|X[-1.5]|X[-1]|X[-0.7]|} \everyrow{\hline}
\hline
\rowfont\bfseries {Attribute} & \multicolumn{5}{|l|}{Value} \\
\tabucline[1.5pt]{}
BrowseName & \multicolumn{5}{|l|}{AlternatingSubClassType} \\
IsAbstract & \multicolumn{5}{|l|}{False} \\
\tabucline[1.5pt]{}
\rowfont \bfseries References & NodeClass & BrowseName & DataType & Type\-Definition & {Modeling\-Rule} \\
\multicolumn{6}{|l|}{Subtype of MTDataItemSubClassType (See section \ref{type:MTDataItemSubClassType})} \\
\end{tabu}
\end{table} 


\FloatBarrier
\subsubsection{Defintion of \texttt{ AScaleSubClassType}}
  \label{type:AScaleSubClassType}

\FloatBarrier
\begin{table}[ht]
\centering 
  \caption{\texttt{AScaleSubClassType} Definition}
  \label{table:AScaleSubClassType}
\fontsize{9pt}{11pt}\selectfont
\tabulinesep=3pt
\begin{tabu} to 6in {|X[-1.35]|X[-0.7]|X[-1.75]|X[-1.5]|X[-1]|X[-0.7]|} \everyrow{\hline}
\hline
\rowfont\bfseries {Attribute} & \multicolumn{5}{|l|}{Value} \\
\tabucline[1.5pt]{}
BrowseName & \multicolumn{5}{|l|}{AScaleSubClassType} \\
IsAbstract & \multicolumn{5}{|l|}{False} \\
\tabucline[1.5pt]{}
\rowfont \bfseries References & NodeClass & BrowseName & DataType & Type\-Definition & {Modeling\-Rule} \\
\multicolumn{6}{|l|}{Subtype of MTDataItemSubClassType (See section \ref{type:MTDataItemSubClassType})} \\
\end{tabu}
\end{table} 


\FloatBarrier
\subsubsection{Defintion of \texttt{ AuxiliarySubClassType}}
  \label{type:AuxiliarySubClassType}

\FloatBarrier
\begin{table}[ht]
\centering 
  \caption{\texttt{AuxiliarySubClassType} Definition}
  \label{table:AuxiliarySubClassType}
\fontsize{9pt}{11pt}\selectfont
\tabulinesep=3pt
\begin{tabu} to 6in {|X[-1.35]|X[-0.7]|X[-1.75]|X[-1.5]|X[-1]|X[-0.7]|} \everyrow{\hline}
\hline
\rowfont\bfseries {Attribute} & \multicolumn{5}{|l|}{Value} \\
\tabucline[1.5pt]{}
BrowseName & \multicolumn{5}{|l|}{AuxiliarySubClassType} \\
IsAbstract & \multicolumn{5}{|l|}{False} \\
\tabucline[1.5pt]{}
\rowfont \bfseries References & NodeClass & BrowseName & DataType & Type\-Definition & {Modeling\-Rule} \\
\multicolumn{6}{|l|}{Subtype of MTDataItemSubClassType (See section \ref{type:MTDataItemSubClassType})} \\
\end{tabu}
\end{table} 


\FloatBarrier
\subsubsection{Defintion of \texttt{ BadSubClassType}}
  \label{type:BadSubClassType}

\FloatBarrier
\begin{table}[ht]
\centering 
  \caption{\texttt{BadSubClassType} Definition}
  \label{table:BadSubClassType}
\fontsize{9pt}{11pt}\selectfont
\tabulinesep=3pt
\begin{tabu} to 6in {|X[-1.35]|X[-0.7]|X[-1.75]|X[-1.5]|X[-1]|X[-0.7]|} \everyrow{\hline}
\hline
\rowfont\bfseries {Attribute} & \multicolumn{5}{|l|}{Value} \\
\tabucline[1.5pt]{}
BrowseName & \multicolumn{5}{|l|}{BadSubClassType} \\
IsAbstract & \multicolumn{5}{|l|}{False} \\
\tabucline[1.5pt]{}
\rowfont \bfseries References & NodeClass & BrowseName & DataType & Type\-Definition & {Modeling\-Rule} \\
\multicolumn{6}{|l|}{Subtype of MTDataItemSubClassType (See section \ref{type:MTDataItemSubClassType})} \\
\end{tabu}
\end{table} 


\FloatBarrier
\subsubsection{Defintion of \texttt{ BrinellSubClassType}}
  \label{type:BrinellSubClassType}

\FloatBarrier
\begin{table}[ht]
\centering 
  \caption{\texttt{BrinellSubClassType} Definition}
  \label{table:BrinellSubClassType}
\fontsize{9pt}{11pt}\selectfont
\tabulinesep=3pt
\begin{tabu} to 6in {|X[-1.35]|X[-0.7]|X[-1.75]|X[-1.5]|X[-1]|X[-0.7]|} \everyrow{\hline}
\hline
\rowfont\bfseries {Attribute} & \multicolumn{5}{|l|}{Value} \\
\tabucline[1.5pt]{}
BrowseName & \multicolumn{5}{|l|}{BrinellSubClassType} \\
IsAbstract & \multicolumn{5}{|l|}{False} \\
\tabucline[1.5pt]{}
\rowfont \bfseries References & NodeClass & BrowseName & DataType & Type\-Definition & {Modeling\-Rule} \\
\multicolumn{6}{|l|}{Subtype of MTDataItemSubClassType (See section \ref{type:MTDataItemSubClassType})} \\
\end{tabu}
\end{table} 


\FloatBarrier
\subsubsection{Defintion of \texttt{ BScaleSubClassType}}
  \label{type:BScaleSubClassType}

\FloatBarrier
\begin{table}[ht]
\centering 
  \caption{\texttt{BScaleSubClassType} Definition}
  \label{table:BScaleSubClassType}
\fontsize{9pt}{11pt}\selectfont
\tabulinesep=3pt
\begin{tabu} to 6in {|X[-1.35]|X[-0.7]|X[-1.75]|X[-1.5]|X[-1]|X[-0.7]|} \everyrow{\hline}
\hline
\rowfont\bfseries {Attribute} & \multicolumn{5}{|l|}{Value} \\
\tabucline[1.5pt]{}
BrowseName & \multicolumn{5}{|l|}{BScaleSubClassType} \\
IsAbstract & \multicolumn{5}{|l|}{False} \\
\tabucline[1.5pt]{}
\rowfont \bfseries References & NodeClass & BrowseName & DataType & Type\-Definition & {Modeling\-Rule} \\
\multicolumn{6}{|l|}{Subtype of MTDataItemSubClassType (See section \ref{type:MTDataItemSubClassType})} \\
\end{tabu}
\end{table} 


\FloatBarrier
\subsubsection{Defintion of \texttt{ CommandedSubClassType}}
  \label{type:CommandedSubClassType}

\FloatBarrier
\begin{table}[ht]
\centering 
  \caption{\texttt{CommandedSubClassType} Definition}
  \label{table:CommandedSubClassType}
\fontsize{9pt}{11pt}\selectfont
\tabulinesep=3pt
\begin{tabu} to 6in {|X[-1.35]|X[-0.7]|X[-1.75]|X[-1.5]|X[-1]|X[-0.7]|} \everyrow{\hline}
\hline
\rowfont\bfseries {Attribute} & \multicolumn{5}{|l|}{Value} \\
\tabucline[1.5pt]{}
BrowseName & \multicolumn{5}{|l|}{CommandedSubClassType} \\
IsAbstract & \multicolumn{5}{|l|}{False} \\
\tabucline[1.5pt]{}
\rowfont \bfseries References & NodeClass & BrowseName & DataType & Type\-Definition & {Modeling\-Rule} \\
\multicolumn{6}{|l|}{Subtype of MTDataItemSubClassType (See section \ref{type:MTDataItemSubClassType})} \\
\end{tabu}
\end{table} 


\FloatBarrier
\subsubsection{Defintion of \texttt{ ControlSubClassType}}
  \label{type:ControlSubClassType}

\FloatBarrier
\begin{table}[ht]
\centering 
  \caption{\texttt{ControlSubClassType} Definition}
  \label{table:ControlSubClassType}
\fontsize{9pt}{11pt}\selectfont
\tabulinesep=3pt
\begin{tabu} to 6in {|X[-1.35]|X[-0.7]|X[-1.75]|X[-1.5]|X[-1]|X[-0.7]|} \everyrow{\hline}
\hline
\rowfont\bfseries {Attribute} & \multicolumn{5}{|l|}{Value} \\
\tabucline[1.5pt]{}
BrowseName & \multicolumn{5}{|l|}{ControlSubClassType} \\
IsAbstract & \multicolumn{5}{|l|}{False} \\
\tabucline[1.5pt]{}
\rowfont \bfseries References & NodeClass & BrowseName & DataType & Type\-Definition & {Modeling\-Rule} \\
\multicolumn{6}{|l|}{Subtype of MTDataItemSubClassType (See section \ref{type:MTDataItemSubClassType})} \\
\end{tabu}
\end{table} 


\FloatBarrier
\subsubsection{Defintion of \texttt{ CScaleSubClassType}}
  \label{type:CScaleSubClassType}

\FloatBarrier
\begin{table}[ht]
\centering 
  \caption{\texttt{CScaleSubClassType} Definition}
  \label{table:CScaleSubClassType}
\fontsize{9pt}{11pt}\selectfont
\tabulinesep=3pt
\begin{tabu} to 6in {|X[-1.35]|X[-0.7]|X[-1.75]|X[-1.5]|X[-1]|X[-0.7]|} \everyrow{\hline}
\hline
\rowfont\bfseries {Attribute} & \multicolumn{5}{|l|}{Value} \\
\tabucline[1.5pt]{}
BrowseName & \multicolumn{5}{|l|}{CScaleSubClassType} \\
IsAbstract & \multicolumn{5}{|l|}{False} \\
\tabucline[1.5pt]{}
\rowfont \bfseries References & NodeClass & BrowseName & DataType & Type\-Definition & {Modeling\-Rule} \\
\multicolumn{6}{|l|}{Subtype of MTDataItemSubClassType (See section \ref{type:MTDataItemSubClassType})} \\
\end{tabu}
\end{table} 


\FloatBarrier
\subsubsection{Defintion of \texttt{ DelaySubClassType}}
  \label{type:DelaySubClassType}

\FloatBarrier
\begin{table}[ht]
\centering 
  \caption{\texttt{DelaySubClassType} Definition}
  \label{table:DelaySubClassType}
\fontsize{9pt}{11pt}\selectfont
\tabulinesep=3pt
\begin{tabu} to 6in {|X[-1.35]|X[-0.7]|X[-1.75]|X[-1.5]|X[-1]|X[-0.7]|} \everyrow{\hline}
\hline
\rowfont\bfseries {Attribute} & \multicolumn{5}{|l|}{Value} \\
\tabucline[1.5pt]{}
BrowseName & \multicolumn{5}{|l|}{DelaySubClassType} \\
IsAbstract & \multicolumn{5}{|l|}{False} \\
\tabucline[1.5pt]{}
\rowfont \bfseries References & NodeClass & BrowseName & DataType & Type\-Definition & {Modeling\-Rule} \\
\multicolumn{6}{|l|}{Subtype of MTDataItemSubClassType (See section \ref{type:MTDataItemSubClassType})} \\
\end{tabu}
\end{table} 


\FloatBarrier
\subsubsection{Defintion of \texttt{ DirectSubClassType}}
  \label{type:DirectSubClassType}

\FloatBarrier
\begin{table}[ht]
\centering 
  \caption{\texttt{DirectSubClassType} Definition}
  \label{table:DirectSubClassType}
\fontsize{9pt}{11pt}\selectfont
\tabulinesep=3pt
\begin{tabu} to 6in {|X[-1.35]|X[-0.7]|X[-1.75]|X[-1.5]|X[-1]|X[-0.7]|} \everyrow{\hline}
\hline
\rowfont\bfseries {Attribute} & \multicolumn{5}{|l|}{Value} \\
\tabucline[1.5pt]{}
BrowseName & \multicolumn{5}{|l|}{DirectSubClassType} \\
IsAbstract & \multicolumn{5}{|l|}{False} \\
\tabucline[1.5pt]{}
\rowfont \bfseries References & NodeClass & BrowseName & DataType & Type\-Definition & {Modeling\-Rule} \\
\multicolumn{6}{|l|}{Subtype of MTDataItemSubClassType (See section \ref{type:MTDataItemSubClassType})} \\
\end{tabu}
\end{table} 


\FloatBarrier
\subsubsection{Defintion of \texttt{ DryRunSubClassType}}
  \label{type:DryRunSubClassType}

\FloatBarrier
\begin{table}[ht]
\centering 
  \caption{\texttt{DryRunSubClassType} Definition}
  \label{table:DryRunSubClassType}
\fontsize{9pt}{11pt}\selectfont
\tabulinesep=3pt
\begin{tabu} to 6in {|X[-1.35]|X[-0.7]|X[-1.75]|X[-1.5]|X[-1]|X[-0.7]|} \everyrow{\hline}
\hline
\rowfont\bfseries {Attribute} & \multicolumn{5}{|l|}{Value} \\
\tabucline[1.5pt]{}
BrowseName & \multicolumn{5}{|l|}{DryRunSubClassType} \\
IsAbstract & \multicolumn{5}{|l|}{False} \\
\tabucline[1.5pt]{}
\rowfont \bfseries References & NodeClass & BrowseName & DataType & Type\-Definition & {Modeling\-Rule} \\
\multicolumn{6}{|l|}{Subtype of MTDataItemSubClassType (See section \ref{type:MTDataItemSubClassType})} \\
\end{tabu}
\end{table} 


\FloatBarrier
\subsubsection{Defintion of \texttt{ DScaleSubClassType}}
  \label{type:DScaleSubClassType}

\FloatBarrier
\begin{table}[ht]
\centering 
  \caption{\texttt{DScaleSubClassType} Definition}
  \label{table:DScaleSubClassType}
\fontsize{9pt}{11pt}\selectfont
\tabulinesep=3pt
\begin{tabu} to 6in {|X[-1.35]|X[-0.7]|X[-1.75]|X[-1.5]|X[-1]|X[-0.7]|} \everyrow{\hline}
\hline
\rowfont\bfseries {Attribute} & \multicolumn{5}{|l|}{Value} \\
\tabucline[1.5pt]{}
BrowseName & \multicolumn{5}{|l|}{DScaleSubClassType} \\
IsAbstract & \multicolumn{5}{|l|}{False} \\
\tabucline[1.5pt]{}
\rowfont \bfseries References & NodeClass & BrowseName & DataType & Type\-Definition & {Modeling\-Rule} \\
\multicolumn{6}{|l|}{Subtype of MTDataItemSubClassType (See section \ref{type:MTDataItemSubClassType})} \\
\end{tabu}
\end{table} 


\FloatBarrier
\subsubsection{Defintion of \texttt{ FixtureSubClassType}}
  \label{type:FixtureSubClassType}

\FloatBarrier
\begin{table}[ht]
\centering 
  \caption{\texttt{FixtureSubClassType} Definition}
  \label{table:FixtureSubClassType}
\fontsize{9pt}{11pt}\selectfont
\tabulinesep=3pt
\begin{tabu} to 6in {|X[-1.35]|X[-0.7]|X[-1.75]|X[-1.5]|X[-1]|X[-0.7]|} \everyrow{\hline}
\hline
\rowfont\bfseries {Attribute} & \multicolumn{5}{|l|}{Value} \\
\tabucline[1.5pt]{}
BrowseName & \multicolumn{5}{|l|}{FixtureSubClassType} \\
IsAbstract & \multicolumn{5}{|l|}{False} \\
\tabucline[1.5pt]{}
\rowfont \bfseries References & NodeClass & BrowseName & DataType & Type\-Definition & {Modeling\-Rule} \\
\multicolumn{6}{|l|}{Subtype of MTDataItemSubClassType (See section \ref{type:MTDataItemSubClassType})} \\
\end{tabu}
\end{table} 


\FloatBarrier
\subsubsection{Defintion of \texttt{ GoodSubClassType}}
  \label{type:GoodSubClassType}

\FloatBarrier
\begin{table}[ht]
\centering 
  \caption{\texttt{GoodSubClassType} Definition}
  \label{table:GoodSubClassType}
\fontsize{9pt}{11pt}\selectfont
\tabulinesep=3pt
\begin{tabu} to 6in {|X[-1.35]|X[-0.7]|X[-1.75]|X[-1.5]|X[-1]|X[-0.7]|} \everyrow{\hline}
\hline
\rowfont\bfseries {Attribute} & \multicolumn{5}{|l|}{Value} \\
\tabucline[1.5pt]{}
BrowseName & \multicolumn{5}{|l|}{GoodSubClassType} \\
IsAbstract & \multicolumn{5}{|l|}{False} \\
\tabucline[1.5pt]{}
\rowfont \bfseries References & NodeClass & BrowseName & DataType & Type\-Definition & {Modeling\-Rule} \\
\multicolumn{6}{|l|}{Subtype of MTDataItemSubClassType (See section \ref{type:MTDataItemSubClassType})} \\
\end{tabu}
\end{table} 


\FloatBarrier
\subsubsection{Defintion of \texttt{ IncrementalSubClassType}}
  \label{type:IncrementalSubClassType}

\FloatBarrier
\begin{table}[ht]
\centering 
  \caption{\texttt{IncrementalSubClassType} Definition}
  \label{table:IncrementalSubClassType}
\fontsize{9pt}{11pt}\selectfont
\tabulinesep=3pt
\begin{tabu} to 6in {|X[-1.35]|X[-0.7]|X[-1.75]|X[-1.5]|X[-1]|X[-0.7]|} \everyrow{\hline}
\hline
\rowfont\bfseries {Attribute} & \multicolumn{5}{|l|}{Value} \\
\tabucline[1.5pt]{}
BrowseName & \multicolumn{5}{|l|}{IncrementalSubClassType} \\
IsAbstract & \multicolumn{5}{|l|}{False} \\
\tabucline[1.5pt]{}
\rowfont \bfseries References & NodeClass & BrowseName & DataType & Type\-Definition & {Modeling\-Rule} \\
\multicolumn{6}{|l|}{Subtype of MTDataItemSubClassType (See section \ref{type:MTDataItemSubClassType})} \\
\end{tabu}
\end{table} 


\FloatBarrier
\subsubsection{Defintion of \texttt{ JobSubClassType}}
  \label{type:JobSubClassType}

\FloatBarrier
\begin{table}[ht]
\centering 
  \caption{\texttt{JobSubClassType} Definition}
  \label{table:JobSubClassType}
\fontsize{9pt}{11pt}\selectfont
\tabulinesep=3pt
\begin{tabu} to 6in {|X[-1.35]|X[-0.7]|X[-1.75]|X[-1.5]|X[-1]|X[-0.7]|} \everyrow{\hline}
\hline
\rowfont\bfseries {Attribute} & \multicolumn{5}{|l|}{Value} \\
\tabucline[1.5pt]{}
BrowseName & \multicolumn{5}{|l|}{JobSubClassType} \\
IsAbstract & \multicolumn{5}{|l|}{False} \\
\tabucline[1.5pt]{}
\rowfont \bfseries References & NodeClass & BrowseName & DataType & Type\-Definition & {Modeling\-Rule} \\
\multicolumn{6}{|l|}{Subtype of MTDataItemSubClassType (See section \ref{type:MTDataItemSubClassType})} \\
\end{tabu}
\end{table} 


\FloatBarrier
\subsubsection{Defintion of \texttt{ KineticSubClassType}}
  \label{type:KineticSubClassType}

\FloatBarrier
\begin{table}[ht]
\centering 
  \caption{\texttt{KineticSubClassType} Definition}
  \label{table:KineticSubClassType}
\fontsize{9pt}{11pt}\selectfont
\tabulinesep=3pt
\begin{tabu} to 6in {|X[-1.35]|X[-0.7]|X[-1.75]|X[-1.5]|X[-1]|X[-0.7]|} \everyrow{\hline}
\hline
\rowfont\bfseries {Attribute} & \multicolumn{5}{|l|}{Value} \\
\tabucline[1.5pt]{}
BrowseName & \multicolumn{5}{|l|}{KineticSubClassType} \\
IsAbstract & \multicolumn{5}{|l|}{False} \\
\tabucline[1.5pt]{}
\rowfont \bfseries References & NodeClass & BrowseName & DataType & Type\-Definition & {Modeling\-Rule} \\
\multicolumn{6}{|l|}{Subtype of MTDataItemSubClassType (See section \ref{type:MTDataItemSubClassType})} \\
\end{tabu}
\end{table} 


\FloatBarrier
\subsubsection{Defintion of \texttt{ LateralSubClassType}}
  \label{type:LateralSubClassType}

\FloatBarrier
\begin{table}[ht]
\centering 
  \caption{\texttt{LateralSubClassType} Definition}
  \label{table:LateralSubClassType}
\fontsize{9pt}{11pt}\selectfont
\tabulinesep=3pt
\begin{tabu} to 6in {|X[-1.35]|X[-0.7]|X[-1.75]|X[-1.5]|X[-1]|X[-0.7]|} \everyrow{\hline}
\hline
\rowfont\bfseries {Attribute} & \multicolumn{5}{|l|}{Value} \\
\tabucline[1.5pt]{}
BrowseName & \multicolumn{5}{|l|}{LateralSubClassType} \\
IsAbstract & \multicolumn{5}{|l|}{False} \\
\tabucline[1.5pt]{}
\rowfont \bfseries References & NodeClass & BrowseName & DataType & Type\-Definition & {Modeling\-Rule} \\
\multicolumn{6}{|l|}{Subtype of MTDataItemSubClassType (See section \ref{type:MTDataItemSubClassType})} \\
\end{tabu}
\end{table} 


\FloatBarrier
\subsubsection{Defintion of \texttt{ LeebSubClassType}}
  \label{type:LeebSubClassType}

\FloatBarrier
\begin{table}[ht]
\centering 
  \caption{\texttt{LeebSubClassType} Definition}
  \label{table:LeebSubClassType}
\fontsize{9pt}{11pt}\selectfont
\tabulinesep=3pt
\begin{tabu} to 6in {|X[-1.35]|X[-0.7]|X[-1.75]|X[-1.5]|X[-1]|X[-0.7]|} \everyrow{\hline}
\hline
\rowfont\bfseries {Attribute} & \multicolumn{5}{|l|}{Value} \\
\tabucline[1.5pt]{}
BrowseName & \multicolumn{5}{|l|}{LeebSubClassType} \\
IsAbstract & \multicolumn{5}{|l|}{False} \\
\tabucline[1.5pt]{}
\rowfont \bfseries References & NodeClass & BrowseName & DataType & Type\-Definition & {Modeling\-Rule} \\
\multicolumn{6}{|l|}{Subtype of MTDataItemSubClassType (See section \ref{type:MTDataItemSubClassType})} \\
\end{tabu}
\end{table} 


\FloatBarrier
\subsubsection{Defintion of \texttt{ LengthSubClassType}}
  \label{type:LengthSubClassType}

\FloatBarrier
\begin{table}[ht]
\centering 
  \caption{\texttt{LengthSubClassType} Definition}
  \label{table:LengthSubClassType}
\fontsize{9pt}{11pt}\selectfont
\tabulinesep=3pt
\begin{tabu} to 6in {|X[-1.35]|X[-0.7]|X[-1.75]|X[-1.5]|X[-1]|X[-0.7]|} \everyrow{\hline}
\hline
\rowfont\bfseries {Attribute} & \multicolumn{5}{|l|}{Value} \\
\tabucline[1.5pt]{}
BrowseName & \multicolumn{5}{|l|}{LengthSubClassType} \\
IsAbstract & \multicolumn{5}{|l|}{False} \\
\tabucline[1.5pt]{}
\rowfont \bfseries References & NodeClass & BrowseName & DataType & Type\-Definition & {Modeling\-Rule} \\
\multicolumn{6}{|l|}{Subtype of MTDataItemSubClassType (See section \ref{type:MTDataItemSubClassType})} \\
\end{tabu}
\end{table} 


\FloatBarrier
\subsubsection{Defintion of \texttt{ LinearSubClassType}}
  \label{type:LinearSubClassType}

\FloatBarrier
\begin{table}[ht]
\centering 
  \caption{\texttt{LinearSubClassType} Definition}
  \label{table:LinearSubClassType}
\fontsize{9pt}{11pt}\selectfont
\tabulinesep=3pt
\begin{tabu} to 6in {|X[-1.35]|X[-0.7]|X[-1.75]|X[-1.5]|X[-1]|X[-0.7]|} \everyrow{\hline}
\hline
\rowfont\bfseries {Attribute} & \multicolumn{5}{|l|}{Value} \\
\tabucline[1.5pt]{}
BrowseName & \multicolumn{5}{|l|}{LinearSubClassType} \\
IsAbstract & \multicolumn{5}{|l|}{False} \\
\tabucline[1.5pt]{}
\rowfont \bfseries References & NodeClass & BrowseName & DataType & Type\-Definition & {Modeling\-Rule} \\
\multicolumn{6}{|l|}{Subtype of MTDataItemSubClassType (See section \ref{type:MTDataItemSubClassType})} \\
\end{tabu}
\end{table} 


\FloatBarrier
\subsubsection{Defintion of \texttt{ LineSubClassType}}
  \label{type:LineSubClassType}

\FloatBarrier
\begin{table}[ht]
\centering 
  \caption{\texttt{LineSubClassType} Definition}
  \label{table:LineSubClassType}
\fontsize{9pt}{11pt}\selectfont
\tabulinesep=3pt
\begin{tabu} to 6in {|X[-1.35]|X[-0.7]|X[-1.75]|X[-1.5]|X[-1]|X[-0.7]|} \everyrow{\hline}
\hline
\rowfont\bfseries {Attribute} & \multicolumn{5}{|l|}{Value} \\
\tabucline[1.5pt]{}
BrowseName & \multicolumn{5}{|l|}{LineSubClassType} \\
IsAbstract & \multicolumn{5}{|l|}{False} \\
\tabucline[1.5pt]{}
\rowfont \bfseries References & NodeClass & BrowseName & DataType & Type\-Definition & {Modeling\-Rule} \\
\multicolumn{6}{|l|}{Subtype of MTDataItemSubClassType (See section \ref{type:MTDataItemSubClassType})} \\
\end{tabu}
\end{table} 


\FloatBarrier
\subsubsection{Defintion of \texttt{ LoadedSubClassType}}
  \label{type:LoadedSubClassType}

\FloatBarrier
\begin{table}[ht]
\centering 
  \caption{\texttt{LoadedSubClassType} Definition}
  \label{table:LoadedSubClassType}
\fontsize{9pt}{11pt}\selectfont
\tabulinesep=3pt
\begin{tabu} to 6in {|X[-1.35]|X[-0.7]|X[-1.75]|X[-1.5]|X[-1]|X[-0.7]|} \everyrow{\hline}
\hline
\rowfont\bfseries {Attribute} & \multicolumn{5}{|l|}{Value} \\
\tabucline[1.5pt]{}
BrowseName & \multicolumn{5}{|l|}{LoadedSubClassType} \\
IsAbstract & \multicolumn{5}{|l|}{False} \\
\tabucline[1.5pt]{}
\rowfont \bfseries References & NodeClass & BrowseName & DataType & Type\-Definition & {Modeling\-Rule} \\
\multicolumn{6}{|l|}{Subtype of MTDataItemSubClassType (See section \ref{type:MTDataItemSubClassType})} \\
\end{tabu}
\end{table} 


\FloatBarrier
\subsubsection{Defintion of \texttt{ MachineAxisLockSubClassType}}
  \label{type:MachineAxisLockSubClassType}

\FloatBarrier
\begin{table}[ht]
\centering 
  \caption{\texttt{MachineAxisLockSubClassType} Definition}
  \label{table:MachineAxisLockSubClassType}
\fontsize{9pt}{11pt}\selectfont
\tabulinesep=3pt
\begin{tabu} to 6in {|X[-1.35]|X[-0.7]|X[-1.75]|X[-1.5]|X[-1]|X[-0.7]|} \everyrow{\hline}
\hline
\rowfont\bfseries {Attribute} & \multicolumn{5}{|l|}{Value} \\
\tabucline[1.5pt]{}
BrowseName & \multicolumn{5}{|l|}{MachineAxisLockSubClassType} \\
IsAbstract & \multicolumn{5}{|l|}{False} \\
\tabucline[1.5pt]{}
\rowfont \bfseries References & NodeClass & BrowseName & DataType & Type\-Definition & {Modeling\-Rule} \\
\multicolumn{6}{|l|}{Subtype of MTDataItemSubClassType (See section \ref{type:MTDataItemSubClassType})} \\
\end{tabu}
\end{table} 


\FloatBarrier
\subsubsection{Defintion of \texttt{ MaintenanceSubClassType}}
  \label{type:MaintenanceSubClassType}

\FloatBarrier
\begin{table}[ht]
\centering 
  \caption{\texttt{MaintenanceSubClassType} Definition}
  \label{table:MaintenanceSubClassType}
\fontsize{9pt}{11pt}\selectfont
\tabulinesep=3pt
\begin{tabu} to 6in {|X[-1.35]|X[-0.7]|X[-1.75]|X[-1.5]|X[-1]|X[-0.7]|} \everyrow{\hline}
\hline
\rowfont\bfseries {Attribute} & \multicolumn{5}{|l|}{Value} \\
\tabucline[1.5pt]{}
BrowseName & \multicolumn{5}{|l|}{MaintenanceSubClassType} \\
IsAbstract & \multicolumn{5}{|l|}{False} \\
\tabucline[1.5pt]{}
\rowfont \bfseries References & NodeClass & BrowseName & DataType & Type\-Definition & {Modeling\-Rule} \\
\multicolumn{6}{|l|}{Subtype of MTDataItemSubClassType (See section \ref{type:MTDataItemSubClassType})} \\
\end{tabu}
\end{table} 


\FloatBarrier
\subsubsection{Defintion of \texttt{ ManualUnclampSubClassType}}
  \label{type:ManualUnclampSubClassType}

\FloatBarrier
\begin{table}[ht]
\centering 
  \caption{\texttt{ManualUnclampSubClassType} Definition}
  \label{table:ManualUnclampSubClassType}
\fontsize{9pt}{11pt}\selectfont
\tabulinesep=3pt
\begin{tabu} to 6in {|X[-1.35]|X[-0.7]|X[-1.75]|X[-1.5]|X[-1]|X[-0.7]|} \everyrow{\hline}
\hline
\rowfont\bfseries {Attribute} & \multicolumn{5}{|l|}{Value} \\
\tabucline[1.5pt]{}
BrowseName & \multicolumn{5}{|l|}{ManualUnclampSubClassType} \\
IsAbstract & \multicolumn{5}{|l|}{False} \\
\tabucline[1.5pt]{}
\rowfont \bfseries References & NodeClass & BrowseName & DataType & Type\-Definition & {Modeling\-Rule} \\
\multicolumn{6}{|l|}{Subtype of MTDataItemSubClassType (See section \ref{type:MTDataItemSubClassType})} \\
\end{tabu}
\end{table} 


\FloatBarrier
\subsubsection{Defintion of \texttt{ MaximumSubClassType}}
  \label{type:MaximumSubClassType}

\FloatBarrier
\begin{table}[ht]
\centering 
  \caption{\texttt{MaximumSubClassType} Definition}
  \label{table:MaximumSubClassType}
\fontsize{9pt}{11pt}\selectfont
\tabulinesep=3pt
\begin{tabu} to 6in {|X[-1.35]|X[-0.7]|X[-1.75]|X[-1.5]|X[-1]|X[-0.7]|} \everyrow{\hline}
\hline
\rowfont\bfseries {Attribute} & \multicolumn{5}{|l|}{Value} \\
\tabucline[1.5pt]{}
BrowseName & \multicolumn{5}{|l|}{MaximumSubClassType} \\
IsAbstract & \multicolumn{5}{|l|}{False} \\
\tabucline[1.5pt]{}
\rowfont \bfseries References & NodeClass & BrowseName & DataType & Type\-Definition & {Modeling\-Rule} \\
\multicolumn{6}{|l|}{Subtype of MTDataItemSubClassType (See section \ref{type:MTDataItemSubClassType})} \\
\end{tabu}
\end{table} 


\FloatBarrier
\subsubsection{Defintion of \texttt{ MinimumSubClassType}}
  \label{type:MinimumSubClassType}

\FloatBarrier
\begin{table}[ht]
\centering 
  \caption{\texttt{MinimumSubClassType} Definition}
  \label{table:MinimumSubClassType}
\fontsize{9pt}{11pt}\selectfont
\tabulinesep=3pt
\begin{tabu} to 6in {|X[-1.35]|X[-0.7]|X[-1.75]|X[-1.5]|X[-1]|X[-0.7]|} \everyrow{\hline}
\hline
\rowfont\bfseries {Attribute} & \multicolumn{5}{|l|}{Value} \\
\tabucline[1.5pt]{}
BrowseName & \multicolumn{5}{|l|}{MinimumSubClassType} \\
IsAbstract & \multicolumn{5}{|l|}{False} \\
\tabucline[1.5pt]{}
\rowfont \bfseries References & NodeClass & BrowseName & DataType & Type\-Definition & {Modeling\-Rule} \\
\multicolumn{6}{|l|}{Subtype of MTDataItemSubClassType (See section \ref{type:MTDataItemSubClassType})} \\
\end{tabu}
\end{table} 


\FloatBarrier
\subsubsection{Defintion of \texttt{ MohsSubClassType}}
  \label{type:MohsSubClassType}

\FloatBarrier
\begin{table}[ht]
\centering 
  \caption{\texttt{MohsSubClassType} Definition}
  \label{table:MohsSubClassType}
\fontsize{9pt}{11pt}\selectfont
\tabulinesep=3pt
\begin{tabu} to 6in {|X[-1.35]|X[-0.7]|X[-1.75]|X[-1.5]|X[-1]|X[-0.7]|} \everyrow{\hline}
\hline
\rowfont\bfseries {Attribute} & \multicolumn{5}{|l|}{Value} \\
\tabucline[1.5pt]{}
BrowseName & \multicolumn{5}{|l|}{MohsSubClassType} \\
IsAbstract & \multicolumn{5}{|l|}{False} \\
\tabucline[1.5pt]{}
\rowfont \bfseries References & NodeClass & BrowseName & DataType & Type\-Definition & {Modeling\-Rule} \\
\multicolumn{6}{|l|}{Subtype of MTDataItemSubClassType (See section \ref{type:MTDataItemSubClassType})} \\
\end{tabu}
\end{table} 


\FloatBarrier
\subsubsection{Defintion of \texttt{ MoleSubClassType}}
  \label{type:MoleSubClassType}

\FloatBarrier
\begin{table}[ht]
\centering 
  \caption{\texttt{MoleSubClassType} Definition}
  \label{table:MoleSubClassType}
\fontsize{9pt}{11pt}\selectfont
\tabulinesep=3pt
\begin{tabu} to 6in {|X[-1.35]|X[-0.7]|X[-1.75]|X[-1.5]|X[-1]|X[-0.7]|} \everyrow{\hline}
\hline
\rowfont\bfseries {Attribute} & \multicolumn{5}{|l|}{Value} \\
\tabucline[1.5pt]{}
BrowseName & \multicolumn{5}{|l|}{MoleSubClassType} \\
IsAbstract & \multicolumn{5}{|l|}{False} \\
\tabucline[1.5pt]{}
\rowfont \bfseries References & NodeClass & BrowseName & DataType & Type\-Definition & {Modeling\-Rule} \\
\multicolumn{6}{|l|}{Subtype of MTDataItemSubClassType (See section \ref{type:MTDataItemSubClassType})} \\
\end{tabu}
\end{table} 


\FloatBarrier
\subsubsection{Defintion of \texttt{ MotionSubClassType}}
  \label{type:MotionSubClassType}

\FloatBarrier
\begin{table}[ht]
\centering 
  \caption{\texttt{MotionSubClassType} Definition}
  \label{table:MotionSubClassType}
\fontsize{9pt}{11pt}\selectfont
\tabulinesep=3pt
\begin{tabu} to 6in {|X[-1.35]|X[-0.7]|X[-1.75]|X[-1.5]|X[-1]|X[-0.7]|} \everyrow{\hline}
\hline
\rowfont\bfseries {Attribute} & \multicolumn{5}{|l|}{Value} \\
\tabucline[1.5pt]{}
BrowseName & \multicolumn{5}{|l|}{MotionSubClassType} \\
IsAbstract & \multicolumn{5}{|l|}{False} \\
\tabucline[1.5pt]{}
\rowfont \bfseries References & NodeClass & BrowseName & DataType & Type\-Definition & {Modeling\-Rule} \\
\multicolumn{6}{|l|}{Subtype of MTDataItemSubClassType (See section \ref{type:MTDataItemSubClassType})} \\
\end{tabu}
\end{table} 


\FloatBarrier
\subsubsection{Defintion of \texttt{ NoScaleSubClassType}}
  \label{type:NoScaleSubClassType}

\FloatBarrier
\begin{table}[ht]
\centering 
  \caption{\texttt{NoScaleSubClassType} Definition}
  \label{table:NoScaleSubClassType}
\fontsize{9pt}{11pt}\selectfont
\tabulinesep=3pt
\begin{tabu} to 6in {|X[-1.35]|X[-0.7]|X[-1.75]|X[-1.5]|X[-1]|X[-0.7]|} \everyrow{\hline}
\hline
\rowfont\bfseries {Attribute} & \multicolumn{5}{|l|}{Value} \\
\tabucline[1.5pt]{}
BrowseName & \multicolumn{5}{|l|}{NoScaleSubClassType} \\
IsAbstract & \multicolumn{5}{|l|}{False} \\
\tabucline[1.5pt]{}
\rowfont \bfseries References & NodeClass & BrowseName & DataType & Type\-Definition & {Modeling\-Rule} \\
\multicolumn{6}{|l|}{Subtype of MTDataItemSubClassType (See section \ref{type:MTDataItemSubClassType})} \\
\end{tabu}
\end{table} 


\FloatBarrier
\subsubsection{Defintion of \texttt{ OperatingSubClassType}}
  \label{type:OperatingSubClassType}

\FloatBarrier
\begin{table}[ht]
\centering 
  \caption{\texttt{OperatingSubClassType} Definition}
  \label{table:OperatingSubClassType}
\fontsize{9pt}{11pt}\selectfont
\tabulinesep=3pt
\begin{tabu} to 6in {|X[-1.35]|X[-0.7]|X[-1.75]|X[-1.5]|X[-1]|X[-0.7]|} \everyrow{\hline}
\hline
\rowfont\bfseries {Attribute} & \multicolumn{5}{|l|}{Value} \\
\tabucline[1.5pt]{}
BrowseName & \multicolumn{5}{|l|}{OperatingSubClassType} \\
IsAbstract & \multicolumn{5}{|l|}{False} \\
\tabucline[1.5pt]{}
\rowfont \bfseries References & NodeClass & BrowseName & DataType & Type\-Definition & {Modeling\-Rule} \\
\multicolumn{6}{|l|}{Subtype of MTDataItemSubClassType (See section \ref{type:MTDataItemSubClassType})} \\
\end{tabu}
\end{table} 


\FloatBarrier
\subsubsection{Defintion of \texttt{ OperatorSubClassType}}
  \label{type:OperatorSubClassType}

\FloatBarrier
\begin{table}[ht]
\centering 
  \caption{\texttt{OperatorSubClassType} Definition}
  \label{table:OperatorSubClassType}
\fontsize{9pt}{11pt}\selectfont
\tabulinesep=3pt
\begin{tabu} to 6in {|X[-1.35]|X[-0.7]|X[-1.75]|X[-1.5]|X[-1]|X[-0.7]|} \everyrow{\hline}
\hline
\rowfont\bfseries {Attribute} & \multicolumn{5}{|l|}{Value} \\
\tabucline[1.5pt]{}
BrowseName & \multicolumn{5}{|l|}{OperatorSubClassType} \\
IsAbstract & \multicolumn{5}{|l|}{False} \\
\tabucline[1.5pt]{}
\rowfont \bfseries References & NodeClass & BrowseName & DataType & Type\-Definition & {Modeling\-Rule} \\
\multicolumn{6}{|l|}{Subtype of MTDataItemSubClassType (See section \ref{type:MTDataItemSubClassType})} \\
\end{tabu}
\end{table} 


\FloatBarrier
\subsubsection{Defintion of \texttt{ OptionalStopSubClassType}}
  \label{type:OptionalStopSubClassType}

\FloatBarrier
\begin{table}[ht]
\centering 
  \caption{\texttt{OptionalStopSubClassType} Definition}
  \label{table:OptionalStopSubClassType}
\fontsize{9pt}{11pt}\selectfont
\tabulinesep=3pt
\begin{tabu} to 6in {|X[-1.35]|X[-0.7]|X[-1.75]|X[-1.5]|X[-1]|X[-0.7]|} \everyrow{\hline}
\hline
\rowfont\bfseries {Attribute} & \multicolumn{5}{|l|}{Value} \\
\tabucline[1.5pt]{}
BrowseName & \multicolumn{5}{|l|}{OptionalStopSubClassType} \\
IsAbstract & \multicolumn{5}{|l|}{False} \\
\tabucline[1.5pt]{}
\rowfont \bfseries References & NodeClass & BrowseName & DataType & Type\-Definition & {Modeling\-Rule} \\
\multicolumn{6}{|l|}{Subtype of MTDataItemSubClassType (See section \ref{type:MTDataItemSubClassType})} \\
\end{tabu}
\end{table} 


\FloatBarrier
\subsubsection{Defintion of \texttt{ OverrideSubClassType}}
  \label{type:OverrideSubClassType}

\FloatBarrier
\begin{table}[ht]
\centering 
  \caption{\texttt{OverrideSubClassType} Definition}
  \label{table:OverrideSubClassType}
\fontsize{9pt}{11pt}\selectfont
\tabulinesep=3pt
\begin{tabu} to 6in {|X[-1.35]|X[-0.7]|X[-1.75]|X[-1.5]|X[-1]|X[-0.7]|} \everyrow{\hline}
\hline
\rowfont\bfseries {Attribute} & \multicolumn{5}{|l|}{Value} \\
\tabucline[1.5pt]{}
BrowseName & \multicolumn{5}{|l|}{OverrideSubClassType} \\
IsAbstract & \multicolumn{5}{|l|}{False} \\
\tabucline[1.5pt]{}
\rowfont \bfseries References & NodeClass & BrowseName & DataType & Type\-Definition & {Modeling\-Rule} \\
\multicolumn{6}{|l|}{Subtype of MTDataItemSubClassType (See section \ref{type:MTDataItemSubClassType})} \\
\end{tabu}
\end{table} 


\FloatBarrier
\subsubsection{Defintion of \texttt{ PoweredSubClassType}}
  \label{type:PoweredSubClassType}

\FloatBarrier
\begin{table}[ht]
\centering 
  \caption{\texttt{PoweredSubClassType} Definition}
  \label{table:PoweredSubClassType}
\fontsize{9pt}{11pt}\selectfont
\tabulinesep=3pt
\begin{tabu} to 6in {|X[-1.35]|X[-0.7]|X[-1.75]|X[-1.5]|X[-1]|X[-0.7]|} \everyrow{\hline}
\hline
\rowfont\bfseries {Attribute} & \multicolumn{5}{|l|}{Value} \\
\tabucline[1.5pt]{}
BrowseName & \multicolumn{5}{|l|}{PoweredSubClassType} \\
IsAbstract & \multicolumn{5}{|l|}{False} \\
\tabucline[1.5pt]{}
\rowfont \bfseries References & NodeClass & BrowseName & DataType & Type\-Definition & {Modeling\-Rule} \\
\multicolumn{6}{|l|}{Subtype of MTDataItemSubClassType (See section \ref{type:MTDataItemSubClassType})} \\
\end{tabu}
\end{table} 


\FloatBarrier
\subsubsection{Defintion of \texttt{ PrimarySubClassType}}
  \label{type:PrimarySubClassType}

\FloatBarrier
\begin{table}[ht]
\centering 
  \caption{\texttt{PrimarySubClassType} Definition}
  \label{table:PrimarySubClassType}
\fontsize{9pt}{11pt}\selectfont
\tabulinesep=3pt
\begin{tabu} to 6in {|X[-1.35]|X[-0.7]|X[-1.75]|X[-1.5]|X[-1]|X[-0.7]|} \everyrow{\hline}
\hline
\rowfont\bfseries {Attribute} & \multicolumn{5}{|l|}{Value} \\
\tabucline[1.5pt]{}
BrowseName & \multicolumn{5}{|l|}{PrimarySubClassType} \\
IsAbstract & \multicolumn{5}{|l|}{False} \\
\tabucline[1.5pt]{}
\rowfont \bfseries References & NodeClass & BrowseName & DataType & Type\-Definition & {Modeling\-Rule} \\
\multicolumn{6}{|l|}{Subtype of MTDataItemSubClassType (See section \ref{type:MTDataItemSubClassType})} \\
\end{tabu}
\end{table} 


\FloatBarrier
\subsubsection{Defintion of \texttt{ ProbeSubClassType}}
  \label{type:ProbeSubClassType}

\FloatBarrier
\begin{table}[ht]
\centering 
  \caption{\texttt{ProbeSubClassType} Definition}
  \label{table:ProbeSubClassType}
\fontsize{9pt}{11pt}\selectfont
\tabulinesep=3pt
\begin{tabu} to 6in {|X[-1.35]|X[-0.7]|X[-1.75]|X[-1.5]|X[-1]|X[-0.7]|} \everyrow{\hline}
\hline
\rowfont\bfseries {Attribute} & \multicolumn{5}{|l|}{Value} \\
\tabucline[1.5pt]{}
BrowseName & \multicolumn{5}{|l|}{ProbeSubClassType} \\
IsAbstract & \multicolumn{5}{|l|}{False} \\
\tabucline[1.5pt]{}
\rowfont \bfseries References & NodeClass & BrowseName & DataType & Type\-Definition & {Modeling\-Rule} \\
\multicolumn{6}{|l|}{Subtype of MTDataItemSubClassType (See section \ref{type:MTDataItemSubClassType})} \\
\end{tabu}
\end{table} 


\FloatBarrier
\subsubsection{Defintion of \texttt{ ProcessSubClassType}}
  \label{type:ProcessSubClassType}

\FloatBarrier
\begin{table}[ht]
\centering 
  \caption{\texttt{ProcessSubClassType} Definition}
  \label{table:ProcessSubClassType}
\fontsize{9pt}{11pt}\selectfont
\tabulinesep=3pt
\begin{tabu} to 6in {|X[-1.35]|X[-0.7]|X[-1.75]|X[-1.5]|X[-1]|X[-0.7]|} \everyrow{\hline}
\hline
\rowfont\bfseries {Attribute} & \multicolumn{5}{|l|}{Value} \\
\tabucline[1.5pt]{}
BrowseName & \multicolumn{5}{|l|}{ProcessSubClassType} \\
IsAbstract & \multicolumn{5}{|l|}{False} \\
\tabucline[1.5pt]{}
\rowfont \bfseries References & NodeClass & BrowseName & DataType & Type\-Definition & {Modeling\-Rule} \\
\multicolumn{6}{|l|}{Subtype of MTDataItemSubClassType (See section \ref{type:MTDataItemSubClassType})} \\
\end{tabu}
\end{table} 


\FloatBarrier
\subsubsection{Defintion of \texttt{ ProgrammedSubClassType}}
  \label{type:ProgrammedSubClassType}

\FloatBarrier
\begin{table}[ht]
\centering 
  \caption{\texttt{ProgrammedSubClassType} Definition}
  \label{table:ProgrammedSubClassType}
\fontsize{9pt}{11pt}\selectfont
\tabulinesep=3pt
\begin{tabu} to 6in {|X[-1.35]|X[-0.7]|X[-1.75]|X[-1.5]|X[-1]|X[-0.7]|} \everyrow{\hline}
\hline
\rowfont\bfseries {Attribute} & \multicolumn{5}{|l|}{Value} \\
\tabucline[1.5pt]{}
BrowseName & \multicolumn{5}{|l|}{ProgrammedSubClassType} \\
IsAbstract & \multicolumn{5}{|l|}{False} \\
\tabucline[1.5pt]{}
\rowfont \bfseries References & NodeClass & BrowseName & DataType & Type\-Definition & {Modeling\-Rule} \\
\multicolumn{6}{|l|}{Subtype of MTDataItemSubClassType (See section \ref{type:MTDataItemSubClassType})} \\
\end{tabu}
\end{table} 


\FloatBarrier
\subsubsection{Defintion of \texttt{ RadialSubClassType}}
  \label{type:RadialSubClassType}

\FloatBarrier
\begin{table}[ht]
\centering 
  \caption{\texttt{RadialSubClassType} Definition}
  \label{table:RadialSubClassType}
\fontsize{9pt}{11pt}\selectfont
\tabulinesep=3pt
\begin{tabu} to 6in {|X[-1.35]|X[-0.7]|X[-1.75]|X[-1.5]|X[-1]|X[-0.7]|} \everyrow{\hline}
\hline
\rowfont\bfseries {Attribute} & \multicolumn{5}{|l|}{Value} \\
\tabucline[1.5pt]{}
BrowseName & \multicolumn{5}{|l|}{RadialSubClassType} \\
IsAbstract & \multicolumn{5}{|l|}{False} \\
\tabucline[1.5pt]{}
\rowfont \bfseries References & NodeClass & BrowseName & DataType & Type\-Definition & {Modeling\-Rule} \\
\multicolumn{6}{|l|}{Subtype of MTDataItemSubClassType (See section \ref{type:MTDataItemSubClassType})} \\
\end{tabu}
\end{table} 


\FloatBarrier
\subsubsection{Defintion of \texttt{ RapidSubClassType}}
  \label{type:RapidSubClassType}

\FloatBarrier
\begin{table}[ht]
\centering 
  \caption{\texttt{RapidSubClassType} Definition}
  \label{table:RapidSubClassType}
\fontsize{9pt}{11pt}\selectfont
\tabulinesep=3pt
\begin{tabu} to 6in {|X[-1.35]|X[-0.7]|X[-1.75]|X[-1.5]|X[-1]|X[-0.7]|} \everyrow{\hline}
\hline
\rowfont\bfseries {Attribute} & \multicolumn{5}{|l|}{Value} \\
\tabucline[1.5pt]{}
BrowseName & \multicolumn{5}{|l|}{RapidSubClassType} \\
IsAbstract & \multicolumn{5}{|l|}{False} \\
\tabucline[1.5pt]{}
\rowfont \bfseries References & NodeClass & BrowseName & DataType & Type\-Definition & {Modeling\-Rule} \\
\multicolumn{6}{|l|}{Subtype of MTDataItemSubClassType (See section \ref{type:MTDataItemSubClassType})} \\
\end{tabu}
\end{table} 


\FloatBarrier
\subsubsection{Defintion of \texttt{ RelativeSubClassType}}
  \label{type:RelativeSubClassType}

\FloatBarrier
\begin{table}[ht]
\centering 
  \caption{\texttt{RelativeSubClassType} Definition}
  \label{table:RelativeSubClassType}
\fontsize{9pt}{11pt}\selectfont
\tabulinesep=3pt
\begin{tabu} to 6in {|X[-1.35]|X[-0.7]|X[-1.75]|X[-1.5]|X[-1]|X[-0.7]|} \everyrow{\hline}
\hline
\rowfont\bfseries {Attribute} & \multicolumn{5}{|l|}{Value} \\
\tabucline[1.5pt]{}
BrowseName & \multicolumn{5}{|l|}{RelativeSubClassType} \\
IsAbstract & \multicolumn{5}{|l|}{False} \\
\tabucline[1.5pt]{}
\rowfont \bfseries References & NodeClass & BrowseName & DataType & Type\-Definition & {Modeling\-Rule} \\
\multicolumn{6}{|l|}{Subtype of MTDataItemSubClassType (See section \ref{type:MTDataItemSubClassType})} \\
\end{tabu}
\end{table} 


\FloatBarrier
\subsubsection{Defintion of \texttt{ RemainingSubClassType}}
  \label{type:RemainingSubClassType}

\FloatBarrier
\begin{table}[ht]
\centering 
  \caption{\texttt{RemainingSubClassType} Definition}
  \label{table:RemainingSubClassType}
\fontsize{9pt}{11pt}\selectfont
\tabulinesep=3pt
\begin{tabu} to 6in {|X[-1.35]|X[-0.7]|X[-1.75]|X[-1.5]|X[-1]|X[-0.7]|} \everyrow{\hline}
\hline
\rowfont\bfseries {Attribute} & \multicolumn{5}{|l|}{Value} \\
\tabucline[1.5pt]{}
BrowseName & \multicolumn{5}{|l|}{RemainingSubClassType} \\
IsAbstract & \multicolumn{5}{|l|}{False} \\
\tabucline[1.5pt]{}
\rowfont \bfseries References & NodeClass & BrowseName & DataType & Type\-Definition & {Modeling\-Rule} \\
\multicolumn{6}{|l|}{Subtype of MTDataItemSubClassType (See section \ref{type:MTDataItemSubClassType})} \\
\end{tabu}
\end{table} 


\FloatBarrier
\subsubsection{Defintion of \texttt{ RequestSubClassType}}
  \label{type:RequestSubClassType}

\FloatBarrier
\begin{table}[ht]
\centering 
  \caption{\texttt{RequestSubClassType} Definition}
  \label{table:RequestSubClassType}
\fontsize{9pt}{11pt}\selectfont
\tabulinesep=3pt
\begin{tabu} to 6in {|X[-1.35]|X[-0.7]|X[-1.75]|X[-1.5]|X[-1]|X[-0.7]|} \everyrow{\hline}
\hline
\rowfont\bfseries {Attribute} & \multicolumn{5}{|l|}{Value} \\
\tabucline[1.5pt]{}
BrowseName & \multicolumn{5}{|l|}{RequestSubClassType} \\
IsAbstract & \multicolumn{5}{|l|}{False} \\
\tabucline[1.5pt]{}
\rowfont \bfseries References & NodeClass & BrowseName & DataType & Type\-Definition & {Modeling\-Rule} \\
\multicolumn{6}{|l|}{Subtype of MTDataItemSubClassType (See section \ref{type:MTDataItemSubClassType})} \\
\end{tabu}
\end{table} 


\FloatBarrier
\subsubsection{Defintion of \texttt{ ResponseSubClassType}}
  \label{type:ResponseSubClassType}

\FloatBarrier
\begin{table}[ht]
\centering 
  \caption{\texttt{ResponseSubClassType} Definition}
  \label{table:ResponseSubClassType}
\fontsize{9pt}{11pt}\selectfont
\tabulinesep=3pt
\begin{tabu} to 6in {|X[-1.35]|X[-0.7]|X[-1.75]|X[-1.5]|X[-1]|X[-0.7]|} \everyrow{\hline}
\hline
\rowfont\bfseries {Attribute} & \multicolumn{5}{|l|}{Value} \\
\tabucline[1.5pt]{}
BrowseName & \multicolumn{5}{|l|}{ResponseSubClassType} \\
IsAbstract & \multicolumn{5}{|l|}{False} \\
\tabucline[1.5pt]{}
\rowfont \bfseries References & NodeClass & BrowseName & DataType & Type\-Definition & {Modeling\-Rule} \\
\multicolumn{6}{|l|}{Subtype of MTDataItemSubClassType (See section \ref{type:MTDataItemSubClassType})} \\
\end{tabu}
\end{table} 


\FloatBarrier
\subsubsection{Defintion of \texttt{ RockwellSubClassType}}
  \label{type:RockwellSubClassType}

\FloatBarrier
\begin{table}[ht]
\centering 
  \caption{\texttt{RockwellSubClassType} Definition}
  \label{table:RockwellSubClassType}
\fontsize{9pt}{11pt}\selectfont
\tabulinesep=3pt
\begin{tabu} to 6in {|X[-1.35]|X[-0.7]|X[-1.75]|X[-1.5]|X[-1]|X[-0.7]|} \everyrow{\hline}
\hline
\rowfont\bfseries {Attribute} & \multicolumn{5}{|l|}{Value} \\
\tabucline[1.5pt]{}
BrowseName & \multicolumn{5}{|l|}{RockwellSubClassType} \\
IsAbstract & \multicolumn{5}{|l|}{False} \\
\tabucline[1.5pt]{}
\rowfont \bfseries References & NodeClass & BrowseName & DataType & Type\-Definition & {Modeling\-Rule} \\
\multicolumn{6}{|l|}{Subtype of MTDataItemSubClassType (See section \ref{type:MTDataItemSubClassType})} \\
\end{tabu}
\end{table} 


\FloatBarrier
\subsubsection{Defintion of \texttt{ RotarySubClassType}}
  \label{type:RotarySubClassType}

\FloatBarrier
\begin{table}[ht]
\centering 
  \caption{\texttt{RotarySubClassType} Definition}
  \label{table:RotarySubClassType}
\fontsize{9pt}{11pt}\selectfont
\tabulinesep=3pt
\begin{tabu} to 6in {|X[-1.35]|X[-0.7]|X[-1.75]|X[-1.5]|X[-1]|X[-0.7]|} \everyrow{\hline}
\hline
\rowfont\bfseries {Attribute} & \multicolumn{5}{|l|}{Value} \\
\tabucline[1.5pt]{}
BrowseName & \multicolumn{5}{|l|}{RotarySubClassType} \\
IsAbstract & \multicolumn{5}{|l|}{False} \\
\tabucline[1.5pt]{}
\rowfont \bfseries References & NodeClass & BrowseName & DataType & Type\-Definition & {Modeling\-Rule} \\
\multicolumn{6}{|l|}{Subtype of MTDataItemSubClassType (See section \ref{type:MTDataItemSubClassType})} \\
\end{tabu}
\end{table} 


\FloatBarrier
\subsubsection{Defintion of \texttt{ SetUpSubClassType}}
  \label{type:SetUpSubClassType}

\FloatBarrier
\begin{table}[ht]
\centering 
  \caption{\texttt{SetUpSubClassType} Definition}
  \label{table:SetUpSubClassType}
\fontsize{9pt}{11pt}\selectfont
\tabulinesep=3pt
\begin{tabu} to 6in {|X[-1.35]|X[-0.7]|X[-1.75]|X[-1.5]|X[-1]|X[-0.7]|} \everyrow{\hline}
\hline
\rowfont\bfseries {Attribute} & \multicolumn{5}{|l|}{Value} \\
\tabucline[1.5pt]{}
BrowseName & \multicolumn{5}{|l|}{SetUpSubClassType} \\
IsAbstract & \multicolumn{5}{|l|}{False} \\
\tabucline[1.5pt]{}
\rowfont \bfseries References & NodeClass & BrowseName & DataType & Type\-Definition & {Modeling\-Rule} \\
\multicolumn{6}{|l|}{Subtype of MTDataItemSubClassType (See section \ref{type:MTDataItemSubClassType})} \\
\end{tabu}
\end{table} 


\FloatBarrier
\subsubsection{Defintion of \texttt{ ShoreSubClassType}}
  \label{type:ShoreSubClassType}

\FloatBarrier
\begin{table}[ht]
\centering 
  \caption{\texttt{ShoreSubClassType} Definition}
  \label{table:ShoreSubClassType}
\fontsize{9pt}{11pt}\selectfont
\tabulinesep=3pt
\begin{tabu} to 6in {|X[-1.35]|X[-0.7]|X[-1.75]|X[-1.5]|X[-1]|X[-0.7]|} \everyrow{\hline}
\hline
\rowfont\bfseries {Attribute} & \multicolumn{5}{|l|}{Value} \\
\tabucline[1.5pt]{}
BrowseName & \multicolumn{5}{|l|}{ShoreSubClassType} \\
IsAbstract & \multicolumn{5}{|l|}{False} \\
\tabucline[1.5pt]{}
\rowfont \bfseries References & NodeClass & BrowseName & DataType & Type\-Definition & {Modeling\-Rule} \\
\multicolumn{6}{|l|}{Subtype of MTDataItemSubClassType (See section \ref{type:MTDataItemSubClassType})} \\
\end{tabu}
\end{table} 


\FloatBarrier
\subsubsection{Defintion of \texttt{ StandardSubClassType}}
  \label{type:StandardSubClassType}

\FloatBarrier
\begin{table}[ht]
\centering 
  \caption{\texttt{StandardSubClassType} Definition}
  \label{table:StandardSubClassType}
\fontsize{9pt}{11pt}\selectfont
\tabulinesep=3pt
\begin{tabu} to 6in {|X[-1.35]|X[-0.7]|X[-1.75]|X[-1.5]|X[-1]|X[-0.7]|} \everyrow{\hline}
\hline
\rowfont\bfseries {Attribute} & \multicolumn{5}{|l|}{Value} \\
\tabucline[1.5pt]{}
BrowseName & \multicolumn{5}{|l|}{StandardSubClassType} \\
IsAbstract & \multicolumn{5}{|l|}{False} \\
\tabucline[1.5pt]{}
\rowfont \bfseries References & NodeClass & BrowseName & DataType & Type\-Definition & {Modeling\-Rule} \\
\multicolumn{6}{|l|}{Subtype of MTDataItemSubClassType (See section \ref{type:MTDataItemSubClassType})} \\
\end{tabu}
\end{table} 


\FloatBarrier
\subsubsection{Defintion of \texttt{ SwitchedSubClassType}}
  \label{type:SwitchedSubClassType}

\FloatBarrier
\begin{table}[ht]
\centering 
  \caption{\texttt{SwitchedSubClassType} Definition}
  \label{table:SwitchedSubClassType}
\fontsize{9pt}{11pt}\selectfont
\tabulinesep=3pt
\begin{tabu} to 6in {|X[-1.35]|X[-0.7]|X[-1.75]|X[-1.5]|X[-1]|X[-0.7]|} \everyrow{\hline}
\hline
\rowfont\bfseries {Attribute} & \multicolumn{5}{|l|}{Value} \\
\tabucline[1.5pt]{}
BrowseName & \multicolumn{5}{|l|}{SwitchedSubClassType} \\
IsAbstract & \multicolumn{5}{|l|}{False} \\
\tabucline[1.5pt]{}
\rowfont \bfseries References & NodeClass & BrowseName & DataType & Type\-Definition & {Modeling\-Rule} \\
\multicolumn{6}{|l|}{Subtype of MTDataItemSubClassType (See section \ref{type:MTDataItemSubClassType})} \\
\end{tabu}
\end{table} 


\FloatBarrier
\subsubsection{Defintion of \texttt{ TargetSubClassType}}
  \label{type:TargetSubClassType}

\FloatBarrier
\begin{table}[ht]
\centering 
  \caption{\texttt{TargetSubClassType} Definition}
  \label{table:TargetSubClassType}
\fontsize{9pt}{11pt}\selectfont
\tabulinesep=3pt
\begin{tabu} to 6in {|X[-1.35]|X[-0.7]|X[-1.75]|X[-1.5]|X[-1]|X[-0.7]|} \everyrow{\hline}
\hline
\rowfont\bfseries {Attribute} & \multicolumn{5}{|l|}{Value} \\
\tabucline[1.5pt]{}
BrowseName & \multicolumn{5}{|l|}{TargetSubClassType} \\
IsAbstract & \multicolumn{5}{|l|}{False} \\
\tabucline[1.5pt]{}
\rowfont \bfseries References & NodeClass & BrowseName & DataType & Type\-Definition & {Modeling\-Rule} \\
\multicolumn{6}{|l|}{Subtype of MTDataItemSubClassType (See section \ref{type:MTDataItemSubClassType})} \\
\end{tabu}
\end{table} 


\FloatBarrier
\subsubsection{Defintion of \texttt{ ToolChangeStopSubClassType}}
  \label{type:ToolChangeStopSubClassType}

\FloatBarrier
\begin{table}[ht]
\centering 
  \caption{\texttt{ToolChangeStopSubClassType} Definition}
  \label{table:ToolChangeStopSubClassType}
\fontsize{9pt}{11pt}\selectfont
\tabulinesep=3pt
\begin{tabu} to 6in {|X[-1.35]|X[-0.7]|X[-1.75]|X[-1.5]|X[-1]|X[-0.7]|} \everyrow{\hline}
\hline
\rowfont\bfseries {Attribute} & \multicolumn{5}{|l|}{Value} \\
\tabucline[1.5pt]{}
BrowseName & \multicolumn{5}{|l|}{ToolChangeStopSubClassType} \\
IsAbstract & \multicolumn{5}{|l|}{False} \\
\tabucline[1.5pt]{}
\rowfont \bfseries References & NodeClass & BrowseName & DataType & Type\-Definition & {Modeling\-Rule} \\
\multicolumn{6}{|l|}{Subtype of MTDataItemSubClassType (See section \ref{type:MTDataItemSubClassType})} \\
\end{tabu}
\end{table} 


\FloatBarrier
\subsubsection{Defintion of \texttt{ ToolEdgeSubClassType}}
  \label{type:ToolEdgeSubClassType}

\FloatBarrier
\begin{table}[ht]
\centering 
  \caption{\texttt{ToolEdgeSubClassType} Definition}
  \label{table:ToolEdgeSubClassType}
\fontsize{9pt}{11pt}\selectfont
\tabulinesep=3pt
\begin{tabu} to 6in {|X[-1.35]|X[-0.7]|X[-1.75]|X[-1.5]|X[-1]|X[-0.7]|} \everyrow{\hline}
\hline
\rowfont\bfseries {Attribute} & \multicolumn{5}{|l|}{Value} \\
\tabucline[1.5pt]{}
BrowseName & \multicolumn{5}{|l|}{ToolEdgeSubClassType} \\
IsAbstract & \multicolumn{5}{|l|}{False} \\
\tabucline[1.5pt]{}
\rowfont \bfseries References & NodeClass & BrowseName & DataType & Type\-Definition & {Modeling\-Rule} \\
\multicolumn{6}{|l|}{Subtype of MTDataItemSubClassType (See section \ref{type:MTDataItemSubClassType})} \\
\end{tabu}
\end{table} 


\FloatBarrier
\subsubsection{Defintion of \texttt{ ToolGroupSubClassType}}
  \label{type:ToolGroupSubClassType}

\FloatBarrier
\begin{table}[ht]
\centering 
  \caption{\texttt{ToolGroupSubClassType} Definition}
  \label{table:ToolGroupSubClassType}
\fontsize{9pt}{11pt}\selectfont
\tabulinesep=3pt
\begin{tabu} to 6in {|X[-1.35]|X[-0.7]|X[-1.75]|X[-1.5]|X[-1]|X[-0.7]|} \everyrow{\hline}
\hline
\rowfont\bfseries {Attribute} & \multicolumn{5}{|l|}{Value} \\
\tabucline[1.5pt]{}
BrowseName & \multicolumn{5}{|l|}{ToolGroupSubClassType} \\
IsAbstract & \multicolumn{5}{|l|}{False} \\
\tabucline[1.5pt]{}
\rowfont \bfseries References & NodeClass & BrowseName & DataType & Type\-Definition & {Modeling\-Rule} \\
\multicolumn{6}{|l|}{Subtype of MTDataItemSubClassType (See section \ref{type:MTDataItemSubClassType})} \\
\end{tabu}
\end{table} 


\FloatBarrier
\subsubsection{Defintion of \texttt{ ToolSubClassType}}
  \label{type:ToolSubClassType}

\FloatBarrier
\begin{table}[ht]
\centering 
  \caption{\texttt{ToolSubClassType} Definition}
  \label{table:ToolSubClassType}
\fontsize{9pt}{11pt}\selectfont
\tabulinesep=3pt
\begin{tabu} to 6in {|X[-1.35]|X[-0.7]|X[-1.75]|X[-1.5]|X[-1]|X[-0.7]|} \everyrow{\hline}
\hline
\rowfont\bfseries {Attribute} & \multicolumn{5}{|l|}{Value} \\
\tabucline[1.5pt]{}
BrowseName & \multicolumn{5}{|l|}{ToolSubClassType} \\
IsAbstract & \multicolumn{5}{|l|}{False} \\
\tabucline[1.5pt]{}
\rowfont \bfseries References & NodeClass & BrowseName & DataType & Type\-Definition & {Modeling\-Rule} \\
\multicolumn{6}{|l|}{Subtype of MTDataItemSubClassType (See section \ref{type:MTDataItemSubClassType})} \\
\end{tabu}
\end{table} 


\FloatBarrier
\subsubsection{Defintion of \texttt{ UasbleSubClassType}}
  \label{type:UasbleSubClassType}

\FloatBarrier
\begin{table}[ht]
\centering 
  \caption{\texttt{UasbleSubClassType} Definition}
  \label{table:UasbleSubClassType}
\fontsize{9pt}{11pt}\selectfont
\tabulinesep=3pt
\begin{tabu} to 6in {|X[-1.35]|X[-0.7]|X[-1.75]|X[-1.5]|X[-1]|X[-0.7]|} \everyrow{\hline}
\hline
\rowfont\bfseries {Attribute} & \multicolumn{5}{|l|}{Value} \\
\tabucline[1.5pt]{}
BrowseName & \multicolumn{5}{|l|}{UasbleSubClassType} \\
IsAbstract & \multicolumn{5}{|l|}{False} \\
\tabucline[1.5pt]{}
\rowfont \bfseries References & NodeClass & BrowseName & DataType & Type\-Definition & {Modeling\-Rule} \\
\multicolumn{6}{|l|}{Subtype of MTDataItemSubClassType (See section \ref{type:MTDataItemSubClassType})} \\
\end{tabu}
\end{table} 


\FloatBarrier
\subsubsection{Defintion of \texttt{ VerticalSubClassType}}
  \label{type:VerticalSubClassType}

\FloatBarrier
\begin{table}[ht]
\centering 
  \caption{\texttt{VerticalSubClassType} Definition}
  \label{table:VerticalSubClassType}
\fontsize{9pt}{11pt}\selectfont
\tabulinesep=3pt
\begin{tabu} to 6in {|X[-1.35]|X[-0.7]|X[-1.75]|X[-1.5]|X[-1]|X[-0.7]|} \everyrow{\hline}
\hline
\rowfont\bfseries {Attribute} & \multicolumn{5}{|l|}{Value} \\
\tabucline[1.5pt]{}
BrowseName & \multicolumn{5}{|l|}{VerticalSubClassType} \\
IsAbstract & \multicolumn{5}{|l|}{False} \\
\tabucline[1.5pt]{}
\rowfont \bfseries References & NodeClass & BrowseName & DataType & Type\-Definition & {Modeling\-Rule} \\
\multicolumn{6}{|l|}{Subtype of MTDataItemSubClassType (See section \ref{type:MTDataItemSubClassType})} \\
\end{tabu}
\end{table} 


\FloatBarrier
\subsubsection{Defintion of \texttt{ VickersSubClassType}}
  \label{type:VickersSubClassType}

\FloatBarrier
\begin{table}[ht]
\centering 
  \caption{\texttt{VickersSubClassType} Definition}
  \label{table:VickersSubClassType}
\fontsize{9pt}{11pt}\selectfont
\tabulinesep=3pt
\begin{tabu} to 6in {|X[-1.35]|X[-0.7]|X[-1.75]|X[-1.5]|X[-1]|X[-0.7]|} \everyrow{\hline}
\hline
\rowfont\bfseries {Attribute} & \multicolumn{5}{|l|}{Value} \\
\tabucline[1.5pt]{}
BrowseName & \multicolumn{5}{|l|}{VickersSubClassType} \\
IsAbstract & \multicolumn{5}{|l|}{False} \\
\tabucline[1.5pt]{}
\rowfont \bfseries References & NodeClass & BrowseName & DataType & Type\-Definition & {Modeling\-Rule} \\
\multicolumn{6}{|l|}{Subtype of MTDataItemSubClassType (See section \ref{type:MTDataItemSubClassType})} \\
\end{tabu}
\end{table} 


\FloatBarrier
\subsubsection{Defintion of \texttt{ VolumeSubClassType}}
  \label{type:VolumeSubClassType}

\FloatBarrier
\begin{table}[ht]
\centering 
  \caption{\texttt{VolumeSubClassType} Definition}
  \label{table:VolumeSubClassType}
\fontsize{9pt}{11pt}\selectfont
\tabulinesep=3pt
\begin{tabu} to 6in {|X[-1.35]|X[-0.7]|X[-1.75]|X[-1.5]|X[-1]|X[-0.7]|} \everyrow{\hline}
\hline
\rowfont\bfseries {Attribute} & \multicolumn{5}{|l|}{Value} \\
\tabucline[1.5pt]{}
BrowseName & \multicolumn{5}{|l|}{VolumeSubClassType} \\
IsAbstract & \multicolumn{5}{|l|}{False} \\
\tabucline[1.5pt]{}
\rowfont \bfseries References & NodeClass & BrowseName & DataType & Type\-Definition & {Modeling\-Rule} \\
\multicolumn{6}{|l|}{Subtype of MTDataItemSubClassType (See section \ref{type:MTDataItemSubClassType})} \\
\end{tabu}
\end{table} 


\FloatBarrier
\subsubsection{Defintion of \texttt{ WeightSubClassType}}
  \label{type:WeightSubClassType}

\FloatBarrier
\begin{table}[ht]
\centering 
  \caption{\texttt{WeightSubClassType} Definition}
  \label{table:WeightSubClassType}
\fontsize{9pt}{11pt}\selectfont
\tabulinesep=3pt
\begin{tabu} to 6in {|X[-1.35]|X[-0.7]|X[-1.75]|X[-1.5]|X[-1]|X[-0.7]|} \everyrow{\hline}
\hline
\rowfont\bfseries {Attribute} & \multicolumn{5}{|l|}{Value} \\
\tabucline[1.5pt]{}
BrowseName & \multicolumn{5}{|l|}{WeightSubClassType} \\
IsAbstract & \multicolumn{5}{|l|}{False} \\
\tabucline[1.5pt]{}
\rowfont \bfseries References & NodeClass & BrowseName & DataType & Type\-Definition & {Modeling\-Rule} \\
\multicolumn{6}{|l|}{Subtype of MTDataItemSubClassType (See section \ref{type:MTDataItemSubClassType})} \\
\end{tabu}
\end{table} 


\FloatBarrier
\subsubsection{Defintion of \texttt{ WorkingSubClassType}}
  \label{type:WorkingSubClassType}

\FloatBarrier
\begin{table}[ht]
\centering 
  \caption{\texttt{WorkingSubClassType} Definition}
  \label{table:WorkingSubClassType}
\fontsize{9pt}{11pt}\selectfont
\tabulinesep=3pt
\begin{tabu} to 6in {|X[-1.35]|X[-0.7]|X[-1.75]|X[-1.5]|X[-1]|X[-0.7]|} \everyrow{\hline}
\hline
\rowfont\bfseries {Attribute} & \multicolumn{5}{|l|}{Value} \\
\tabucline[1.5pt]{}
BrowseName & \multicolumn{5}{|l|}{WorkingSubClassType} \\
IsAbstract & \multicolumn{5}{|l|}{False} \\
\tabucline[1.5pt]{}
\rowfont \bfseries References & NodeClass & BrowseName & DataType & Type\-Definition & {Modeling\-Rule} \\
\multicolumn{6}{|l|}{Subtype of MTDataItemSubClassType (See section \ref{type:MTDataItemSubClassType})} \\
\end{tabu}
\end{table} 


\FloatBarrier
\subsubsection{Defintion of \texttt{ WorkpieceSubClassType}}
  \label{type:WorkpieceSubClassType}

\FloatBarrier
\begin{table}[ht]
\centering 
  \caption{\texttt{WorkpieceSubClassType} Definition}
  \label{table:WorkpieceSubClassType}
\fontsize{9pt}{11pt}\selectfont
\tabulinesep=3pt
\begin{tabu} to 6in {|X[-1.35]|X[-0.7]|X[-1.75]|X[-1.5]|X[-1]|X[-0.7]|} \everyrow{\hline}
\hline
\rowfont\bfseries {Attribute} & \multicolumn{5}{|l|}{Value} \\
\tabucline[1.5pt]{}
BrowseName & \multicolumn{5}{|l|}{WorkpieceSubClassType} \\
IsAbstract & \multicolumn{5}{|l|}{False} \\
\tabucline[1.5pt]{}
\rowfont \bfseries References & NodeClass & BrowseName & DataType & Type\-Definition & {Modeling\-Rule} \\
\multicolumn{6}{|l|}{Subtype of MTDataItemSubClassType (See section \ref{type:MTDataItemSubClassType})} \\
\end{tabu}
\end{table} 


\FloatBarrier
\subsection{MTConnect Device Profile} \label{model:MTConnectDeviceProfile}

\begin{figure}[ht]
  \centering
    \includegraphics[width=1.0\textwidth]{./diagrams/types/MTConnectDeviceProfile.png}
  \caption{MTConnect Device Profile Diagram}
  \label{fig:MTConnectDeviceProfile}
\end{figure}

\FloatBarrier

\subsubsection{Defintion of \texttt{<<stereotype>> Deprecated}}
  \label{type:Deprecated}

\FloatBarrier
\FloatBarrier
\subsubsection{Defintion of \texttt{<<stereotype>> Dynamic Type}}
  \label{type:Dynamic Type}

\FloatBarrier
\FloatBarrier
\subsubsection{Defintion of \texttt{<<stereotype>> Mixes In}}
  \label{type:Mixes In}

\FloatBarrier
\FloatBarrier
\subsubsection{Defintion of \texttt{<<stereotype>> Mixes In}}
  \label{type:Mixes In}

\FloatBarrier
\FloatBarrier
\subsubsection{Defintion of \texttt{<<stereotype>> bind}}
  \label{type:bind}

\FloatBarrier
\FloatBarrier
\subsubsection{Defintion of \texttt{<<stereotype>> mixin}}
  \label{type:mixin}

\FloatBarrier
\FloatBarrier
\subsubsection{Defintion of \texttt{<<stereotype>> use}}
  \label{type:use}

\FloatBarrier
\FloatBarrier
\subsubsection{Defintion of \texttt{<<stereotype>> values}}
  \label{type:values}

\FloatBarrier
\FloatBarrier
