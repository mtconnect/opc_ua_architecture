% Generated 2019-05-02 19:41:05 -0700
\subsection{Components} \label{model:Components}

\begin{figure}[ht]
  \centering
    \includegraphics[width=1.0\textwidth]{./diagrams/types/Components.png}
  \caption{Components Diagram}
  \label{fig:Components}
\end{figure}

\FloatBarrier

\subsubsection{Defintion of \texttt{ MTChannelType}}
  \label{type:MTChannelType}

\FloatBarrier
\begin{table}[ht]
\centering 
  \caption{\texttt{MTChannelType} Definition}
  \label{table:MTChannelType}
\fontsize{9pt}{11pt}\selectfont
\tabulinesep=3pt
\begin{tabu} to 6in {|X[-1.35]|X[-0.7]|X[-1.75]|X[-1.5]|X[-1]|X[-0.7]|} \everyrow{\hline}
\hline
\rowfont\bfseries {Attribute} & \multicolumn{5}{|l|}{Value} \\
\tabucline[1.5pt]{}
BrowseName & \multicolumn{5}{|l|}{MTChannelType} \\
IsAbstract & \multicolumn{5}{|l|}{False} \\
\tabucline[1.5pt]{}
\rowfont \bfseries References & NodeClass & BrowseName & DataType & Type\-Definition & {Modeling\-Rule} \\
\multicolumn{6}{|l|}{Subtype of BaseObjectType (See \cite{UAPart05} Documentation)} \\
