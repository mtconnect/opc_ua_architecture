\usetikzlibrary{arrows}
\begin{figure}[ht]
\centering\scalebox{0.7}{
\begin{tikzpicture}[node distance=2cm, font=\small]
\tikzset{
   object/.style={
           rectangle,
           rounded corners,
           draw=black, very thick,
           minimum height=2em,
           inner sep=2pt,
           text centered           
           },
   class/.style={
           rectangle,
           draw=black, very thick,
           minimum height=2em,
           inner sep=2pt,
           text centered           
           },
   ref/.style={
           thick,->,>={Stealth[length=8pt,width=8pt,inset=4pt]},
           rounded corners
           },
   seg/.style={thick,rounded corners}
   }

\node[object] (axes) { %
\begin{tabular}{l l}
  \multicolumn{2}{c}{\underline{\textbf{Axes: AxesType}}} \\[4pt]
  \hline
  NodeId & "mtc_adapter002/Mazak01-base" \\
\end{tabular}
};


\node[object, left=3cm of axes.west] (device) { 
\underline{\textbf{SimpleCnc : \typeref{MTDeviceType}}}
};

\node[object,below=1.5cm of axes.south] (components) { %
\begin{tabular}{l l}
  \multicolumn{2}{c}{\underline{\textbf{Components: FolderType}}} \\[4pt]
  \hline
  NodeId & ".../Conditions" \\
\end{tabular}
};

\node[object, below=1.5cm of components] (linear) { %
\begin{tabular}{l l}
  \multicolumn{2}{c}{\underline{\textbf{Linear[X]: LinearType}}} \\[4pt]
  \hline
  NodeId & "mtc_adapter002/Mazak01-X" \\
\end{tabular}
};

\node[object,below left=1.5cm and 4cm of linear.south] (position) {
\begin{tabular}{l l}
  \multicolumn{2}{c}{\underline{\textbf{ActualPosition: \typeref[MTSampleType}}} \\[4pt]
  \hline
 EngineeringUnits & "MILLIMETER" \\
 EURange & -INF:INF\\
 Name & "Xabs" \\
 Category & "SAMPLE" \\
 MTTypeName & "POSITION" \\
 MTSubTypeName & "ACTUAL" \\
 Units & "MILLIMETER" \\
 NativeUnits & "MILLIMETER" \\
\end{tabular}
};

\node[class,below=6cm of position.north] (position-class) {
\textbf{PositionClassType}
};

\node[object,below right=1.5cm and 4cm of linear.south] (load) {
\begin{tabular}{l l}
  \multicolumn{2}{c}{\underline{\textbf{Load: \typeref{MTSampleType}}}} \\[4pt]
  \hline
 EngineeringUnits & "PERCENT" \\
 EURange & -INF:INF\\
 Name & "Xload" \\
 Category & "SAMPLE" \\
 MTTypeName & "LOAD" \\
 Units & "PERCENT" \\
 NativeUnits & "PERCENT" \\
\end{tabular}
};

\node[class,below=6cm of load.north] (load-class) {
\textbf{LoadClassType}
};

\node[object,below=1.5cm of linear] (position-cond) { %
\begin{tabular}{l l}
  \multicolumn{2}{c}{\underline{\textbf{PositionCondition: \typeref{MTConditionType}}}} \\[4pt]
  \hline
  <<Attribute>> NodeId & "e086dd60" \\
  ConditionClassId & (ns=1;i=365705) \\
  ConditionClassName & "PositionClassType" \\
  ConditionName & "Xtravel" \\
  BranchId & 0 \\
  Retain & false \\
  EnableState & True \\
  Quality & Good \\
  LastSeverity & 0 \\
  Comment & "" \\
  ClientUserId & "" \\
  NormalState & (ns=1;i=981658) \\
  Name & "Xtravel" \\
  MTTypeName & "POSITION" \\
  Category & "CONDITION" \\
  NativeCode & "" \\
  ActiveState & False \\
\end{tabular}
};

\draw[ref] (device) -- (axes)
  node[pos=0.5,above]{<<HasNotifier>>};  

\draw[ref] (axes) -- (components) 
  node[pos=0.5]{<<Organizes>>};

\draw[ref] (components) -- (linear) 
  node[pos=0.5]{<<HasComponent>>};
  
\draw[seg] (axes.east) -| ++(1,-1) coordinate (d1);
\draw[ref] (d1) |- (linear.east)
 node[pos=0.2]{<<HssNotifier>>};

\draw[ref] (position) -- (position-class) 
  node[pos=0.5]{<<HasMTClassType>>};

\draw[ref] (position-cond.-140) -| (position-class) 
  node[pos=0.5,below]{<<HasMTClassType>>};

\draw[ref] (load) -- (load-class) 
  node[pos=0.5]{<<HasMTClassType>>};

\draw[seg] (linear.south) |- ++(0,-0.75) coordinate (d2);
\draw[ref] (d2) -| (position.north)
  node[pos=0.5,above]{<<HasComponent>>};

\draw[seg] (linear.south) |- ++(0,-0.75) coordinate (d3);
\draw[ref] (d3) -| (load.north)
  node[pos=0.5,above]{<<HasComponent>>};
  
\draw[seg] (linear.south) -- (position-cond)
 node[pos=0.7]{<<HasComponent>>};

\draw[seg] (linear.-5) -| ++(6.5,-1) coordinate (d4)
 node[pos=0.3,above]{<<HasCondition>>};
\draw[ref] (d4) |- (position-cond.-40);

\end{tikzpicture}
}
\caption{Linear X Axis Example}
 \label{fig:linear-x-example-1}
\end{figure}