\section{Introduction}\label{intro}

\subsection{Background}

In September 2010, the OPC Foundation and the MTConnect Institute signed a memorandum of understanding to provide a mechanism for OPC and MTConnect to collaborate to extend the reach of the existing manufacturing data exchange standards and implementation technologies in order to:

\begin{itemize}
    \item Evolve the existing standards for each organization to provide complete manufacturing technology interoperability.
    \item Provide the mechanism for continuous improvement of standards and specifications overseen by each body.
    \item Work directly with the end users and suppliers of technology and manufacturing. 
    \item Provide a coordinating function to exchange insights, identify overlaps, and harmonize work where appropriate.
    \item Facilitate clear communication and education for users and others concerning possible overlaps and the ways the standards and specifications can be used.
    \item Provide a solid foundation to develop and deliver specifications, technology and processes to facilitate adoption of the technology into real products.
\end{itemize}

The outcome of that agreement was an initial companion specification called MTConnect-OPC UA Version 1.2.0. MTConnect-OPC UA companion specification describes an architecture for exchanging information for interoperability and consistency between MTConnect specifications and the OPC Unified Architecture (UA) specifications, as well as describing the manufacturing technology equipment, devices, software or other products that may implement those standards.

This document, OPC Unified Architecture for MTConnect Companion Specification Draft Version 2.0, provides an update to the original companion specification to include the latest capabilities and functionality of the standards provided by the MTConnect Institute and the OPC Foundation.

\subsection{MTConnect-OPC UA Goals}

The OPC Unified Architecture for MTConnect Companion Specification is designed with the following goals in mind, in the interest of wide and rapid adoption by vendors of equipment and software:

\begin{itemize}
    \item Incremental adoption:the technical barrier to MTConnect-OPC UA enablement will be greatly reduced with this companion specification and the source code and binaries available in the MTConnect-OPC UA reference port.
    \item Evolution: MTConnect and OPC UA can incrementally evolve without jeopardizing backwards compatibility of previous MTConnect-OPC UA versions.
    \item Customizability: MTConnect-OPC UA's extensibility enables integrators to create value-added software and tools that are machine-specific or installation-specific, without jeopardizing compatibility with other equipment or software.
    \item Non-proprietary: built on open standards, backed by both the OPC Foundation and the MTConnect Institute which represents hundreds of companies, individuals, government organizations and non-profits all working toward the goal of increased productivity in the manufacturing arena.
\end{itemize}

\subsection{Who Will Find Benefit from this Specification?}

To adopt the OPC Unified Architecture for MTConnect Companion Specification one will need to have a clear understanding of both MTConnect and OPC UA. From the technical side, we will discuss MTConnect-OPC UA from:

\begin{itemize}
    \item The backend or OPC UA Server and MTConnect agent/adapter architecture.
    \item The client or software application side, we will discuss how one develops an application that is MTConnect-OPC UA enabled.
    \item Applying MTConnect semantics to devices containing an embedded OPC UA Server.
\end{itemize}

From the business side, we will reference a companion business MTConnect-OPC UA white paper that addresses the concerns from the owners and top management of the business as well as the operations and engineering management. It is the objective of this white paper to provide information primarily to MTConnect and OPC UA software developers. We do not make assumptions about the level of programming expertise beyond what would be considered to be "reasonable" level of expertise. It is for this reason that we include enough details about both MTConnect and OPC UA to provide the ability to implement this companion specification without having references back to other documents. However, the OPC and MTConnect standards are critical and become much more meaningful with the appropriate overview from this document.

\subsection{References}

\printbibliography[title=OPC UA References,keyword=OPC]

\printbibliography[title=MTConnect References,keyword=MTC]

\printbibliography[title=Other References,notkeyword=MTC,notkeyword=OPC]

\subsection{Abbrevations}

The following abbreviation are used in this document:

\begin{itemize}
    \item ERP -- Enterprise Resource Planning
    \item HMI -- Human Machine Interface
    \item HTTP -- Hyper Text Transport Protocol
    \item MES -- Management Execution Systems
    \item PLC -- Programmable Logic Controller
    \item PMS -- Production Management Systems
    \item SCADA -- Supervisory Control And Data Acquisition
    \item TCP/IP -- Transmission Control Protocol/Internet Protocol
    \item XML -- eXtensible Mark-up Language
\end{itemize}

