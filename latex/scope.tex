\section{Scope}\label{scope}

In September 2010, the OPC Foundation and the MTConnect Institute signed a memorandum of understanding to extend the reach of existing manufacturing data exchange standards to:

\begin{itemize}
    \item Evolve the existing standards from each organization to provide complete manufacturing technology interoperability.
    \item Provide a mechanism for continuous improvement of those standards and specifications.
    \item Support the evolution of digital manufacturing systems.  
    \item Provide a coordinating function to harmonize work between the organizations.
    \item Educate customers and suppliers on the standards and specifications.
    \item Provide a foundation for adopting the standards, specifications, and associated technology into real products.
\end{itemize}

The first document produced was the MTConnect-OPC UA Companion Specification, Version 1.0 (2012), which defines a method for interoperability between the standards. It also identifies how the standards can be used together in manufacturing systems.

This document, OPC Unified Architecture for MTConnect Companion Specification, Version 2.0, updates the original companion specification and incorporates the latest capabilities and functions.   

The technologies provided from these two organizations include:

\quad\underline{OPC Foundation}

OPC is an interoperability standard for the secure and reliable exchange of data and information in the industrial automation space and in other industries.  It is platform independent and ensures the seamless flow of information among devices from multiple vendors. The OPC Foundation is responsible for the development and maintenance of this standard.
OPC UA is a platform independent service-oriented architecture that integrates all the functionality of the individual OPC Classic specifications into one extensible framework. This multi-layered approach accomplishes the original design specification goals of:
\begin{itemize}
    \item Platform independence: from an embedded microcontroller to cloud-based infrastructure
    \item Secure: encryption, authentication, authorization and auditing
    \item Extensible: ability to add new features including transports without affecting existing applications
    \item Comprehensive information modelling capabilities: for defining any model from simple to complex 
\end{itemize} 

\quad\underline{MTConnect Institute}



MTConnect is a data and information exchange standard that is based on a data dictionary of terms describing information associated with manufacturing operations.  The standard also defines a series of semantic data models that provide a clear and unambiguous representation of how that information relates to a manufacturing operation.  The MTConnect Standard has been designed to:
\begin{itemize}
    \item Enhance the data acquisition capabilities from equipment in manufacturing facilities
    \item Expand the use of data driven decision making in manufacturing operations
    \item Enable software applications and manufacturing equipment to move toward a plug-and-play environment to reduce the cost of integration of manufacturing software systems.  
\end{itemize}

The MTConnect Institute is responsible for development of the MTConnect Standard.  The Institute is a 501(c)(6) not-for-profit standards development organization and is a subsidiary of \gls{amt} - The Association For Manufacturing Technology. Its mission is to create open standards to foster greater interoperability between devices and clients by defining the structure and terminology used in communications in the discrete parts manufacturing sector.
%%% Local Variables:
%%% mode: latex
%%% TeX-master: "main"
%%% End:
