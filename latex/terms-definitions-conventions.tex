\section{Terms, Definitions and Conventions}\label{termsdefinitionsconventions}

\subsection{Overview}

The basic concepts of OPC UA and MTConnect are pre-requisites for understanding and interpreting the content provided in this companion specification. Additionally, the terms and definitions given in OPC UA Part 1, OPC UA Part 3, OPC UA Part 4, OPC UA Part 5, OPC UA Part 7, OPC UA Part 100, and MTConnect Part 1.0, as well as the following, apply to this document. 


\subsection{OPC UA for MTConnect terms}

\textit{NEEDS TO BE DISCUSSED}
The following terms (1 and 2) are examples. They have the IEC format for term definitions.

\subsubsection{term\_1}
<a short description – max two lines>

\begin{quote}
\footnotesize
Note 1 to entry: Optional additional text if the short description is not considered sufficient.
\end{quote}

\subsubsection{term\_2}
<a short description – max two lines>
....

\subsection{Abbreviations and Symbols}
The following abbreviation are used in this document:
\begin{itemize}
    \item OPC
    \item UA - Unified Architecture
    \item ERP – Enterprise Resource Planning
    \item HMI – Human Machine Interface
    \item HTTP – Hyper Text Transport Protocol
    \item MES – Management Execution Systems
    \item PLC – Programmable Logic Controller
    \item PMS - Production Management Systems
    \item SCADA - Supervisory Control And Data Acquisition
    \item TCP/IP - Transmission Control Protocol/Internet Protocol
    \item XML - eXtensible Mark-up Language
\end{itemize}

\subsection{Conventions}
Following are basic conventions that shall be followed for all formal definitions used.

\subsubsection{Conventions for Node descriptions}

\textit{Nodes} represent objects and their components in an \textit{AddressSpace}.
\textit{Attributes} are defined by providing the Attribute name and a value, or a description of the value.
\textit{References} are defined by providing the \textit{ReferenceType} name, the \textit{BrowseName} of the \textit{TargetNode} and its \textit{NodeClass}.

\begin{itemize}
    \item If the \textit{TargetNode} is a component of the Node being defined in the table the \textit{Attributes} of the composed Node are defined in the same row of the table. 
    \item The \textit{DataType} is only specified for Variables; "[<number>]" indicates a single-dimensional array, for multi-dimensional arrays the expression is repeated for each dimension (e.g. [2][3] for a two-dimensional array). For all arrays the \textit{ArrayDimensions} is set as identified by <number> values. If no <number> is set, the corresponding dimension is set to 0, indicating an unknown size. If no number is provided at all the \textit{ArrayDimensions} can be omitted. If no brackets are provided, it identifies a scalar \textit{DataType} and the \textit{ValueRank} is set to the corresponding value (see \cite{UAPart3}). In addition, \textit{ArrayDimensions} is set to null or is omitted. If it can be Any or ScalarOrOneDimension, the value is put into "{<value>}", so either "{Any}" or "{ScalarOrOneDimension}" and the \textit{ValueRank} is set to the corresponding value (see \cite{UAPart3}) and the \textit{ArrayDimensions} is set to null or is omitted.
    \item The \textit{TypeDefinition} is specified for Objects and Variables.
    \item The \textit{TypeDefinition} column specifies a symbolic name for a \textit{NodeId}, i.e. the specified Node points with a \textit{HasTypeDefinition} Reference to the corresponding Node.
    \item The \textit{ModellingRule} of the referenced component is provided by specifying the symbolic name of the rule in \textit{ModellingRule}. In the \textit{AddressSpace}, the Node shall use a \textit{HasModellingRule} Reference to point to the corresponding \textit{ModellingRule} Object.
\end{itemize}

If the \textit{NodeId} of a \textit{DataType} is provided, the symbolic name of the Node representing the \textit{DataType} shall be used.

Nodes of all other \textit{NodeClasses} cannot be defined in the same table; therefore only the used \textit{ReferenceType}, their \textit{NodeClass} and their \textit{BrowseName} are specified. A reference to another part of this document points to their definition.

If no components are provided, the \textit{DataType}, \textit{TypeDefinition} and \textit{ModellingRule} columns may be omitted and only a Comment column is introduced to point to the Node definition.

Components of \textit{Nodes} can be complex that is containing components by themselves. The \textit{TypeDefinition}, \textit{NodeClass}, \textit{DataType} and \textit{ModellingRule} can be derived from the type definitions, and the symbolic name can be created. Therefore, those containing components are not explicitly specified; they are implicitly specified by the type definitions.

\subsubsection{NodeIds and BrowseNames}

\paragraph{NodeIds}

The NodeIds of all Nodes described in this standard are only symbolic names. Annex A defines the actual NodeIds.
The symbolic name of each Node defined in this specification is its BrowseName, or, when it is part of another Node, the BrowseName of the other Node, a ".", and the BrowseName of itself. In this case "part of" means that the whole has a HasProperty or HasComponent Reference to its part. Since all Nodes not being part of another Node have a unique name in this specification, the symbolic name is unique.
The namespace for all NodeIds defined in this specification is defined in Annex A. The namespace for this NamespaceIndex is Server-specific and depends on the position of the namespace URI in the server namespace table.
Note that this specification not only defines concrete Nodes, but also requires that some Nodes shall be generated, for example one for each Session running on the Server. The NodeIds of those Nodes are Server-specific, including the namespace. But the NamespaceIndex of those Nodes cannot be the NamespaceIndex used for the Nodes defined in this specification, because they are not defined by this specification but generated by the Server.

\paragraph{BrowseNames}
The text part of the BrowseNames for all Nodes defined in this specification is specified in the tables defining the Nodes. The NamespaceIndex for all BrowseNames defined in this specification is defined in Annex A.

\subsubsection{Common Attributes}

\paragraph{General}
The Attributes of Nodes, their DataTypes and descriptions are defined in \cite{UAPart3}. Attributes not marked as optional are mandatory and shall be provided by a Server. The following tables define if the Attribute value is defined by this specification or if it is server-specific.

For all Nodes specified in this specification, the Attributes shall be set as per the OPC UA Standard.

\paragraph{Objects}

For all \textit{Objects} specified in this specification, the Attributes named in Table 4 shall be set as specified in the Table 4. The definitions for the Attributes can be found in OPC \cite{UAPart3}.

\paragraph{Variables}

For all \textit{Variables} specified in this specification, the Attributes named in Table 5 shall be set as specified in the table. The definitions for the Attributes can be found in \cite{UAPart3}.

\paragraph{VariableTypes}
For all \textit{VariableTypes} specified in this specification, the Attributes named in Table 6 shall be set as specified in the table. The definitions for the Attributes can be found in \cite{UAPart3}.

\paragraph{Methods}
For all \textit{Methods} specified in this specification, the Attributes named in Table 6 shall be set as specified in the table. The definitions for the Attributes can be found in \cite{UAPart3}.