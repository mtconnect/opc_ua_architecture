

%% OPC UA Terms

\newglossaryentry{Node}
{
  type=opc,
  name = {Node},
  description = {The elements of the UA model are represented in the AddressSpace as Nodes. Each Node is assigned to a NodeClass and each NodeClass represents a different element of the Object Model.}
}

\newglossaryentry{DataType}
{
  type=opc,
  category=code,
  name = {DataType},
  description = {instance of a DataType Node that is used together with the ValueRank Attribute to define the data type of a Variable}
}

\newglossaryentry{TargetNode}
{
  type=opc,
  name = {TargetNode},
  description = {a Node that is referenced by another Node}
}

\newglossaryentry{TypeDefinitionNode}
{
  type=opc,
  name = {TypeDefinitionNode},
  description = {Node that is used to define the type of another Node}
}


\newglossaryentry{InstanceDeclaration}
{
  type=opc,
  name = {InstanceDeclaration},
  description = {Node that is used by a complex \gls{TypeDefinitionNode} to expose its complex structure}
}

\newglossaryentry{ModellingRule}
{
  type=opc,
  category=code,
  name = {ModellingRule},
  description = {metadata of an InstanceDeclaration that defines how the \gls{InstanceDeclaration} will be used for instantiation and also defines subtyping rules for an \gls{InstanceDeclaration}}
}

\newglossaryentry{TypeDefinition}
{
  type=opc,
  name = {Type Definition},
  description = {The creation of an OPC UA Type}
}

\newglossaryentry{Server}
{
  type=opc,
  name=Server,
  description={An OPC UA Server Application}
}

\newglossaryentry{HaveTypeDefinition}
{
  type=opc,
  category=code,
  name = {HaveTypeDefinition},
  description = {The \gls{Reference} between the \gls{Object} and its \gls{Type}}
}

\newglossaryentry{Variable}
{
  type=opc,
  name = {Variable},
  description = {A UA Variable abstraction}
}

\newglossaryentry{ValueRank}
{
  type=opc,
  category=code,
  name = {ValueRank},
  description = {This Attribute indicates whether the Value Attribute of the Variable is an array and how many dimensions the array has.}
}

\newglossaryentry{NodeClass}
{
  type=opc,
  name=NodeClass,
  plural=NodeClasses,
  description = {The NodeClasses defined to represent Objects fall into three categories: those used to define instances, those used to define types for those instances and those used to define data types}
}

\newglossaryentry{BrowseName}
{
  type=opc,
  category=code,
  name=BrowseName,
  description={Nodes have a BrowseName Attribute that is used as a non-localised human-readable name when browsing the AddressSpace to create paths out of BrowseNames}
}

\newglossaryentry{ConditionType}
{
    type=opc,
    name=ConditionType,
    description={The ConditionType defines all general characteristics of a Condition. All other ConditionTypes derive from it}
}

\newglossaryentry{Condition}
{
    type=opc,
    name=Condition,
    description={The Condition model extends the Event model by defining the ConditionType}
}

\newglossaryentry{FolderType}
{
  type=opc,
  category=code,
  name=FolderType,
  description={Instances of FolderType are used to organise the AddressSpace into a hierarchy of Nodes}
}

\newglossaryentry{DataVariable}{
  type=opc,
  category=code,
  name=DataVariable,
  description={Variables that represent values of Objects, either directly or indirectly for complex Variables, where the Variables are always the TargetNode of a HasComponent Reference}
}

\newglossaryentry{HasTypeDefinition}{
  type=opc,
  category=code,
  name=HasTypeDefinition,
  description={The HasTypeDefinition ReferenceType is a concrete ReferenceType and can be used directly}
}

\newglossaryentry{Attribute}
{
  type=opc,
  name=Attribute,
  description={An intrinsic part of the OPC UA Node}
}

\newglossaryentry{Reference}
{
  type=opc,
  name=Reference,
  description={An OPC UA relation between two \glspl{Node}}
}

\newglossaryentry{NodeId}
{
  type=opc,
  category=code,
  name=NodeId,
  description={The unique identifier for a \gls{Node}}
}
  
\newglossaryentry{ArrayDimension}
{
  type=opc,
  name=ArrayDimension,
  description={The dimensions refers to the number of rows and columns of the associated array}
}

\newglossaryentry{AddressSpace}
{
  type=opc,
  name=AddressSpace,
  description={The instantiation of the all the instances of \glspl{Node} and \glspl{Reference}}
}

\newglossaryentry{HasModellingRule}
{
  type=opc,
  category=code,
  name=HasModellingRule,
  description={A \gls{Reference} to a specific modeling constraint that applies to the \gls{Type}}
}

\newglossaryentry{Type}
{
  type=opc,
  name=Type,
  description={A UA classification of \glspl{Object} or \glspl{Variable}}
}

\newglossaryentry{Object}
{
  type=opc,
  name=Object,
  description={An instantiation of a UA \gls{Type}}
}

\newglossaryentry{ObjectType}
{
  type=opc,
  name=ObjectType,
  description={The abstract \gls{Type} for all \glspl{Object}}
}

\newglossaryentry{ReferenceType}
{
  type=opc,
  category=code,
  name=ReferenceType,
  description={The \gls{Type} representing all \glspl{Reference} \glspl{Object}}
}

\newglossaryentry{ClassType}
{
  type=opc,
  category=code,
  name=ClassType,
  description={The \uamodel{ConditionClassId} \gls{Property} of the \uamodel{ConditionType} is used to assign a \uamodel{Condition} to a \uamodel{ConditionClass}. Clients can use this Property to filter out essential classes. OPC UA defines the base \gls{ObjectType} for all \uamodel{ConditionClasses} and a set of common classes used across many industries.}
}

\newglossaryentry{HasComponent}
{
  type=opc,
  category=code,
  name=HasComponent,
  description={A \gls{Reference} to an \gls{Object}}
}

\newglossaryentry{Property}
{
  type=opc,
  plural=Properties,
  name=Property,
  description={Variables that are the TargetNode for a \gls{HasProperty} Reference. Properties describe the characteristics of a Node.}
}

\newglossaryentry{HasSubtype}
{
  type=opc,
  category=code,
  name=HasSubtype,
  description={The HasSubtype ReferenceType is a concrete ReferenceType that can be used directly. It is a subtype of the HasChild ReferenceType}
}

\newglossaryentry{HasProperty}
{
  type=opc,
  category=code,
  name=HasProperty,
  description={A relationship of an \gls{Object} to a \gls{Variable} that represents a characteristic of the \gls{Object}}
}

\newglossaryentry{HierarchicalReference}
{
  type=opc,
  name={Hierarchical Reference},
  description={Reference that is used to construct hierarchies in the \gls{AddressSpace}. All hierarchical ReferenceTypes are derived from HierarchicalReferences}
}

\newglossaryentry{Organizes}
{
  type=opc,
  category=code,
  name=Organizes,
  description={A \gls{HierarchicalReference} relationship between an \gls{Object} and a collection of owned \glspl{Object}. Represented in \gls{uml} as \xml{<<Organizes>>} \gls{stereotype}}
}

\newglossaryentry{Event}
{
  type=opc,
  category=code,
  name=Event,
  description={An OPC UA Event}
}


%% MTConnect Terms

\newglossaryentry{Agent}
{
  type=mtc,
  name = {Agent},
  description = {The Middleware Broker and Protocol Server for the MTConnect Standard using a REST HTTP Interface}
}

\newglossaryentry{MTComponent}
{
  type=mtc,
  name = Component,
  description = {A functional unit of a piece of manufacturing equipment}
}

\newglossaryentry{MTDataItem}
{
  type=mtc,
  name = DataItem,
  description = {A piece of information regarding the state, event, or continuous variable related to a \gls{MTComponent} semantically identified by the \texttt{type} and \texttt{subType}}
}

\newglossaryentry{Adaper}
{
  type=mtc,
  name=Adapter,
  description = {An application that provides data from a piece of equipment to an MTConnect Agent}
}

\newglossaryentry{ControlledVocab}
{
  type=mtc,
  name={Controlled Vocabulary},
  plural = {Controlled Vocabularies},
  description = {A restricted set of values that may be published as the Valid Data Value for a Data Entity.}
}

\newglossaryentry{SampleRequest}
{
  type=mtc,
  name={Sample Request},
  description = {A request from the MTConnect \gls{Agent} for a stream of time series data.}
}

\newglossaryentry{DataItemType}
{
  type=mtc,
  name={DataItem type},
  description = {The top level taxonomy classification of \glspl{DataItem}}
}

\newglossaryentry{PascalCase}
{
  type=mtc,
  name={Pascal Case},
  description = {The first letter of each word is capitalized and the remaining letters are in lowercase. All space is removed between letters}
}

\newglossaryentry{LowerCamelCase}
{
  type=mtc,
  name={Lower Camel Case},
  description={the first word is lowercase and the remaining words are capitalized and all spaces between words are removed.}
}


\newglossaryentry{MTDevice}
{
  type=mtc,
  name=Device,
  description={The top level component in an MTConnect piece of equipment. Represented as an \mtuatype{MTDeviceType}}
}

\newglossaryentry{Configuration}
{
  type=mtc,
  name=Configuration,
  description={The specifications about a specific component in the MTConnect inforamtion model}
}

\newglossaryentry{Asset}
{
  type=mtc,
  name=Asset,
  description={A complex information model relating to an entity in the manufacturing process that does not directly supply data}
}

\newglossaryentry{Composition}
{
  type=mtc,
  name=Composition,
  description={A sub-classification of a component that represents part of the component heirarchy}
}

\newglossaryentry{type}
{
  type=mtc,
  category=code,
  name=type,
  description={The top level taxonomy classification used in the \glstext{xml} document}
}

\newglossaryentry{subType}
{
  type=mtc,
  category=code,
  name=subType,
  description={The second level taxonomy classification used in the \glstext{xml} document}
}

\newglossaryentry{UMLAssociation}
{
  type=mtc,
  name=UMLAssociation,
  description={A relationship between UML Classes}
}

\newglossaryentry{probe}
{
  type=mtc,
  name={probe request},
  description={An \glstext{http} request to the \gls{Agent} for returning metadata as an \xml{MTConnectDevices} \glstext{xml} document}
}

\newglossaryentry{current}
{
  type=mtc,
  name={current request},
  description={An \glstext{http} request to the \gls{Agent} for returning latest known values for the \glspl{MTDataItem} as an \xml{MTConnectStreams} \glstext{xml} document}
}

\newglossaryentry{sample}
{
  type=mtc,
  name={sample request},
  description={An \glstext{http} request to the \gls{Agent} for returning time series of values for the \glspl{MTDataItem} as an \xml{MTConnectStreams} \glstext{xml} document}
}

\newglossaryentry{Realization}
{
  type=mtc,
  name=Realization,
  description={Realization is a specialized abstraction relationship between two sets of model elements, one representing a specification (the supplier) and the other represents an implementation of the latter (the client). Realization can be used to model stepwise refinement, optimizations, transformations, templates, model synthesis, framework composition, etc.}
}

\newglossaryentry{mixes-in}
{
  type=mtc,
  name={<<Mixes In>>},
  description={A software architecture pattern represented as a \gls{stereotype} that combines the properties of one class with another. Similar to a sub-class, it allows for inheritance in single inheritance type systems.}
}

\newglossaryentry{stereotype}
{
  type=mtc,
  name=stereotype,
  description={A profile class which defines how an existing metaclass may be extended as part of a profile. It enables the use of a platform or domain specific terminology or notation in place of, or in addition to, the ones used for the extended metaclass. A stereotype is denoted by <<[name]>>.}
}

\newglossaryentry{Slot}
{
  type=mtc,
  name=Slot,
  description={Slot is UML element which specifies that an instance has a value or values for a specific structural feature. UML specification also says that a slot gives the value or values of a structural feature of the instance. An instance can have one slot per structural feature of its classifiers, including inherited features.}
}

\newglossaryentry{CDATA}
{
  type=mtc,
  category=code,
  name=CDATA,
  description={The blocks of text that are not parsed by the parser, but are otherwise recognized as markup. The predefined entities such as \&lt;, \&gt;, and \&amp; require typing and are generally difficult to read in the markup}
}

\newglossaryentry{QName}
{
  type=mtc,
  category=code,
  name=QName,
  description={A QName, or qualified name, is the fully qualified name of an element, attribute, or identifier in an XML document. A QName concisely associates the URI of an XML namespace with the local name of an element, attribute, or identifier in that namespace.}
}

\newglossaryentry{category}
{
  type=mtc,
  category=code,
  name=category,
  description={The top level classification of all MTConnect \glspl{MTDataItem}}
}


\newglossaryentry{MTEvent}
{
  type=mtc,
  category=code,
  name=Event,
  description={A category of MTConnect \gls{MTDataItem} that represents discrete or state related data from the device}
}

\newglossaryentry{Sample}
{
  type=mtc,
  category=code,
  name=Sample,
  description={A category of MTConnect \gls{MTDataItem} that data that represents a numeric continuous variable}
}

\newglossaryentry{MTCondition}
{
  type=mtc,
  category=code,
  name=Condition,
  description={A category of MTConnect \gls{MTDataItem} that data that represents the alarms and functional state of a component with respect to a \gls{type}}
}

\newglossaryentry{ontology}
{
  type=mtc,
  name=ontology,
  description={logical structure of the terms used to describe a domain of knowledge, including both the definitions of the applicable terms and their relationships ISO 20534:2018}
}

\newglossaryentry{modbus}
{
  type=mtc,
  name=MODBUS,
  description={Modbus is a communication protocol developed by Modicon systems and is a method used for transmitting information over serial lines between electronic devices}
}

\newglossaryentry{interfaces}
{
  type=mtc,
  name=interfaces,
  description={\cite{MTCPart5} provides an interaction model for coordinating activities between manufacturing devices}
}

\newglossaryentry{Constraint}
{
  type=mtc,
  name=Constraint,
  description={a restriction on the values or range of a \gls{MTDataItem}}
}

\newglossaryentry{buffer}
{
  type=mtc,
  name=buffer,
  description={a set of entities that are stored in memory often limited in size}
}

\newglossaryentry{TimeSeries}
{
  type=mtc,
  name={Time Series},
  description={A \gls{MTDataItem} representation of a contiguous vector of values supporting high frequency data rates}
}


%%% Local Variables:
%%% mode: latex
%%% TeX-master: "main"
%%% End:
