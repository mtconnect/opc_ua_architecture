\subsection{MixinType} \label{model:MixinType}
\subsubsection{Defintion of \texttt{ ComponentType}}
  \label{type:ComponentType}

\FloatBarrier

ComponentType Docs

\begin{table}[ht]
\centering 
  \caption{\texttt{ComponentType} Definition}
  \label{table:ComponentType}
\fontsize{9pt}{11pt}\selectfont
\tabulinesep=3pt
\begin{tabu} to 6in {|X[-1.35]|X[-0.7]|X[-1.75]|X[-1.5]|X[-1]|X[-0.7]|} \everyrow{\hline}
\hline
\rowfont\bfseries {Attribute} & \multicolumn{5}{|l|}{Value} \\
\tabucline[1.5pt]{}
BrowseName & \multicolumn{5}{|l|}{ComponentType} \\
IsAbstract & \multicolumn{5}{|l|}{True} \\
\tabucline[1.5pt]{}
\rowfont \bfseries References & NodeClass & BrowseName & DataType & Type\-Definition & {Modeling\-Rule} \\
\multicolumn{6}{|l|}{Subtype of BaseObjectType (See \cite{UAPart05} Documentation)} \\
Has\-Property & Variable & id & Id\-Type & Property\-Type & Mandatory \\
Has\-Property & Variable & name & String & Property\-Type & Optional \\
Has\-Property & Variable & uuid & String & Property\-Type & Optional \\
Has\-Notifier & Object & <Base\-Object> & \multicolumn{2}{l|}{BaseObjectType} & Optional \\
Has\-Component & Variable & Active\-State & Base\-Data\-Type & Two\-State\-Variable\-Type & Mandatory \\
Organizes & Object & Compositions & Composition\-Type[] & Folder\-Type & Optional \\
Has\-Component & Variable & <Integer\-Event> & Int32[] & Integer\-Event\-Type & Optional \\
\end{tabu}
\end{table} 


\FloatBarrier
\paragraph{Referenced Properties and Objects}

\begin{itemize}
\item \texttt{name::String:} Optional name

\item \texttt{uuid::String:} Only required with Device

\end{itemize}
\FloatBarrier
\subsubsection{Defintion of \texttt{ CompositionFolder}}
  \label{type:CompositionFolder}

\FloatBarrier
\begin{table}[ht]
\centering 
  \caption{\texttt{CompositionFolder} Definition}
  \label{table:CompositionFolder}
\fontsize{9pt}{11pt}\selectfont
\tabulinesep=3pt
\begin{tabu} to 6in {|X[-1.35]|X[-0.7]|X[-1.75]|X[-1.5]|X[-1]|X[-0.7]|} \everyrow{\hline}
\hline
\rowfont\bfseries {Attribute} & \multicolumn{5}{|l|}{Value} \\
\tabucline[1.5pt]{}
BrowseName & \multicolumn{5}{|l|}{CompositionFolder} \\
IsAbstract & \multicolumn{5}{|l|}{False} \\
\tabucline[1.5pt]{}
\rowfont \bfseries References & NodeClass & BrowseName & DataType & Type\-Definition & {Modeling\-Rule} \\
\multicolumn{6}{|l|}{Subtype of FolderType (See \cite{UAPart05} Documentation)} \\
Has\-Property & Variable & Count & Integer & Property\-Type & Mandatory \\
Organizes & Object & <Folder> & Composition\-Type & Folder\-Type & Mandatory \\
\end{tabu}
\end{table} 


\FloatBarrier
\subsubsection{Defintion of \texttt{ CompositionItem}}
  \label{type:CompositionItem}

\FloatBarrier
\begin{table}[ht]
\centering 
  \caption{\texttt{CompositionItem} Definition}
  \label{table:CompositionItem}
\fontsize{9pt}{11pt}\selectfont
\tabulinesep=3pt
\begin{tabu} to 6in {|X[-1.35]|X[-0.7]|X[-1.75]|X[-1.5]|X[-1]|X[-0.7]|} \everyrow{\hline}
\hline
\rowfont\bfseries {Attribute} & \multicolumn{5}{|l|}{Value} \\
\tabucline[1.5pt]{}
BrowseName & \multicolumn{5}{|l|}{CompositionItem} \\
IsAbstract & \multicolumn{5}{|l|}{False} \\
\tabucline[1.5pt]{}
\rowfont \bfseries References & NodeClass & BrowseName & DataType & Type\-Definition & {Modeling\-Rule} \\
\multicolumn{6}{|l|}{Subtype of BaseObjectType (See \cite{UAPart05} Documentation)} \\
Has\-Property & Variable & id & String & Property\-Type & Mandatory \\
Has\-Property & Variable & name & String & Property\-Type & Mandatory \\
\end{tabu}
\end{table} 


\FloatBarrier
\subsubsection{Defintion of \texttt{ CompositionType}}
  \label{type:CompositionType}

\FloatBarrier
\begin{table}[ht]
\centering 
  \caption{\texttt{CompositionType} Definition}
  \label{table:CompositionType}
\fontsize{9pt}{11pt}\selectfont
\tabulinesep=3pt
\begin{tabu} to 6in {|X[-1.35]|X[-0.7]|X[-1.75]|X[-1.5]|X[-1]|X[-0.7]|} \everyrow{\hline}
\hline
\rowfont\bfseries {Attribute} & \multicolumn{5}{|l|}{Value} \\
\tabucline[1.5pt]{}
BrowseName & \multicolumn{5}{|l|}{CompositionType} \\
IsAbstract & \multicolumn{5}{|l|}{False} \\
\tabucline[1.5pt]{}
\rowfont \bfseries References & NodeClass & BrowseName & DataType & Type\-Definition & {Modeling\-Rule} \\
\multicolumn{6}{|l|}{Subtype of BaseObjectType (See \cite{UAPart05} Documentation)} \\
Has\-Property & Variable & type & String & Property\-Type & Mandatory \\
Has\-Property & Variable & Enum\-Strings & Vocab & Vocab & Mandatory \\
Organizes & Object & Items & Composition\-Item[] & Folder\-Type & Optional \\
\end{tabu}
\end{table} 


\FloatBarrier
\paragraph{Referenced Properties and Objects}

\begin{itemize}
\item \textbf{Allowable Values} for \texttt{Vocab}
\FloatBarrier
\begin{table}[ht]
\centering 
  \caption{\texttt{Vocab} Enumeration}
  \label{enum:Vocab}
\tabulinesep=3pt
\begin{tabu} to 6in {|l|r|} \everyrow{\hline}
\hline
\rowfont\bfseries {Name} & {Index} \\
\tabucline[1.5pt]{}
\texttt{CAT} & \texttt{0} \\
\texttt{DOG} & \texttt{1} \\
\texttt{HORSE} & \texttt{2} \\
\end{tabu}
\end{table} 
\FloatBarrier
\end{itemize}
\FloatBarrier
\subsubsection{Defintion of \texttt{<<mixin>> DataItemMixin}}
  \label{type:DataItemMixin}

\FloatBarrier
\begin{table}[ht]
\centering 
  \caption{\texttt{DataItemMixin} Definition}
  \label{table:DataItemMixin}
\fontsize{9pt}{11pt}\selectfont
\tabulinesep=3pt
\begin{tabu} to 6in {|X[-1.35]|X[-0.7]|X[-1.75]|X[-1.5]|X[-1]|X[-0.7]|} \everyrow{\hline}
\hline
\rowfont\bfseries {Attribute} & \multicolumn{5}{|l|}{Value} \\
\tabucline[1.5pt]{}
BrowseName & \multicolumn{5}{|l|}{DataItemMixin} \\
IsAbstract & \multicolumn{5}{|l|}{False} \\
\tabucline[1.5pt]{}
\rowfont \bfseries References & NodeClass & BrowseName & DataType & Type\-Definition & {Modeling\-Rule} \\
Has\-Property & Variable & Category & String & Property\-Type & Mandatory \\
Has\-Property & Variable & Name & String & Property\-Type & Optional \\
Has\-Property & Variable & Native\-Name & String & Property\-Type & Optional \\
Has\-Property & Variable & Sub\-Type & String & Property\-Type & Mandatory \\
Has\-Property & Variable & Type & String & Property\-Type & Mandatory \\
Has\-Property & Variable & Xml\-Id & String & Property\-Type & Mandatory \\
\end{tabu}
\end{table} 


\FloatBarrier
\subsubsection{Defintion of \texttt{ IntegerEventType}}
  \label{type:IntegerEventType}

\FloatBarrier
\begin{table}[ht]
\centering 
  \caption{\texttt{IntegerEventType} Definition}
  \label{table:IntegerEventType}
\fontsize{9pt}{11pt}\selectfont
\tabulinesep=3pt
\begin{tabu} to 6in {|X[-1.35]|X[-0.7]|X[-1.75]|X[-1.5]|X[-1]|X[-0.7]|} \everyrow{\hline}
\hline
\rowfont\bfseries {Attribute} & \multicolumn{5}{|l|}{Value} \\
\tabucline[1.5pt]{}
BrowseName & \multicolumn{5}{|l|}{IntegerEventType} \\
IsAbstract & \multicolumn{5}{|l|}{False} \\
ValueRank & \multicolumn{5}{|l|}{-1} \\
DataType & \multicolumn{5}{|l|}{Int32} \\
\tabucline[1.5pt]{}
\rowfont \bfseries References & NodeClass & BrowseName & DataType & Type\-Definition & {Modeling\-Rule} \\
\multicolumn{6}{|l|}{Subtype of DataItemType (See \cite{UAPart08} Documentation)} \\
Has\-Property & Variable & Category & String & Property\-Type & Mandatory \\
Has\-Property & Variable & Name & String & Property\-Type & Optional \\
Has\-Property & Variable & Native\-Name & String & Property\-Type & Optional \\
Has\-Property & Variable & Sub\-Type & String & Property\-Type & Mandatory \\
Has\-Property & Variable & Type & String & Property\-Type & Mandatory \\
Has\-Property & Variable & Xml\-Id & String & Property\-Type & Mandatory \\
Has\-Property & Variable & EUInformation & EUInformation & Property\-Type & Optional \\
\end{tabu}
\end{table} 


\FloatBarrier
\paragraph{Referenced Properties and Objects}

\begin{itemize}
\item \texttt{Supertype::DataItemType:} For EVENT with Integer

\end{itemize}
\paragraph{Dependencies and Relationships}

\begin{itemize}
\item Mixes in \texttt{DataItemMixin}, see See section \ref{type:DataItemMixin}
\end{itemize}
\FloatBarrier
\subsubsection{Defintion of \texttt{ ThreeSpace}}
  \label{type:ThreeSpace}

\FloatBarrier
\begin{table}[ht]
\centering 
  \caption{\texttt{ThreeSpace} DataType}
  \label{data-type:ThreeSpace}
\tabulinesep=3pt
\begin{tabu} to 6in {|l|l|l|} \everyrow{\hline}
\hline
\rowfont\bfseries {Field} & {Type} & {Optional} \\
\tabucline[1.5pt]{}
\texttt{X} & \texttt{Double} & \texttt{Mandatory} \\
\texttt{Y} & \texttt{Double} & \texttt{Mandatory} \\
\texttt{Z} & \texttt{Double} & \texttt{Mandatory} \\
\end{tabu}
\end{table} 

\FloatBarrier
\paragraph{Data Type Fields}

\begin{itemize}
\item \texttt{X::Double:} Inf if null

\item \texttt{Y::Double:} Inf if null

\item \texttt{Z::Double:} Inf if null

\end{itemize}
\FloatBarrier
\subsubsection{Defintion of \texttt{ ThreeSpaceSampleType}}
  \label{type:ThreeSpaceSampleType}

\FloatBarrier
\begin{table}[ht]
\centering 
  \caption{\texttt{ThreeSpaceSampleType} Definition}
  \label{table:ThreeSpaceSampleType}
\fontsize{9pt}{11pt}\selectfont
\tabulinesep=3pt
\begin{tabu} to 6in {|X[-1.35]|X[-0.7]|X[-1.75]|X[-1.5]|X[-1]|X[-0.7]|} \everyrow{\hline}
\hline
\rowfont\bfseries {Attribute} & \multicolumn{5}{|l|}{Value} \\
\tabucline[1.5pt]{}
BrowseName & \multicolumn{5}{|l|}{ThreeSpaceSampleType} \\
IsAbstract & \multicolumn{5}{|l|}{False} \\
ValueRank & \multicolumn{5}{|l|}{-1} \\
DataType & \multicolumn{5}{|l|}{ThreeSpace} \\
\tabucline[1.5pt]{}
\rowfont \bfseries References & NodeClass & BrowseName & DataType & Type\-Definition & {Modeling\-Rule} \\
\multicolumn{6}{|l|}{Subtype of DataItemType (See \cite{UAPart08} Documentation)} \\
Has\-Property & Variable & Category & String & Property\-Type & Mandatory \\
Has\-Property & Variable & Name & String & Property\-Type & Optional \\
Has\-Property & Variable & Native\-Name & String & Property\-Type & Optional \\
Has\-Property & Variable & Sub\-Type & String & Property\-Type & Mandatory \\
Has\-Property & Variable & Type & String & Property\-Type & Mandatory \\
Has\-Property & Variable & Xml\-Id & String & Property\-Type & Mandatory \\
\end{tabu}
\end{table} 


\paragraph{Dependencies and Relationships}

\begin{itemize}
\item Mixes in \texttt{DataItemMixin}, see See section \ref{type:DataItemMixin}
\end{itemize}
\FloatBarrier
