\subsection{SimpleType} \label{model:SimpleType}
\subsubsection{Defintion of \texttt{ ComponentType}}
  \label{type:ComponentType}

\FloatBarrier

The base \gls{MTComponent} Type from which all MTConnect Components are derived.
The component types will be created once for all \gls{MTComponent} \glspl{Object}
of that type based on the \gls{QName} of the MTConnect XML element. 

The Component Objects will be created and inserted into the \mtmodel{Components} 
folder with a \gls{BrowseName} of the Component \gls{QName} and the \mtmodel{name} element if specified surrounded 
by square brackets, \texttt{[]}. For example if the MTConnect Element is:

\xml{<Linear name='X'>...</...>}

The OPC UA Object with \gls{BrowseName} \xml{Linear[X]} will be created with the \uamodel{HasTypeDefinition}
referencing the \mtmodel{Linear} OPC UA \gls{Type}. 

The meta data for the component and its relationships are static. The dynamic data will be 
represented using the \cite{UAPart8}.

\begin{table}[ht]
\centering 
  \caption{\texttt{ComponentType} Definition}
  \label{table:ComponentType}
\fontsize{9pt}{11pt}\selectfont
\tabulinesep=3pt
\begin{tabu} to 6in {|X[-1.35]|X[-0.7]|X[-1.75]|X[-1.5]|X[-1]|X[-0.7]|} \everyrow{\hline}
\hline
\rowfont\bfseries {Attribute} & \multicolumn{5}{|l|}{Value} \\
\tabucline[1.5pt]{}
BrowseName & \multicolumn{5}{|l|}{ComponentType} \\
IsAbstract & \multicolumn{5}{|l|}{False} \\
\tabucline[1.5pt]{}
\rowfont \bfseries References & NodeClass & BrowseName & DataType & Type\-Definition & {Modeling\-Rule} \\
\multicolumn{6}{|l|}{Subtype of BaseObjectType (See \cite{UAPart05} Documentation)} \\
Has\-Property & Variable & id & Id\-Type & Property\-Type & Mandatory \\
Has\-Property & Variable & name & String & Property\-Type & Mandatory \\
Has\-Property & Variable & uuid & String & Property\-Type & Mandatory \\
Organizes & Object & Compositions & Composition\-Type[] & Folder\-Type & Optional \\
\end{tabu}
\end{table} 


\FloatBarrier
\paragraph{Referenced Properties and Objects}

\begin{itemize}
\item \texttt{name::String:} Optional name

\item \texttt{uuid::String:} Only required with Device

\end{itemize}
\paragraph{Dependencies and Relationships}

\begin{itemize}
\item Dependency on FolderType

This class relates to \texttt{FolderType} (See section \ref{type:FolderType}) for a(n) \texttt{Organizes} relationship.

\end{itemize}
\FloatBarrier
\subsubsection{Defintion of \texttt{ CompositionType}}
  \label{type:CompositionType}

\FloatBarrier
\begin{table}[ht]
\centering 
  \caption{\texttt{CompositionType} Definition}
  \label{table:CompositionType}
\fontsize{9pt}{11pt}\selectfont
\tabulinesep=3pt
\begin{tabu} to 6in {|X[-1.35]|X[-0.7]|X[-1.75]|X[-1.5]|X[-1]|X[-0.7]|} \everyrow{\hline}
\hline
\rowfont\bfseries {Attribute} & \multicolumn{5}{|l|}{Value} \\
\tabucline[1.5pt]{}
BrowseName & \multicolumn{5}{|l|}{CompositionType} \\
IsAbstract & \multicolumn{5}{|l|}{False} \\
\tabucline[1.5pt]{}
\rowfont \bfseries References & NodeClass & BrowseName & DataType & Type\-Definition & {Modeling\-Rule} \\
\multicolumn{6}{|l|}{Subtype of BaseObjectType (See \cite{UAPart05} Documentation)} \\
\end{tabu}
\end{table} 


\FloatBarrier
